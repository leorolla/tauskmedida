% !TeX spellcheck = pt_BR
% !TEX encoding = UTF-8 Unicode

\NeedsTeXFormat{LaTeX2e}

\documentclass[oneside,final,11pt]{amsbook}

\usepackage[]{bookml}

\usepackage[brazil]{babel}
\usepackage[utf8]{inputenc}

%\usepackage{multind}
\usepackage{imakeidx}

\usepackage{dsfont}

%\makeindex{A}
%\makeindex{B}

\iflatexml
\makeindex
\else
\makeindex[name=simbolos, title=Lista de Símbolos,columns=2, options=-s simbolos]{}
\makeindex[name=indice,title=Índice Remissivo,columns=2, options=-s medida]
% This fixes a tiny little conflict between amsbook and imakeidx in the page heading:
\makeatletter
\def\imki@indexheaders{%
  \begingroup\edef\x{\endgroup
    \noexpand\@mkboth{\noexpand\MakeUppercase{\indexname}}{\noexpand\MakeUppercase{\indexname}}%
  }\x
}
\makeatother
\fi

%\newcommand{\boldR}{I\!R}
%\newcommand{\R}{I\!\!R}
%\newcommand{\C}{\mathbb C}
%\newcommand{\N}{I\!\!N}
%\newcommand{\Q}{\mathbb Q}
%\newcommand{\Z}{\mathbb Z}
%\newcommand{\K}{\mathbb K}

\newcommand{\R}{\mathds R}
\newcommand{\C}{\mathds C}
\newcommand{\boldR}{\pmb{\mathds R}}
\newcommand{\N}{\mathds N}
\newcommand{\Q}{\mathds Q}
\newcommand{\Z}{\mathds Z}
\newcommand{\K}{\mathds K}

\newcommand{\Times}{\pmb{\times}}
\newcommand{\leb}{\mathfrak m}
\newcommand{\Int}{\mathrm{int}}
\newcommand{\compl}{\mathrm c}
\newcommand{\Img}{\mathrm{Im}}
\newcommand{\infsup}{\mathop{\vphantom{\sup}\inf}\limits}
\newcommand{\dd}{\mathrm d}
\newcommand{\Gr}{\mathrm{gr}}
\newcommand{\qs}{\hbox{q.$\,$s.}}
\newcommand{\sen}{\mathrm{sen}}
\newcommand{\Dim}{\mathrm{dim}}
\newcommand{\Bounded}{\mathrm{Bd}}
\newcommand{\Cb}{C_{\mathrm b}}
\newcommand{\Sup}{\mathrm{sup}}
\newcommand{\Lin}{\mathrm{Lin}}
\newcommand{\Ker}{\mathrm{Ker}}
\newcommand{\ca}{\mathfrak c}

\newcommand{\chilow}[1]{\chi_{\lower2pt\hbox{$\scriptstyle#1$}}}
\newcommand{\subchilow}[1]{\chi_{\lower2pt\hbox{$\scriptscriptstyle#1$}}}

\newcommand{\norma}[2]{\Vert#1\Vert_{\lower2pt\hbox{$\scriptstyle#2$}}}
\newcommand{\bignorma}[2]{\big\Vert#1\big\Vert_{\lower2pt\hbox{$\scriptstyle#2$}}}
\newcommand{\subnorma}[2]{\Vert#1\Vert_{\lower2pt\hbox{$\scriptscriptstyle#2$}}}

\newcommand{\To}[1]{\xrightarrow{\;\mathrm{#1}\;}}

\newcommand{\intinfd}{\raise2pt\hbox{$\scriptstyle(R)$}\!\!\int_-}
\newcommand{\intsupd}{\raise2pt\hbox{$\scriptstyle(R)$}\!\!\int^-}
\newcommand{\intRd}{\raise2pt\hbox{$\scriptstyle(R)$}\!\!\int}
\newcommand{\intinf}{\raise2pt\hbox{$\scriptscriptstyle(R)$}\!\int_-}
\newcommand{\intsup}{\raise2pt\hbox{$\scriptscriptstyle(R)$}\!\int^-}
\newcommand{\intR}{\raise2pt\hbox{$\scriptscriptstyle(R)$}\!\int}

\newcommand{\Borel}{\mathcal B}
\newcommand{\Lebmens}{\mathcal M}

\newcounter{contador}
\newenvironment{bulletindent}{\setcounter{contador}{0}
\begin{list} {$\bullet$}
{\usecounter{contador}
\setlength{\leftmargin}{10pt}
\setlength{\rightmargin}{10pt}
\setlength{\labelsep}{5pt}
\setlength{\itemsep}{10pt}
\setlength{\topsep}{10pt}}}
{\end{list}}

\newcounter{contastep}
\newenvironment{stepindent}{\setcounter{contastep}{1}
\begin{list} {Passo \arabic{contastep}.}
{\usecounter{contastep}
\setlength{\leftmargin}{10pt}
\setlength{\rightmargin}{10pt}
\setlength{\labelsep}{5pt}
\setlength{\itemsep}{10pt}
\setlength{\topsep}{10pt}}}
{\end{list}}

\title{Notas Para o Curso de Medida e Integração}

\author{Daniel V. Tausk}

\date{9 de agosto de 2004}

\theoremstyle{remark}\newtheorem{exercise}{Exercício}[chapter]
\theoremstyle{remark}\newtheorem{*exercise}[exercise]{\hbox to 0pt{\hskip 0pt minus 1fil*}Exercício}
\theoremstyle{definition}\newtheorem{exdefin}{Definição}[chapter]
\swapnumbers
\theoremstyle{plain}\newtheorem{teo}{Teorema}[section]
\theoremstyle{plain}\newtheorem{lem}[teo]{Lema}
\theoremstyle{plain}\newtheorem{prop}[teo]{Proposição}
\theoremstyle{plain}\newtheorem{cor}[teo]{Corolário}
\theoremstyle{definition}\newtheorem{defin}[teo]{Definição}
\theoremstyle{remark}\newtheorem{rem}[teo]{Observação}
\theoremstyle{definition}\newtheorem{notation}[teo]{Notação}
\theoremstyle{definition}\newtheorem{convention}[teo]{Convenção}
\theoremstyle{definition}\newtheorem{example}[teo]{Exemplo}

\numberwithin{section}{chapter}
\numberwithin{equation}{section}

\begin{document}

\frontmatter
\maketitle

\chapter*{Licença e Créditos}

\noindent
\copyright\ 2004 Daniel V. Tausk.
\\
Permitido o uso nos termos da licença CC~BY-SA 4.0.
\\

\noindent
\textbf{Créditos}
\\
Para fazer atribuição em derivados deste material, incluir a referência:
\\
D. V. Tausk. Notas para o Curso de Medida e Integração (BY-SA).
\\
com um link para
https://github.com/leorolla/tauskmedida

\renewcommand{\contentsline}[3]{\csname novo#1\endcsname{#2}{#3}}
\newcommand{\novochapter}[2]{\bigskip\hbox to \hsize{\vbox{\advance\hsize by -1cm\baselineskip=12pt\parfillskip=0pt\leftskip=3cm\noindent\hskip -2cm #1\leaders\hbox{.}\hfil\hfil\par}$\,$#2\hfil}}
\newcommand{\novosection}[2]{\medskip\hbox to \hsize{\vbox{\advance\hsize by -1cm\baselineskip=12pt\parfillskip=0pt\leftskip=3.5cm\noindent\hskip -2cm #1\leaders\hbox{.}\hfil\hfil\par}$\,$#2\hfil}}
\newcommand{\novosubsection}[2]{\baselineskip=12pt}%{\medskip\hbox to \hsize{\vbox{\advance\hsize by -1cm\baselineskip=12pt\parfillskip=0pt\leftskip=3.5cm\noindent\hskip -2cm #1\leaders\hbox{.}\hfil\hfil\par}$\,$#2\hfil}}

\iflatexml
\else
\tableofcontents
\fi

\mainmatter

\begin{chapter}{Medida de Lebesgue e Espaços de Medida}
\label{CHP:LEBESGUE}

\begin{section}{Aritmética na Reta Estendida}
\label{sec:AritRetaEstend}

Medidas associam números reais não negativos a conjuntos, mas a alguns conjuntos fica
associado o valor {\em infinito}. Precisamos então tratar infinitudes como objetos
que podem ser operados com somas e produtos. Introduzimos então formalmente a {\em reta estendida\/}
que é a reta real usual acrescida de dois objetos $+\infty$, $-\infty$ e com operações e relação de ordem
definidas de maneira natural. Por uma questão de completude, listamos nesta seção
em detalhes várias definições e propriedades relacionadas à reta estendida. Na Subseção~\ref{sub:limseq}
definimos o conceito de limite de uma seqüência na reta estendida e na Subseção~\ref{sub:somasinfinitas}
formalizamos o conceito de soma de uma família (possivelmente infinita) de elementos não
negativos da reta estendida.

As noções formalizadas nesta seção são de caráter bastante intuitivo e acreditamos que o leitor
pode optar pela omissão de sua leitura sem prejuízo significativo de compreensão das seções
seguintes.

\begin{notation}
Denotamos por $\R$\index[simbolos]{$\R$} o corpo ordenado dos números reais.
\end{notation}

Escolha dois objetos quaisquer não pertencentes à reta real $\R$ e denote-os por $+\infty$\index[simbolos]{$+\infty$}
e $-\infty$\index[simbolos]{$-\infty$}.
\begin{defin}
O conjunto $\overline\R=\R\cup\{+\infty,-\infty\}$\index[simbolos]{$\overline\R$} será chamado a {\em reta estendida}\index[indice]{reta estendida}.
Um elemento $a\in\overline\R$ é dito {\em finito\/}\index[indice]{finito} (resp., {\em infinito}\index[indice]{infinito})
quando $a\in\R$ (resp., $a\not\in\R$).
\end{defin}
A natureza dos objetos $+\infty$ e $-\infty$ é totalmente irrelevante; o que importa é a forma
como eles interagem com os números reais através das operações e relações que definiremos a seguir
em $\overline\R$.

\begin{defin}
Dados $a,b\in\overline\R$, escrevemos $a<b$ e dizemos que $a$ é {\em menor que\/}\index[indice]{menor que} $b$ quando
uma das seguintes condições é satisfeita:
\begin{itemize}
\item $a,b\in\R$ e $a<b$ na ordem usual de $\R$;
\item $b=+\infty$ e $a\ne+\infty$;
\item $a=-\infty$ e $b\ne-\infty$.
\end{itemize}
Escrevemos $a>b$ quando $b<a$, $a\le b$ quando $a<b$ ou $a=b$ e escrevemos $a\ge b$ quando $b\le a$.
\end{defin}
A relação binária $<$ define uma {\em relação de ordem total\/}\index[indice]{relacao de ordem@relação de ordem!total}
na reta estendida $\overline\R$\index[indice]{relacao de ordem@relação de ordem!na reta estendida},
ou seja, possui as seguintes propriedades:
\begin{itemize}
\item ({\em anti-reflexividade})\index[indice]{anti-reflexividade}\index[indice]{relacao@relação!anti-reflexiva}
para todo $a\in\overline\R$, não é o caso que $a<a$;
\item ({\em transitividade})\index[indice]{transitividade}\index[indice]{relacao@relação!transitiva}
para todos $a,b,c\in\overline\R$, se $a<b$ e $b<c$ então $a<c$;
\item ({\em tricotomia})\index[indice]{tricotomia} dados $a,b\in\overline\R$ então $a<b$, $b<a$ ou $a=b$.
\end{itemize}
A relação de ordem em $\overline\R$ nos permite introduzir as notações de intervalo\index[indice]{intervalo!na reta estendida}
$[a,b]$, $\left]a,b\right]$, $\left[a,b\right[$
e $\left]a,b\right[$, com $a,b\in\overline\R$, da maneira usual. Se $A$ é um subconjunto de $\overline\R$
podemos definir também o {\em supremo\/}\index[indice]{supremo} (resp., o {\em ínfimo\/}\index[indice]{infimo@ínfimo})
de $A$ em $\overline\R$ como sendo a menor cota superior (resp., a maior cota inferior) de $A$ em $\overline\R$.
O supremo (resp., o ínfimo) de um conjunto $A\subset\overline\R$ é denotado por $\sup A$\index[simbolos]{$\sup$} (resp., $\inf A$\index[simbolos]{$\inf$});
se $(a_i)_{i\in I}$ é uma família em $\overline\R$, denotamos também o supremo (resp., o ínfimo) do conjunto
$\{a_i:i\in I\}$ por $\sup_{i\in I}a_i$ (resp., $\inf_{i\in I}a_i$).
No Exercício~\ref{exe:supinf} pedimos ao leitor para mostrar que todo subconjunto de $\overline\R$
possui supremo e ínfimo.

\begin{defin}
A {\em soma\/}\index[indice]{soma!na reta estendida} na reta estendida é definida da seguinte forma:
\begin{itemize}
\item se $a,b\in\R$ então $a+b$ é igual à soma usual de $a$ e $b$ em $\R$;
\item $(+\infty)+a=a+(+\infty)=+\infty$, se $a\in\overline\R$ e $a\ne-\infty$;
\item $(-\infty)+a=a+(-\infty)=-\infty$, se $a\in\overline\R$ e $a\ne+\infty$.
\end{itemize}
As somas $(+\infty)+(-\infty)$ e $(-\infty)+(+\infty)$ são consideradas {\em indefinidas}.
Para $a\in\overline\R$ denotamos por $-a$ o elemento de $\overline\R$ definido pelas condições:
\begin{itemize}
\item se $a\in\R$ então $-a$ é o inverso de $a$ com relação à soma de $\R$;
\item se $a=+\infty$ então $-a=-\infty$;
\item se $a=-\infty$ então $-a=+\infty$.
\end{itemize}
Para $a,b\in\overline\R$, escrevemos $a-b=a+(-b)$.
Definimos também o {\em módulo\/}\index[indice]{modulo@módulo} de $a\in\overline\R$ fazendo
$\vert a\vert=a$ para $a\ge0$ e $\vert a\vert=-a$ para $a<0$.
O {\em produto\/}\index[indice]{produto!na reta estendida} na reta estendida é definido da seguinte forma:
\begin{itemize}
\item se $a,b\in\R$ então $a\cdot b$ (ou, simplesmente, $ab$)
é igual ao produto usual de $a$ e $b$ em $\R$;
\item $ab=0$ se $a,b\in\overline\R$ e $a=0$ ou $b=0$;
\item $ab=ba=a$, se $a\in\{+\infty,-\infty\}$ e $b>0$;
\item $ab=ba=-a$, se $a\in\{+\infty,-\infty\}$ e $b<0$.
\end{itemize}
\end{defin}
Note que o produto é uma operação binária no conjunto $\overline\R$, mas a soma é apenas uma
{\em operação binária parcialmente definida\/} em $\overline\R$, já que não atribuímos significado
para $(+\infty)+(-\infty)$ e $(-\infty)+(+\infty)$. Note também que, de acordo com nossas convenções,
$0\cdot(\pm\infty)=(\pm\infty)\cdot0=0$; essa convenção é conveniente em teoria da medida, embora possa
parecer estranha para quem está acostumado com as propriedades usuais de limites de funções.

Na proposição abaixo resumimos as propriedades da ordem e das operações de $\overline\R$;
a demonstração é obtida simplesmente por uma verificação tediosa de diversos casos.
\begin{prop}
A ordem e as operações da reta estendida satisfazem as seguintes propriedades:
\begin{itemize}
\item a soma é associativa\index[indice]{associatividade}\index[indice]{operacao@operação!associativa}
onde estiver bem-definida, i.e., $(a+b)+c=a+(b+c)$, para todos
$a,b,c\in\overline\R$, desde que ou $a,b,c\ne+\infty$ ou $a,b,c\ne-\infty$;
\item a soma é comutativa\index[indice]{comutatividade}\index[indice]{operacao@operação!comutativa}
onde estiver bem-definida, i.e., $a+b=b+a$, para todos
$a,b\in\overline\R$, desde que ou $a,b\ne+\infty$ ou $a,b\ne-\infty$;
\item o zero de $\R$ é o elemento neutro\index[indice]{elemento neutro}
para a soma de $\overline\R$, i.e., $a+0=0+a=a$,
para todo $a\in\overline\R$;
\item o produto é associativo, i.e., $(ab)c=a(bc)$, para todos $a,b,c\in\overline\R$;
\item o produto é comutativo, i.e., $ab=ba$, para todos $a,b\in\overline\R$;
\item a unidade de $\R$ é o elemento neutro para o produto de $\overline\R$, i.e.,
$a\cdot1=1\cdot a=a$, para todo $a\in\overline\R$;
\item a soma é distributiva com relação ao produto, i.e., $(a+b)c=ac+bc$, para todos
$a,b,c\in\overline\R$, desde que as somas $a+b$ e $ac+bc$ estejam bem-definidas;
\item a ordem é compatível com a soma, i.e., se $a\le b$ então $a+c\le b+c$,
para todos $a,b,c\in\overline\R$, desde que as somas $a+c$ e $b+c$ estejam bem-definidas;
\item a ordem é compatível com o produto, i.e., se $a\le b$ então $ac\le bc$, para todos
$a,b,c\in\overline\R$ com $c\ge0$.\qed
\end{itemize}
\end{prop}

Algumas observações importantes seguem. A identidade $a+(-a)=0$ é válida {\em apenas para\/}
$a\in\R$; os elementos $+\infty$ e $-\infty$ não possuem inverso com respeito à soma.
Em particular, as implicações:
\[a+c=b+c\Longrightarrow a=b\quad\text{e}\quad a=b+c\Longrightarrow a-c=b\]
{\em são válidas apenas quando $c\in\R$}.
A implicação:
\[a<b\Longrightarrow a+c<b+c\]
é também {\em apenas válida para $c\in\R$} e a implicação:
\[a<b\Longrightarrow ac<bc\]
é válida {\em apenas para $0<c<+\infty$}.

\begin{subsection}{Limites de seqüências na reta estendida}
\label{sub:limseq}

Limites de se\-qüên\-cias em $\overline\R$ podem ser definidos através da introdução de uma
topologia em $\overline\R$ (veja Exercício~\ref{exe:topologiaRbar}).
Para o leitor não familiarizado com a noção de espaço topológico,
definimos a noção de limite de seqüência em $\overline\R$ diretamente.

\begin{defin}\label{thm:defconvseq}
Seja $(a_k)_{k\ge1}$ uma seqüência em $\overline\R$. Dizemos que $(a_k)_{k\ge1}$
{\em converge\/}\index[indice]{convergencia@convergência!em R barra@em $\overline\R$}
para um elemento $a\in\overline\R$ e escrevemos $a_k\to a$\index[simbolos]{$a_k\to a$} se uma das
situações abaixo ocorre:
\begin{itemize}
\item $a\in\R$ e para todo $\varepsilon>0$ existe $k_0\ge1$ tal que $a_k\in\left]a-\varepsilon,a+\varepsilon\right[$
para todo $k\ge k_0$;
\item $a=+\infty$ e para todo $M<+\infty$ existe $k_0\ge1$ tal que $a_k>M$ para todo $k\ge k_0$;
\item $a=-\infty$ e para todo $M>-\infty$ existe $k_0\ge1$ tal que $a_k<M$ para todo $k\ge k_0$.
\end{itemize}
\end{defin}
Quando existe $a\in\overline\R$ com $a_k\to a$ dizemos que a seqüência $(a_k)_{k\ge1}$
é {\em convergente\/}\index[indice]{sequencia@seqüência!convergente em R barra@convergente em $\overline\R$}
em $\overline\R$. Nesse caso, é fácil mostrar que tal $a\in\overline\R$ é
único e é chamado o {\em limite\/}\index[indice]{limite!em R barra@em $\overline\R$}
da seqüência $(a_k)_{k\ge1}$; denotâmo-lo por $\lim_{k\to\infty}a_k$\index[simbolos]{$\lim_{k\to\infty}a_k$}.

Uma seqüência $(a_k)_{k\ge1}$ em $\overline\R$ é dita {\em crescente\/}\index[indice]{crescente!sequencia@seqüência}\index[indice]{sequencia@seqüência!crescente}
(resp., {\em decrescente}\index[indice]{decrescente!sequencia@seqüência}\index[indice]{sequencia@seqüência!decrescente}) se $a_k\le a_{k+1}$
(resp., se $a_k\ge a_{k+1}$), para todo $k\ge1$. Uma seqüência que é ou crescente ou decrescente é dita
{\em monótona}.\index[indice]{monotona@monótona!sequencia@seqüência}\index[indice]{sequencia@seqüência!monotona@monótona}
Deixamos a demonstração do seguinte resultado simples a cargo do leitor.
\begin{lem}\label{thm:seqmonot}
Toda seqüência monótona em $\overline\R$ é convergente em $\overline\R$. Mais
especificamente, se $(a_k)_{k\ge1}$ é uma seqüência crescente (resp., decrescente)
em $\overline\R$ então $\lim_{k\to\infty}a_k=\sup_{k\ge1}a_k$ (resp.,
$\lim_{k\to\infty}a_k=\inf_{k\ge1}a_k$).
\end{lem}
\begin{proof}
Veja Exercício~\ref{exe:seqmonot}.
\end{proof}

Enunciamos a seguir as propriedades operatórias dos limites na reta estendida:
\begin{lem}\label{thm:oplimbarR}
Sejam $(a_k)_{k\ge1}$, $(b_k)_{k\ge1}$ seqüências convergentes em $\overline\R$, com
$\lim_{k\to\infty}a_k=a$ e $\lim_{k\to\infty}b_k=b$. Então:
\begin{itemize}
\item se a soma $a+b$ estiver bem-definida então a soma $a_k+b_k$ está bem-definida
para todo $k$ suficientemente grande e:
\[\lim_{k\to\infty}a_k+b_k=a+b;\]
\item se $\{\vert a\vert,\vert b\vert\}\ne\{0,+\infty\}$ então $\lim_{k\to\infty}a_kb_k=ab$.
\end{itemize}
\end{lem}
\begin{proof}
Veja Exercício~\ref{exe:oplimbarR}.
\end{proof}

\begin{defin}
Seja $(a_k)_{k\ge1}$ uma seqüência em $\overline\R$. O {\em limite superior\/}\index[indice]{limite!superior}
e o {\em limite inferior\/}\index[indice]{limite!inferior}
da seqüência $(a_k)_{k\ge1}$, denotados respectivamente por $\limsup_{k\to\infty}a_k$\index[simbolos]{$\limsup_{k\to\infty}a_k$}
e $\liminf_{k\to\infty}a_k$\index[simbolos]{$\liminf_{k\to\infty}a_k$}, são definidos por:
\[\limsup_{k\to\infty}a_k=\infsup_{k\ge1}\sup_{r\ge k}a_r,\quad\liminf_{k\to\infty}a_k=\sup_{k\ge1}\infsup_{r\ge k}a_r.\]
\end{defin}

Temos a seguinte:
\begin{prop}\label{thm:limlimsupliminf}
Seja $(a_k)_{k\ge1}$ uma seqüência em $\overline\R$. Então:
\[\liminf_{k\to\infty}a_k\le\limsup_{k\to\infty}a_k,\]
sendo que a igualdade vale se e somente se a seqüência $(a_k)_{k\ge1}$ é convergente; nesse caso:
\[\lim_{k\to\infty}a_k=\liminf_{k\to\infty}a_k=\limsup_{k\to\infty}a_k.\]
\end{prop}
\begin{proof}
Veja Exercício~\ref{exe:liminflimsup}
\end{proof}

\end{subsection}

\begin{subsection}{Somas infinitas em ${[0,+\infty]}$}
\label{sub:somasinfinitas}

Se $(a_i)_{i\in I}$ é uma família finita
em $\overline\R$ então, já que a soma de $\overline\R$ é associativa e comutativa,
podemos definir a soma $\sum_{i\in I}a_i$ de maneira óbvia, desde que
$a_i\ne+\infty$ para todo $i\in I$ ou $a_i\ne-\infty$ para todo $i\in I$.
Definiremos a seguir um significado para somas de famílias infinitas de elementos não negativos
de $\overline\R$. É possível também definir somas de famílias que contenham elementos
negativos de $\overline\R$, mas esse conceito não será necessário no momento.

\begin{defin}
Seja $(a_i)_{i\in I}$ uma família arbitrária em $[0,+\infty]$. A {\em soma\/}\index[indice]{soma!de uma familia@de uma família}
$\sum_{i\in I}a_i$ é definida por:
\[\sum_{i\in I}a_i=\sup\Big\{\sum_{i\in F}a_i:\text{$F\subset I$ um subconjunto finito}\Big\}.\]
\end{defin}
Se $I$ é o conjunto dos inteiros positivos então denotamos a soma $\sum_{i\in I}a_i$ também
por $\sum_{i=1}^\infty a_i$; segue facilmente do Lema~\ref{thm:seqmonot} que:
\[\sum_{i=1}^\infty a_i=\lim_{k\to\infty}\,\sum_{i=1}^ka_i.\]

Deixamos a demonstração do seguinte resultado a cargo do leitor.
\begin{prop}\label{thm:propsomasinf}
Somas de famílias em $[0,+\infty]$ satisfazem as seguintes propriedades:
\begin{itemize}
\item se $(a_i)_{i\in I}$ e $(b_i)_{i\in I}$ são famílias em $[0,+\infty]$
então:
\[\sum_{i\in I}(a_i+b_i)=\sum_{i\in I}a_i+\sum_{i\in I}b_i;\]
\item se $(a_i)_{i\in I}$ é uma família em $[0,+\infty]$ e $c\in[0,+\infty]$ então
\[\sum_{i\in I}c\,a_i=c\sum_{i\in I}a_i;\]
\item se $(a_i)_{i\in I}$ é uma família em $[0,+\infty]$ e se $\phi:I'\to I$
é uma função bijetora então:
\[\sum_{i\in I'}a_{\phi(i)}=\sum_{i\in I}a_i;\]
\item se $(a_\lambda)_{\lambda\in\Lambda}$ é uma família em $[0,+\infty]$
e se $(J_i)_{i\in I}$ é uma família de conjuntos dois a dois disjuntos
com $\Lambda=\bigcup_{i\in I}J_i$ então:
\[\sum_{\lambda\in\Lambda}a_\lambda=\sum_{i\in I}\Big(\sum_{\lambda\in J_i}a_\lambda\Big).\]
\end{itemize}
\end{prop}
\begin{proof}
Veja Exercício~\ref{exe:propsomasinf}.
\end{proof}

A última propriedade no enunciado da Proposição~\ref{thm:propsomasinf} implica em particular que:
\[\sum_{i\in I}\Big(\sum_{j\in J}a_{ij}\Big)=\!\!\!\sum_{(i,j)\in I\times J}\!\!\!a_{ij}=
\sum_{j\in J}\Big(\sum_{i\in I}a_{ij}\Big),\]
onde $(a_{ij})_{(i,j)\in I\times J}$ é uma família em $[0,+\infty]$. Basta tomar
$\Lambda=I\times J$ e $J_i=\{i\}\times J$, para todo $i\in I$.

\end{subsection}

\end{section}

\begin{section}{O Problema da Medida}

\begin{notation}
Denotamos por $\wp(X)$\index[simbolos]{$\wp(X)$} o conjuntos de todas as partes de um conjunto $X$,
por $\Q$\index[simbolos]{$\Q$} o corpo ordenado dos números racionais e por $\Z$\index[simbolos]{$\Z$} o anel dos números inteiros.
\end{notation}

Queremos investigar a existência de uma função $\mu:\wp(\R)\to[0,+\infty]$ satisfazendo as seguintes
propriedades:
\begin{itemize}
\item[(a)] dada uma seqüência $(A_n)_{n\ge1}$ de subconjuntos de $\R$ dois a dois disjuntos então:
\[\mu\Big(\bigcup_{n=1}^\infty A_n\Big)=\sum_{n=1}^{\infty}\mu(A_n);\]
\item[(b)] $\mu(A+x)=\mu(A)$, para todo $A\subset\R$ e todo $x\in\R$, onde:
\index[simbolos]{$A+x$}\[A+x=\big\{a+x:a\in A\big\}\]
denota a {\em translação\/}\index[indice]{translacao@translação} de $A$ por $x$;
\item[(c)] $0<\mu\big([0,1]\big)<+\infty$.
\end{itemize}
Nosso objetivo é mostrar que tal função $\mu$ {\em não existe}. Antes disso, observamos algumas
conseqüências simples das propriedades (a), (b) e (c) acima.

\begin{lem}\label{thm:lemdefg}
Se uma função $\mu:\wp(\R)\to[0,+\infty]$ satisfaz as propriedades (a), (b) e (c) acima então
ela também satisfaz as seguintes propriedades:
\begin{itemize}
\item[(d)] $\mu(\emptyset)=0$;
\item[(e)] dada uma coleção finita $(A_k)_{k=1}^n$ de subconjuntos de $\R$ dois a dois disjuntos
então:
\[\mu\Big(\bigcup_{k=1}^n A_k\Big)=\sum_{k=1}^n\mu(A_k);\]
\item[(f)] se $A\subset B\subset\R$ então $\mu(A)\le\mu(B)$;
\item[(g)] dados $a,b\in\R$ com $a\le b$ então $\mu\big([a,b]\big)<+\infty$.
\end{itemize}
\end{lem}
\begin{proof}\
\begin{bulletindent}
\item {\em Prova de\/} (d).

Tome $A_1=[0,1]$ e $A_n=\emptyset$ para $n\ge2$ na propriedade (a) e use a propriedade (c).

\item {\em Prova de\/} (e).

Tome $A_k=\emptyset$ para $k>n$ e use as propriedades (a) e (d).

\item {\em Prova de\/} (f).

Basta observar que a propriedade (e) implica que:
\[\mu(B)=\mu(A)+\mu(B\setminus A),\]
onde $\mu(B\setminus A)\ge0$.

\item {\em Prova de\/} (g).

Seja $n$ um inteiro positivo tal que $b<a+n$. As propriedades
(e) e (f) implicam que:
\begin{multline*}
\mu\big([a,b]\big)\le\mu\big(\left[a,a+n\right[\big)=\sum_{k=0}^{n-1}\mu\big(\left[a+k,a+k+1\right[\big)\\
\le\sum_{k=0}^{n-1}\mu\big([a+k,a+k+1]\big),
\end{multline*}
e as propriedades (b) e (c) implicam que:
\[\mu\big([a+k,a+k+1]\big)=\mu\big([0,1]\big)<+\infty,\]
para todo $k$.\qedhere
\end{bulletindent}
\end{proof}

Finalmente, mostramos a seguinte:
\begin{prop}\label{thm:naoexistemu}
Não existe uma função $\mu:\wp(\R)\to[0,+\infty]$ satisfazendo as propriedades (a), (b)
e (c) acima.
\end{prop}
\begin{proof}
Pelo Lema~\ref{thm:lemdefg}, as propriedades (a), (b) e (c) implicam as propriedades
(d), (e), (f) e (g).
Considere a relação binária $\sim$ no intervalo $[0,1]$ definida por:
\[x\sim y\Longleftrightarrow x-y\in\Q,\]
para todos $x,y\in[0,1]$. É fácil ver que $\sim$ é uma relação de equivalência\index[indice]{relacao@relação!de equivalencia@de equivalência}
em $[0,1]$.
Seja $A\subset[0,1]$ um {\em conjunto escolha\/}\index[indice]{conjunto!escolha}
para $\sim$, i.e., $A$ possui exatamente um
elemento de cada classe de equivalência. Temos então que $x-y\not\in\Q$, para todos $x,y\in A$
com $x\ne y$. Em particular, os conjuntos $(A+q)_{q\in\Q}$ são dois a dois disjuntos.
Note também que para todo $x\in[0,1]$ existe $y\in A$ com $x-y\in\Q$; na verdade, temos
$x-y\in\Q\cap[-1,1]$, já que $x,y\in[0,1]$. Segue então que:
\[[0,1]\subset\!\!\!\!\!\!\!\bigcup_{q\in\Q\cap[-1,1]}\!\!\!\!(A+q)\subset[-1,2].\]
Como $\Q\cap[-1,1]$ é enumerável, as propriedades (a), (b) e (f) implicam:
\[\mu\big([0,1]\big)\le\!\!\!\!\!\!\sum_{q\in\Q\cap[-1,1]}\!\!\!\mu(A+q)
=\!\!\!\!\!\!\sum_{q\in\Q\cap[-1,1]}\!\!\!\mu(A)\le\mu\big([-1,2]\big).\]
Agora, se $\mu(A)=0$ concluímos que $\mu\big([0,1]\big)=0$, contradizendo (c);
se $\mu(A)>0$ concluímos que $\mu\big([-1,2]\big)=+\infty$, contradizendo (g).
\end{proof}

\end{section}

\begin{section}{Volume de Blocos Retangulares}

\begin{defin}\label{thm:defbloco}
Um {\em bloco retangular $n$-dimensional\/}\index[indice]{bloco retangular}
é um subconjunto $B$ de $\R^n$ ($n\ge1$) que é ou vazio, ou da forma:
\[B=\prod_{i=1}^n\,[a_i,b_i]=[a_1,b_1]\times\cdots\times[a_n,b_n],\]
onde $a_i,b_i\in\R$, $a_i\le b_i$, para $i=1,2,\ldots,n$. O {\em volume\/}\index[indice]{volume}\index[indice]{bloco retangular!volume de}
do bloco $B$ acima é definido por:
\index[simbolos]{$\vert B\vert$}\[\vert B\vert=\prod_{i=1}^n\,(b_i-a_i)=(b_1-a_1)\cdots(b_n-a_n),\]
e por $\vert B\vert=0$, caso $B=\emptyset$.
\end{defin}
Quando $n=1$ então um bloco retangular $n$-dimensional $B$ é simplesmente um intervalo
fechado e limitado (possivelmente um conjunto unitário ou vazio) e o escalar $\vert B\vert$ será
chamado também o {\em comprimento\/}\index[indice]{comprimento!de um intervalo}\index[indice]{intervalo!comprimento de} de $B$.
Quando $n=2$, um bloco retangular $n$-dimensional $B$ será chamado também um {\em retângulo\/} e o
escalar $\vert B\vert$ será chamado também a
{\em área\/}\index[indice]{area@área}\index[indice]{retangulo@retângulo!area de@área de} de $B$.

\begin{defin}\label{thm:defcoisasbloco}
Dados $a,b\in\R$, $a<b$, então uma {\em partição\/}\index[indice]{particao@partição} do intervalo $[a,b]$ é um subconjunto finito $P\subset[a,b]$
com $a,b\in P$; tipicamente escrevemos $P:a=t_0<t_1<\cdots<t_k=b$ quando $P=\{t_0,t_1,\ldots,t_k\}$. Os
{\em sub-intervalos\/}\index[indice]{sub-intervalo!determinado por uma particao@determinado por\hfil\break
uma partição}
de $[a,b]$ determinados pela partição $P$ são os intervalos $[t_i,t_{i+1}]$, $i=0,\ldots,k-1$. Denotamos
por $\overline P$ o conjunto dos sub-intervalos de $[a,b]$ deterninados por $P$, ou seja:
\[\overline P=\big\{[t_i,t_{i+1}];i=0,1,\ldots,k-1\big\}.\]
Se $B=\prod_{i=1}^n[a_i,b_i]$ é um bloco retangular $n$-dimensional com $\vert B\vert>0$ (ou seja,
$a_i<b_i$, para $i=1,\ldots,n$), então uma {\em partição\/}
de $B$ é uma $n$-upla $P=(P_1,\ldots,P_n)$, onde $P_i$ é uma partição do intervalo $[a_i,b_i]$, para cada $i=1,\ldots,n$.
Os {\em sub-blocos\/}\index[indice]{sub-bloco!determinado por uma particao@determinado por\hfil\break
uma partição}
de $B$ determinados pela partição $P$ são os blocos retangulares $n$-dimensionais da forma $\prod_{r=1}^nI_r$,
onde $I_r$ é um sub-intervalo de $[a_r,b_r]$ determinado pela partição $P_r$, para $r=1,\ldots,n$. Denotamos por $\overline P$
o conjunto dos sub-blocos de $B$ determinados por $P$, ou seja:
\index[simbolos]{$\overline P$}\[\overline P=\big\{I_1\times\cdots\times I_n:I_r\in\overline{P_r},\ r=1,\ldots,n\big\}.\]
\end{defin}

\begin{lem}\label{thm:particionabloco}
Se $B=\prod_{i=1}^n[a_i,b_i]$ é um bloco retangular $n$-dimensional com $\vert B\vert>0$
e se $P=(P_1,\ldots,P_n)$ é uma partição de $B$ então:
\[\vert B\vert=\sum_{\mathfrak b\in\overline P}\vert\mathfrak b\vert.\]
\end{lem}
\begin{proof}
Usamos indução em $n$. O caso $n=1$ é trivial. Suponha então que $n>1$ e que o resultado é válido para blocos retangulares de dimensão menor
que $n$. Sejam $B'=\prod_{i=1}^{n-1}[a_i,b_i]$ e $P'=(P_1,\ldots,P_{n-1})$, de modo que $P'$ é uma partição
do bloco retangular $(n-1)$-dimensional $B'$. Escrevendo $P_n:a_n=t_0<t_1<\cdots<t_k=b_n$ temos:
\[\vert B\vert=\vert B'\vert(b_n-a_n)=\Big(\sum_{\mathfrak b'\in\overline{P'}}\vert\mathfrak b'\vert\Big)
\Big(\sum_{i=0}^{k-1}(t_{i+1}-t_i)\Big)
=\!\!\!\!\sum_{\substack{\mathfrak b'\in\overline{P'}\\
i=0,\ldots,k-1}}\!\!\!\big\vert\mathfrak b'\times[t_i,t_{i+1}]\big\vert.\]
A conclusão segue observando que os blocos $\mathfrak b'\times[t_i,t_{i+1}]$ com $\mathfrak b'\in\overline{P'}$ e
$i=0,\ldots,k-1$ são precisamente os sub-blocos de $B$ determinados pela partição $P$.
\end{proof}

\begin{rem}\label{thm:remintermonot}
Note que a interseção de dois blocos retangulares $n$-di\-men\-sio\-nais é também um bloco retangular $n$-dimensional.
Note também que se $B$ e $B'$ são blocos retangulares $n$-dimensionais com $B\subset B'$ então
$\vert B\vert\le\vert B'\vert$.
\end{rem}

\begin{lem}\label{thm:dificilvolblocos}
Sejam $B$, $B_1$, \ldots, $B_t$ blocos retangulares $n$-dimensionais com $B\subset\bigcup_{r=1}^tB_r$. Então
$\vert B\vert\le\sum_{r=1}^t\vert B_r\vert$.
\end{lem}
\begin{proof}
Em vista da Observação~\ref{thm:remintermonot}, substituindo cada bloco $B_r$ por $B_r\cap B$ e descartando os índices
$r$ com $B_r\cap B=\emptyset$, podemos supor sem perda de generalidade que $B=\bigcup_{r=1}^tB_r$ e que $B_r\ne\emptyset$
para todo $r=1,\ldots,t$. Podemos supor também
que $\vert B\vert>0$, senão o resultado é trivial. Escreva então $B=\prod_{i=1}^n[a_i,b_i]$ com $a_i<b_i$, $i=1,\ldots,n$,
e $B_r=\prod_{i=1}^n[a_i^r,b_i^r]$ com $a_i^r\le b_i^r$, $i=1,\ldots,n$. Para cada $i=1,\ldots,n$, o conjunto
\[P_i=\{a_i,b_i\}\cup\{a_i^r,b_i^r;r=1,\ldots,t\}\]
é uma partição do intervalo $[a_i,b_i]$ e portanto $P=(P_1,\ldots,P_n)$
é uma partição do bloco $B$. Para cada $r=1,\ldots,t$ com $\vert B_r\vert>0$, tomamos $P_i^r=P_i\cap[a_i^r,b_i^r]$, $i=1,\ldots,n$ e $P^r=(P_1^r,\ldots,P_n^r)$,
de modo que $P^r$ é uma partição do bloco $B_r$. Temos que se $\mathfrak b=\prod_{i=1}^n[\alpha_i,\beta_i]$ é um sub-bloco de $B$
determinado pela partição $P$ então existe um índice $r=1,\ldots,t$ tal que $\vert B_r\vert>0$ e $\mathfrak b$ é um sub-bloco de $B_r$ determinado
pela partiação $P^r$. De fato, como $B=\bigcup_{r=1}^tB_r$ então $\prod_{i=1}^n\left]\alpha_i,\beta_i\right[$ intercepta
$B_r$, para algum $r=1,\ldots,t$ tal que\footnote{%
Os blocos de volume zero são conjuntos fechados de interior vazio e portanto a união de um número finito deles também tem interior vazio. Assim, o aberto
não vazio $\prod_{i=1}^n\left]\alpha_i,\beta_i\right[$ não pode estar contido na união dos blocos $B_r$ de volume zero.}
$\vert B_r\vert>0$. Daí é fácil ver que $[\alpha_i,\beta_i]$ é um sub-intervalo de $[a_i^r,b_i^r]$
determinado pela partição $P_i^r$ para $i=1,\ldots,n$ e portanto $\mathfrak b$ é um sub-bloco de $B_r$ determinado
pela partição $P^r$. Mostramos então que:
\[\overline P\subset\!\!\!\bigcup_{\substack{r=1,\ldots,t\\\vert B_r\vert>0}}\!\overline{P^r}.\]
A conclusão segue agora do Lema~\ref{thm:particionabloco} observando que:
\[\vert B\vert=\sum_{\mathfrak b\in\overline P}\vert\mathfrak b\vert\le\!\!\!\sum_{\substack{
r=1,\ldots,t\\\vert B_r\vert>0}}\;\sum_{\mathfrak b\in\overline{P^r}}\vert\mathfrak b\vert
=\sum_{r=1}^t\vert B_r\vert.\qedhere\]
\end{proof}

\end{section}

\begin{section}[Medida de Lebesgue em $\R^n$]{Medida de Lebesgue em ${\R^n}$}
\label{sec:MedLebRn}

\begin{defin}
Seja $A\subset\R^n$ um subconjunto arbitrário. A {\em medida exterior de Lebesgue\/}\index[indice]{medida!exterior!de Lebesgue}\index[indice]{Lebesgue!medida exterior de}
de $A$, denotada por
$\leb^*(A)$, é definida como sendo o ínfimo do conjunto de todas as somas da forma $\sum_{k=1}^\infty\vert B_k\vert$,
onde $(B_k)_{k\ge1}$ é uma seqüência de blocos retangulares $n$-dimensionais com $A\subset\bigcup_{k=1}^\infty B_k$;
em símbolos:
\index[simbolos]{$\leb^*(A)$}\[\leb^*(A)=\inf\,\mathcal C(A),\]
onde:
\begin{multline}\label{eq:defCA}
\mathcal C(A)=\Big\{\sum_{k=1}^\infty\vert B_k\vert:A\subset\bigcup_{k=1}^\infty B_k,\ \text{$B_k$ bloco retangular
$n$-dimensional},\\[-10pt]
\text{para todo $k\ge1$}\Big\}.
\end{multline}\index[simbolos]{$\mathcal C(A)$}
\end{defin}
Note que é sempre possível cobrir um subconjunto $A$ de $\R^n$ com uma coleção enumerável de blocos retangulares $n$-dimensionais
(i.e., $\mathcal C(A)\ne\emptyset$),
já que, por exemplo, $\R^n=\bigcup_{k=1}^\infty[-k,k]^n$.
Obviamente temos $\leb^*(A)\in[0,+\infty]$, para todo $A\subset\R^n$.

\begin{rem}\label{thm:boundfinite}
Todo subconjunto limitado de $\R^n$ possui medida exterior finita. De fato,
se $A\subset\R^n$ é limitado então existe um bloco retangular $n$-dimensional $B$ contendo $A$.
Tomando $B_1=B$ e $B_k=\emptyset$ para $k\ge2$, temos $A\subset\bigcup_{k=1}^\infty B_k$
e portanto $\leb^*(A)\le\sum_{k=1}^\infty\vert B_k\vert=\vert B\vert<+\infty$.
Veremos logo adiante (Corolários~\ref{thm:uniaonula} e \ref{thm:hiperplanonula}) que a recíproca dessa afirmação não é verdadeira,
i.e., subconjuntos de $\R^n$ com medida exterior finita não precisam ser limitados.
\end{rem}

\begin{lem}\label{thm:medidabloco}
Se $B\subset\R^n$ é um bloco retangular $n$-dimensional então:
\[\leb^*(B)=\vert B\vert,\]
ou seja, a medida exterior de um bloco retangular $n$-dimensional coincide com seu volume.
\end{lem}
\begin{proof}
Tomando $B_1=B$ e $B_k=\emptyset$ para $k\ge2$, obtemos uma cobertura $(B_k)_{k\ge1}$ de $B$ por blocos retangulares
com $\sum_{k=1}^\infty\vert B_k\vert=\vert B\vert$;
isso mostra que $\leb^*(B)\le\vert B\vert$. Para mostrar a desigualdade oposta, devemos escolher uma cobertura arbitrária
$B\subset\bigcup_{k=1}^\infty B_k$ de $B$ por blocos retangulares $B_k$ e mostrar que $\vert B\vert\le\sum_{k=1}^\infty\vert B_k\vert$.
Seja dado $\varepsilon>0$ e seja para cada $k\ge1$, $B'_k$ um bloco retangular $n$-dimensional que contém $B_k$ no seu interior
e tal que $\vert B'_k\vert\le\vert B_k\vert+\frac\varepsilon{2^k}$. Os interiores dos blocos $B'_k$, $k\ge1$, constituem então
uma cobertura aberta do compacto $B$ e dessa cobertura aberta podemos extrair uma subcobertura finita; existe portanto
$t\ge1$ tal que $B\subset\bigcup_{k=1}^tB'_k$. Usando o Lema~\ref{thm:dificilvolblocos} obtemos:
\[\vert B\vert\le\sum_{k=1}^t\vert B'_k\vert\le\sum_{k=1}^t\Big(\vert B_k\vert+\frac\varepsilon{2^k}\Big)\le\Big(\sum_{k=1}^\infty\vert B_k\vert\Big)+\varepsilon.\]
Como $\varepsilon>0$ é arbitrário, a conclusão segue.
\end{proof}

\begin{lem}\label{thm:lebmonotonica}
Se $A_1\subset A_2\subset\R^n$ então $\leb^*(A_1)\le\leb^*(A_2)$.
\end{lem}
\begin{proof}
Basta observar que $\mathcal C(A_2)\subset\mathcal C(A_1)$ (recorde \eqref{eq:defCA}).
\end{proof}

\begin{lem}\label{thm:sigmasubad}
Se $A_1$, \dots, $A_t$ são subconjuntos de $\R^n$ então:
\[\leb^*\Big(\bigcup_{k=1}^tA_k\Big)\le\sum_{k=1}^t\leb^*(A_k).\]
Além do mais, se $(A_k)_{k\ge1}$ é uma seqüência de subconjuntos de $\R^n$ então:
\[\leb^*\Big(\bigcup_{k=1}^\infty A_k\Big)\le\sum_{k=1}^\infty\leb^*(A_k).\]
\end{lem}
\begin{proof}
Como $\leb^*(\emptyset)=0$, tomando $A_k=\emptyset$ para $k>t$, podemos considerar apenas
o caso de uma seqüência infinita de subconjuntos de $\R^n$.
Seja dado $\varepsilon>0$. Para cada $k\ge1$ existe uma cobertura $A_k\subset\bigcup_{j=1}^\infty B_k^j$
de $A_k$ por blocos retangulares $n$-dimensionais $B_k^j$ de modo que:
\[\sum_{j=1}^\infty\vert B_k^j\vert\le\leb^*(A_k)+\frac\varepsilon{2^k}.\]
Daí $(B_k^j)_{k,j\ge1}$ é uma cobertura enumerável do conjunto $\bigcup_{k=1}^\infty A_k$ por blocos retangulares $n$-dimensionais
e portanto:
\[\leb^*\Big(\bigcup_{k=1}^\infty A_k\Big)\le\sum_{k=1}^\infty\sum_{j=1}^\infty\vert B_k^j\vert\le
\sum_{k=1}^\infty\Big(\leb^*(A_k)+\frac\varepsilon{2^k}\Big)=\Big(\sum_{k=1}^\infty\leb^*(A_k)\Big)+\varepsilon.\]
Como $\varepsilon>0$ é arbitrário, a conclusão segue.
\end{proof}

\begin{cor}\label{thm:uniaonula}
A união de uma coleção enumerável de conjuntos de medida exterior nula tem medida exterior nula.
Em particular, todo conjunto enumerável tem medida exterior nula.\qed
\end{cor}

\begin{cor}\label{thm:hiperplanonula}
Dado $i=1,\ldots,n$ e $c\in\R$ então todo subconjunto do hiperplano afim $\big\{x=(x_1,\ldots,x_n)\in\R^n:x_i=c\big\}$
tem medida exterior nula.
\end{cor}
\begin{proof}
Basta observar que $\big\{x\in\R^n:x_i=c\big\}=\bigcup_{k=1}^\infty B_k$, onde:
\[B_k=\big\{x\in\R^n:\text{$x_i=c$ e $\vert x_j\vert\le k$, $j=1,\ldots,n$, $j\ne i$}\big\}\]
é um bloco retangular $n$-dimensional de volume zero.
\end{proof}

\begin{cor}\label{thm:frontbloconula}
Todo subconjunto da fronteira de um bloco retangular $n$-dimensional tem medida exterior nula.
\end{cor}
\begin{proof}
Basta observar que a fronteira de um bloco retangular $n$-dimensional é uma união finita de blocos retangulares $n$-dimensionais
de volume zero.
\end{proof}

\begin{cor}\label{thm:AminusB}
Sejam $A_1,A_2\subset\R^n$ tais que $\leb^*(A_1)<+\infty$ ou $\leb^*(A_2)<+\infty$; então:
\begin{equation}\label{eq:A1minusA2}
\leb^*(A_1)-\leb^*(A_2)\le\leb^*(A_1\setminus A_2).
\end{equation}
\end{cor}
\begin{proof}
Como $A_1\subset(A_1\setminus A_2)\cup A_2$, os Lemas~\ref{thm:lebmonotonica}
e \ref{thm:sigmasubad} implicam que:
\begin{equation}\label{eq:preA1minusA2}
\leb^*(A_1)\le\leb^*(A_1\setminus A_2)+\leb^*(A_2).
\end{equation}
Se $\leb^*(A_2)=+\infty$ e $\leb^*(A_1)<+\infty$, a desigualdade \eqref{eq:A1minusA2} é
trivial; se $\leb^*(A_2)<+\infty$, ela segue de \eqref{eq:preA1minusA2}.
\end{proof}

\begin{lem}\label{thm:extmeastransinv}
A medida exterior é {\em invariante por translação}, i.e., dados um subconjunto $A$ de $\R^n$ e $x\in\R^n$ então:
\index[simbolos]{$A+x$}\[\leb^*(A+x)=\leb^*(A),\]
onde $A+x=\big\{a+x:a\in A\big\}$ denota a {\em translação}\index[indice]{translacao@translação}
de $A$ por $x$.
\end{lem}
\begin{proof}
É fácil ver que se $B$ é um bloco retangular $n$-di\-men\-sio\-nal então $B+x$ também é um bloco retangular $n$-dimensional
e:
\[\vert B+x\vert=\vert B\vert;\]
em particular, se $A\subset\bigcup_{k=1}^\infty B_k$ é uma cobertura de $A$ por
blocos retangulares $n$-dimensionais então $A+x\subset\bigcup_{k=1}^\infty(B_k+x)$ é uma cobertura de $A+x$ por blocos
retangulares $n$-dimensionais e $\sum_{k=1}^\infty\vert B_k+x\vert=\sum_{k=1}^\infty\vert B_k\vert$. Isso
mostra que $\mathcal C(A)\subset\mathcal C(A+x)$ (recorde \eqref{eq:defCA}). Como $A=(A+x)+({-x})$, o mesmo
argumento mostra que $\mathcal C(A+x)\subset\mathcal C(A)$; logo:
\[\leb^*(A)=\inf\,\mathcal C(A)=\inf\,\mathcal C(A+x)=\leb^*(A+x).\qedhere\]
\end{proof}

\begin{notation}
Dado um subconjunto $A\subset\R^n$, denotamos por $\overset{\;\circ}A$\index[simbolos]{$\overset{\;\circ}A$}
ou por $\Int(A)$\index[simbolos]{$\Int(A)$} o interior do conjunto\index[indice]{interior de um conjunto} $A$.
\end{notation}

\begin{lem}\label{thm:aproxaberto}
Dados $A\subset\R^n$ e $\varepsilon>0$ então existe um aberto $U\subset\R^n$ com $A\subset U$
e $\leb^*(U)\le\leb^*(A)+\varepsilon$.
\end{lem}
\begin{proof}
Seja $A\subset\bigcup_{k=1}^\infty B_k$ uma cobertura de $A$ por blocos retangulares $n$-dimensionais
tal que $\sum_{k=1}^\infty\vert B_k\vert\le\leb^*(A)+\frac\varepsilon2$. Para cada $k\ge1$, seja
$B'_k$ um bloco retangular que contém $B_k$ no seu interior e tal que $\vert B'_k\vert\le\vert B_k\vert+\frac\varepsilon{2^{k+1}}$.
Seja $U=\bigcup_{k=1}^\infty\Int(B'_k)$. Temos que $U$ é aberto e $U\supset A$; além do mais,
usando os Lemas~\ref{thm:lebmonotonica} e \ref{thm:sigmasubad} obtemos:
\begin{multline*}\leb^*(U)\le\leb^*\Big(\bigcup_{k=1}^\infty B'_k\Big)\le\sum_{k=1}^\infty\leb^*(B'_k)=
\sum_{k=1}^\infty\vert B'_k\vert\le\sum_{k=1}^\infty\Big(\vert B_k\vert+\frac\varepsilon{2^{k+1}}\Big)\\
=\Big(\sum_{k=1}^\infty\vert B_k\vert\Big)+\frac\varepsilon2\le\leb^*(A)+\varepsilon.\qedhere
\end{multline*}
\end{proof}

Note que {\em não podemos\/} concluir do Lema~\ref{thm:aproxaberto} que $\leb^*(U\setminus A)\le\varepsilon$,
nem mesmo se $\leb^*(A)<+\infty$; quando $A$ tem medida exterior finita, o Corolário~\ref{thm:AminusB} nos garante que
$\leb^*(U)-\leb^*(A)\le\leb^*(U\setminus A)$, mas veremos adiante que é possível que a desigualdade estrita ocorra.
\begin{defin}\label{thm:defmens}
Um subconjunto $A\subset\R^n$ é dito ({\em Lebesgue}) {\em mensurável\/}\index[indice]{mensuravel@mensurável}\index[indice]{Lebesgue!mensuravel@mensurável}
\index[indice]{conjunto!mensuravel@mensurável} se para todo $\varepsilon>0$, existe um aberto $U\subset\R^n$
contendo $A$ tal que $\leb^*(U\setminus A)<\varepsilon$.
\end{defin}

\begin{rem}\label{thm:abertomens}
Obviamente, todo aberto em $\R^n$ é mensurável; de fato, se $A\subset\R^n$ é aberto, podemos tomar $U=A$ na Definição~\ref{thm:defmens},
para todo $\varepsilon>0$.
\end{rem}

\begin{lem}\label{thm:uniaoenummens}
A união de uma coleção enumerável de subconjuntos mensuráveis de $\R^n$ é mensurável.
\end{lem}
\begin{proof}
Seja $(A_k)_{k\ge1}$ uma seqüência de subconjuntos mensuráveis de $\R^n$. Dado $\varepsilon>0$ então, para cada
$k\ge1$, podemos encontrar um aberto $U_k$ contendo $A_k$ tal que $\leb^*(U_k\setminus A_k)<\frac\varepsilon{2^k}$.
Tomando $U=\bigcup_{k=1}^\infty U_k$ então $U$ é aberto, $U$ contém $A=\bigcup_{k=1}^\infty A_k$ e:
\[\leb^*(U\setminus A)\le\leb^*\Big(\bigcup_{k=1}^\infty(U_k\setminus A_k)\Big)\le
\sum_{k=1}^\infty\leb^*(U_k\setminus A_k)<\sum_{k=1}^\infty\frac\varepsilon{2^k}=\varepsilon.\qedhere\]
\end{proof}

\begin{lem}\label{thm:nulamens}
Todo subconjunto de $\R^n$ com medida exterior nula é mensurável.
\end{lem}
\begin{proof}
Seja $A\subset\R^n$ com $\leb^*(A)=0$. Dado $\varepsilon>0$ então, pelo Lema~\ref{thm:aproxaberto}, existe
um aberto $U\subset\R^n$ contendo $A$ tal que $\leb^*(U)\le\varepsilon$. Concluímos então que:
\[\leb^*(U\setminus A)\le\leb^*(U)\le\varepsilon.\qedhere\]
\end{proof}

\begin{notation}
No que segue, $d(x,y)$\index[simbolos]{$d(x,y)$} denota a {\em distância Euclideana\/}\index[indice]{distancia@distância!Euclideana}
entre os pontos $x,y\in\R^n$, i.e., $d(x,y)=\Vert x-y\Vert$, onde $\Vert x\Vert$ denota a {\em norma Euclideana\/}\index[indice]{norma!Euclideana}
de um vetor $x\in\R^n$, definida por $\Vert x\Vert=\big(\sum_{i=1}^nx_i^2\big)^{\frac12}$.\index[simbolos]{$\Vert x\Vert$} Dados
$x\in\R^n$ e um subconjunto não vazio $A\subset\R^n$ denotamos por $d(x,A)$\index[simbolos]{$d(x,A)$}
a {\em distância entre $x$ e $A$\/}\index[indice]{distancia@distância!entre ponto e conjunto} definida por:
\[d(x,A)=\inf\big\{d(x,y):y\in A\big\},\]
e dados subconjuntos não vazios $A,B\subset\R^n$ denotamos por $d(A,B)$\index[simbolos]{$d(A,B)$}
a {\em distância entre os conjuntos $A$ e $B$\/}\index[indice]{distancia@distância!entre conjuntos}
definida por:
\[d(A,B)=\inf\big\{d(x,y):x\in A,\ y\in B\big\}.\]
\end{notation}

\begin{lem}\label{eq:distanciapositiva}
Dados subconjuntos $A_1,A_2\subset\R^n$ com $d(A_1,A_2)>0$ então $\leb^*(A_1\cup A_2)=\leb^*(A_1)+\leb^*(A_2)$.
\end{lem}
\begin{proof}
Em vista do Lema~\ref{thm:sigmasubad} é suficiente mostrar a desigualdade:
\[\leb^*(A_1\cup A_2)\ge\leb^*(A_1)+\leb^*(A_2).\]
Para isso, seja $A_1\cup A_2\subset\bigcup_{k=1}^\infty B_k$
uma cobertura de $A_1\cup A_2$ por blocos retangulares $n$-dimensionais $B_k$ e vamos mostrar que:
\begin{equation}\label{eq:A1A2Bk}
\leb^*(A_1)+\leb^*(A_2)\le\sum_{k=1}^\infty\vert B_k\vert.
\end{equation}
Como $d(A_1,A_2)>0$, existe $\varepsilon>0$ tal que $d(x,y)\ge\varepsilon$, para todos $x\in A_1$, $y\in A_2$.
Para cada $k\ge1$ com $\vert B_k\vert>0$, podemos escolher uma partição $P_k$ de $B_k$ de modo que os sub-blocos
de $B_k$ determinados por $P_k$ tenham todos diâmetro menor do que $\varepsilon$. Seja $\overline{P_k^1}$ (respectivamente,
$\overline{P_k^2}$) o conjunto dos sub-blocos de $B_k$ determinados por $P_k$ que interceptam $A_1$ (respectivamente,
interceptam $A_2$). Um bloco de diâmetro menor do que $\varepsilon$ não pode interceptar ambos os conjuntos $A_1$ e $A_2$
e portanto $\overline{P_k^1}$ e $\overline{P_k^2}$ são subconjuntos disjuntos de $\overline{P_k}$. Segue do Lema~\ref{thm:particionabloco}
que:
\begin{equation}\label{eq:somaPk12}
\sum_{\mathfrak b\in\overline{P_k^1}}\vert\mathfrak b\vert+\sum_{\mathfrak b\in\overline{P_k^2}}\vert\mathfrak b\vert\le
\vert B_k\vert.
\end{equation}
Como $A_1\subset\bigcup_{k=1}^\infty B_k$, temos que a coleção formada pelos blocos $B_k$ com $\vert B_k\vert=0$ e pelos
blocos pertencentes a $\overline{P_k^1}$ para algum $k$ com $\vert B_k\vert>0$ constitui uma cobertura enumerável de $A_1$
por blocos retangulares $n$-dimensionais; logo:
\begin{equation}\label{eq:A1lessPk1bar}
\leb^*(A_1)\le\sum_{\substack{k\ge1\\\vert B_k\vert>0}}\sum_{\mathfrak b\in\overline{P_k^1}}\vert\mathfrak b\vert.
\end{equation}
Similarmente:
\begin{equation}\label{eq:A2lessPk1bar}
\leb^*(A_2)\le\sum_{\substack{k\ge1\\\vert B_k\vert>0}}\sum_{\mathfrak b\in\overline{P_k^2}}\vert\mathfrak b\vert.
\end{equation}
Somando as desigualdades \eqref{eq:A1lessPk1bar} e \eqref{eq:A2lessPk1bar} e usando \eqref{eq:somaPk12} obtemos
\eqref{eq:A1A2Bk}, o que completa a demonstração.
\end{proof}

\begin{cor}\label{thm:medidacompactosdisjuntos}
Se $K_1$, \dots, $K_t$ são subconjuntos compactos dois a dois disjuntos de $\R^n$ então $\leb^*\big(\bigcup_{i=1}^tK_i\big)
=\sum_{i=1}^t\leb^*(K_i)$.
\end{cor}
\begin{proof}
O caso $t=2$ segue do Lema~\ref{eq:distanciapositiva}, observando que a distância entre compactos disjuntos é positiva.
O caso geral segue por indução.
\end{proof}

\begin{cor}\label{thm:aditivafinitablocos}
Se $B_1$, \dots, $B_t$ são blocos retangulares $n$-di\-men\-sio\-nais com interiores dois a dois disjuntos então
$\leb^*\big(\bigcup_{r=1}^tB_r\big)=\sum_{r=1}^t\vert B_r\vert$.
\end{cor}
\begin{proof}
Dado $\varepsilon>0$, podemos para cada $r=1,\ldots,t$ encontrar um bloco retangular $n$-dimensional $B_r'$ contido
no interior de $B_r$ e satisfazendo $\vert B_r'\vert\ge(1-\varepsilon)\vert B_r\vert$ (note que no caso $\vert B_r\vert=0$ podemos
tomar $B_r'=\emptyset$). Os blocos $B_r'$, $r=1,\ldots,t$ são subconjuntos compactos dois a dois disjuntos de $\R^n$ e portanto
o Corolário~\ref{thm:medidacompactosdisjuntos} nos dá:
\[\leb^*\Big(\bigcup_{r=1}^tB_r\Big)\ge\leb^*\Big(\bigcup_{r=1}^tB_r'\Big)=\sum_{r=1}^t\leb^*(B_r')
=\sum_{r=1}^t\vert B_r'\vert\ge(1-\varepsilon)\sum_{r=1}^t\vert B_r\vert.\]
Como $\varepsilon>0$ é arbitrário, concluímos que:
\[\leb^*\Big(\bigcup_{r=1}^tB_r\Big)\ge\sum_{r=1}^t\vert B_r\vert.\]
A desigualdade oposta segue do Lema~\ref{thm:sigmasubad}.
\end{proof}

\begin{cor}\label{thm:corinfinitosblocos}
Se $(B_r)_{r\ge1}$ é uma seqüência de blocos retangulares $n$-dimensionais com interiores dois a dois disjuntos então:
\[\leb^*\Big(\bigcup_{r=1}^\infty B_r\Big)=\sum_{r=1}^\infty\vert B_r\vert.\]
\end{cor}
\begin{proof}
O Corolário~\ref{thm:aditivafinitablocos} nos dá:
\[\leb^*\Big(\bigcup_{r=1}^\infty B_r\Big)\ge\leb^*\Big(\bigcup_{r=1}^t B_r\Big)=\sum_{r=1}^t\vert B_r\vert,\]
para todo $t\ge1$. Fazendo $t\to\infty$ obtemos:
\[\leb^*\Big(\bigcup_{r=1}^\infty B_r\Big)\ge\sum_{r=1}^\infty\vert B_r\vert.\]
A desigualdade oposta segue do Lema~\ref{thm:sigmasubad}.
\end{proof}

\begin{defin}\label{thm:defcubo}
Um {\em cubo $n$-dimensional\/}\index[indice]{cubo!$n$-dimensional} é um bloco
retangular $n$-dimensional não vazio $B=\prod_{i=1}^n[a_i,b_i]$ tal que:
\[b_1-a_1=b_2-a_2=\cdots=b_n-a_n;\]
o valor comum aos escalares $b_i-a_i$ é chamado a {\em aresta\/}\index[indice]{aresta!de um cubo}
de $B$.
\end{defin}

\begin{lem}\label{thm:abertocubos}
Se $U\subset\R^n$ é um aberto então existe um conjunto enumerável $\mathcal R$ de cubos $n$-dimensionais
com interiores dois a dois disjuntos tal que $U=\bigcup_{B\in\mathcal R}B$. Em particular,
$U$ é igual à união de uma coleção enumerável de blocos retangulares $n$-dimensionais
com interiores dois a dois disjuntos.
\end{lem}
\begin{proof}
Para cada $k\ge1$ seja $\mathcal R_k$ o conjunto de todos os cubos $n$-dimensionais de aresta $\frac1{2^k}$ e com vértices em pontos
de $\R^n$ cujas coordenadas são múltiplos inteiros de $\frac1{2^k}$; mais precisamente:
\[\mathcal R_k=\Big\{\big[\tfrac{a_1}{2^k},\tfrac{a_1+1}{2^k}\big]\times\cdots\times\big[\tfrac{a_n}{2^k},\tfrac{a_n+1}{2^k}\big]:
a_1,\ldots,a_n\in\Z\Big\}.\]
Cada $\mathcal R_k$ é portanto um conjunto enumerável de cubos $n$-dimensionais.
As seguintes propriedades são de fácil verificação:
\begin{itemize}
\item[(a)] os cubos pertencentes a $\mathcal R_k$ possuem interiores dois a dois disjuntos, para todo $k\ge1$;
\item[(b)] $\R^n=\bigcup_{B\in\mathcal R_k}B$, para todo $k\ge1$;
\item[(c)] dados $k,l\ge1$ com $k\ge l$ então todo cubo pertencente a $\mathcal R_k$ está contido em algum cubo
pertencente a $\mathcal R_l$;
\item[(d)] todo cubo pertencente a $\mathcal R_k$ tem diâmetro igual a $\frac{\sqrt n}{2^k}$.
\end{itemize}
Construiremos agora indutivamente uma seqüência $(\mathcal R'_k)_{k\ge1}$ onde cada $\mathcal R'_k$ é um subconjunto de $\mathcal R_k$.
Seja $\mathcal R'_1$ o conjunto dos cubos $B\in\mathcal R_1$ tais que $B\subset U$. Supondo $\mathcal R'_i$ construído
para $i=1,\ldots,k$, seja $\mathcal R'_{k+1}$ o conjunto dos cubos $B\in\mathcal R_{k+1}$ que estão contidos em $U$
e que tem interior disjunto do interior de todos os cubos pertencentes a $\bigcup_{i=1}^k\mathcal R'_i$.
Tome $\mathcal R=\bigcup_{k=1}^\infty\mathcal R'_k$. Como cada $\mathcal R_k$ é enumerável, segue que $\mathcal R$ é enumerável.
Afirmamos que os cubos pertencentes a $\mathcal R$ possuem interiores dois a dois disjuntos.
De fato, sejam $B_1,B_2\in\mathcal R$ cubos distintos, digamos $B_1\in\mathcal R'_k$ e $B_2\in\mathcal R'_l$ com $k\ge l$.
Se $k>l$ então, por construção, o interior de $B_1$ é disjunto do interior de qualquer cubo pertencente a
$\bigcup_{i=1}^{k-1}\mathcal R'_i$; em particular, o interior de $B_1$ é disjunto do interior de $B_2$.
Se $k=l$, segue da propriedade (a) acima que os cubos $B_1$ e $B_2$ possuem interiores disjuntos.
Para terminar a demonstração, verifiquemos que $U=\bigcup_{B\in\mathcal R}B$. Obviamente temos $\bigcup_{B\in\mathcal R}B\subset U$.
Seja $x\in U$. Como $U$ é aberto, existe $k\ge1$ tal que a bola fechada de centro $x$ e raio $\frac{\sqrt n}{2^k}$
está contida em $U$. Em vista das propriedades (b) e (d) acima, vemos que existe $B\in\mathcal R_k$ com $x\in B$ e,
além disso, $B\subset U$. Se $B\in\mathcal R'_k$ então $x\in B\in\mathcal R$; caso contrário, existem $l<k$ e um cubo
$B_1\in\mathcal R'_l$ tal que os interiores de $B$ e $B_1$ se interceptam. Em vista da propriedade (c), existe um cubo
$B_2\in\mathcal R_l$ contendo $B$. Daí $B_1,B_2\in\mathcal R_l$ e os interiores de $B_1$ e $B_2$ se interceptam;
a propriedade (a) implica então que $B_1=B_2$ e portanto $x\in B\subset B_2=B_1\in\mathcal R$. Em qualquer caso,
mostramos que $x\in\bigcup_{B\in\mathcal R}B$, o que completa a demonstração.
\end{proof}

\begin{lem}
Todo subconjunto compacto de $\R^n$ é mensurável.
\end{lem}
\begin{proof}
Seja $K\subset\R^n$ um subconjunto compacto e seja dado $\varepsilon>0$. Pelo Lema~\ref{thm:aproxaberto} existe
um aberto $U\supset K$ tal que $\leb^*(U)\le\leb^*(K)+\varepsilon$. Vamos mostrar que $\leb^*(U\setminus K)\le\varepsilon$.
Pelo Lema~\ref{thm:abertocubos}, o aberto $U\setminus K$ pode ser escrito como uma união enumerável $U\setminus K=\bigcup_{k=1}^\infty B_k$
de blocos retangulares $n$-dimensionais com interiores dois a dois disjuntos. Para cada $t\ge1$
os conjuntos $K$ e $\bigcup_{k=1}^tB_k$ são compactos e disjuntos; os Corolários~\ref{thm:medidacompactosdisjuntos}
e \ref{thm:aditivafinitablocos} implicam então que:
\[\leb^*(K)+\sum_{k=1}^t\vert B_k\vert=\leb^*(K)+\leb^*\Big(\bigcup_{k=1}^tB_k\Big)
=\leb^*\Big(K\cup\bigcup_{k=1}^tB_k\Big)\le\leb^*(U).\]
Como $K$ é limitado,
a Observação~\ref{thm:boundfinite} nos diz que $\leb^*(K)<+\infty$ e portanto a desigualdade acima implica que:
\[\sum_{k=1}^t\vert B_k\vert\le\leb^*(U)-\leb^*(K)\le\varepsilon.\]
Como $t\ge1$ é arbitrário, concluímos que $\sum_{k=1}^\infty\vert B_k\vert\le\varepsilon$
e, finalmente, o Corolário~\ref{thm:corinfinitosblocos} nos dá $\leb^*(U\setminus K)\le\varepsilon$.
\end{proof}

\begin{cor}\label{thm:fechadomens}
Todo subconjunto fechado de $\R^n$ é mensurável.
\end{cor}
\begin{proof}
Se $F\subset\R^n$ é fechado então $F=\bigcup_{k=1}^\infty\big(F\cap[-k,k]^n\big)$ é uma união
enumerável de compactos. A conclusão segue do Lema~\ref{thm:uniaoenummens}.
\end{proof}

\begin{defin}
Um subconjunto de $\R^n$ é dito {\em de tipo $G_\delta$\/}\index[indice]{conjunto!de tipo $G_\delta$}
(ou, simplesmente, um {\em conjunto $G_\delta$})
se pode ser escrito como uma interseção de uma coleção enumerável de abertos de $\R^n$. Similarmente,
um subconjunto de $\R^n$ é dito {\em de tipo $F_\sigma$\/}\index[indice]{conjunto!de tipo $F_\sigma$}
(ou, simplesmente, um {\em conjunto $F_\sigma$})
se pode ser escrito como uma união de uma coleção enumerável de fechados de $\R^n$.
\end{defin}
Obviamente o complementar de um conjunto de tipo $G_\delta$ é de tipo $F_\sigma$ (e vice-versa).

\begin{cor}\label{thm:Fsigmamens}
Todo subconjunto de $\R^n$ de tipo $F_\sigma$ é mensurável.
\end{cor}
\begin{proof}
Segue do Corolário~\ref{thm:fechadomens} e do Lema~\ref{thm:uniaoenummens}.
\end{proof}

\begin{lem}\label{thm:Gdeltazero}
Se $A\subset\R^n$ é mensurável então existe um subconjunto $Z$ de $\R^n$ de tipo $G_\delta$
tal que $A\subset Z$ e $\leb^*(Z\setminus A)=0$.
\end{lem}
\begin{proof}
Para todo $k\ge1$ existe um aberto $U_k\subset\R^n$ contendo $A$ tal que $\leb^*(U_k\setminus A)<\frac1k$.
Daí o conjunto $Z=\bigcap_{k=1}^\infty U_k$ é um $G_\delta$ que contém $A$ e:
\[\leb^*(Z\setminus A)\le\leb^*(U_k\setminus A)<\frac1k,\]
para todo $k\ge1$. Logo $\leb^*(Z\setminus A)=0$.
\end{proof}

\begin{cor}\label{thm:complmens}
O complementar de um subconjunto mensurável de $\R^n$ também é mensurável.
\end{cor}
\begin{proof}
Seja $A\subset\R^n$ um subconjunto mensurável. Pelo Lema~\ref{thm:Gdeltazero} existe um conjunto
$Z$ de tipo $G_\delta$ contendo $A$ tal que $\leb^*(Z\setminus A)=0$. Daí $Z^\compl\subset A^\compl$
e $A^\compl\setminus Z^\compl=Z\setminus A$; logo:
\[A^\compl=Z^\compl\cup(Z\setminus A).\]
O conjunto $Z^\compl$ é de tipo $F_\sigma$ e portanto mensurável, pelo Corolário~\ref{thm:Fsigmamens}.
A conclusão segue dos Lemas~\ref{thm:uniaoenummens} e \ref{thm:nulamens}.
\end{proof}

\begin{cor}\label{thm:innerfechado}
Se $A\subset\R^n$ é mensurável então para todo $\varepsilon>0$ existe um subconjunto
fechado $F\subset\R^n$ contido em $A$ tal que $\leb^*(A\setminus F)<\varepsilon$.
\end{cor}
\begin{proof}
Pelo Corolário~\ref{thm:complmens}, $A^\compl$ é mensurável e portanto existe um aberto
$U\subset\R^n$ contendo $A^\compl$ tal que $\leb^*(U\setminus A^\compl)<\varepsilon.$
Tomando $F=U^\compl$ então $F$ é fechado e $F\subset A$. Como $A\setminus F=U\setminus A^\compl$,
segue que $\leb^*(A\setminus F)<\varepsilon$.
\end{proof}

\begin{cor}\label{thm:innerFsigma}
Se $A\subset\R^n$ é mensurável então existe um subconjunto $W$ de $\R^n$ de tipo $F_\sigma$ tal que
$W\subset A$ e $\leb^*(A\setminus W)=0$.
\end{cor}
\begin{proof}
Pelo Corolário~\ref{thm:complmens}, $A^\compl$ também é mensurável e portanto, pelo Lema~\ref{thm:Gdeltazero}
existe um subconjunto $Z$ de $\R^n$ de tipo $G_\delta$ tal que $A^\compl\subset Z$ e $\leb^*(Z\setminus A^\compl)=0$.
Tomando $W=Z^\compl$ então $W$ é de tipo $F_\sigma$ e $W\subset A$. Como $A\setminus W=Z\setminus A^\compl$,
segue que $\leb^*(A\setminus W)=0$.
\end{proof}

\begin{defin}\label{thm:defalgsigmaalg}
Seja $X$ um conjunto arbitrário. Uma {\em álgebra\/}\index[indice]{algebra@álgebra}
de partes de $X$ é um subconjunto não vazio $\mathcal A\subset\wp(X)$ satisfazendo as seguintes condições:
\begin{itemize}
\item[(a)] se $A\in\mathcal A$ então $A^\compl\in\mathcal A$;
\item[(b)] se $A,B\in\mathcal A$ então $A\cup B\in\mathcal A$.
\end{itemize}
Uma {\em $\sigma$-álgebra\/}\index[indice]{sigma algebra@$\sigma$-álgebra}
de partes de $X$ é um subconjunto não vazio $\mathcal A\subset\wp(X)$ satisfazendo a condição (a) acima
e também a condição:
\begin{itemize}
\item[(b')] se $(A_k)_{k\ge1}$ é uma seqüência de elementos de $\mathcal A$
então $\bigcup_{k=1}^\infty A_k\in\mathcal A$.
\end{itemize}
\end{defin}
Note que toda $\sigma$-álgebra de partes de $X$ é também uma álgebra de partes de $X$.
De fato, se $\mathcal A$ é uma $\sigma$-álgebra
de partes de $X$ e se $A,B\in\mathcal A$, podemos tomar $A_1=A$ e $A_k=B$ para todo $k\ge2$ na condição (b');
daí $A\cup B=\bigcup_{k=1}^\infty A_k\in\mathcal A$.

\begin{rem}\label{thm:obsXnaalg}
Se $\mathcal A$ é uma álgebra (em particular, se $\mathcal A$ é uma $\sigma$-álgebra) de partes
de $X$ então $X\in\mathcal A$ e $\emptyset\in\mathcal A$.
De fato, como $\mathcal A\ne\emptyset$, existe algum elemento $A\in\mathcal A$. Daí $A^\compl\in\mathcal A$
e portanto $X=A\cup A^\compl\in\mathcal A$; além do mais, $\emptyset=X^\compl\in\mathcal A$.
\end{rem}

\begin{teo}\label{thm:menssigmalgebra}
A coleção de todos os subconjuntos mensuráveis de $\R^n$ é uma $\sigma$-álgebra de partes
de $\R^n$ que contém todos os subconjuntos abertos de $\R^n$ e todos os subconjuntos
de $\R^n$ com medida exterior nula.
\end{teo}
\begin{proof}
Segue da Observação~\ref{thm:abertomens}, dos Lemas~\ref{thm:uniaoenummens} e \ref{thm:nulamens}
e do Corolário~\ref{thm:complmens}.
\end{proof}

\begin{defin}\label{thm:defsigmagerada}
Se $X$ é um conjunto arbitrário e se $\mathcal C\subset\wp(X)$ é uma coleção arbitrária de partes de $X$ então
a {\em $\sigma$-álgebra de partes de $X$ gerada por $\mathcal C$\/}\index[indice]{sigma algebra@$\sigma$-álgebra!gerada por uma colecao de conjuntos@gerada por uma coleção\hfil\break
de conjuntos}, denotada por $\sigma[\mathcal C]$\index[simbolos]{$\sigma[\mathcal C]$}, é a menor $\sigma$-álgebra
de partes de $X$ que contém $\mathcal C$, i.e., $\sigma[\mathcal C]$ é uma $\sigma$-álgebra de partes de $X$ tal que:
\begin{enumerate}
\item\label{itm:sigma1} $\mathcal C\subset\sigma[\mathcal C]$;
\item\label{itm:sigma2} se $\mathcal A$ é uma $\sigma$-álgebra de partes de $X$ tal que $\mathcal C\subset\mathcal A$ então
$\sigma[\mathcal C]\subset\mathcal A$.
\end{enumerate}
Dizemos também que $\mathcal C$ é um
{\em conjunto de geradores\/}\index[indice]{conjunto!de geradores!para uma sigma-algebra@para uma $\sigma$-álgebra}\index[indice]{geradores!para uma sigma-algebra@para uma $\sigma$-álgebra}
para a $\sigma$-álgebra $\sigma[\mathcal C]$.
A $\sigma$-álgebra de partes de $\R^n$ gerada pela coleção de todos os subconjuntos abertos de $\R^n$ é chamada a {\em
$\sigma$-álgebra de Borel\/}\index[indice]{sigma algebra@$\sigma$-álgebra!de Borel}\index[indice]{Borel!sigma algebra de@$\sigma$-álgebra de} de
$\R^n$ e é denotada por $\Borel(\R^n)$. Os elementos de $\Borel(\R^n)$\index[simbolos]{$\Borel(\R^n)$}
são chamados {\em conjuntos Boreleanos\/}\index[indice]{conjunto!Boreleano}\index[indice]{Boreleano} de $\R^n$.
\end{defin}
No Exercício~\ref{exe:sigmagerada} pedimos ao leitor para justificar o fato de que a $\sigma$-álgebra
gerada por uma coleção $\mathcal C\subset\wp(X)$ está de fato bem definida,
ou seja, existe uma única $\sigma$-álgebra $\sigma[\mathcal C]$ satisfazendo as propriedades \eqref{itm:sigma1} e \eqref{itm:sigma2}
acima.

\begin{cor}\label{thm:corBormens}
Todo conjunto Boreleano de $\R^n$ é mensurável.
\end{cor}
\begin{proof}
Pelo Teorema~\ref{thm:menssigmalgebra}, os conjuntos mensuráveis formam uma $\sigma$-álgebra que contém os abertos
de $\R^n$; portanto, deve conter também a $\sigma$-álgebra de Borel.
\end{proof}

\begin{lem}\label{thm:propalgebras}
Se $\mathcal A$ é uma álgebra de partes de um conjunto $X$ e se $A,B\in\mathcal A$ então
$A\cap B$ e $A\setminus B$ pertencem a $\mathcal A$. Além do mais, se $\mathcal A$ é uma
$\sigma$-álgebra de partes de $X$ e se $(A_k)_{k\ge1}$ é uma seqüência de elementos de $\mathcal A$
então $\bigcap_{k=1}^\infty A_k\in\mathcal A$.
\end{lem}
\begin{proof}
Se $\mathcal A$ é uma álgebra e $A,B\in\mathcal A$ então $A^\compl,B^\compl\in\mathcal A$
e portanto $A\cap B=(A^\compl\cup B^\compl)^\compl\in\mathcal A$; além do mais,
$A\setminus B=A\cap B^\compl\in\mathcal A$. Se $\mathcal A$ é uma $\sigma$-álgebra
e $(A_k)_{k\ge1}$ é uma seqüência de elementos de $\mathcal A$ então $A_k^\compl\in\mathcal A$
para todo $k\ge1$ e portanto $\bigcap_{k=1}^\infty A_k=\big(\bigcup_{k=1}^\infty A_k^\compl\big)^\compl\in\mathcal A$.
\end{proof}

\begin{cor}\label{thm:interdifmens}
A interseção de uma coleção enumerável de subconjuntos mensuráveis de $\R^n$ é mensurável e a
diferença de dois subconjuntos mensuráveis de $\R^n$ é mensurável.
\end{cor}
\begin{proof}
Segue do Teorema~\ref{thm:menssigmalgebra} e do Lema~\ref{thm:propalgebras}.
\end{proof}

\begin{lem}\label{thm:A0limitado}
Para todo $A\subset\R^n$ com $\leb^*(A)<+\infty$ e para todo $\varepsilon>0$
existe um subconjunto limitado $A_0\subset A$ tal que:
\[\leb^*(A)-\leb^*(A_0)\le\leb^*(A\setminus A_0)<\varepsilon.\]
Além do mais, se $A$ é mensurável, podemos escolher o conjunto $A_0$ também mensurável.
\end{lem}
\begin{proof}
Pelo Lema~\ref{thm:aproxaberto} existe um aberto $U\subset\R^n$ contendo $A$ tal que
$\leb^*(U)\le\leb^*(A)+1<+\infty$. O Lema~\ref{thm:abertocubos} nos permite escrever
$U=\bigcup_{k=1}^\infty B_k$, onde $(B_k)_{k\ge1}$ é uma seqüência de blocos retangulares
$n$-dimensionais com interiores dois a dois disjuntos. O Corolário~\ref{thm:corinfinitosblocos} nos dá:
\[\sum_{k=1}^\infty\vert B_k\vert=\leb^*(U)<+\infty;\]
portanto a série $\sum_{k=1}^\infty\vert B_k\vert$ é convergente e
existe $t\ge1$ tal que:
\[\sum_{k=t+1}^\infty\vert B_k\vert<\varepsilon.\]
Seja $A_0=A\cap\big(\bigcup_{k=1}^tB_k\big)$. Temos que $A_0\subset A$ e $A_0$ é limitado. Note
que se $A$ é mensurável então $A_0$ também é mensurável. Como
$A\subset\bigcup_{k=1}^\infty B_k$ segue que $A\setminus A_0\subset\bigcup_{k=t+1}^\infty B_k$
e portanto:
\[\leb^*(A\setminus A_0)\le\leb^*\Big(\bigcup_{k=t+1}^\infty B_k\Big)
\le\sum_{k=t+1}^\infty\vert B_k\vert<\varepsilon.\]
A desigualdade $\leb^*(A)-\leb^*(A_0)\le\leb^*(A\setminus A_0)$ segue do Corolário~\ref{thm:AminusB}.
\end{proof}

\begin{cor}\label{thm:innercompacto}
Se $A\subset\R^n$ é mensurável e $\leb^*(A)<+\infty$ então para todo $\varepsilon>0$
existe um subconjunto compacto $K\subset\R^n$ contido em $A$ tal que:
\[\leb^*(A)-\leb^*(K)\le\leb^*(A\setminus K)<\varepsilon.\]
\end{cor}
\begin{proof}
Pelo Lema~\ref{thm:A0limitado}, existe um subconjunto limitado mensurável $A_0\subset A$ tal que
$\leb^*(A\setminus A_0)<\frac\varepsilon2$ e pelo Corolário~\ref{thm:innerfechado} existe um
subconjunto fechado $K\subset\R^n$ contido em $A_0$ tal que $\leb^*(A_0\setminus K)<\frac\varepsilon2$.
Obviamente $K\subset A$ e $K$ é compacto. Como $A\setminus K=(A\setminus A_0)\cup(A_0\setminus K)$, obtemos:
\[\leb^*(A\setminus K)\le\leb^*(A\setminus A_0)+\leb^*(A_0\setminus K)<\varepsilon.\]
A desigualdade $\leb^*(A)-\leb^*(K)\le\leb^*(A\setminus K)$ segue do Corolário~\ref{thm:AminusB}.
\end{proof}

\begin{prop}\label{thm:sigmaditiva}
Se $A_1$, \dots, $A_t$ são subconjuntos mensuráveis dois a dois disjuntos de $\R^n$ então:
\begin{equation}\label{eq:finiteadit}
\leb^*\Big(\bigcup_{r=1}^tA_r\Big)=\sum_{r=1}^t\leb^*(A_r).
\end{equation}
Além do mais, se $(A_r)_{r\ge1}$ é uma seqüência de subconjuntos mensuráveis dois a dois
disjuntos de $\R^n$ então:
\begin{equation}\label{eq:sigmadit}
\leb^*\Big(\bigcup_{r=1}^\infty A_r\Big)=\sum_{r=1}^\infty\leb^*(A_r).
\end{equation}
\end{prop}
\begin{proof}
Comecemos provando \eqref{eq:finiteadit}. Se $\leb^*(A_r)=+\infty$ para algum $r=1,\ldots,t$
então também $\leb^*\big(\bigcup_{r=1}^tA_r\big)=+\infty$
e portanto não há nada a mostrar. Se $\leb^*(A_r)<+\infty$ para todo $r=1,\ldots,t$
então para todo $\varepsilon>0$ o Corolário~\ref{thm:innercompacto} nos dá um subconjunto compacto
$K_r$ de $A_r$ tal que $\leb^*(A_r)-\leb^*(K_r)<\frac\varepsilon t$. Usando o Corolário~\ref{thm:medidacompactosdisjuntos} obtemos:
\begin{multline*}
\leb^*\Big(\bigcup_{r=1}^tA_r\Big)\ge\leb^*\Big(\bigcup_{r=1}^tK_r\Big)=\sum_{r=1}^t\leb^*(K_r)>
\sum_{r=1}^t\big(\leb^*(A_r)-\tfrac\varepsilon t\big)\\
=\Big(\sum_{r=1}^t\leb^*(A_r)\Big)-\varepsilon.
\end{multline*}
Como $\varepsilon>0$ é arbitrário, concluímos que:
\[\leb^*\Big(\bigcup_{r=1}^tA_r\Big)\ge\sum_{r=1}^t\leb^*(A_r).\]
O Lema~\ref{thm:sigmasubad} nos dá a desigualdade oposta, provando \eqref{eq:finiteadit}.
Passemos então à prova de \eqref{eq:sigmadit}. A identidade \eqref{eq:finiteadit} nos dá:
\[\leb^*\Big(\bigcup_{r=1}^\infty A_r\Big)\ge\leb^*\Big(\bigcup_{r=1}^tA_r\Big)=\sum_{r=1}^t\leb^*(A_r),\]
para todo $t\ge1$. Fazendo $t\to\infty$ concluímos que:
\[\leb^*\Big(\bigcup_{r=1}^\infty A_r\Big)\ge\sum_{r=1}^\infty\leb^*(A_r).\]
Novamente a desigualdade oposta segue do Lema~\ref{thm:sigmasubad}, o que prova \eqref{eq:sigmadit}.
\end{proof}

\begin{defin}\label{thm:defespacomedida}
Sejam $X$ um conjunto e $\mathcal A$ uma $\sigma$-álgebra de partes de $X$.
O par $(X,\mathcal A)$ é chamado um {\em espaço mensurável};\index[indice]{mensuravel@mensurável!espaco@espaço}
\index[indice]{espaco@espaço!mensuravel@mensurável} uma {\em medida\/}\index[indice]{medida} no espaço mensurável
$(X,\mathcal A)$ é uma função $\mu:\mathcal A\to[0,+\infty]$ tal que $\mu(\emptyset)=0$ e tal que, se $(A_k)_{k\ge1}$ é
uma seqüência de elementos dois a dois disjuntos de $\mathcal A$ então:
\begin{equation}\label{eq:musigmaditiva}
\mu\Big(\bigcup_{k=1}^\infty A_k\Big)=\sum_{k=1}^\infty\mu(A_k).
\end{equation}
Os elementos da $\sigma$-álgebra $\mathcal A$ são ditos {\em subconjuntos mensuráveis\/}\index[indice]{conjunto!mensuravel@mensurável}
de $X$.
A trinca $(X,\mathcal A,\mu)$ é chamada um {\em espaço de medida}\index[indice]{medida!espaco de@espaço de}
\index[indice]{espaco@espaço!de medida}.
\end{defin}
Se $(X,\mathcal A,\mu)$ é um espaço de medida e se $A_1$, \dots, $A_t$ é uma
coleção finita de elementos dois a dois disjuntos de $\mathcal A$ então
$\mu\big(\bigcup_{k=1}^tA_k\big)=\sum_{k=1}^t\mu(A_k)$. De fato, basta tomar
$A_k=\emptyset$ para $k>t$ e usar \eqref{eq:musigmaditiva}.

\begin{notation}
Denotaremos por $\Lebmens(\R^n)$\index[simbolos]{$\Lebmens(\R^n)$} a $\sigma$-álgebra de
todos os subconjuntos Lebesgue mensuráveis de $\R^n$ e por $\leb:\Lebmens(\R^n)
\to[0,+\infty]$\index[simbolos]{$\leb(A)$} a restrição à $\Lebmens(\R^n)$ da função
$\leb^*:\wp(\R^n)\to[0,+\infty]$ que associa a cada parte de $\R^n$ sua medida
exterior de Lebesgue.
\end{notation}

\begin{defin}
Se $A\subset\R^n$ é um subconjunto mensurável então o escalar
$\leb(A)\in[0,+\infty]$ é chamado a {\em medida de Lebesgue\/}\index[indice]{medida!de Lebesgue}\index[indice]{Lebesgue!medida de}
de $A$.
\end{defin}
Note que $\leb(A)=\leb^*(A)$ para todo $A\in\Lebmens(\R^n)$, i.e.,
a medida de Lebesgue de um conjunto mensurável simplesmente coincide com sua
medida exterior de Lebesgue; apenas nos permitimos remover o adjetivo
``exterior'' quando lidamos com conjuntos mensuráveis.

Provamos o seguinte:
\begin{teo}\label{thm:Lebesguemedida}
A trinca $\big(\R^n,\Lebmens(\R^n),\leb\big)$ é um espaço de medida.
\end{teo}
\begin{proof}
Segue do Teorema~\ref{thm:menssigmalgebra} e da Proposição~\ref{thm:sigmaditiva}.
\end{proof}

\begin{lem}\label{thm:muAminusB}
Seja $(X,\mathcal A,\mu)$ um espaço de medida e sejam $A_1,A_2\in\mathcal A$
com $A_1\subset A_2$. Então $\mu(A_1)\le\mu(A_2)$; além do mais, se $\mu(A_1)<+\infty$ então:
\[\mu(A_2\setminus A_1)=\mu(A_2)-\mu(A_1).\]
\end{lem}
\begin{proof}
Basta observar que $A_2=A_1\cup(A_2\setminus A_1)$ é uma união disjunta de elementos de $\mathcal A$ e
portanto $\mu(A_2)=\mu(A_1)+\mu(A_2\setminus A_1)$.
\end{proof}

\begin{notation}
Se $(A_k)_{k\ge1}$ é uma seqüência de conjuntos então a notação $A_k\nearrow A$\index[simbolos]{$A_k\nearrow A$} indica que
$A_k\subset A_{k+1}$ para todo $k\ge1$ (i.e., a seqüência $(A_k)_{k\ge1}$ é {\em crescente})
e que $A=\bigcup_{k=1}^\infty A_k$. Analogamente, escreveremos $A_k\searrow A$\index[simbolos]{$A_k\searrow A$} para indicar que
$A_k\supset A_{k+1}$ para todo $k\ge1$ (i.e., a seqüência $(A_k)_{k\ge1}$ é {\em decrescente})
e que $A=\bigcap_{k=1}^\infty A_k$.
\end{notation}

\begin{lem}\label{thm:setlimits}
Seja $(X,\mathcal A,\mu)$ um espaço de medida e seja $(A_k)_{k\ge1}$ uma se\-qüên\-cia de elementos de $\mathcal A$. Temos:
\begin{itemize}
\item[(a)] se $A_k\nearrow A$ então $\mu(A)=\lim_{k\to\infty}\mu(A_k)$;
\item[(b)] se $A_k\searrow A$ e se $\mu(A_1)<+\infty$ então $\mu(A)=\lim_{k\to\infty}\mu(A_k)$.
\end{itemize}
\end{lem}
\begin{proof}
Provemos inicialmente o item (a). Defina $A_0=\emptyset$ e $B_k=A_k\setminus A_{k-1}$, para todo $k\ge1$;
evidentemente $B_k\in\mathcal A$, para todo $k\ge1$. É fácil ver que os conjuntos $B_k$ são dois a dois disjuntos e
que:
\[A=\bigcup_{k=1}^\infty B_k,\quad A_r=\bigcup_{k=1}^r B_k,\]
para todo $r\ge1$; logo:
\[\mu(A)=\sum_{k=1}^\infty\mu(B_k)=\lim_{r\to\infty}\sum_{k=1}^r\mu(B_k)=\lim_{r\to\infty}\mu(A_r).\]
Passemos à prova do item (b). Se $\mu(A_1)<+\infty$ então
$\mu(A_k)<+\infty$ para todo $k\ge1$. Como $(A_1\setminus A_k)_{k\ge1}$ é uma
seqüência de elementos de $\mathcal A$ e $(A_1\setminus A_k)\nearrow
(A_1\setminus A)$, segue do item (a) que:
\[\lim_{k\to\infty}\mu(A_1\setminus A_k)=\mu(A_1\setminus A).\]
Usando o Lema~\ref{thm:muAminusB} obtemos:
\[\lim_{k\to\infty}\big(\mu(A_1)-\mu(A_k)\big)=\mu(A_1)-\mu(A).\]
Como $\mu(A_1)<+\infty$, a conclusão segue.
\end{proof}

\begin{defin}
Um {\em envelope mensurável\/}\index[indice]{envelope mensuravel@envelope mensurável}%
\index[indice]{mensuravel@mensurável!envelope} de um subconjunto $A$ de $\R^n$ é um subconjunto
mensurável $E$ de $\R^n$ tal que $A\subset E$ e $\leb^*(A)=\leb(E)$.
\end{defin}

\begin{lem}\label{thm:aproxGdelta}
Para todo $A\subset\R^n$ existe um subconjunto $E$ de $\R^n$ de tipo $G_\delta$ contendo $A$
tal que $\leb^*(A)=\leb(E)$.
\end{lem}
\begin{proof}
Para cada $k\ge1$ o Lema~\ref{thm:aproxaberto} nos dá um aberto $U_k$ contendo $A$ tal que
$\leb(U_k)\le\leb^*(A)+\frac1k$. Daí $E=\bigcap_{k=1}^\infty U_k$ é um $G_\delta$ contendo $A$ e:
\[\leb^*(A)\le\leb(E)\le\leb(U_k)\le\leb^*(A)+\frac1k,\]
para todo $k\ge1$. A conclusão segue.
\end{proof}

\begin{cor}
Todo subconjunto de $\R^n$ admite um envelope mensurável.
\end{cor}
\begin{proof}
Basta observar que todo $G_\delta$ é mensurável (vide Corolário~\ref{thm:interdifmens}).
\end{proof}

\begin{lem}\label{thm:envelopesdisjuntos}
Sejam $A_1$, \dots, $A_t$ subconjuntos de $\R^n$ e suponha que existam subconjuntos mensuráveis dois a dois disjuntos
$E_1$, \dots, $E_t$ de $\R^n$ de modo que $A_k\subset E_k$, para $k=1,\ldots,t$. Então:
\[\leb^*\Big(\bigcup_{k=1}^tA_k\Big)=\sum_{k=1}^t\leb^*(A_k).\]
Além do mais, se $(A_k)_{k\ge1}$ é uma seqüência de subconjuntos de $\R^n$ tal que existe uma seqüência
$(E_k)_{k\ge1}$ de subconjuntos mensuráveis de $\R^n$ dois a dois disjuntos de modo que $A_k\subset E_k$ para todo $k\ge1$
então:
\[\leb^*\Big(\bigcup_{k=1}^\infty A_k\Big)=\sum_{k=1}^\infty\leb^*(A_k).\]
\end{lem}
\begin{proof}
Tomando $A_k=E_k=\emptyset$ para $k>t$, podemos considerar apenas o caso de uma seqüência infinita de subconjuntos
de $\R^n$. Seja $E$ um envelope mensurável do conjunto $\bigcup_{k=1}^\infty A_k$. Daí, para todo $k\ge1$, o conjunto
$E'_k=E\cap E_k$ é mensurável e $A_k\subset E'_k$.
Como os conjuntos $E'_k$ são dois a dois disjuntos e $\bigcup_{k=1}^\infty E'_k\subset E$, temos:
\[\leb^*\Big(\bigcup_{k=1}^\infty A_k\Big)=\leb(E)\ge\leb\Big(\bigcup_{k=1}^\infty E'_k\Big)
=\sum_{k=1}^\infty\leb(E'_k)\ge\sum_{k=1}^\infty\leb^*(A_k).\]
A desigualdade $\leb^*\big(\bigcup_{k=1}^\infty A_k\big)\le\sum_{k=1}^\infty\leb^*(A_k)$ segue do Lema~\ref{thm:sigmasubad}.
\end{proof}

\begin{prop}[Carathéodory]\label{thm:Caratheodory}\index[indice]{Caratheodory@Carathéodory}
Um subconjunto $E\subset\R^n$ é mensurável se e somente se para todo $A\subset\R^n$ vale:
\begin{equation}\label{eq:Caratheodory}
\leb^*(A)=\leb^*(A\cap E)+\leb^*(A\cap E^\compl).
\end{equation}
\end{prop}
\begin{proof}
Se $E$ é mensurável então $A=(A\cap E)\cup(A\cap E^\compl)$, onde $A\cap E$ e $A\cap E^\compl$ estão respectivamente
contidos nos conjuntos mensuráveis disjuntos $E$ e $E^\compl$. A identidade \eqref{eq:Caratheodory} segue portanto do Lema~\ref{thm:envelopesdisjuntos}.
Reciprocamente, suponha que a identidade \eqref{eq:Caratheodory} vale para todo $A\subset\R^n$. Para cada $k\ge1$
seja $E_k=E\cap[-k,k]^n$ e seja $Z_k$ um envelope mensurável para $E_k$. A identidade \eqref{eq:Caratheodory} com
$A=Z_k$ nos dá:
\[\leb^*(E_k)=\leb(Z_k)=\leb^*(Z_k\cap E)+\leb^*(Z_k\cap E^\compl).\]
Como $Z_k\cap E\supset E_k$ vemos que:
\[\leb^*(E_k)\ge\leb^*(E_k)+\leb^*(Z_k\cap E^\compl)\ge\leb^*(E_k);\]
como $E_k$ é limitado, temos que $\leb^*(E_k)<+\infty$ (vide Observação~\ref{thm:boundfinite}) e portanto
$\leb^*(Z_k\cap E^\compl)=0$. Em particular, pelo Lema~\ref{thm:nulamens}, $Z_k\cap E^\compl$ é mensurável.
Tomando $Z=\bigcup_{k\ge1}Z_k$ vemos que $E\subset Z$, $Z$ é mensurável e:
\[Z\setminus E=Z\cap E^\compl=\bigcup_{k\ge1}(Z_k\cap E^\compl).\]
Daí $Z\setminus E$ é mensurável e portanto $E=Z\setminus(Z\setminus E)$ também é mensurável.
\end{proof}

\begin{rem}
Na verdade, a demonstração apresentada para a Proposição~\ref{thm:Caratheodory} mostra algo mais forte:
se a identidade \eqref{eq:Caratheodory} vale para todo conjunto mensurável $A\subset\R^n$ então $E$ é mensurável.
Em vista do Lema~\ref{thm:aproxGdelta}, todo subconjunto de $\R^n$ admite um envelope mensurável de tipo $G_\delta$
e portanto a demonstração que apresentamos para a Proposição~\ref{thm:Caratheodory} mostra até mesmo o seguinte:
se a identidade \eqref{eq:Caratheodory} vale para todo subconjunto $A$ de $\R^n$ de tipo $G_\delta$ então $E$ é mensurável.
\end{rem}

\begin{lem}
Seja $(A_k)_{k\ge1}$ uma seqüência de subconjuntos (não necessariamente mensuráveis) de $\R^n$ tal que
$A_k\nearrow A$. Então:
\[\leb^*(A)=\lim_{k\to\infty}\leb^*(A_k).\]
\end{lem}
\begin{proof}
Temos que a seqüência $\big(\leb^*(A_k)\big)_{k\ge1}$ é crescente e limitada superiormente
por $\leb^*(A)$, donde o limite $\lim_{k\to\infty}\leb^*(A_k)$ existe (em $[0,+\infty]$) e é menor ou igual a $\leb^*(A)$.
Para provar que $\leb^*(A)$ é menor ou igual a $\lim_{k\to\infty}\leb^*(A_k)$, escolha um envelope mensurável $E_k$ para $A_k$ e defina
$F_k=\bigcap_{r\ge k}E_r$, para todo $k\ge1$. Daí cada $F_k$ é mensurável e $A_k\subset F_k\subset E_k$,
donde também $F_k$ é um envelope mensurável de $A_k$. Além do mais, temos
$F_k\nearrow F$, onde $F$ é um conjunto mensurável que contém $A$. A conclusão segue agora do Lema~\ref{thm:setlimits} observando que:
\[\leb^*(A)\le\leb(F)=\lim_{k\to\infty}\leb(F_k)=\lim_{k\to\infty}\leb^*(A_k).\qedhere\]
\end{proof}

\begin{subsection}{Medida interior}

O conceito de medida interior é útil para entender melhor o fenômeno da não mensurabilidade de um subconjunto
de $\R^n$.

\begin{defin}
Seja $A$ um subconjunto de $\R^n$. A {\em medida interior de Lebesgue\/}\index[indice]{medida!interior}\index[indice]{Lebesgue!medida interior de}
de $A$ é definida por:
\index[simbolos]{$\leb_*(A)$}\[\leb_*(A)=\sup\big\{\leb(K):K\subset A,\ \text{$K$ compacto}\big\}\in[0,+\infty].\]
\end{defin}

\begin{lem}\label{thm:intextmens}
Se $A\subset\R^n$ é mensurável então $\leb_*(A)=\leb^*(A)$. Reciprocamente, dado $A\subset\R^n$ com
$\leb_*(A)=\leb^*(A)<+\infty$ então $A$ é mensurável.
\end{lem}
\begin{proof}
Suponha que $A$ é mensurável e mostremos que as medidas interior e exterior de $A$ coincidem. Em primeiro lugar, se $A$ tem medida
exterior finita isso segue diretamente do Corolário~\ref{thm:innercompacto}.
Suponha então que $\leb^*(A)=+\infty$. Pelo Corolário~\ref{thm:innerfechado},
existe um subconjunto fechado $F\subset\R^n$ contido em $A$ tal que $\leb^*(A\setminus F)<1$. Daí:
\[\leb^*(A)=\leb^*\big(F\cup(A\setminus F)\big)\le\leb^*(F)+\leb^*(A\setminus F)\le\leb^*(F)+1,\]
e portanto $\leb^*(F)=+\infty$. Para cada $r\ge1$, seja $K_r=F\cap[-r,r]^n$. Daí cada $K_r$ é compacto e $K_r\nearrow F$;
o Lema~\ref{thm:setlimits} nos dá:
\[\lim_{r\to\infty}\leb(K_r)=\leb(F)=+\infty.\]
Logo $\leb_*(A)\ge\sup_{r\ge1}\leb(K_r)=+\infty=\leb^*(A)$. Suponha agora que as medidas interior e exterior
de $A$ são iguais e finitas e mostremos que $A$ é mensurável. Seja dado $\varepsilon>0$. Temos que existe um subconjunto
compacto $K\subset A$ tal que:
\[\leb(K)\ge\leb_*(A)-\frac\varepsilon2=\leb^*(A)-\frac\varepsilon2.\]
Pelo Lema~\ref{thm:aproxaberto}, existe um aberto $U\subset\R^n$ contendo $A$ tal que:
\[\leb(U)\le\leb^*(A)+\frac\varepsilon2.\]
Portanto:
\begin{multline*}
\leb^*(U\setminus A)\le\leb(U\setminus K)=\leb(U)-\leb(K)\\
=\big(\leb(U)-\leb^*(A)\big)+\big(\leb^*(A)-\leb(K)\big)\le\varepsilon.
\end{multline*}
A conclusão segue.
\end{proof}

\begin{cor}\label{thm:corsupcompactosdentro}
Se $A\subset\R^n$ é mensurável então:
\[\leb(A)=\sup\big\{\leb(K):K\subset A,\ \text{\em$K$ compacto}\big\}.\eqno{\qed}\]
\end{cor}

\begin{cor}\label{thm:mensdentronaomens}
Dados $A\subset\R^n$ e $E$ um subconjunto mensurável de $A$ então $\leb(E)\le\leb_*(A)$.
\end{cor}
\begin{proof}
O Lema~\ref{thm:intextmens} nos dá $\leb(E)=\leb_*(E)$. Como $E\subset A$, segue do resultado do Exercício~\ref{exe:medintmonot}
que $\leb_*(E)\le\leb_*(A)$.
\end{proof}

\begin{lem}\label{thm:preintext}
Dados $A_1,A_2\subset\R^n$ então vale a desigualdade:
\[\leb_*(A_1\cup A_2)\le\leb^*(A_1)+\leb_*(A_2);\]
se $A_1\cap A_2=\emptyset$ então vale também a desigualdade:
\[\leb^*(A_1)+\leb_*(A_2)\le\leb^*(A_1\cup A_2).\]
\end{lem}
\begin{proof}
Pelo resultado do Exercício~\ref{exe:internalmens}, existe $W\subset\R^n$ de tipo $F_\sigma$ tal que
$W\subset A_1\cup A_2$ e $\leb(W)=\leb_*(A_1\cup A_2)$. Seja $Z$ um envelope mensurável de $A_1$. Temos
$W\subset(W\setminus Z)\cup Z$ e portanto:
\[\leb_*(A_1\cup A_2)=\leb(W)\le\leb(Z)+\leb(W\setminus Z)\le\leb^*(A_1)+\leb_*(A_2),\]
onde usamos o Corolário~\ref{thm:mensdentronaomens} e o fato que $W\setminus Z$ é um subconjunto mensurável
de $A_2$. Suponha agora que $A_1\cap A_2=\emptyset$. Seja $E$ um envelope mensurável de $A_1\cup A_2$
e seja $F\subset\R^n$ de tipo $F_\sigma$ tal que $F\subset A_2$ e $\leb(F)=\leb_*(A_2)$ (Exercício~\ref{exe:internalmens}).
Como $A_1\cap A_2=\emptyset$, temos $A_1\subset E\setminus F$ e portanto:
\[\leb^*(A_1\cup A_2)=\leb(E)=\leb(E\setminus F)+\leb(F)\ge\leb^*(A_1)+\leb_*(A_2).\qedhere\]
\end{proof}

\begin{cor}\label{thm:intext}
Seja $E\subset\R^n$ um subconjunto mensurável e sejam $A_1$, $A_2$ tais que $E=A_1\cup A_2$ e $A_1\cap A_2=\emptyset$. Então:
\[\leb(E)=\leb^*(A_1)+\leb_*(A_2).\]
\end{cor}
\begin{proof}
O Lema~\ref{thm:preintext} nos dá:
\[\leb(E)\le\leb^*(A_1)+\leb_*(A_2)\le\leb(E).\qedhere\]
\end{proof}

\end{subsection}

\end{section}

\begin{section}{Conjuntos de Cantor}
\label{sec:Cantor}

Seja $I=[a,b]$, $a<b$, um intervalo fechado e limitado de comprimento positivo.
Dado um escalar $\alpha>0$, $\alpha<b-a=\vert I\vert$, consideramos o intervalo aberto $J$ de comprimento $\alpha$
que possui o mesmo centro que $I$; denotamos então por $\lambda(I,\alpha;0)$ e $\lambda(I,\alpha;1)$ os dois
intervalos remanescentes após remover $J$ de $I$. Mais precisamente, sejam $c=\frac12(a+b-\alpha)$ e $d=\frac12(a+b+\alpha)$,
de modo que $J=\left]c,d\right[$; definimos:
\begin{equation}\label{eq:deflambdaI}
\lambda(I,\alpha;0)=[a,c],\quad\lambda(I,\alpha;1)=[d,b].
\end{equation}
Note que $a<c<d<b$, de modo que $\lambda(I,\alpha;0)$ e $\lambda(I,\alpha;1)$ são dois intervalos
fechados e limitados disjuntos de comprimento positivo contidos em $I$; mais especificamente:
\[\big\vert\lambda(I,\alpha;0)\big\vert=\big\vert\lambda(I,\alpha;1)\big\vert=\frac12\,(\vert I\vert-\alpha).\]
Dados um intervalo fechado e limitado $I$ de comprimento positivo, um inteiro
$n\ge1$, escalares positivos $\alpha_1$, \dots, $\alpha_n$ com $\sum_{i=1}^n\alpha_i<\vert I\vert$
e $\epsilon_1,\ldots,\epsilon_n\in\{0,1\}$, vamos definir um intervalo limitado e fechado
$\lambda\big(I,(\alpha_i)_{i=1}^n;(\epsilon_i)_{i=1}^n\big)$\index[simbolos]{$\lambda\big(I,(\alpha_i)_{i=1}^n;(\epsilon_i)_{i=1}^n\big)$} tal que:
\begin{equation}\label{eq:lengthlambda}
\Big\vert\lambda\big(I,(\alpha_i)_{i=1}^n;(\epsilon_i)_{i=1}^n\big)\Big\vert=
\frac1{2^n}\Big(\vert I\vert-\sum_{i=1}^n\alpha_i\Big)>0.
\end{equation}
A definição será feita recursivamente. Para $n=1$, a definição já foi dada em \eqref{eq:deflambdaI}.
Dados um intervalo fechado e limitado $I$ de comprimento positivo, escalares positivos
$\alpha_1$, \dots, $\alpha_{n+1}$ com $\sum_{i=1}^{n+1}\alpha_i<\vert I\vert$
e $\epsilon_1,\ldots,\epsilon_{n+1}\in\{0,1\}$, definimos:
\[\lambda\big(I,(\alpha_i)_{i=1}^{n+1};(\epsilon_i)_{i=1}^{n+1}\big)=\lambda\Big(\lambda\big(I,
(\alpha_i)_{i=1}^n;(\epsilon_i)_{i=1}^n\big),\frac{\alpha_{n+1}}{2^n};\epsilon_{n+1}\Big).\]
Assumindo \eqref{eq:lengthlambda}, é fácil ver que $\lambda\big(I,(\alpha_i)_{i=1}^{n+1};(\epsilon_i)_{i=1}^{n+1}\big)$
está bem definido e que:
\[\Big\vert\lambda\big(I,(\alpha_i)_{i=1}^{n+1};(\epsilon_i)_{i=1}^{n+1}\big)\Big\vert=
\frac1{2^{n+1}}\Big(\vert I\vert-\sum_{i=1}^{n+1}\alpha_i\Big)>0.\]
Segue então por indução que temos uma família de intervalos fechados e limitados
$\lambda\big(I,(\alpha_i)_{i=1}^n;(\epsilon_i)_{i=1}^n\big)$ satisfazendo \eqref{eq:lengthlambda}.

Fixemos então um intervalo fechado e limitado $I$ de comprimento positivo e uma seqüência
$(\alpha_i)_{i\ge1}$ de escalares positivos tal que $\sum_{i=1}^\infty\alpha_i\le\vert I\vert$.
Note que $\sum_{i=1}^n\alpha_i<\vert I\vert$, para todo $n\ge1$. Para simplificar a notação,
escrevemos:
\index[simbolos]{$I(\epsilon)$}\[I(\epsilon)=I(\epsilon_1,\ldots,\epsilon_n)=\lambda\big(I,(\alpha_i)_{i=1}^n;(\epsilon_i)_{i=1}^n\big),\]
para todo $n\ge1$ e todo $\epsilon=(\epsilon_1,\ldots,\epsilon_n)\in\{0,1\}^n$. Dada
uma seqüência $(\epsilon_i)_{i\ge1}$ em $\{0,1\}$ obtemos uma seqüência decrescente
de intervalos fechados e limitados:
\begin{equation}\label{eq:Iepsilondecr}
I\supset I(\epsilon_1)\supset I(\epsilon_1,\epsilon_2)\supset\cdots\supset I(\epsilon_1,\ldots,\epsilon_n)\supset\cdots
\end{equation}
Afirmamos que, para todo $n\ge1$, os intervalos $I(\epsilon)$, $\epsilon\in\{0,1\}^n$, são dois a dois disjuntos.
De fato, sejam dados $\epsilon,\epsilon'\in\{0,1\}^n$, com $\epsilon\ne\epsilon'$. Seja
$k\in\{1,\ldots,n\}$ o menor índice tal que $\epsilon_k\ne\epsilon'_k$. Temos
$I(\epsilon)\subset I(\epsilon_1,\ldots,\epsilon_k)$, $I(\epsilon')\subset I(\epsilon'_1,\ldots,\epsilon'_k)$,
$J=I(\epsilon_1,\ldots,\epsilon_{k-1})=I(\epsilon'_1,\ldots,\epsilon'_{k-1})$ e:
\[I(\epsilon_1,\ldots,\epsilon_k)=\lambda\Big(J,\frac{\alpha_k}{2^{k-1}};\epsilon_k\Big),\quad
I(\epsilon'_1,\ldots,\epsilon'_k)=\lambda\Big(J,\frac{\alpha_k}{2^{k-1}};\epsilon'_k\Big).\]
Como $\epsilon_k\ne\epsilon'_k$, os intervalos $\lambda\big(J,\frac{\alpha_k}{2^{k-1}};\epsilon_k\big)$
e $\lambda\big(J,\frac{\alpha_k}{2^{k-1}};\epsilon'_k\big)$ são disjuntos e portanto também
$I(\epsilon)\cap I(\epsilon')=\emptyset$. Para cada $n\ge1$ definimos:
\[K_n=\!\!\!\bigcup_{\epsilon\in\{0,1\}^n}\!\!I(\epsilon).\]
Note que cada $K_n$ é uma união disjunta de $2^n$ intervalos fechados e limitados
de comprimento $\frac1{2^n}\big(\vert I\vert-\sum_{i=1}^n\alpha_i\big)$. Em particular,
cada $K_n$ é compacto e sua medida de Lebesgue é dada por:
\begin{equation}\label{eq:medCantorn}
\leb(K_n)=\vert I\vert-\sum_{i=1}^n\alpha_i.
\end{equation}
\begin{defin}
O conjunto $K=\bigcap_{n=1}^\infty K_n$ é chamado o {\em conjunto de Cantor\/}\index[indice]{Cantor!conjunto de}\index[indice]{conjunto!de Cantor}
determinado pelo intervalo fechado e limitado $I$ e pela seqüência $(\alpha_i)_{i\ge1}$
de escalares positivos com $\sum_{i=1}^\infty\alpha_i\le\vert I\vert$.
\end{defin}

Para cada seqüência $(\epsilon_i)_{i\ge1}$ em $\{0,1\}$ temos que \eqref{eq:Iepsilondecr}
é uma seqüência decrescente de intervalos fechados e limitados cujos comprimentos tendem
a zero; de fato:
\begin{equation}\label{eq:lengthKnzero}
\big\vert I(\epsilon_1,\ldots,\epsilon_n)\big\vert=\frac1{2^n}\Big(\vert I\vert-\sum_{i=1}^n\alpha_i\Big)
\le\frac1{2^n}\,\vert I\vert\xrightarrow[\;n\to\infty\;]{}0.
\end{equation}
Pelo princípio dos intervalos encaixantes, existe exatamente um ponto pertencente à interseção
de todos os intervalos em \eqref{eq:Iepsilondecr}. Definimos então uma aplicação:
\[\phi:\{0,1\}^\infty=\prod_{i=1}^\infty\{0,1\}\ni\epsilon=(\epsilon_i)_{i\ge1}\longmapsto
\phi(\epsilon)\in K,\]
de modo que:
\begin{equation}\label{eq:defphiCantor}
\bigcap_{n=1}^\infty I(\epsilon_1,\ldots,\epsilon_n)=\big\{\phi(\epsilon)\big\},
\end{equation}
para todo $\epsilon=(\epsilon_i)_{i\ge1}\in\{0,1\}^\infty$.

As principais propriedades do conjunto $K$ podem ser sumarizadas no seguinte:
\begin{teo}
Seja $I$ um intervalo fechado e limitado de comprimento positivo e seja $(\alpha_i)_{i\ge1}$
uma seqüência de escalares positivos tal que:
\[\sum_{i=1}^\infty\alpha_i\le\vert I\vert.\]
Seja $K$ o conjunto de Cantor determinado por $I$ e por $(\alpha_i)_{i\ge1}$. Então:
\begin{itemize}
\item[(a)] $K$ é um subconjunto compacto de $I$;
\item[(b)] a medida de Lebesgue de $K$ é $\leb(K)=\vert I\vert-\sum_{i=1}^\infty\alpha_i$;
\item[(c)] $K$ tem interior vazio;
\item[(d)] $K$ tem a mesma cardinalidade que a reta $\R$ (e é portanto não enumerável);
\item[(e)] $K$ não tem pontos isolados.
\end{itemize}
\end{teo}
\begin{proof}\
\begin{bulletindent}
\item {\em Prova de\/} (a).

Basta observar que $K$ é uma interseção de subconjuntos compactos de $I$.

\item {\em Prova de\/} (b).

Segue de \eqref{eq:medCantorn} e do Lema~\ref{thm:setlimits}, observando que
$K_n\searrow K$.

\item {\em Prova de\/} (c).

Um intervalo contido em $K_n$ deve estar contido em algum dos intervalos $I(\epsilon)$, $\epsilon\in\{0,1\}^n$,
e portanto deve ter comprimento menor ou igual a $\frac1{2^n}\big(\vert I\vert-\sum_{i=1}^n\alpha_i\big)$.
Segue de \eqref{eq:lengthKnzero} que nenhum intervalo de comprimento positivo pode estar contido
em $K_n$ para todo $n\ge1$. Logo $K=\bigcap_{n=1}^\infty K_n$ não pode conter um intervalo
aberto não vazio.

\item {\em Prova de\/} (d).

É fácil ver que a função $\phi$ definida em \eqref{eq:defphiCantor} é bijetora.
A conclusão segue do fato bem conhecido que $\{0,1\}^\infty$ tem a mesma cardinalidade de $\R$.

\item {\em Prova de\/} (e).

Seja $x\in K$. Como $\phi$ é bijetora, existe $\epsilon\in\{0,1\}^\infty$ tal que
$x=\phi(\epsilon)$. Escolhendo $\epsilon'\in\{0,1\}^\infty$ com $\epsilon'\ne\epsilon$
e $(\epsilon'_1,\ldots,\epsilon'_n)=(\epsilon_1,\ldots,\epsilon_n)$ então
$\phi(\epsilon')$ é um ponto de $K$ distinto de $x$. Além do mais,
$\phi(\epsilon')$ e $x$ ambos pertencem ao intervalo $I(\epsilon_1,\ldots,\epsilon_n)$
e portanto:
\[\big\vert x-\phi(\epsilon')\big\vert\le\big\vert I(\epsilon_1,\ldots,\epsilon_n)\big\vert=
\frac1{2^n}\Big(\vert I\vert-\sum_{i=1}^n\alpha_i\Big)\le\frac1{2^n}\,\vert I\vert.\]
Concluímos que toda vizinhança de $x$ contém um ponto de $K$ distinto de $x$, i.e., $x$ é um
ponto de acumulação de $K$.\qedhere
\end{bulletindent}
\end{proof}

\begin{example}
Escolhendo os escalares $\alpha_i$ com $\sum_{i=1}^\infty\alpha_i=\vert I\vert$ então
o conjunto de Cantor $K$ correspondente nos fornece um exemplo de um subconjunto
não enumerável de $\R$ (com a mesma cardinalidade de $\R$) e com medida de Lebesgue zero.
\end{example}

\begin{example}\label{exa:raroquaseI}
Escolhendo os escalares $\alpha_i$ com $\sum_{i=1}^\infty\alpha_i<\vert I\vert$ então o conjunto
de Cantor $K$ correspondente nos fornece um exemplo de um subconjunto compacto de $\R$
com interior vazio e medida de Lebesgue positiva. Na verdade, para todo $\varepsilon>0$
podemos escolher os escalares $\alpha_i$ com $\sum_{i=1}^\infty\alpha_i<\varepsilon$
e daí o conjunto de Cantor $K$ correspondente nos fornece um exemplo de um subconjunto
compacto do intervalo $I$ com interior vazio e $\leb(K)>\vert I\vert-\varepsilon$.
\end{example}

\end{section}

\begin{section}{Conjuntos não Mensuráveis}

Uma forma de construir um exemplo de um subconjunto não mensurável de $\R^n$ é repetir os passos da demonstração
da Proposição~\ref{thm:naoexistemu}.
\begin{example}
Considere a relação binária $\sim$ no bloco $[0,1]^n$ definida por:
\[x\sim y\Longleftrightarrow x-y\in\Q^n,\]
para todos $x,y\in[0,1]^n$. É fácil ver que $\sim$ é uma relação de equivalência em $[0,1]^n$.
Seja $A$ um conjunto escolha para $\sim$. Como na demonstração da Proposição~\ref{thm:naoexistemu},
vemos que os conjuntos $(A+q)_{q\in\Q^n}$ são dois a dois disjuntos e que:
\[[0,1]^n\subset\!\!\!\!\!\!\!\!\!\bigcup_{q\in\Q^n\cap[-1,1]^n}\!\!\!\!\!\!(A+q)\subset[-1,2]^n.\]
Usando o Lema~\ref{thm:extmeastransinv} e o resultado do Exercício~\ref{exe:translmens}, vemos que
a mensurabilidade de $A$ implicaria em:
\[0<1=\leb\big([0,1]^n\big)\le\!\!\!\!\!\!\!\!\!\sum_{q\in\Q^n\cap[-1,1]^n}\!\!\!\!\!\!\leb(A)\le\leb\big([-1,2]^n\big)=3^n<+\infty,\]
já que $\Q^n\cap[-1,1]^n$ é enumerável. Obtemos então uma contradição, o que mostra que $A$ é um subconjunto
não mensurável do bloco $[0,1]^n$.
\end{example}

No que segue, investigaremos mais a fundo o fenômeno da não mensurabilidade, produzindo alguns exemplos mais
radicais de conjuntos não mensuráveis. Começamos com alguns lemas.

\begin{lem}\label{thm:UquaseUmaisx}
Seja $U\subset\R^n$ um aberto. Então, dado $\varepsilon>0$, existe $\delta>0$ tal que para
todo $x\in\R^n$ com $\Vert x\Vert<\delta$, temos:
\begin{equation}\label{eq:lebUUplusx}
\leb\big(U\cup(U+x)\big)\le\leb(U)+\varepsilon.
\end{equation}
\end{lem}
\begin{proof}
A desigualdade \eqref{eq:lebUUplusx} é trivial para $\leb(U)=+\infty$, de modo que podemos supor
que $\leb(U)<+\infty$. Para cada $k\ge1$, consideramos o conjunto $U_k$ definido por:
\[U_k=\big\{x\in\R^n:d(x,U^\compl)>\tfrac1k\big\}.\]
Como $U$ é aberto, temos que $d(x,U^\compl)>0$ se e somente se $x\in U$; isso implica
que $U=\bigcup_{k=1}^\infty U_k$ e portanto $U_k\nearrow U$. A continuidade da função $x\mapsto d(x,U^\compl)$
implica que cada $U_k$ é aberto e portanto mensurável. Pelo Lema~\ref{thm:setlimits},
temos $\leb(U)=\lim_{k\to\infty}\leb(U_k)$ e portanto existe $k\ge1$ tal que:
\[\leb(U_k)\ge\leb(U)-\varepsilon.\]
Tome $\delta=\frac1k$ e seja $x\in\R^n$ com $\Vert x\Vert<\delta$. Para todo $y\in U_k$, temos
$d(y,y-x)=\Vert x\Vert<\frac1k$ e portanto $y-x\in U$, i.e., $y\in U+x$. Segue então que $U_k\subset U\cap(U+x)$
e portanto:
\[\leb\big(U\cap(U+x)\big)\ge\leb(U)-\varepsilon.\]
A conclusão é obtida agora do cálculo abaixo:
\begin{multline*}
\leb\big(U\cup(U+x)\big)=\leb(U)+\leb(U+x)-\leb\big(U\cap(U+x)\big)\\
=2\leb(U)-\leb\big(U\cap(U+x)\big)\le\leb(U)+\varepsilon,
\end{multline*}
onde usamos o Lema~\ref{thm:extmeastransinv} e o resultado do Exercício~\ref{exe:muAcupB}.
\end{proof}

\begin{defin}
Se $A$ é um subconjunto de $\R^n$, então o {\em conjunto das diferenças\/}\index[indice]{conjunto!das diferencas@das diferenças}\index[indice]{diferencas@diferenças!conjunto das}
de $A$ é definido por:
\index[simbolos]{$A^-$}\[A^-=\big\{x-y:x,y\in A\big\}.\]
\end{defin}

\begin{lem}\label{thm:Emenosvizorig}
Se $A\subset\R^n$ é um conjunto mensurável com medida de Lebesgue positiva então
$A^-$ contém uma vizinhança da origem.
\end{lem}
\begin{proof}
Se $\leb(A)=+\infty$ então $A$ contém um conjunto mensurável $A_0$ tal que
$0<\leb(A_0)<+\infty$ (isso segue, por exemplo, do Corolário~\ref{thm:corsupcompactosdentro}).
Como $A_0^-\subset A^-$, podemos considerar apenas o caso em que $\leb(A)<+\infty$.
Pelo Lema~\ref{thm:aproxaberto}, existe um aberto $U\subset\R^n$ contendo $A$
tal que $\leb(U)<2\leb(A)$. Seja $\varepsilon>0$ tal que $\leb(U)+\varepsilon<2\leb(A)$.
Pelo Lema~\ref{thm:UquaseUmaisx}, existe $\delta>0$ tal que $\leb\big(U\cup(U+x)\big)\le\leb(U)+\varepsilon$,
para todo $x\in\R^n$ com $\Vert x\Vert<\delta$. Afirmamos que $A^-$ contém a bola aberta
de centro na origem e raio $\delta$. Senão, existiria $x\in\R^n$ com $\Vert x\Vert<\delta$
e $x\not\in A^-$; daí $A$ e $A+x$ seriam conjuntos mensuráveis disjuntos (veja Exercício~\ref{exe:translmens})
e portanto, usando o Lema~\ref{thm:extmeastransinv}, concluiríamos que:
\begin{multline*}
2\leb(A)=\leb(A)+\leb(A+x)=\leb\big(A\cup(A+x)\big)\le\leb(U\cup(U+x)\big)\\
\le\leb(U)+\varepsilon<2\leb(A),
\end{multline*}
e obteríamos portanto uma contradição.
\end{proof}

\begin{cor}\label{thm:cormedintzero}
Seja $A$ um subconjunto de $\R^n$. Se $A^-$ não contém uma vizinhança da origem então
$\leb_*(A)=0$.
\end{cor}
\begin{proof}
Dado um compacto $K\subset A$ então $K$ é mensurável e $K^-$ não contém uma vizinhança da origem.
Segue então do Lema~\ref{thm:Emenosvizorig} que $\leb(K)=0$.
\end{proof}

Para construir exemplos de conjuntos não mensuráveis, vamos aplicar algumas técnicas da teoria de colorimento de grafos.

\begin{defin}
Um {\em grafo\/}\index[indice]{grafo} é um par ordenado $G=(V,\mathcal E)$, onde $V$ é um conjunto
arbitrário e $\mathcal E$ é uma relação binária anti-reflexiva\index[indice]{relacao@relação!anti-reflexiva}
e simétrica\index[indice]{relacao@relação!simetrica@simétrica} em $V$;
mais precisamente, $\mathcal E$ é um subconjunto de $V\times V$ tal que:
\begin{itemize}
\item $(x,x)\not\in\mathcal E$, para todo $x\in V$;
\item $(x,y)\in\mathcal E$ implica $(y,x)\in\mathcal E$, para todos $x,y\in V$.
\end{itemize}
Os elementos de $V$ são chamados os {\em vértices\/}\index[indice]{vertices@vértices!de um grafo}
do grafo $G$. Dados vértices $x,y\in V$
com $(x,y)\in\mathcal E$ então dizemos que $x$ e $y$ são {\em vértices adjacentes\/}\index[indice]{vertices@vértices!adjacentes}
no grafo $G$.
\end{defin}
Se $V'$ é um subconjunto de $V$ então $\mathcal E'=\mathcal E\cap(V'\times V')$
é um relação binária anti-reflexiva e simétrica em $V'$, de modo que $G'=(V',\mathcal E')$ é um grafo.
Dizemos que $G'=(V',\mathcal E')$ é o {\em subgrafo cheio\/}\index[indice]{subgrafo!cheio} de $G$ determinado pelo
conjunto de vértices $V'$.

\begin{defin}
Seja $G=(V,\mathcal E)$ um grafo. Um {\em colorimento\/}\index[indice]{colorimento}\index[indice]{grafo!colorimento de}
para $G$ é uma função $f$ definida em
$V$ tal que $f(x)\ne f(y)$, para todo $(x,y)\in\mathcal E$. Para cada $x\in V$, dizemos que
$f(x)$ é a {\em cor\/} do vértice $x$. Se $k$ é um inteiro positivo então um {\em $k$-colorimento\/}\index[indice]{k-colorimento@$k$-colorimento}
de $G$ é um colorimento $f:V\to\{0,1,\ldots,k-1\}$ de $G$. Quando $G$ admite um $k$-colorimento
dizemos que $G$ é {\em $k$-colorível}\index[indice]{grafo!k-colorivel@$k$-colorível}\index[indice]{k-colorivel@$k$-colorível}.
\end{defin}

\begin{defin}
Seja $G=(V,\mathcal E)$ um grafo. Um {\em caminho\/}\index[indice]{caminho} em $G$ é uma seqüência finita $(x_i)_{i=0}^p$, $p\ge0$,
de vértices de $G$ tal que $(x_i,x_{i+1})\in\mathcal E$ para todo $i=0,\ldots,p-1$; dizemos também
que $(x_i)_{i=0}^p$ é um caminho {\em começando\/} em $x_0$ e {\em terminando\/} em $x_p$.
O caminho $(x_i)_{i=0}^p$ é dito {\em de comprimento\/}\index[indice]{comprimento!de um caminho} $p$.
Por convenção, uma seqüência unitária formada por um único vértice $x_0\in V$ é um caminho de
comprimento zero começando em $x_0$ e terminando em $x_0$. Quando existe um caminho em $G$ começando
em $x$ e terminando em $y$ para todos $x,y\in V$, dizemos que $G$ é um {\em grafo conexo}\index[indice]{grafo!conexo}.
Um {\em circuito\/}\index[indice]{circuito} em $G$ é um caminho $(x_i)_{i=0}^p$ em $G$ tal que $x_0=x_p$.
\end{defin}
É fácil ver que a relação binária $\sim$ em $V$ definida por:
\[x\sim y\Longleftrightarrow\text{existe um caminho em $G$ começando em $x$ e terminando em $y$},\]
é uma relação de equivalência em $V$. Seja $V_0\subset V$ uma classe de equivalência determinada
por $\sim$. Verifica-se facilmente que o subgrafo cheio $G_0$ de $G$ determinado por $V_0$ é conexo;
dizemos que $G_0$ é uma {\em componente conexa\/}\index[indice]{componente conexa!de um grafo} do
grafo $G$.

\begin{lem}\label{thm:circuitoimpar}
Um grafo $G=(V,\mathcal E)$ é $2$-colorível se e somente se não possui circuitos de comprimento
ímpar.
\end{lem}
\begin{proof}
Assuma que o grafo $G$ é $2$-colorível, i.e., existe um $2$-colorimento $f:V\to\{0,1\}$ de $G$.
Seja $(x_i)_{i=0}^p$ um circuito de $G$.
Mostremos que $p$ é par. Para fixar as idéias, assuma que $f(x_0)=0$. Como os vértices $x_0$ e $x_1$
são adjacentes, temos $f(x_1)\ne f(x_0)$ e portanto $f(x_1)=1$. Similarmente, vemos que $f(x_2)=0$ e, mais geralmente,
$f(x_i)=0$ para $i$ par e $f(x_i)=1$ para $i$ ímpar. Como $f(x_p)=f(x_0)=0$, concluímos que
$p$ deve ser par. Reciprocamente, assuma agora que o grafo $G$ não possui circuito de comprimento
ímpar e mostremos que $G$ é $2$-colorível. É fácil ver que:
\begin{itemize}
\item nenhuma componente conexa de $G$ possui um circuito de comprimento ímpar;
\item se cada componente conexa de $G$ é $2$-colorível então $G$ é $2$-colorível.
\end{itemize}
Podemos então supor que $G$ é conexo. Dados vértices $x,y\in V$ de $G$ então os comprimentos
de dois caminhos em $G$ começando em $x$ e terminando em $y$ têm a mesma paridade. De fato,
se $(x_i)_{i=0}^p$ e $(x'_i)_{i=0}^q$ são caminhos em $G$ começando em $x$ e terminando em $y$
então:
\[x=x_0,\ x_1,\ \ldots,\ x_p=y=x'_q,\ x'_{q-1},\ \ldots,\ x'_0=x,\]
é um circuito em $G$ de comprimento $p+q$. Logo $p+q$ é par e portanto $p$ e $q$ possuem
a mesma paridade. Fixamos agora um vértice $x_0\in V$ e definimos $f:V\to\{0,1\}$ fazendo
$f(x)=0$ se todo caminho começando em $x_0$ e terminando em $x$ tem comprimento par e $f(x)=1$
se todo caminho começando em $x_0$ e terminando em $x$ tem comprimento ímpar. É fácil ver que
$f$ é um $2$-colorimento para $G$.
\end{proof}

\begin{defin}
Seja $S$ um subconjunto de $\R^n$ que não contém a origem. O {\em grafo de Cayley\/}\index[indice]{grafo!de Cayley}\index[indice]{Cayley!grafo de}
associado ao par $(\R^n,S)$, denotado por $G(\R^n,S)$\index[simbolos]{$G(\R^n,S)$}, é o grafo $(V,\mathcal E)$
tal que $V=\R^n$ e:
\[\mathcal E=\big\{(x,y)\in\R^n\times\R^n:\text{$x-y\in S$ ou $y-x\in S$}\big\}.\]
\end{defin}

\begin{lem}\label{thm:conjuntoadmissivel}
Seja $S$ um subconjunto de $\R^n$ que não contém a origem. O grafo de Cayley $G(\R^n,S)$ é\/ $2$-colorível
se e somente se $S$ possui a seguinte propriedade:
\begin{itemize}
\item[($*$)] dados $s_1,\ldots,s_k\in S$ e $n_1,\ldots,n_k\in\Z$ com $\sum_{i=1}^kn_is_i=0$ então
$\sum_{i=1}^kn_i$ é par.\index[indice]{propriedade ($*$)}
\end{itemize}
\end{lem}
\begin{proof}
Em vista do Lema~\ref{thm:circuitoimpar}, basta mostrar que $G(\R^n,S)$ não possui circuito
de comprimento ímpar se e somente se $S$ possui a propriedade ($*$). Assuma que $S$ possui a propriedade
($*$) e que $(x_i)_{i=0}^p$ é um circuito de $G(\R^n,S)$. Mostremos que $p$ é par.
Para cada $i=0,\ldots,p-1$ temos que $x_{i+1}-x_i\in S$ ou $x_i-x_{i+1}\in S$; podemos então
escrever $x_{i+1}-x_i=n_is_i$, com $n_i\in\{\pm1\}$ e $s_i\in S$. Daí:
\[\sum_{i=0}^{p-1}n_is_i=\sum_{i=0}^{p-1}(x_{i+1}-x_i)=x_p-x_0=0\]
e logo $\sum_{i=0}^{p-1}n_i$ é par. Mas $\sum_{i=0}^{p-1}\vert n_i\vert$ tem a mesma paridade
que $\sum_{i=0}^{p-1}n_i$ e portanto $\sum_{i=0}^{p-1}\vert n_i\vert=p$ é par. Reciprocamente, suponha que
$G(\R^n,S)$ não possui circuito de comprimento ímpar e mostremos que $S$ possui a propriedade
($*$). Sejam $s_1,\ldots,s_k\in S$ e $n_1,\ldots,n_k\in\Z$ com $\sum_{i=1}^kn_is_i=0$. Escreva
$s'_i=s_i$ se $n_i\ge0$ e $s'_i=-s_i$ se $n_i<0$, de modo que $n_is_i=\vert n_i\vert s'_i$ e
$s'_i\in S$ ou $-s'_i\in S$, para todo $i=1,\ldots,k$. Temos que $\sum_{i=1}^k\vert n_i\vert s'_i=0$, ou seja:
\begin{equation}\label{eq:somacircuito}
\underbrace{s'_1+s'_1+\cdots+s'_1}_{\text{$\vert n_1\vert$ termos}}+
\underbrace{s'_2+s'_2+\cdots+s'_2}_{\text{$\vert n_2\vert$ termos}}+\cdots+
\underbrace{s'_k+s'_k+\cdots+s'_k}_{\text{$\vert n_k\vert$ termos}}=0.
\end{equation}
Sejam $p=\sum_{i=1}^k\vert n_i\vert$, $x_0=0$ e, para $j=1,2,\ldots,p$, seja $x_j$ a soma dos primeiros $j$ termos da
soma que aparece do lado esquerdo da identidade \eqref{eq:somacircuito}. Temos que $(x_j)_{j=0}^p$ é um circuito
em $G(\R^n,S)$ de comprimento $p$ e portanto $p$ é par. Finalmente,
como $\sum_{i=1}^k\vert n_i\vert$ e $\sum_{i=1}^kn_i$ têm a mesma paridade, segue que
$\sum_{i=1}^kn_i$ é par.
\end{proof}

\begin{lem}\label{thm:particaomedintzero}
Seja $S\subset\R^n\setminus\{0\}$ e suponha que exista um $2$-colorimento $f:\R^n\to\{0,1\}$
do grafo de Cayley $G(\R^n,S)$. Se a origem é um ponto de acumulação de $S$ então os conjuntos
$A=f^{-1}(0)$ e $B=f^{-1}(1)$ possuem medida interior nula.
\end{lem}
\begin{proof}
Dados $x,y\in A$ então $f(x)=f(y)=0$ e portanto os vértices $x$ e $y$ não podem ser adjacentes
no grafo $G(\R^n,S)$. Em particular, $x-y\not\in S$, o que mostra que o conjunto das diferenças
$A^-$ é disjunto de $S$. Como a origem é um ponto de acumulação de $S$, segue que $A^-$ não pode
conter uma vizinhança da origem e portanto, pelo Corolário~\ref{thm:cormedintzero},
$A$ tem medida interior nula. Analogamente, vemos que $B^-\cap S=\emptyset$ e portanto
$\leb_*(B)=0$.
\end{proof}

\begin{example}\label{exa:partbizarraRn}
Em vista dos Lemas~\ref{thm:conjuntoadmissivel} e \ref{thm:particaomedintzero}, se exibirmos
um subconjunto $S\subset\R^n\setminus\{0\}$ com a propriedade ($*$) e que possui a origem como ponto de acumulação
então obteremos uma partição $\R^n=A\cup B$ de $\R^n$ tal que $\leb_*(A)=\leb_*(B)=0$. Por exemplo, é fácil
mostrar que o conjunto:
\[S=\big\{\tfrac1m:\text{$m$ inteiro ímpar}\big\}\subset\R\setminus\{0\}\]
tem a propriedade ($*$) e obviamente a origem é ponto de acumulução de $S$. Em $\R^n$, podemos considerar
o conjunto $S^n$ (ou até mesmo $S\times\{0\}^{n-1}$), que também tem a propriedade ($*$) e a origem como ponto
de acumulação.
\end{example}

\begin{example}
Sejam $A,B\subset\R^n$ conjuntos disjuntos de medida interior nula tais que $\R^n=A\cup B$ (vide Exemplo~\ref{exa:partbizarraRn}).
Definindo:
\[A'=A\cap[0,1]^n,\quad B'=B\cap[0,1]^n,\]
obtemos uma partição $[0,1]^n=A'\cup B'$ do bloco $[0,1]^n$ em conjuntos
$A'$, $B'$ de medida interior nula. Usando o Corolário~\ref{thm:intext} vemos que:
\[1=\leb\big([0,1]^n\big)=\leb^*(A')+\leb_*(B')=\leb^*(A')\]
e portanto $\leb^*(A')=1$. Similarmente, vemos que $\leb^*(B')=1$. Obtivemos então subconjuntos do bloco $[0,1]^n$
com medida interior nula e medida exterior igual a $1$. Obtivemos também uma partição do bloco $[0,1]^n$ em dois
conjuntos de medida exterior igual a $1$; note que:
\[1=\leb\big([0,1]^n\big)<\leb^*(A')+\leb^*(B')=2,\]
com $[0,1]^n=A'\cup B'$ e $A'$, $B'$ disjuntos!
\end{example}

\end{section}

\section*{Exercícios para o Capítulo~\ref{CHP:LEBESGUE}}

\subsection*{Aritmética na Reta Estendida}

\begin{exercise}\label{exe:supinf}
Mostre que todo subconjunto da reta estendida possui supremo e ínfimo.
\end{exercise}

\begin{exercise}\label{exe:seqmonot}
Prove o Lema~\ref{thm:seqmonot}.
\end{exercise}

\begin{exercise}
Dadas famílias $(a_i)_{i\in I}$ e $(b_j)_{j\in J}$ em $\overline\R$ tais que a soma
$a_i+b_j$ é bem definida para todos $i\in I$, $j\in J$, mostre que:
\[\sup\big\{a_i+b_j:i\in I,\ j\in J\big\}=\sup_{i\in I}a_i+\sup_{j\in J}b_j,\]
desde que a soma $\sup_{i\in I}a_i+\sup_{j\in J}b_j$ esteja bem
definida. Mostre também que:
\[\inf\big\{a_i+b_j:i\in I,\ j\in J\big\}=\inf_{i\in I}a_i+\inf_{j\in J}b_j,\]
desde que a soma $\inf_{i\in I}a_i+\inf_{j\in J}b_j$ esteja bem definida.
\end{exercise}

\begin{exercise}\label{exe:oplimbarR}
Prove o Lema~\ref{thm:oplimbarR}.
\end{exercise}

\begin{exercise}\label{exe:prodcresseq}
Sejam $(a_k)_{k\ge1}$ e $(b_k)_{k\ge1}$ seqüências crescentes no intervalo $[0,+\infty]$.
Mostre que:
\[\lim_{k\to\infty}a_kb_k=\big(\lim_{k\to\infty}a_k\big)\big(\lim_{k\to\infty}b_k\big).\]
\end{exercise}

\begin{exercise}\label{exe:liminflimsup}
Prove a Proposição~\ref{thm:limlimsupliminf}.
\end{exercise}

\begin{exercise}\label{exe:propsomasinf}
	Prove a Proposição~\ref{thm:propsomasinf}.
\end{exercise}

\begin{*exercise}\label{exe:topologiaRbar}\
\begin{itemize}
\item Mostre que os conjuntos:
\begin{gather*}
\left]a,b\right[,\quad a,b\in\overline\R,\ a<b,\\
\left[-\infty,a\right[,\quad a\in\overline\R,\ a>-\infty,\\
\left]a,+\infty\right],\quad a\in\overline\R,\ a<+\infty,
\end{gather*}
constituem uma base de abertos para uma topologia em $\overline\R$.
\item Mostre que a aplicação $f:[-1,1]\to\overline\R$ definida por:
\[f(x)=\begin{cases}
\hfil-\infty,&\text{se $x=-1$},\\[3pt]
\dfrac x{1-x^2},&\text{se $x\in\left]-1,1\right[$},\\[5pt]
\hfil+\infty,&\text{se $x=1$},
\end{cases}\]
é um homeomorfismo.
\item Mostre que uma seqüência $(a_k)_{k\ge1}$ em $\overline\R$ converge para um elemento
$a\in\overline\R$ com respeito à topologia introduzida acima se e somente se $(a_k)_{k\ge1}$
converge para $a$ de acordo com a Definição~\ref{thm:defconvseq}.
\item Mostre que a função $D_+\ni(a,b)\mapsto a+b\in\overline\R$ é contínua, onde:
\[D_+=(\,\overline\R\times\overline\R\,)\setminus\big\{(-\infty,+\infty),(+\infty,-\infty)\big\}\]
é munido da topologia induzida pela topologia produto de $\overline\R\times\overline\R$.
\item Mostre que a função $\overline\R\times\overline\R\ni(a,b)\mapsto ab\in\overline\R$
é contínua, {\em exceto\/} nos pontos $(+\infty,0)$, $(-\infty,0)$, $(0,+\infty)$ e $(0,-\infty)$.
\end{itemize}
\end{*exercise}

\subsection*{Medida de Lebesgue em ${\R^n}$}

\begin{exercise}\label{exe:lebouterreg}
Dado $A\subset\R^n$, mostre que:
\[\leb^*(A)=\inf\big\{\leb(U):\text{$U$ aberto em $\R^n$ e $A\subset U$}\big\}.\]
\end{exercise}

\begin{exercise}\label{exe:translmens}
Se $A\subset\R^n$ é um conjunto mensurável, mostre que $A+x$ também é mensurável
para todo $x\in\R^n$.
\end{exercise}

\begin{exercise}\label{exe:permutacao}
Seja $\sigma$ uma {\em permutação de $n$ elementos}\index[indice]{permutacao@permutação},
ou seja, uma bijeção do conjunto $\{1,\ldots,n\}$ sobre si próprio.
Considere o isomorfismo linear $\widehat\sigma:\R^n\to\R^n$ definido por:\index[simbolos]{$\widehat\sigma$}
\[\widehat\sigma(x_1,\ldots,x_n)=(x_{\sigma(1)},\ldots,x_{\sigma(n)}),\]
para todo $(x_1,\ldots,x_n)\in\R^n$. Mostre que:
\begin{itemize}
\item[(a)] se $B$ é um bloco retangular $n$-dimensional então $\widehat\sigma(B)$ é também um bloco retangular $n$-dimensional
e $\vert\widehat\sigma(B)\vert=\vert B\vert$;
\item[(b)] para todo $A\subset\R^n$, vale a igualdade $\leb^*\big(\widehat\sigma(A)\big)=\leb^*(A)$;
\item[(c)] se $A\subset\R^n$ é mensurável então $\widehat\sigma(A)$ também é mensurável.
\end{itemize}
\end{exercise}

\begin{exercise}\label{exe:diagonal}
Dado um vetor $\lambda=(\lambda_1,\ldots,\lambda_n)\in\R^n$ com todas as coordenadas não nulas, consideramos
o isomorfismo linear $D_\lambda:\R^n\to\R^n$ definido por:\index[simbolos]{$D_\lambda$}
\[D_\lambda(x_1,\ldots,x_n)=(\lambda_1x_1,\ldots,\lambda_nx_n),\]
para todo $(x_1,\ldots,x_n)\in\R^n$. Mostre que:
\begin{itemize}
\item[(a)] se $B$ é um bloco retangular $n$-dimensional então $D_\lambda(B)$ é também um bloco retangular $n$-dimensional
e:
\[\vert D_\lambda(B)\vert=\vert\lambda_1\vert\cdots\vert\lambda_n\vert\,\vert B\vert=
\vert\det D_\lambda\vert\,\vert B\vert;\]
\item[(b)] para todo $A\subset\R^n$, vale a igualdade $\leb^*\big(D_\lambda(A)\big)=\vert\det D_\lambda\vert\,\leb^*(A)$;
\item[(c)] se $A\subset\R^n$ é mensurável então $D_\lambda(A)$ também é mensurável.
\end{itemize}
\end{exercise}

\begin{exdefin}
Dados conjuntos $A$ e $B$ então a {\em diferença simétrica\/}\index[indice]{diferenca simetrica@diferença simétrica}
de $A$ e $B$ é definida por:
\index[simbolos]{$A\bigtriangleup B$}\[A\bigtriangleup B=(A\setminus B)\cup(B\setminus A).\]
\end{exdefin}

\begin{exercise}\label{exe:AtriangleBzero}
Sejam $A,B\subset\R^n$ tais que $\leb^*(A\bigtriangleup B)=0$. Mostre que:
\begin{itemize}
\item $\leb^*(A)=\leb^*(B)$;
\item $A$ é mensurável se e somente se $B$ é mensurável.
\end{itemize}
\end{exercise}

\begin{exercise}\label{exe:mensaproxblocos}
Dado um subconjunto mensurável $A\subset\R^n$ tal que $\leb(A)<+\infty$, mostre que,
para todo $\varepsilon>0$, existem blocos retangulares $n$-di\-men\-sio\-nais $B_1$, \dots, $B_t$ com
interiores dois a dois disjuntos de modo que:
\[\leb\Big(\big({\textstyle\bigcup_{k=1}^t B_k}\big)\bigtriangleup A\Big)<\varepsilon.\]
\end{exercise}

\begin{exercise}\label{exe:mstarweakcontr}
Dados subconjuntos $A,B\subset\R^n$ com $\leb^*(A)<+\infty$ ou $\leb^*(B)<+\infty$, mostre que:
\[\big\vert\leb^*(A)-\leb^*(B)\big\vert\le\leb^*(A\bigtriangleup B).\]
\end{exercise}

\begin{exercise}\label{exe:novoenvelope}
Seja $A\subset\R^n$ e seja $E\subset\R^n$ um envelope mensurável de $A$. Se $E'$ é um conjunto mensurável
tal que $A\subset E'\subset E$, mostre que $E'$ também é um envelope mensurável de $A$.
\end{exercise}

\begin{exercise}\label{exe:LebcomplBorel}
Seja $E$ um subconjunto de $\R^n$. Mostre que as seguintes condições são equivalentes:
\begin{itemize}
\item[(a)] $E$ é Lebesgue mensurável;
\item[(b)] existem Boreleanos $A,M\subset\R^n$ e um subconjunto $N$ de $M$ de modo que
$E=A\cup N$ e $\leb(M)=0$.
\end{itemize}
\end{exercise}

\begin{exercise}\label{exe:muAcupB}
Seja $(X,\mathcal A,\mu)$ um espaço de medida. Dados $A,B\in\mathcal A$ com $\mu(A\cap B)<+\infty$,
mostre que:
\[\mu(A\cup B)=\mu(A)+\mu(B)-\mu(A\cap B).\]
\end{exercise}

\begin{exercise}\label{exe:disjuntar}
Seja $(A_k)_{k\ge1}$ uma seqüência de conjuntos. Defina:
\[B_k=A_k\setminus\bigcup_{i=0}^{k-1}A_i,\]
para todo $k\ge1$, onde $A_0=\emptyset$. Mostre que
os conjuntos $(B_k)_{k\ge1}$ são dois a dois disjuntos e que:
\[\bigcup_{k=1}^\infty A_k=\bigcup_{k=1}^\infty B_k.\]
\end{exercise}

\begin{exercise}\label{exe:musubad}
Seja $(X,\mathcal A,\mu)$ um espaço de medida e seja $(A_k)_{k\ge1}$ uma seqüência de elementos
de $\mathcal A$. Mostre que $\mu\big(\bigcup_{k=1}^\infty A_k\big)\le\sum_{k=1}^\infty\mu(A_k)$.
\end{exercise}

\begin{exercise}\label{exe:quasedisjuntos}
Seja $(X,\mathcal A,\mu)$ um espaço de medida e seja $(A_k)_{k\ge1}$ uma seqüência de elementos
de $\mathcal A$ tal que $\mu(A_k\cap A_l)=0$, para todos $k,l\ge1$ com $k\ne l$. Mostre que
$\mu\big(\bigcup_{k=1}^\infty A_k\big)=\sum_{k=1}^\infty\mu(A_k)$.
\end{exercise}

\begin{exercise}\label{exe:sigmagerada}
Seja $X$ um conjunto arbitrário.
\begin{itemize}
\item[(a)] Se $(\mathcal A_i)_{i\in I}$ é uma família não vazia de $\sigma$-álgebras de partes de $X$, mostre que
$\mathcal A=\bigcap_{i\in I}\mathcal A_i$ também é uma $\sigma$-álgebra de partes de $X$.
\item[(b)] Mostre que, fixada uma coleção $\mathcal C\subset\wp(X)$ de partes de $X$, existe {\em no máximo uma\/}
$\sigma$-álgebra $\sigma[\mathcal C]$ de partes de $X$ satisfazendo as propriedades \eqref{itm:sigma1} e
\eqref{itm:sigma2} que aparecem na Definição~\ref{thm:defsigmagerada}.
\item[(c)] Dada uma coleção arbitrária $\mathcal C\subset\wp(X)$ de partes de $X$, mostre que a interseção de todas
as $\sigma$-álgebras de partes de $X$ que contém $\mathcal C$ é uma $\sigma$-álgebra de partes
de $X$ que satisfaz as propriedades \eqref{itm:sigma1} e
\eqref{itm:sigma2} que aparecem na Definição~\ref{thm:defsigmagerada} (note que sempre existe
ao menos uma $\sigma$-álgebra de partes de $X$ contendo $\mathcal C$, a saber, $\wp(X)$).
\end{itemize}
\end{exercise}

\begin{exercise}\label{exe:relacionargeradas}
Seja $X$ um conjunto arbitrário e sejam $\mathcal C_1,\mathcal C_2\subset\wp(X)$ coleções arbitrárias de partes de $X$.
Se $\mathcal C_1\subset\sigma[\mathcal C_2]$ e $\mathcal C_2\subset\sigma[\mathcal C_1]$, mostre que
$\sigma[\mathcal C_1]=\sigma[\mathcal C_2]$.
\end{exercise}

\begin{exercise}\label{exe:GdeltaFsigmaBorel}
Mostre que todo subconjunto de $\R^n$ de tipo $G_\delta$ ou de tipo $F_\sigma$ é Boreleano.
\end{exercise}

\begin{exercise}\label{exe:BorelRgeradores}
Mostre que:
\begin{itemize}
\item[(a)] a $\sigma$-álgebra de Borel de $\R$ coincide com a $\sigma$-álgebra gerada pelos intervalos da forma
$\left]a,b\right]$, com $a<b$, $a,b\in\R$;
\item[(b)] a $\sigma$-álgebra de Borel de $\R$ coincide com a $\sigma$-álgebra gerada pelos intervalos da forma
$\left]-\infty,c\right]$, $c\in\R$.
\end{itemize}
\end{exercise}

\begin{exercise}\label{exe:rarotudonaoda}
Se $I$ é um intervalo fechado e limitado de comprimento positivo, mostre que o único subconjunto
fechado $F\subset I$ com $\leb(F)=\vert I\vert$ é $F=I$. Conclua que não existe um subconjunto
fechado com interior vazio $F\subset I$ tal que $\leb(F)=\vert I\vert$ (compare com o
Exemplo~\ref{exa:raroquaseI}).
\end{exercise}

\begin{*exercise}\label{exe:medLebproduto}
Sejam dados conjuntos $A\subset\R^m$, $B\subset\R^n$, de modo que $A\times B\subset\R^m\times\R^n\cong\R^{m+n}$.
\begin{itemize}
\item[(a)] Mostre que $\leb^*(A\times B)\le\leb^*(A)\leb^*(B)$.
\item[(b)] Mostre que se $A$ e $B$ são mensuráveis então $A\times B$ também é mensurável.
\item[(c)] Mostre que se $A$ e $B$ são mensuráveis então $\leb(A\times B)=\leb(A)\leb(B)$.
\end{itemize}
\end{*exercise}

\subsection*{Medida Interior}

\begin{exercise}\label{exe:lebintlelebext}
Dado $A\subset\R^n$, mostre que $\leb_*(A)\le\leb^*(A)$.
\end{exercise}

\begin{exercise}\label{exe:medintmonot}
Mostre que a medida interior de Lebesgue é mo\-no\-tô\-ni\-ca, i.e., se $A_1\subset A_2\subset\R^n$
então $\leb_*(A_1)\le\leb_*(A_2)$.
\end{exercise}

\begin{exercise}\label{exe:alternativamedint}
Dado $A\subset\R^n$, mostre que:
\[\leb_*(A)=\sup\big\{\leb(E):E\subset A,\ \text{$E$ mensurável}\big\}.\]
Mais geralmente, mostre que se $\mathcal M'$ é um subconjunto de $\Lebmens(\R^n)$ que contém
todos os subconjuntos compactos de $\R^n$ então:
\[\leb_*(A)=\sup\big\{\leb(E):E\subset A,\ E\in\mathcal M'\big\}.\]
\end{exercise}

\begin{exercise}\label{exe:internalmens}
Dado um subconjunto $A\subset\R^n$, mostre que existe um subconjunto $W$ de $\R^n$ de tipo $F_\sigma$
tal que $W\subset A$ e $\leb(W)=\leb_*(A)$.
\end{exercise}

\begin{exercise}\label{exe:intersuperadd}
Seja $(A_k)_{k\ge1}$ uma seqüência de subconjuntos dois a dois disjuntos de $\R^n$. Mostre que:
\[\leb_*\Big(\bigcup_{k=1}^\infty A_k\Big)\ge\sum_{k=1}^\infty\leb_*(A_k).\]
\end{exercise}

\begin{exercise}\label{exe:AksearrowA}
Seja $(A_k)_{k\ge1}$ uma seqüência de subconjuntos de $\R^n$ tal que $A_k\searrow A$ e
$\leb_*(A_k)<+\infty$ para algum $k\ge1$. Mostre que:
\[\leb_*(A)=\lim_{k\to\infty}\leb_*(A_k).\]
\end{exercise}

\subsection*{Conjuntos de Cantor}

\begin{exdefin}
Um subconjunto de $\R^n$ é dito {\em magro\/}\index[indice]{conjunto!magro}\index[indice]{magro} quando
está contido numa reunião enumerável de subconjuntos fechados de $\R^n$ com interior vazio.
\end{exdefin}
O famoso {\em Teorema de Baire\/}\index[indice]{Baire!teorema de}\index[indice]{teorema!de Baire}
implica que todo subconjunto magro de $\R^n$ tem interior vazio.

\begin{exercise}
Mostre que:
\begin{itemize}
\item existe um subconjunto magro e mensurável $A\subset[0,1]$ tal que $\leb(A)=1$
(compare com o Exercício~\ref{exe:rarotudonaoda});
\item se $A$ é o conjunto do item anterior, mostre que $[0,1]\setminus A$ é um conjunto
de medida de Lebesgue zero que não é magro.
\end{itemize}
\end{exercise}

\begin{exercise}
Considere o intervalo $I=[0,1]$ e a seqüência $(\alpha_i)_{i\ge1}$ definida por:
\[\alpha_i=\frac{2^{i-1}}{3^i},\]
para todo $i\ge1$. O conjunto de Cantor $K$ associado a $I$ e à seqüência $(\alpha_i)_{i\ge1}$ é conhecido como o
{\em conjunto ternário de Cantor}\index[indice]{conjunto!de Cantor!ternario@ternário}\index[indice]{Cantor!conjunto ternario de@conjunto ternário de}.
Mostre que:
\begin{itemize}
\item $\leb(K)=0$;
\item para todo $n\ge1$ e todo $\epsilon=(\epsilon_1,\ldots,\epsilon_n)\in\{0,1\}^n$
o intervalo $I(\epsilon)$ é dado por:
\[I(\epsilon)=\Big[\sum_{i=1}^n\frac{2\epsilon_i}{3^i},\frac1{3^n}+\sum_{i=1}^n\frac{2\epsilon_i}{3^i}\Big];\]
\item a bijeção $\phi:\{0,1\}^\infty\to K$ definida em \eqref{eq:defphiCantor} é dada por:
\[\phi(\epsilon)=\sum_{i=1}^\infty\frac{2\epsilon_i}{3^i},\]
para todo $\epsilon=(\epsilon_i)_{i\ge1}\in\{0,1\}^\infty$.
\end{itemize}
\end{exercise}

\begin{exercise}
Considere a relação de {\em ordem lexicográfica\/}\index[indice]{relacao de ordem@relação de ordem!lexicografica@lexicográfica}
no conjunto $\{0,1\}^\infty$, i.e.,
para $\epsilon=(\epsilon_i)_{i\ge1},\epsilon'=(\epsilon'_i)_{i\ge1}\in\{0,1\}^\infty$ dizemos
que $\epsilon<\epsilon'$ quando existe um índice $i\ge1$
tal que $(\epsilon_1,\ldots,\epsilon_{i-1})=(\epsilon'_1,\ldots,\epsilon'_{i-1})$
e $\epsilon_i<\epsilon'_i$. Mostre que a função $\phi:\{0,1\}^\infty\to K$ definida em
\eqref{eq:defphiCantor} é {\em estritamente crescente}\index[indice]{funcao@função!estritamente crescente},
i.e., se $\epsilon<\epsilon'$ então $\phi(\epsilon)<\phi(\epsilon')$.
\end{exercise}

\begin{exercise}
Utilizando a notação da Seção~\ref{sec:Cantor}, mostre que para todo $n\ge1$ e todo
$\epsilon=(\epsilon_i)_{i=1}^n\in\{0,1\}^n$, a extremidade esquerda do intervalo $I(\epsilon)$ é
$\phi(\epsilon_1,\ldots,\epsilon_n,0,0,\ldots)$ e a extremidade direita de $I(\epsilon)$
é $\phi(\epsilon_1,\ldots,\epsilon_n,1,1,\ldots)$.
\end{exercise}

\subsection*{Conjuntos não Mensuráveis}

\begin{exercise}
Mostre que existe um subconjunto não mensurável $A$ de $\R^n$ tal que $\leb_*(A)=\leb^*(A)=+\infty$.
\end{exercise}

\end{chapter}

\begin{chapter}{Integrando Funções em Espaços de Medida}
\label{CHP:INTEGRAL}

\begin{section}{Funções Mensuráveis}

Recorde da Definição~\ref{thm:defespacomedida} que um espaço mensurável\index[indice]{espaco@espaço!mensuravel@mensurável}\index[indice]{mensuravel@mensurável!espaco@espaço}
é um conjunto $X$ do qual destacamos uma certa coleção de subconjuntos $\mathcal A\subset\wp(X)$
(mais precisamente, uma $\sigma$-álgebra de partes de $X$) aos quais damos o nome de mensuráveis.
A palavra ``mensurável'' nesse contexto não indica que os conjuntos possam ser medidos de alguma
forma ou que estamos assumindo a existência de alguma medida não trivial definida em $\mathcal A$.
Um mesmo conjunto $X$ admite em geral diversas $\sigma$-álgebras; por exemplo, $\{\emptyset,X\}$
e $\wp(X)$ são sempre exemplos (triviais) de $\sigma$-álgebras de partes de $X$. Portanto, o termo
``mensurável'' só deve ser usado quando uma $\sigma$-álgebra específica
estiver fixada pelo contexto. No conjunto $\R^n$, temos dois exemplos importantes de $\sigma$-álgebras;
a $\sigma$-álgebra de Borel $\Borel(\R^n)$ e a $\sigma$-álgebra $\Lebmens(\R^n)$ de conjuntos
Lebesgue mensuráveis. No que segue, precisaremos também introduzir uma $\sigma$-álgebra de Borel
para a reta estendida $\overline\R$; temos a seguinte:
\begin{defin}
Um subconjunto $A\subset\overline\R$ é dito {\em Boreleano\/}\index[indice]{conjunto!Boreleano!em R@em $\overline\R$}\index[indice]{Boreleano!em R@em $\overline\R$}%
\index[indice]{reta estendida!Boreleanos da} quando $A\cap\R$ for um Boreleano de $\R$.
\end{defin}
É fácil ver que os subconjuntos Boreleanos de $\overline\R$ constituem de fato uma $\sigma$-álgebra de partes
de $\overline\R$. Tal $\sigma$-álgebra será chamada a {\em $\sigma$-álgebra de Borel\/}\index[indice]{sigma algebra@$\sigma$-álgebra!de Borel!de R@de $\overline\R$}%
\index[indice]{Borel!sigma algebra de@$\sigma$-álgebra de!de R@de $\overline\R$}
de $\overline\R$ e será denotada por $\Borel(\overline\R)$\index[simbolos]{$\Borel(\overline\R)$}.

A $\sigma$-álgebra $\mathcal A$ de um espaço mensurável $(X,\mathcal A)$ pode ser entendida
como uma {\em estrutura\/} que colocamos no conjunto subjacente $X$ (assim como, digamos,
as operações de um espaço vetorial constituem uma estrutura no conjunto subjacente).
Devemos então introduzir uma noção de {\em função que preserva a estrutura\/} de um espaço mensurável.
\begin{defin}\label{thm:deffuncmens}
Sejam $(X,\mathcal A)$, $(X',\mathcal A')$ espaços mensuráveis. Uma
{\em função mensurável\/}\index[indice]{mensuravel@mensurável!funcao@função}\index[indice]{funcao@função!mensuravel@mensurável}
$f:(X,\mathcal A)\to(X',\mathcal A')$ é uma função $f:X\to X'$ tal que
para todo conjunto $E\in\mathcal A'$ temos que $f^{-1}(E)$ pertence a $\mathcal A$.
\end{defin}
Em outras palavras, uma função é mensurável se a imagem inversa de conjuntos mensuráveis é mensurável.
Quando as $\sigma$-álgebras em questão estiverem subentendidas pelo contexto,
nos referiremos apenas à mensurabilidade da função $f:X\to X'$, omitindo a menção explícita
a $\mathcal A$ e $\mathcal A'$.

O conjunto $\R^n$ aparecerá com muita freqüência como domínio ou contra-domínio de nossas funções
e introduzimos abaixo uma convenção que evita a necessidade de especificar a $\sigma$-álgebra
considerada em $\R^n$ em cada situação.
\begin{convention}\label{cnv:menscontrRn}
A menos de menção explícita em contrário, o conjunto $\R^n$ será considerado munido da $\sigma$-álgebra
de Borel $\Borel(\R^n)$ sempre que o mesmo aparecer no {\em contra-domínio\/} de uma função; mais explicitamente,
se $(X,\mathcal A)$ é um espaço mensurável então por uma {\em função mensurável\/}
$f:(X,\mathcal A)\to\R^n$ entenderemos uma função $f:X\to\R^n$ tal que $f^{-1}(E)\in\mathcal A$,
para todo Boreleano $E\in\Borel(\R^n)$. Similarmente, a reta estendida $\overline\R$ será
considerada munida da $\sigma$-álgebra de Borel $\Borel(\overline\R)$, sempre que a mesma
aparecer no contra-domínio de uma função.\index[indice]{mensuravel@mensurável!funcao@função!a valores em Rn ou R@a valores em $\R^n$ ou $\overline\R$}%
\index[indice]{funcao@função!mensuravel@mensurável!a valores em Rn ou R@a valores em $\R^n$ ou $\overline\R$}
Por outro lado, o conjunto $\R^n$ será sempre considerado munido da $\sigma$-álgebra $\Lebmens(\R^n)$
de conjuntos Lebesgue mensuráveis, quando o mesmo aparecer no {\em domínio\/} de uma função;
mais explicitamente, uma {\em função mensurável\/}\index[indice]{mensuravel@mensurável!funcao@função!definida em Rn@definida em $\R^n$}%
\index[indice]{funcao@função!mensuravel@mensurável!definida em Rn@definida em $\R^n$} $f:\R^n\to(X,\mathcal A)$ é uma função
$f:\R^n\to X$ tal que $f^{-1}(E)\in\Lebmens(\R^n)$, para todo $E\in\mathcal A$.
\end{convention}
Por exemplo, em vista da convenção~\ref{cnv:menscontrRn} acima, uma função mensurável $f:\R\to\R$
é uma função tal que $f^{-1}(E)\in\Lebmens(\R)$, para todo $E\in\Borel(\R)$.

Nós dificilmente teremos qualquer interesse em considerar a $\sigma$-álgebra $\Lebmens(\R^n)$
em $\R^n$ quando o mesmo aparece no contra-domínio de uma função; por outro lado, em algumas
situações é interessante considerar a $\sigma$-álgebra $\Borel(\R^n)$ em $\R^n$ quando o mesmo
aparece no domínio de uma função (contrariando, portanto, a convenção~\ref{cnv:menscontrRn}).
Introduzimos então a seguinte terminologia.
\begin{defin}\label{thm:Borelmens}
Seja $(X,\mathcal A)$ um espaço mensurável. Uma {\em função Borel mensurável\/}\index[indice]{mensuravel@mensurável!funcao Borel@função Borel}%
\index[indice]{funcao@função!Borel mensuravel@Borel mensurável}\index[indice]{Borel mensuravel@Borel mensurável!funcao@função}
$f:\R^n\to(X,\mathcal A)$ é uma função $f:\R^n\to X$ tal que $f:\big(\R^n,\Borel(\R^n)\big)\to(X,\mathcal A)$
é uma função mensurável, i.e., tal que $f^{-1}(E)$ é um Boreleano de $\R^n$ para todo $E\in\mathcal A$.
Similarmente, uma {\em função Borel mensurável\/} $f:\overline\R\to(X,\mathcal A)$ é uma função
$f:\overline\R\to X$ tal que $f:\big(\overline\R,\Borel(\overline\R)\big)\to(X,\mathcal A)$
é uma função mensurável.
\end{defin}

Para verificar a mensurabilidade de uma função $f:(X,\mathcal A)\to(X',\mathcal A')$ não é necessário
verificar que $f^{-1}(E)\in\mathcal A$ {\em para todo\/} $E\in\mathcal A'$, mas apenas para $E$
pertencente a um conjunto de geradores de $\mathcal A'$. Esse é o conteúdo do seguinte:
\begin{lem}\label{thm:funcmensgeradores}
Sejam $(X,\mathcal A)$, $(X',\mathcal A')$ espaços mensuráveis e seja $\mathcal C$ um conjunto de geradores
para a $\sigma$-álgebra $\mathcal A'$. Uma função $f:X\to X'$ é mensurável se e somente se $f^{-1}(E)\in\mathcal A$,
para todo $E\in\mathcal C$.
\end{lem}
\begin{proof}
Como $\mathcal C\subset\mathcal A'$, temos obviamente que $f^{-1}(E)\in\mathcal A$ para todo $E\in\mathcal C$, caso
$f$ seja mensurável. Suponha então que $f^{-1}(E)\in\mathcal A$ para todo $E\in\mathcal C$.
Verifica-se diretamente que a coleção:
\begin{equation}\label{eq:colecaopartesXlinha}
\big\{E\in\wp(X'):f^{-1}(E)\in\mathcal A\big\}
\end{equation}
é uma $\sigma$-álgebra de partes de $X'$. Por hipótese, \eqref{eq:colecaopartesXlinha}
contém $\mathcal C$ e portanto contém $\mathcal A'=\sigma[\mathcal C]$.
Isso mostra que $f^{-1}(E)\in\mathcal A$ para todo $E\in\mathcal A'$, i.e., $f$ é mensurável.
\end{proof}

\begin{cor}\label{thm:cormenscontrRn}
Se $(X,\mathcal A)$ é um espaço mensurável então uma função $f:X\to\R^n$ é mensurável se e somente se
$f^{-1}(U)\in\mathcal A$, para todo aberto $U\subset\R^n$.\qed
\end{cor}

\begin{cor}
Se $(X,\mathcal A)$ é um espaço mensurável então uma função $f:X\to\R$ é mensurável se e somente se o conjunto:
\[f^{-1}\big(\left]-\infty,c\right]\big)=\big\{x\in X:f(x)\le c\big\}\]
está em $\mathcal A$ para todo $c\in\R$.
\end{cor}
\begin{proof}
Segue do Lema~\ref{thm:funcmensgeradores}, tendo em mente o resultado do Exercício~\ref{exe:BorelRgeradores}.
\end{proof}

\begin{cor}\label{thm:cormenscontrbarR}
Se $(X,\mathcal A)$ é um espaço mensurável então uma função $f:X\to\overline\R$ é mensurável se e somente se o conjunto:
\[f^{-1}\big([-\infty,c]\big)=\big\{x\in X:f(x)\le c\big\}\]
está em $\mathcal A$ para todo $c\in\R$.
\end{cor}
\begin{proof}
Segue do Lema~\ref{thm:funcmensgeradores}, tendo em mente o resultado do Exercício~\ref{exe:geradoresbarR}.
\end{proof}

\begin{lem}\label{thm:compmens}
A composta de duas funções mensuráveis é uma função mensurável, i.e., se $(X,\mathcal A)$, $(X',\mathcal A')$, $(X'',\mathcal A'')$ são
espaços mensuráveis e se $f:(X,\mathcal A)\to(X',\mathcal A')$, $g:(X',\mathcal A')\to(X'',\mathcal A'')$ são funções
mensuráveis então a função $g\circ f:(X,\mathcal A)\to(X'',\mathcal A'')$ também é mensurável.
\end{lem}
\begin{proof}
Dado $E\in\mathcal A''$ devemos verificar que $(g\circ f)^{-1}(E)\in\mathcal A$. Mas $(g\circ f)^{-1}(E)=f^{-1}\big(g^{-1}(E)\big)$;
temos $g^{-1}(E)\in\mathcal A'$, pois $g$ é mensurável, e $f^{-1}\big(g^{-1}(E)\big)\in\mathcal A$, pois $f$ é mensurável.
\end{proof}
É necessário cüidado na utilização do Lema~\ref{thm:compmens}; para concluir a mensurabilidade
de $g\circ f$ a partir da mensurabilidade de $f$ e de $g$ é necessário que a $\sigma$-álgebra
fixada para o contra-domínio de $f$ e para o domínio de $g$ sejam as mesmas. Em vista da convenção~\ref{cnv:menscontrRn},
se $f:(X,\mathcal A)\to\R^n$ e $g:\R^n\to(X',\mathcal A')$ são funções mensuráveis então
{\em não podemos\/} usar o Lema~\ref{thm:compmens} para concluir que $g\circ f$ é mensurável
já que adotamos a $\sigma$-álgebra de Borel para o contra-domínio de $f$ e a $\sigma$-álgebra
de conjuntos Lebesgue mensuráveis para o domínio de $g$. Nós poderíamos utilizar o Lema~\ref{thm:compmens}
para concluir que $g\circ f$ é mensurável caso soubéssemos, por exemplo, que $f$ é mensurável e que $g$ é {\em Borel mensurável}.

Se $f$ é uma função definida num espaço mensurável $(X,\mathcal A)$ então em muitas situações
é interessante considerar restrições de $f$ a subconjuntos de $X$ e gostaríamos que tais subconjuntos
de $X$ pudessem ser encarados como espaços mensuráveis. Dado então um subconjunto $Y\subset X$, definimos:
\begin{equation}\label{eq:ArestrY}
\mathcal A\vert_Y=\big\{E\cap Y:E\in\mathcal A\big\}\index[simbolos]{$\mathcal A\vert_Y$};
\end{equation}
é fácil ver que $\mathcal A\vert_Y$ é uma $\sigma$-álgebra de partes de $Y$
(veja Exercício~\ref{exe:AvertY}).
\begin{defin}
Se $\mathcal A$ é uma $\sigma$-álgebra de partes de um conjunto $X$ e se $Y$ é um subconjunto de $X$ então
a $\sigma$-álgebra $\mathcal A\vert_Y$ de partes de $Y$ definida em \eqref{eq:ArestrY} é chamada a {\em
$\sigma$-álgebra induzida em $Y$ por $\mathcal A$}.\index[indice]{sigma algebra@$\sigma$-álgebra!induzida num subconjunto}
Dizemos então que $(Y,\mathcal A\vert_Y)$ é um
{\em subespaço\/}\index[indice]{subespaco@subespaço!de um espaco mensuravel@de um espaço mensurável}%
\index[indice]{espaco@espaço!mensuravel@mensurável!subespaco de@subespaço de}\index[indice]{mensuravel@mensurável!subespaco@subespaço}
do espaço mensurável $(X,\mathcal A)$.
\end{defin}
Observe que se $(X,\mathcal A)$ é um espaço mensurável e se $Y\in\mathcal A$ então os elementos da $\sigma$-álgebra
induzida $\mathcal A\vert_Y$ são precisamente os elementos de $\mathcal A$ que estão contidos em $Y$; em símbolos:
\[\mathcal A\vert_Y=\mathcal A\cap\wp(Y).\]
Em outras palavras, se $Y$ é mensurável então os subconjuntos mensuráveis do subespaço mensurável $Y$ de $X$ são
precisamente os subconjuntos mensuráveis de $X$ que estão contidos em $Y$.

\begin{convention}\label{cnv:subespacomens}
Se $(X,\mathcal A)$ é um espaço mensurável e se $Y$ é um subconjunto de $X$ então, a menos
de menção explícita em contrário, consideraremos sempre o conjunto $Y$ munido da $\sigma$-álgebra
induzida $\mathcal A\vert_Y$.
\end{convention}
Em vista das convenções~\ref{cnv:subespacomens} e \ref{cnv:menscontrRn}, observamos que:
\begin{itemize}
\item se um subconjunto $Y$ de $\R^n$ (resp., um subconjunto $Y$ de $\overline\R$) aparece
no contra-domínio de uma função, consideramo-lo munido da $\sigma$-álgebra $\Borel(\R^n)\vert_Y$
induzida da $\sigma$-álgebra de Borel de $\R^n$ (resp., da $\sigma$-álgebra $\Borel(\overline\R)\vert_Y$
induzida da $\sigma$-álgebra de Borel de $\overline\R$);

\item se um subconjunto $Y$ de $\R^n$ aparece no domínio de uma função, consideramo-lo munido
da $\sigma$-álgebra $\Lebmens(\R^n)\vert_Y$ induzida da $\sigma$-ál\-ge\-bra de subconjuntos
Lebesgue mensuráveis de $\R^n$;

\item se $Y$ é um subconjunto de $\R^n$ (resp., um subconjunto de $\overline\R$) e se
$(X,\mathcal A)$ é um espaço mensurável então uma função $f:Y\to(X,\mathcal A)$ é dita
{\em Borel mensurável\/} quando a função $f:\big(Y,\Borel(\R^n)\vert_Y\big)\to(X,\mathcal A)$
(resp., a função $f:\big(Y,\Borel(\overline\R)\vert_Y\big)\to(X,\mathcal A)$) for mensurável.
\end{itemize}

\begin{lem}\label{thm:propsrestr}
Sejam $(X,\mathcal A)$, $(X',\mathcal A')$ espaços mensuráveis e $Y\subset X$ um subconjunto. Então:
\begin{itemize}
\item[(a)] a aplicação inclusão $i:Y\to X$ é mensurável;
\item[(b)] se $f:X\to X'$ é uma função mensurável então $f\vert_Y:Y\to X'$ também é mensurável;
\item[(c)] dada uma função $f:X'\to X$ com imagem contida em $Y$, se $f_0:X'\to Y$ denota a função que difere de $f$ apenas
pelo contra-domínio então $f$ é mensurável se e somente se $f_0$ é mensurável.
\end{itemize}
\end{lem}
\begin{proof}\
\begin{bulletindent}
\item {\em Prova de\/} (a).

Basta observar que $i^{-1}(E)=E\cap Y\in\mathcal A\vert_Y$, para todo $E\in\mathcal A$.

\item {\em Prova de\/} (b).

Basta observar que $f\vert_Y=f\circ i$ e usar o Lema~\ref{thm:compmens} juntamente com o item (a) acima.

\item {\em Prova de\/} (c).

Se $f_0$ é mensurável então $f=i\circ f_0$ é mensurável, pelo Lema~\ref{thm:compmens} e pelo item (a) acima.
Reciprocamente, suponha que $f$ é mensurável. Dado $E_1\in\mathcal A\vert_Y$, devemos mostrar que $f_0^{-1}(E_1)$
(que é igual a $f^{-1}(E_1)$) pertence a $\mathcal A'$. Mas $E_1=E\cap Y$ para algum $E\in\mathcal A$ e portanto,
como $\Img(f)\subset Y$, temos $f^{-1}(E_1)=f^{-1}(E)\in\mathcal A'$.\qedhere
\end{bulletindent}
\end{proof}

\begin{lem}\label{thm:cobremensf}
Sejam $(X,\mathcal A)$, $(X',\mathcal A')$ espaços mensuráveis e seja dada
$X=\bigcup_{i\in I}X_i$ uma cobertura enumerável de $X$ por conjuntos mensuráveis $X_i\in\mathcal A$.
Então uma função $f:X\to X'$ é mensurável se e somente se
$f\vert_{X_i}:X_i\to X'$ é mensurável para todo $i\in I$.
\end{lem}
\begin{proof}
Se $f$ é mensurável então $f\vert_{X_i}$ é mensurável para todo $i\in I$, pelo Lema~\ref{thm:propsrestr}.
Reciprocamente, suponha que $f\vert_{X_i}$ seja mensurável para todo $i\in I$. Dado $E\in\mathcal A'$,
temos:
\[(f\vert_{X_i})^{-1}(E)=f^{-1}(E)\cap X_i\in\mathcal A\vert_{X_i},\]
para todo $i\in I$.
Como $X_i\in\mathcal A$, temos $\mathcal A\vert_{X_i}=\mathcal A\cap\wp(X_i)$ e portanto
$f^{-1}(E)\cap X_i\in\mathcal A$, para todo $i\in I$. Como $I$ é enumerável segue que:
\[f^{-1}(E)=\bigcup_{i\in I}\big(f^{-1}(E)\cap X_i\big)\in\mathcal A,\]
e portanto $f$ é uma função mensurável.
\end{proof}

\begin{cor}\label{thm:BorelrestrR}
Sejam $(X,\mathcal A)$ um espaço mensurável e $Y$ um subconjunto de $\overline\R$. Uma função
$f:Y\to X$ é Borel mensurável se e somente se $f\vert_{Y\cap\R}:Y\cap\R\to X$ é Borel mensurável.
\end{cor}
\begin{proof}
Temos que $Y=(Y\setminus\R)\cup(Y\cap\R)$, onde:
\[Y\cap\R\in\Borel(\overline\R)\vert_Y,\quad
Y\setminus\R=Y\cap\{+\infty,-\infty\}\in\Borel(\overline\R)\vert_Y.\]
Segue do Lema~\ref{thm:cobremensf} que $f$ é Borel mensurável se e somente se
suas restrições a $Y\setminus\R$ e a $Y\cap\R$ são Borel mensuráveis.
Mas todos os quatro subconjuntos de $\{+\infty,-\infty\}$ são Boreleanos de $\overline\R$ e portanto
a $\sigma$-álgebra induzida por $\Borel(\overline\R)\vert_Y$ em $Y\setminus\R$
é $\wp(Y\setminus\R)$. Em particular, a restrição de $f$ a $Y\setminus\R$ é Borel mensurável,
seja qual for $f:Y\to X$. A conclusão segue.
\end{proof}

\begin{lem}\label{thm:contmens}
Dado um subconjunto arbitrário $Y\subset\R^m$, então toda função contínua
$f:Y\to\R^n$ é Borel mensurável.
\end{lem}
\begin{proof}
Pelo Corolário~\ref{thm:cormenscontrRn}, é suficiente mostrar que:
\[f^{-1}(U)\in\Borel(\R^m)\vert_Y,\]
para todo aberto $U\subset\R^n$.
Mas, como $f$ é contínua, temos que $f^{-1}(U)$ é aberto relativamente a $Y$, i.e., existe
um aberto $V\subset\R^m$ com:
\[f^{-1}(U)=V\cap Y;\]
daí $V\in\Borel(\R^m)$ e portanto $f^{-1}(U)=V\cap Y\in\Borel(\R^m)\vert_Y$.
\end{proof}

\begin{lem}\label{thm:menscoordbycoord}
Seja $(X,\mathcal A)$ um espaço mensurável e seja $f:X\to\R^n$ uma função com funções coordenadas
$f_i:X\to\R$, $i=1,\ldots,n$. Então $f:X\to\R^n$ é mensurável se e somente se $f_i:X\to\R$ for mensurável,
para todo $i=1,\ldots,n$.
\end{lem}
\begin{proof}
Temos $f_i=\pi_i\circ f$, onde $\pi_i:\R^n\to\R$ denota a $i$-ésima projeção. A função $\pi_i$
é contínua e portanto Borel mensurável, pelo Lema~\ref{thm:contmens}; segue então do Lema~\ref{thm:compmens}
que a mensurabilidade de $f$ implica na mensurabilidade de cada $f_i$.
Reciprocamente, suponha que cada $f_i$ é mensurável. Em vista do Lema~\ref{thm:abertocubos}, a $\sigma$-álgebra de Borel de $\R^n$
coincide com a $\sigma$-álgebra gerada pelos blocos retangulares $n$-dimensionais. Segue então do Lema~\ref{thm:funcmensgeradores}
que, para mostrar a mensurabilidade de $f$, é suficiente mostrar que $f^{-1}(B)\in\mathcal A$ para todo bloco retangular
$n$-dimensional $B$. Se $B=\prod_{i=1}^n[a_i,b_i]$, então:
\[f^{-1}(B)=\big\{x\in X:f_i(x)\in[a_i,b_i],\ i=1,\ldots,n\big\}=\bigcap_{i=1}^nf_i^{-1}\big([a_i,b_i]\big).\]
Como cada $f_i$ é mensurável, temos $f_i^{-1}\big([a_i,b_i]\big)\in\mathcal A$ para todo $i$ e portanto
$f^{-1}(B)\in\mathcal A$.
\end{proof}

\begin{cor}\label{thm:corphidefs}
Sejam $(X,\mathcal A)$, $(X',\mathcal A')$ espaços mensuráveis e sejam $f_i:X\to\R$, $i=1,\ldots,n$,
funções mensuráveis. Dada uma função Borel mensurável
$\phi:Y\to X'$ definida num subconjunto $Y\subset\R^n$ tal que:
\[\big(f_1(x),\ldots,f_n(x)\big)\in Y,\]
para todo $x\in X$ então a função:
\[\phi\circ(f_1,\ldots,f_n):X\ni x\longmapsto\phi\big(f_1(x),\ldots,f_n(x)\big)\in X'\]
é mensurável.
\end{cor}
\begin{proof}
Pelo Lema~\ref{thm:menscoordbycoord} e pelo item (c) do Lema~\ref{thm:propsrestr} temos
que a função $(f_1,\ldots,f_n):X\to Y$ é mensurável. A conclusão segue do Lema~\ref{thm:compmens}.
\end{proof}

Se $f:X\to\R^n$, $g:X\to\R^n$ são funções definidas num conjunto arbitrário $X$ então, como é usual,
definimos a {\em soma\/}\index[indice]{soma!de funcoes@de funções}\index[indice]{funcoes@funções!soma de}
$f+g:X\to\R^n$ das funções $f$ e $g$ fazendo
$(f+g)(x)=f(x)+g(x)$, para todo $x\in X$; para $n=1$, podemos definir também
o {\em produto\/}\index[indice]{produto!de funcoes@de funções}\index[indice]{funcoes@funções!produto de}
$fg:X\to\R^n$ fazendo $(fg)(x)=f(x)g(x)$, para todo $x\in X$.

\begin{cor}\label{thm:corsomaprodmens}
Seja $(X,\mathcal A)$ um espaço mensurável. Dadas funções mensuráveis $f:X\to\R^n$, $g:X\to\R^n$
então:
\begin{itemize}
\item a soma $f+g:X\to\R^n$ é uma função mensurável;
\item se $n=1$, o produto $fg:X\to\R$ é uma função mensurável.
\end{itemize}
\end{cor}
\begin{proof}
As funções:
\[\R^n\times\R^n\ni(x,y)\longmapsto x+y\in\R^n\quad\text{e}\quad\R\times\R\ni(x,y)\longmapsto xy\in\R\]
são contínuas e portanto Borel mensuráveis, pelo Lema~\ref{thm:contmens}. A conclusão
segue do Corolário~\ref{thm:corphidefs}.
\end{proof}

Note que para funções $f:X\to\overline\R$, $g:X\to\overline\R$ a valores na reta estendida, também podemos
definir a soma $f+g:X\to\overline\R$\index[indice]{soma!de funcoes@de funções}\index[indice]{funcoes@funções!soma de},
desde que a soma $f(x)+g(x)$ esteja bem definida (i.e., não seja da forma
$(+\infty)+(-\infty)$ ou $(-\infty)+(+\infty)$) para todo $x\in X$. O produto $fg:X\to\overline\R$\index[indice]{produto!de funcoes@de funções}\index[indice]{funcoes@funções!produto de}
pode ser definido sempre, sem nenhuma restrição sobre $f$ e $g$.

\begin{prop}\label{thm:somaprodmensRest}
Seja $(X,\mathcal A)$ um espaço mensurável. Sejam dadas funções mensuráveis
$f:X\to\overline\R$ e $g:X\to\overline\R$. Então:
\begin{itemize}
\item se a soma $f(x)+g(x)$ estiver bem definida para todo $x\in X$ então a função
$f+g:X\to\overline\R$ é uma função mensurável;
\item o produto $fg:X\to\overline\R$ é uma função mensurável.
\end{itemize}
\end{prop}
\begin{proof}
Considere os seguintes subconjuntos de $X$:
\begin{gather*}
f^{-1}(\R)\cap g^{-1}(\R),\\
f^{-1}(+\infty)\cup g^{-1}(+\infty),\\
f^{-1}(-\infty)\cup g^{-1}(-\infty);
\end{gather*}
todos eles pertencem a $\mathcal A$ e sua união é igual a $X$. A restrição de $f+g$ a cada um
deles é mensurável; de fato, a restrição de $f+g$ ao primeiro deles é mensurável pelo Corolário~\ref{thm:corsomaprodmens}
e a restrição de $f+g$ aos outros é uma função constante (veja Exercício~\ref{exe:constmens}). Segue então do Lema~\ref{thm:cobremensf}
que $f+g$ é mensurável. A mensurabilidade de $fg$ é mostrada de forma similar considerando
as restrições de $fg$ aos conjuntos:
\begin{gather*}
f^{-1}(\R)\cap g^{-1}(\R),\\
f^{-1}(0)\cup g^{-1}(0),\\
\big[f^{-1}(+\infty)\cap g^{-1}\big(\left]0,+\infty\right]\big)\big]\cup
\big[f^{-1}\big(\left]0,+\infty\right]\big)\cap g^{-1}(+\infty)\big],\\
\big[f^{-1}(-\infty)\cap g^{-1}\big(\left[-\infty,0\right[\big)\big]\cup
\big[f^{-1}\big(\left[-\infty,0\right[\big)\cap g^{-1}(-\infty)\big],\\
\big[f^{-1}(+\infty)\cap g^{-1}\big(\left[-\infty,0\right[\big)\big]\cup
\big[f^{-1}\big(\left[-\infty,0\right[\big)\cap g^{-1}(+\infty)\big],\\
\big[f^{-1}(-\infty)\cap g^{-1}\big(\left]0,+\infty\right]\big)\big]\cup
\big[f^{-1}\big(\left]0,+\infty\right]\big)\cap g^{-1}(-\infty)\big].\qedhere
\end{gather*}
\end{proof}

\begin{defin}
Dado $x\in\overline\R$ então a {\em parte positiva\/}\index[indice]{parte positiva} e a {\em parte negativa\/}\index[indice]{parte negativa}
de $x$, denotadas respectivamente por $x^+$ e $x^-$, são definidas por:
\[x^+=\begin{cases}
x,&\text{se $x\ge0$},\\
0,&\text{se $x<0$},
\end{cases}
\qquad
x^-=\begin{cases}
\hfil0,&\text{se $x>0$},\\
-x,&\text{se $x\le0$}.
\end{cases}\]
Se $f$ é uma função tomando valores em $\overline\R$ então a {\em parte positiva\/}\index[indice]{parte positiva!de uma funcao@de uma função}
e a {\em parte negativa\/}\index[indice]{parte negativa!de uma funcao@de uma função}
de $f$, denotadas respectivamente por $f^+$ e $f^-$\index[simbolos]{$f^+$}\index[simbolos]{$f^-$},
são definidas por $f^+(x)=[f(x)]^+$ e $f^-(x)=[f(x)]^-$, para todo $x$ no domínio de $f$.
\end{defin}
É fácil ver que $x=x^+-x^-$ e $\vert x\vert=x^++x^-$, para todo $x\in\overline\R$;
em particular, se $f$ é uma função tomando valores em $\overline\R$ então:
\[f=f^+-f^-\quad\text{e}\quad\vert f\vert=f^++f^-,\]
onde, obviamente, $\vert f\vert$ denota a função $\vert f\vert(x)=\vert f(x)\vert$.

\begin{lem}\label{thm:fmaismenosmens}
Seja $(X,\mathcal A)$ um espaço mensurável. Se $f:X\to\overline\R$ é uma função mensurável
então as funções $f^+$, $f^-$ e $\vert f\vert$ também são mensuráveis.
\end{lem}
\begin{proof}
Segue do Lema~\ref{thm:contmens} e do Corolário~\ref{thm:BorelrestrR} que as funções:
\[\overline\R\ni x\longmapsto x^+\in\overline\R,\quad
\overline\R\ni x\longmapsto x^-\in\overline\R,\quad
\overline\R\ni x\longmapsto \vert x\vert\in\overline\R\]
são Borel mensuráveis; de fato, observe que suas restrições a $\R$ são funções contínuas.
A conclusão segue do Lema~\ref{thm:compmens}.
\end{proof}

\begin{lem}
Seja $(X,\mathcal A)$ um espaço mensurável e seja $(f_k)_{k\ge1}$ uma seqüência de funções
mensuráveis $f_k:X\to\overline\R$. Então as funções:
\[\sup_{k\ge1}f_k:X\ni x\longmapsto\sup_{k\ge1}f_k(x)\in\overline\R\quad\text{e}\quad
\inf_{k\ge1}f_k:X\ni x\longmapsto\inf_{k\ge1}f_k(x)\in\overline\R\]
são mensuráveis.
\end{lem}
\begin{proof}
Note que para todo $x\in X$ temos $\sup_{k\ge1}f_k(x)\le c$ se e somente se
$f_k(x)\le c$ para todo $k\ge1$; logo:
\[\Big\{x\in X:\sup_{k\ge1}f_k(x)\le c\Big\}=\bigcap_{k=1}^\infty f_k^{-1}\big([-\infty,c]\big)\in\mathcal A,\]
para todo $c\in\R$. Além do mais, para todo $x\in X$, temos $\inf_{k\ge1}f_k(x)\le c$
se e somente se para todo $r\ge1$ existe $k\ge1$ tal que $f_k(x)\le c+\frac1r$; logo:
\[\Big\{x\in X:\inf_{k\ge1}f_k(x)\le c\Big\}=\bigcap_{r=1}^\infty\,\bigcup_{k=1}^\infty f_k^{-1}\big(\big[-\infty,c+\tfrac1r\big]\big)\in\mathcal A,\]
para todo $c\in\R$. A conclusão segue do Corolário~\ref{thm:cormenscontrbarR}.
\end{proof}

\begin{cor}\label{thm:liminfsupmens}
Seja $(X,\mathcal A)$ um espaço mensurável e seja $(f_k)_{k\ge1}$ uma seqüência de funções
mensuráveis $f_k:X\to\overline\R$. Então as funções:
\begin{gather*}
\limsup_{k\to\infty}f_k:X\ni x\longmapsto\limsup_{k\to\infty}f_k(x)\in\overline\R,\\
\liminf_{k\to\infty}f_k:X\ni x\longmapsto\liminf_{k\to\infty}f_k(x)\in\overline\R
\end{gather*}
são mensuráveis.
\end{cor}
\begin{proof}
Basta observar que:
\[\limsup_{k\to\infty}f_k=\infsup_{r\ge1}\,\sup_{k\ge r}f_k,\quad
\liminf_{k\to\infty}f_k=\sup_{r\ge1}\,\infsup_{k\ge r}f_k.\qedhere\]
\end{proof}

\begin{cor}\label{thm:limmensmens}
Seja $(X,\mathcal A)$ um espaço mensurável e seja $(f_k)_{k\ge1}$ uma seqüência de funções
mensuráveis $f_k:X\to\overline\R$. Se para todo $x\in X$ a seqüência
$\big(f_k(x)\big)_{k\ge1}$ converge em $\overline\R$ então a função:
\[\lim_{k\to\infty}f_k:X\ni x\longmapsto\lim_{k\to\infty}f_k(x)\in\overline\R\]
é mensurável.
\end{cor}
\begin{proof}
Basta observar que:
\[\lim_{k\to\infty}f_k=\liminf_{k\to\infty}f_k=\limsup_{k\to\infty}f_k.\qedhere\]
\end{proof}

\begin{subsection}{Funções Simples}

\begin{defin}\label{thm:deffuncsimples}
Uma função é dita {\em simples\/}\index[indice]{funcao@função!simples} quando sua imagem é um conjunto finito.
\end{defin}

\begin{lem}\label{thm:somaprodsimples}
Seja $X$ um conjunto e sejam $f:X\to\overline\R$, $g:X\to\overline\R$ funções simples.
\begin{itemize}
\item se a soma $f(x)+g(x)$ estiver bem definida para todo $x\in X$ então a função $f+g$ é simples;
\item o produto $fg$ é uma função simples.
\end{itemize}
\end{lem}
\begin{proof}
A imagem de $f+g$ está contida no conjunto:
\[\big\{a+b:\text{$a\in\Img(f)$, $b\in\Img(g)$ e a soma $a+b$ está bem definida}\big\};\]
tal conjunto é obviamente finito. Similarmente, a imagem de $fg$ está contida no conjunto finito
$\{ab:\text{$a\in\Img(f)$, $b\in\Img(g)$}\big\}$.
\end{proof}

\begin{lem}\label{thm:condsimplesmens}
Sejam $(X,\mathcal A)$ um espaço mensurável e $f:X\to\overline\R$ uma função simples.
Então $f$ é mensurável se e somente se $f^{-1}(c)\in\mathcal A$ para todo $c\in\Img(f)$.
\end{lem}
\begin{proof}
Se $f$ é uma função mensurável então $f^{-1}(c)\in\mathcal A$ para todo $c\in\Img(f)$, já que
$\{c\}$ é um Boreleano de $\overline\R$. Reciprocamente, se $f^{-1}(c)\in\mathcal A$ para todo
$c\in\Img(f)$ então a mensurabilidade de $f$ segue do Lema~\ref{thm:cobremensf}, já que:
\[X=\!\!\bigcup_{c\in\Img(f)}\!\!f^{-1}(c)\]
é uma cobertura finita de $X$ por conjuntos mensuráveis e a restrição de $f$ a cada conjunto
$f^{-1}(c)$ é mensurável (veja Exercício~\ref{exe:constmens}).
\end{proof}

\begin{defin}
Seja $X$ um conjunto e seja $A\subset X$ um subconjunto de $X$. A {\em função característica\/}\index[indice]{funcao@função!caracteristica@característica} de $A$,
definida em $X$, é a função $\chi_A:X\to\R$\index[simbolos]{$\chi_A$} definida por $\chi_A(x)=1$ para $x\in A$ e $\chi_A(x)=0$ para $x\in X\setminus A$.
\end{defin}
Observe que a notação $\chi_A$ não deixa explícito qual seja o domínio $X$ da função característica de $A$ que está sendo considerada;
em geral, tal domínio deve ser deixado claro pelo contexto.

\begin{rem}\label{thm:remchiAmens}
Se $(X,\mathcal A)$ é um espaço mensurável e se $A\subset X$ é um subconjunto então a função característica $\chi_A:X\to\R$
é uma função simples. Segue do Lema~\ref{thm:condsimplesmens} que $\chi_A$ é uma função mensurável
se e somente se $A\in\mathcal A$.
\end{rem}

\begin{rem}\label{thm:funcsimples}
Se $(X,\mathcal A)$ é um espaço mensurável então, dados $A_1,\ldots,A_k\in\mathcal A$ e $c_1,\ldots,c_k\in\overline\R$,
temos que a função:
\begin{equation}\label{eq:combcarac}
\sum_{i=1}^kc_i\chi_{A_i}:X\longrightarrow\overline\R
\end{equation}
é simples e mensurável, desde que esteja bem definida (i.e., desde que não ocorra $A_i\cap A_j\ne\emptyset$
com $c_i=+\infty$ e $c_j=-\infty$). De fato, isso segue da Proposição~\ref{thm:somaprodmensRest},
do Lema~\ref{thm:somaprodsimples} e da Observação~\ref{thm:remchiAmens}. Reciprocamente, se $f:X\to\overline\R$
é uma função simples e mensurável, podemos escrevê-la na forma \eqref{eq:combcarac}, com $A_i\in\mathcal A$
e $c_i\in\overline\R$, $i=1,\ldots,k$. De fato, basta tomar $A_i=f^{-1}(c_i)$, onde $c_1$, \dots, $c_k$ são os
elementos (distintos) do conjunto finito $\Img(f)$. Note que os conjuntos $A_i$ assim construídos
constituem uma partição de $X$.
\end{rem}

\begin{lem}\label{thm:fYfchiY}
Sejam $(X,\mathcal A)$ um espaço mensurável, $f:X\to\overline\R$ uma função e $Y\in\mathcal A$.
Então:
\begin{itemize}
\item[(a)] $f\vert_Y$ é mensurável se e somente se $f\chilow Y$ é mensurável;
\item[(b)] $f\vert_Y$ é simples se e somente se $f\chilow Y$ é simples.
\end{itemize}
\end{lem}
\begin{proof}
Temos $X=Y\cup Y^\compl$, com $Y,Y^\compl\in\mathcal A$; além do mais, $f\vert_Y=(f\chilow Y)\vert_Y$
e $(f\chilow Y)\vert_{Y^\compl}\equiv0$. Tendo em mente essas observações, o item (a) segue do Lema~\ref{thm:cobremensf}.
O item (b) segue da igualdade:
\[f(Y)\setminus\{0\}=\Img(f\chilow Y)\setminus\{0\}.\qedhere\]
\end{proof}

\begin{notation}
Seja $(f_k)_{k\ge1}$ uma seqüência de funções $f_k:X\to\overline\R$
e seja $f:X\to\overline\R$ uma função, onde $X$ é um conjunto arbitrário. Escrevemos $f_k\nearrow f$\index[simbolos]{$f_k\nearrow f$} quando
$f_k(x)\le f_{k+1}(x)$ para todo $x\in X$ e todo $k\ge1$ e $\lim_{k\to\infty}f_k(x)=f(x)$
para todo $x\in X$. Similarmente,
escrevemos $f_k\searrow f$\index[simbolos]{$f_k\searrow f$} quando $f_k(x)\ge f_{k+1}(x)$
para todo $x\in X$ e todo $k\ge1$ e $\lim_{k\to\infty}f_k(x)=f(x)$ para todo $x\in X$.
\end{notation}

\begin{prop}\label{thm:aproxmonotsimples}
Sejam $(X,\mathcal A)$ um espaço mensurável. Para toda função mensurável $f:X\to[0,+\infty]$
existe uma seqüência $(f_k)_{k\ge1}$
de funções simples e mensuráveis $f_k:X\to\left[0,+\infty\right[$ tal que $f_k\nearrow f$.
\end{prop}
\begin{proof}
Para cada $k\ge1$ particionamos o intervalo $\left[0,k\right[$ em intervalos disjuntos
de comprimento $\frac1{2^k}$; mais explicitamente, consideramos os intervalos:
\begin{equation}\label{eq:particao0k}
\left[\tfrac r{2^k},\tfrac{r+1}{2^k}\right[,\quad r=0,1,\ldots,k2^k-1.
\end{equation}
Para cada $x\in X$ temos $f(x)\ge k$ ou então $f(x)$ pertence a exatamente um dos intervalos
\eqref{eq:particao0k}; se $f(x)\ge k$ definimos $f_k(x)=k$ e, caso contrário,
tomamos $f_k(x)$ como sendo a extremidade esquerda do intervalo da coleção \eqref{eq:particao0k}
ao qual $f(x)$ pertence. Em símbolos, temos:
\[f_k=k\,\chi_{f^{-1}\big([k,+\infty]\big)}+
\sum_{r=0}^{k2^k-1}\frac r{2^k}\,\chi_{f^{-1}\big(\left[\tfrac r{2^k},\tfrac{r+1}{2^k}\right[\big)}.\]
Temos então que $f_k$ é uma função simples e mensurável para todo $k\ge1$ (veja Observação~\ref{thm:funcsimples}). Note que:
\begin{equation}\label{eq:distfkf}
\big\vert f_k(x)-f(x)\big\vert<\frac1{2^k},
\end{equation}
para todo $x\in X$ com $f(x)<k$. Afirmamos que $\lim_{k\to\infty}f_k=f$. De fato, seja $x\in X$ fixado.
Se $f(x)<+\infty$ então vale \eqref{eq:distfkf} para $k>f(x)$ e portanto $\lim_{k\to\infty}f_k(x)=f(x)$.
Se $f(x)=+\infty$ então $f_k(x)=k$ para todo $k\ge1$ e portanto $\lim_{k\to\infty}f_k(x)=+\infty=f(x)$.
Para completar a demonstração, vamos mostrar agora que:
\begin{equation}\label{eq:fkcresce}
f_k(x)\le f_{k+1}(x),
\end{equation}
para todos $x\in X$ e $k\ge1$. Sejam $x\in X$ e $k\ge1$ fixados.
Se $f(x)\ge k+1$, então $f_k(x)=k$ e $f_{k+1}(x)=k+1$, donde \eqref{eq:fkcresce} é satisfeita.
Senão, seja $r=0,\ldots,(k+1)2^{k+1}-1$
o único inteiro tal que $\frac r{2^{k+1}}\le f(x)<\frac{r+1}{2^{k+1}}$; temos
$f_{k+1}(x)=\frac r{2^{k+1}}$. Seja $s$ o maior inteiro menor ou igual a $\frac r2$;
daí $s\le\frac r2<\frac{r+1}2\le s+1$ e portanto:
\[\frac s{2^k}\le\frac r{2^{k+1}}\le f(x)<\frac{r+1}{2^{k+1}}\le\frac{s+1}{2^k}.\]
Se $f(x)\in\left[0,k\right[$, segue que $f_k(x)=\frac s{2^k}\le\frac r{2^{k+1}}=f_{k+1}(x)$.
Caso contrário, se $f(x)\in\left[k,k+1\right[$ então $r\ge k2^{k+1}$ e
$f_{k+1}(x)=\frac r{2^{k+1}}\ge k=f_k(x)$. Em todo caso, a desigualdade \eqref{eq:fkcresce}
é satisfeita.
\end{proof}

\end{subsection}

\end{section}

\begin{section}{Integrando Funções Simples não Negativas}

Ao longo de toda esta seção consideramos fixado um espaço de medida $(X,\mathcal A,\mu)$.
Recorde que uma função $f:X\to[0,+\infty]$ é simples e mensurável se e somente se
$\Img(f)$ é um subconjunto finito de $[0,+\infty]$ e $f^{-1}(c)\in\mathcal A$ para todo
$c\in\Img(f)$ (vide Definição~\ref{thm:deffuncsimples} e Lema~\ref{thm:condsimplesmens}).

\begin{defin}
Se $f:X\to[0,+\infty]$ é uma função
simples, mensurável e não negativa então a {\em integral\/}\index[indice]{integral!de uma funcao simples nao negativa@de uma função simples não negativa}%
\index[indice]{funcao@função!simples!integral de}
de $f$ é definida por:
\index[simbolos]{$\int_Xf\,\dd\mu$}\[\int_Xf\,\dd\mu=\!\!\!\sum_{c\in\Img(f)}\!\!c\,\mu\big(f^{-1}(c)\big).\]
A integral $\int_Xf\,\dd\mu$ será também às vezes denotada por:
\index[simbolos]{$\int_Xf(x)\,\dd\mu(x)$}\[\int_Xf(x)\,\dd\mu(x).\]
\end{defin}
Obviamente, para toda função simples mensurável $f:X\to[0,+\infty]$,
temos:
\[\int_Xf\,\dd\mu\ge0.\]

Se $Y\in\mathcal A$ é um conjunto mensurável então é fácil ver que a restrição de $\mu$ à $\sigma$-álgebra
$\mathcal A\vert_Y=\mathcal A\cap\wp(Y)$ é também uma medida, de modo que a trinca
$(Y,\mathcal A\vert_Y,\mu\vert_{(\mathcal A\vert_Y)})$ é um espaço de medida. Se $f$
é uma função a valores em $\overline\R$ cujo domínio contém $Y$ e tal que
$f\vert_Y$ é simples, mensurável e não negativa então
a integral de $f\vert_Y$ será denotada por:
\[\int_Yf\,\dd\mu=\int_Yf(x)\,\dd\mu(x).\]
\begin{lem}\label{thm:intsubsimples}
Seja $f:X\to\overline\R$ uma função e seja $Y\in\mathcal A$. Suponha que
$f\vert_Y$ é simples, mensurável e não negativa (pelo Lema~\ref{thm:fYfchiY} isso equivale a dizer
que $f\chilow Y$ é simples, mensurável e não negativa). Então:
\[\int_Yf\,\dd\mu=\int_Xf\chilow Y\,\dd\mu.\]
\end{lem}
\begin{proof}
Temos:
\begin{gather*}
\int_Yf\,\dd\mu=\!\!\!\sum_{c\in f(Y)}\!\!c\,\mu\big((f\vert_Y)^{-1}(c)\big)=\!\!\!
\sum_{\substack{c\in f(Y)\\c\ne0}}\!\!c\,\mu\big((f\vert_Y)^{-1}(c)\big),\\
\int_Xf\chilow Y\,\dd\mu=\!\!\!\!\!\sum_{c\in\Img(f\subchilow Y)}\!\!\!\!\!c\,\mu\big((f\chilow Y)^{-1}(c)\big)
=\!\!\!\!\!\sum_{\substack{c\in\Img(f\subchilow Y)\\c\ne0}}\!\!\!\!\!c\,\mu\big((f\chilow Y)^{-1}(c)\big).
\end{gather*}
A conclusão segue das igualdades acima observando que para todo $c\ne0$, temos
$c\in f(Y)$ se e somente se $c\in\Img(f\chilow Y)$ e, nesse caso:
\[(f\vert_Y)^{-1}(c)=f^{-1}(c)\cap Y=(f\chilow Y)^{-1}(c).\qedhere\]
\end{proof}

\begin{lem}\label{thm:ciAi}
Sejam $A_1,\ldots,A_k\in\mathcal A$ conjuntos dois a dois disjuntos e sejam $c_1,\ldots,c_k\in[0,+\infty]$.
Então:
\begin{equation}\label{eq:teseintsimples}
\int_X\sum_{i=1}^kc_i\chi_{A_i}\,\dd\mu=\sum_{i=1}^kc_i\mu(A_i).
\end{equation}
\end{lem}
\begin{proof}
Eliminando os índices $i$ tais que $c_i=0$ ou $A_i=\emptyset$
não alteramos o resultado de nenhum dos dois lados da igualdade \eqref{eq:teseintsimples};
podemos portanto supor que $c_i\ne0$ e $A_i\ne\emptyset$ para todo $i=1,\ldots,k$.
Seja $f=\sum_{i=1}^kc_i\chi_{A_i}$. Temos $\Img(f)\setminus\{0\}=\{c_1,\ldots,c_k\}$;
note que é possível ter $c_i=c_j$ para $i\ne j$. Para $c\in\Img(f)$, $c\ne0$, temos:
\[f^{-1}(c)=\bigcup_{\substack{i=1\\c_i=c}}^kA_i\]
e portanto:
\[\mu\big(f^{-1}(c)\big)=\sum_{\substack{i=1\\c_i=c}}^k\mu(A_i).\]
Logo:
\begin{multline*}
\int_Xf\,\dd\mu=\!\!\!\sum_{c\in\Img(f)}\!\!c\,\mu\big(f^{-1}(c)\big)
=\!\!\!\sum_{\substack{c\in\Img(f)\\c\ne0}}\!\!c\,\mu\big(f^{-1}(c)\big)
=\!\!\!\sum_{\substack{c\in\Img(f)\\c\ne0}}\sum_{\substack{i=1\\c_i=c}}^kc\mu(A_i)
\\=\!\!\!\sum_{\substack{c\in\Img(f)\\c\ne0}}\sum_{\substack{i=1\\c_i=c}}^kc_i\mu(A_i)
=\sum_{i=1}^kc_i\mu(A_i),
\end{multline*}
onde na última igualdade usamos o fato que o conjunto $\{1,\ldots,k\}$ é união disjunta
dos conjuntos $\big\{i\in\{1,\ldots,k\}:c_i=c\big\}$, com $c\in\Img(f)$, $c\ne0$.
\end{proof}

\begin{lem}\label{thm:intsomasimples}
Sejam $f:X\to[0,+\infty]$, $g:X\to[0,+\infty]$ funções simples e mensuráveis. Então:
\[\int_X(f+g)\,\dd\mu=\int_Xf\,\dd\mu+\int_Xg\,\dd\mu.\]
\end{lem}
\begin{proof}
Podemos escrever:
\[f=\sum_{i=1}^kc_i\chi_{A_i},\quad g=\sum_{j=1}^ld_j\chi_{B_j},\]
onde tanto os conjuntos $A_1,\ldots,A_k\in\mathcal A$ como os conjuntos $B_1,\ldots,B_l\in\mathcal A$
constituem uma partição de $X$ (veja Observação~\ref{thm:funcsimples}).
Temos:
\[\sum_{j=1}^l\chi_{B_j}=1\]
e portanto:
\[\chi_{A_i}=\sum_{j=1}^l\chi_{A_i}\chi_{B_j}=\sum_{j=1}^l\chi_{A_i\cap B_j},\]
para todo $i=1,\ldots,k$; daí:
\begin{equation}\label{eq:fquebrada}
f=\sum_{i=1}^k\sum_{j=1}^lc_i\chi_{A_i\cap B_j}.
\end{equation}
Como os conjuntos $A_i\cap B_j$, $i=1,\ldots,k$, $j=1,\ldots,l$ são dois a dois disjuntos,
o Lema~\ref{thm:ciAi} nos dá:
\begin{equation}\label{eq:intfquebrada}
\int_Xf\,\dd\mu=\sum_{i=1}^k\sum_{j=1}^lc_i\mu(A_i\cap B_j).
\end{equation}
Analogamente, mostra-se que:
\begin{equation}\label{eq:gquebrada}
g=\sum_{j=1}^l\sum_{i=1}^kd_j\chi_{B_j\cap A_i}
\end{equation}
e portanto:
\begin{equation}\label{eq:intgquebrada}
\int_Xg\,\dd\mu=\sum_{j=1}^l\sum_{i=1}^kd_j\mu(B_j\cap A_i).
\end{equation}
De \eqref{eq:fquebrada} e \eqref{eq:gquebrada} obtemos:
\[f+g=\sum_{i=1}^k\sum_{j=1}^l(c_i+d_j)\chi_{A_i\cap B_j};\]
novamente, o Lema~\ref{thm:ciAi} nos dá:
\begin{equation}\label{eq:intfgquebrada}
\int_X(f+g)\,\dd\mu=\sum_{i=1}^k\sum_{j=1}^l(c_i+d_j)\mu(A_i\cap B_j).
\end{equation}
A conclusão segue de \eqref{eq:intfquebrada}, \eqref{eq:intgquebrada} e \eqref{eq:intfgquebrada}.
\end{proof}

\begin{cor}
Dados $A_1,\ldots,A_k\in\mathcal A$ (conjuntos não necessariamente disjuntos)
e $c_1,\ldots,c_k\in[0,+\infty]$ então:
\[\int_X\sum_{i=1}^kc_i\chi_{A_i}\,\dd\mu=\sum_{i=1}^kc_i\mu(A_i).\]
\end{cor}
\begin{proof}
Basta observar que:
\[\int_X\sum_{i=1}^kc_i\chi_{A_i}\,\dd\mu=\sum_{i=1}^k\int_Xc_i\chi_{A_i}\,\dd\mu
=\sum_{i=1}^kc_i\mu(A_i).\qedhere\]
\end{proof}

\begin{notation}
Se $f:X\to\overline\R$, $g:X\to\overline\R$ são funções então escrevemos $f\le g$\index[simbolos]{$f\le g$} quando
$f(x)\le g(x)$, para todo $x\in X$.
\end{notation}

\begin{cor}\label{thm:simplesfmenorg}
Sejam $f:X\to[0,+\infty]$, $g:X\to[0,+\infty]$ funções simples mensuráveis. Se
$f\le g$ então:
\[\int_Xf\,\dd\mu\le\int_Xg\,\dd\mu.\]
\end{cor}
\begin{proof}
Defina $h:X\to[0,+\infty]$ fazendo:
\[h(x)=\begin{cases}
g(x)-f(x),&\text{se $x\in f^{-1}\big(\left[0,+\infty\right[\big)$},\\
\hfil0,&\text{se $x\in f^{-1}(+\infty)$},
\end{cases}\]
para todo $x\in X$. Temos $g=f+h$. A função $h$ é mensurável, pelo Lema~\ref{thm:cobremensf}
e pela Proposição~\ref{thm:somaprodmensRest}. Além do mais, a função $h$ é simples já que sua
imagem está contida no conjunto finito:
\[\{0\}\cup\big\{a-b:\text{$a\in\Img(g)$, $b\in\Img(f)$ e $b<+\infty$}\big\}.\]
Segue então do Lema~\ref{thm:intsomasimples} que:
\[\int_Xg\,\dd\mu=\int_Xf\,\dd\mu+\int_Xh\,\dd\mu\ge\int_Xf\,\dd\mu,\]
já que $\int_Xh\,\dd\mu\ge0$.
\end{proof}

\begin{lem}\label{thm:cintsimples}
Sejam $f:X\to[0,+\infty]$ uma função simples mensurável e $c\in[0,+\infty]$. Então:
\[\int_Xcf\,\dd\mu=c\int_Xf\,\dd\mu.\]
\end{lem}
\begin{proof}
Escreva:
\[f=\sum_{i=1}^kc_i\chi_{A_i},\]
onde os conjuntos $A_1,\ldots,A_k\in\mathcal A$ constituem uma partição de $X$. Daí:
\[cf=\sum_{i=1}^kcc_i\chi_{A_i}.\]
O Lema~\ref{thm:ciAi} nos dá então:
\[\int_Xcf\,\dd\mu=\sum_{i=1}^kcc_i\mu(A_i)=c\sum_{i=1}^kc_i\mu(A_i)=c\int_Xf\,\dd\mu.\qedhere\]
\end{proof}

\end{section}

\begin{section}{Integrando Funções Mensuráveis não Negativas}
\label{sec:intmensnneg}

Ao longo de toda esta seção consideramos fixado um espaço de medida $(X,\mathcal A,\mu)$.
Dada uma função mensurável não negativa $f:X\to[0,+\infty]$ consideramos o conjunto:
\begin{multline}\label{eq:calIf}
\mathcal I(f)=\Big\{\int_X\phi\,\dd\mu:
\text{$\phi:X\to[0,+\infty]$ é função simples mensurável}\\
\text{tal que $\phi\le f$}\Big\}\subset[0,+\infty].
\end{multline}\index[simbolos]{$\mathcal I(f)$}
Observe que o conjunto $\mathcal I(f)$ não é vazio, já que a função $\phi\equiv0$
é simples, mensurável, não negativa e menor ou igual a $f$, de modo que $0\in\mathcal I(f)$.
Afirmamos que se $f:X\to[0,+\infty]$ é uma função simples mensurável então:
\[\int_Xf\,\dd\mu=\sup\mathcal I(f).\]
De fato, nesse caso $f$ é uma função simples, mensurável, não negativa e menor ou igual a $f$,
de modo que $\int_Xf\,\dd\mu\in\mathcal I(f)$ e $\sup\mathcal I(f)\ge\int_Xf\,\dd\mu$.
Por outro lado, o Corolário~\ref{thm:simplesfmenorg} implica que $\int_X\phi\,\dd\mu\le\int_Xf\,\dd\mu$
para toda função simples mensurável $\phi:X\to[0,+\infty]$ tal que $\phi\le f$; portanto
$\int_Xf\,\dd\mu$ é uma cota superior de $\mathcal I(f)$ e $\sup\mathcal I(f)\le\int_Xf\,\dd\mu$.

Em vista das considerações acima podemos introduzir a seguinte:
\begin{defin}
Se $f:X\to[0,+\infty]$ é uma função mensurável não negativa então a
{\em integral\/}\index[indice]{integral!de uma funcao mensuravel nao negativa@de uma função mensurável não negativa}%
\index[indice]{funcao@função!mensuravel@mensurável!integral de}
de $f$ é definida por:
\index[simbolos]{$\int_Xf\,\dd\mu$}\[\int_Xf\,\dd\mu=\sup\mathcal I(f)\in[0,+\infty],\]
onde $\mathcal I(f)$ é o conjunto definido em \eqref{eq:calIf}.
\end{defin}
Como no caso de funções simples, a integral $\int_Xf\,\dd\mu$ será também às vezes denotada por:
\index[simbolos]{$\int_Xf(x)\,\dd\mu(x)$}\[\int_Xf(x)\,\dd\mu(x).\]
Além do mais, se $Y\in\mathcal A$ e se $f$ é uma função a valores em $\overline\R$
cujo domínio contém $Y$ e tal que
$f\vert_Y$ é mensurável e não negativa então a integral de $f\vert_Y$ com respeito à medida
$\mu\vert_{(\mathcal A\vert_Y)}$ será denotada por:
\[\int_Yf\,\dd\mu=\int_Yf(x)\,\dd\mu(x).\]

\begin{lem}\label{thm:intmensnnegcresce}
Sejam $f:X\to[0,+\infty]$, $g:X\to[0,+\infty]$ funções mensuráveis. Se $f\le g$ então:
\[\int_Xf\,\dd\mu\le\int_Xg\,\dd\mu.\]
\end{lem}
\begin{proof}
Se $\phi:X\to[0,+\infty]$ é uma função simples mensurável tal que $\phi\le f$ então também $\phi\le g$;
isso implica que $\mathcal I(f)\subset\mathcal I(g)$ e portanto $\sup\mathcal I(f)\le\sup\mathcal I(g)$.
\end{proof}

\begin{teo}[da convergência monotônica]\label{thm:monotonicanneg}
\index[indice]{teorema!da convergencia monotonica@da convergência monotô-\hfil\break nica}\index[indice]{convergencia monotonica@convergência monotônica!teorema da}%
Seja $(f_n)_{n\ge1}$ uma se\-qüên\-cia de funções mensuráveis não negativas $f_n:X\to[0,+\infty]$.
Se $f_n\nearrow f$ então $f:X\to[0,+\infty]$ é mensurável e:
\[\int_Xf\,\dd\mu=\lim_{n\to\infty}\int_Xf_n\,\dd\mu.\]
\end{teo}
\begin{proof}
A mensurabilidade de $f$ segue do Corolário~\ref{thm:limmensmens}. O Lema~\ref{thm:intmensnnegcresce}
implica que $\big(\int_Xf_n\,\dd\mu\big)_{n\ge1}$ é uma seqüência crescente e que:
\[\lim_{n\to\infty}\int_Xf_n\,\dd\mu\le\int_Xf\,\dd\mu.\]
Para mostrar a desigualdade oposta, é suficiente verificar que:
\begin{equation}\label{eq:limintfnge}
\lim_{n\to\infty}\int_Xf_n\,\dd\mu\ge\int_X\phi\,\dd\mu,
\end{equation}
para toda função simples mensurável $\phi:X\to[0,+\infty]$ tal que $\phi\le f$.
Escreva $\phi=\sum_{i=1}^kc_i\chi_{A_i}$, com $c_1,\ldots,c_k\in\left]0,+\infty\right]$
e $A_1,\ldots,A_k\in\mathcal A$ dois a dois disjuntos e não vazios. Fixados $c'_1,\ldots,c'_k>0$
com $c'_i<c_i$, $i=1,\ldots,k$, definimos:
\[A_i^n=\big\{x\in A_i:f_n(x)\ge c'_i\big\}=f_n^{-1}\big([c'_i,+\infty]\big)\cap A_i\in\mathcal A,\]
para todo $n\ge1$. Para $n\ge1$ fixado, os conjuntos $A_i^n$, $i=1,\ldots,k$ são dois a dois
disjuntos e:
\[f_n\ge\sum_{i=1}^kc'_i\chi_{A_i^n};\]
os Lemas~\ref{thm:intmensnnegcresce} e \ref{thm:ciAi} nos dão então:
\begin{equation}\label{eq:intfnge}
\int_Xf_n\,\dd\mu\ge\sum_{i=1}^kc'_i\mu(A_i^n).
\end{equation}
Note que para todo $x\in A_i$ temos $f(x)\ge\phi(x)=c_i>c'_i$ e portanto,
como $f_n\nearrow f$, temos que $A_i^n\nearrow A_i$. O Lema~\ref{thm:setlimits}
nos dá então:
\[\lim_{n\to\infty}\mu(A_i^n)=\mu(A_i);\]
fazendo $n\to\infty$ em \eqref{eq:intfnge} obtemos (veja Exercício~\ref{exe:prodcresseq}):
\begin{equation}\label{eq:limintfngequase}
\lim_{n\to\infty}\int_Xf_n\,\dd\mu\ge\sum_{i=1}^kc'_i\mu(A_i).
\end{equation}
Como a desigualdade \eqref{eq:limintfngequase} vale para quaisquer $c'_i\in\left]0,c_i\right[$,
temos:
\begin{equation}\label{eq:limintfngequasem}
\lim_{n\to\infty}\int_Xf_n\,\dd\mu\ge\sum_{i=1}^kc'_{i,m}\mu(A_i),
\end{equation}
para todo $m\ge1$, onde $(c'_{i,m})_{m\ge1}$ é uma seqüência crescente (arbitrariamente escolhida) em $\left]0,c_i\right[$
que converge para $c_i$. Fazendo $m\to\infty$ em \eqref{eq:limintfngequasem} obtemos:
\[\lim_{n\to\infty}\int_Xf_n\,\dd\mu\ge\sum_{i=1}^kc_i\mu(A_i)=\int_X\phi\,\dd\mu,\]
o que prova \eqref{eq:limintfnge} e completa a demonstração.
\end{proof}

\begin{lem}\label{thm:vertYchiYnneg}
Seja $f:X\to\overline\R$ uma função e seja $Y\in\mathcal A$. Suponha que $f\vert_Y$
é mensurável e não negativa (pelo Lema~\ref{thm:fYfchiY} isso equivale a dizer que $f\chilow Y$ é mensurável
e não negativa). Então:
\[\int_Yf\,\dd\mu=\int_Xf\chilow Y\,\dd\mu.\]
\end{lem}
\begin{proof}
Pela Proposição~\ref{thm:aproxmonotsimples} existe uma seqüência $(f_n)_{n\ge1}$
de funções simples mensuráveis $f_n:X\to\left[0,+\infty\right[$ tal que $f_n\nearrow f\chilow Y$.
Como cada $f_n$ é simples o Lema~\ref{thm:intsubsimples} nos dá:
\[\int_Yf_n\,\dd\mu=\int_Xf_n\chilow Y\,\dd\mu,\]
para todo $n\ge1$. Obviamente $f_n\vert_Y\nearrow f\vert_Y$ e $(f_n\chilow Y)\nearrow(f\chilow Y)$.
A conclusão segue portanto do Teorema~\ref{thm:monotonicanneg} fazendo $n\to\infty$ na
igualdade acima.
\end{proof}

\begin{cor}\label{thm:cornoYdiminui}
Se $f:X\to[0,+\infty]$ é uma função mensurável então:
\[\int_Yf\,\dd\mu\le\int_Xf\,\dd\mu,\]
para todo $Y\in\mathcal A$.
\end{cor}
\begin{proof}
Temos:
\[\int_Yf\,\dd\mu=\int_Xf\chilow Y\,\dd\mu\le\int_Xf\,\dd\mu,\]
onde na última desigualdade usamos o Lema~\ref{thm:intmensnnegcresce}.
\end{proof}

\begin{lem}\label{thm:intlinearnneg}
Sejam $f:X\to[0,+\infty]$, $g:X\to[0,+\infty]$ funções mensuráveis. Então:
\[\int_X(f+g)\,\dd\mu=\int_Xf\,\dd\mu+\int_Xg\,\dd\mu,\quad\int_Xcf\,\dd\mu=c\int_Xf\,\dd\mu,\]
para qualquer $c\in[0,+\infty]$.
\end{lem}
\begin{proof}
Pela Proposição~\ref{thm:aproxmonotsimples} existem seqüências $(f_n)_{n\ge1}$, $(g_n)_{n\ge1}$
de funções simples mensuráveis $f_n:X\to\left[0,+\infty\right[$, $g_n:X\to\left[0,+\infty\right[$
tais que $f_n\nearrow f$ e $g_n\nearrow g$. Como as funções $f_n$ e $g_n$ são simples, os Lemas~\ref{thm:intsomasimples}
e \ref{thm:cintsimples} nos dão:
\[\int_X(f_n+g_n)\,\dd\mu=\int_Xf_n\,\dd\mu+\int_Xg_n\,\dd\mu,\quad\int_Xcf_n\,\dd\mu=c\int_Xf_n\,\dd\mu.\]
Temos $(f_n+g_n)\nearrow(f+g)$ e $(cf_n)\nearrow(cf)$ (veja Lema~\ref{thm:oplimbarR}
e Exercício~\ref{exe:prodcresseq}). A conclusão segue portanto
do Teorema~\ref{thm:monotonicanneg} fazendo $n\to\infty$ nas igualdades acima.
\end{proof}

\end{section}

\begin{section}{Definição da Integral: o Caso Geral}

Ao longo de toda esta seção consideramos fixado um espaço de medida $(X,\mathcal A,\mu)$.
Dada uma função mensurável arbitrária $f:X\to\overline\R$ então, como vimos no Lema~\ref{thm:fmaismenosmens},
temos $f=f^+-f^-$, onde a parte positiva $f^+$ e a parte negativa $f^-$ de $f$ são funções mensuráveis
não negativas definidas em $X$. Obviamente, se $f$ já é não negativa então $f^+=f$ e $f^-=0$,
de modo que $\int_Xf\,\dd\mu=\int_Xf^+\,\dd\mu-\int_Xf^-\,\dd\mu$. Em vista dessa observação,
introduzimos a seguinte:
\begin{defin}
Diremos que uma função $f:X\to\overline\R$ é
{\em quase integrável\/}\index[indice]{funcao@função!quase integravel@quase integrável}\index[indice]{quase integravel@quase integrável!funcao@função}
quando $f$ for mensurável e a diferença $\int_Xf^+\,\dd\mu-\int_Xf^-\,\dd\mu$
estiver bem-definida, ou seja, quando $\int_Xf^+\,\dd\mu<+\infty$ ou $\int_Xf^-\,\dd\mu<+\infty$;
nesse caso, definimos a
{\em integral\/}\index[indice]{integral!de uma funcao mensuravel@de uma função mensurável}%
\index[indice]{funcao@função!mensuravel@mensurável!integral de}
de $f$ fazendo:\index[simbolos]{$\int_Xf\,\dd\mu$}
\[\int_Xf\,\dd\mu=\int_Xf^+\,\dd\mu-\int_Xf^-\,\dd\mu\in\overline\R.\]
Quando $f$ é quase integrável e $\int_Xf\,\dd\mu\in\R$ (ou seja, se $\int_Xf^+\,\dd\mu<+\infty$
{\em e\/} $\int_Xf^-\,\dd\mu<+\infty$) então dizemos que a função $f$
é {\em integrável}\index[indice]{funcao@função!integravel@integrável}\index[indice]{integravel@integrável!funcao@função}.
\end{defin}

Como na Seção~\ref{sec:intmensnneg}, introduzimos também a notação alternativa:
\index[simbolos]{$\int_Xf(x)\,\dd\mu(x)$}\[\int_Xf(x)\,\dd\mu(x),\]
para a integral de $f$. Também, se $Y\in\mathcal A$ e se $f$ é uma função a valores em $\overline\R$
cujo domínio contém $Y$ então dizemos que $f$ é
{\em quase integrável em $Y$\/}\index[indice]{funcao@função!quase integravel@quase integrável!num subespaco@num subespaço}%
\index[indice]{quase integravel@quase integrável!funcao@função!num subespaco@num subespaço}
(resp., {\em integrável em $Y$}\index[indice]{integral!de uma funcao mensuravel@de uma função mensurável!num subespaco@num subespaço}%
\index[indice]{funcao@função!mensuravel@mensurável!integral num subespaco@integral num subespaço})
se a função $f\vert_Y$ for quase integrável (resp., integrável)
com respeito à medida $\mu\vert_{(\mathcal A\vert_Y)}$; quando $f$ for quase integrável em $Y$,
a integral de $f\vert_Y$ com respeito à medida $\mu\vert_{(\mathcal A\vert_Y)}$ será denotada por:
\[\int_Yf\,\dd\mu=\int_Yf(x)\,\dd\mu(x).\]

\begin{convention}
Seja $X\in\Lebmens(\R^n)$ um subconjunto Lebesgue mensurável de $\R^n$ e seja
$f:X\to\overline\R$ uma função mensurável; como sempre (recorde Convenções~\ref{cnv:menscontrRn} e
\ref{cnv:subespacomens}) assumimos que $X$ é munido da $\sigma$-álgebra $\Lebmens(\R^n)\vert_X$
constituída pelos subconjuntos Lebesgue mensuráveis de $\R^n$ que estão contidos em $X$.
Nesse contexto, a menos de menção explícita em contrário, quando usamos os adjetivos {\em quase integrável\/}
e {\em integrável\/}, subentendemos que a $\sigma$-álgebra $\Lebmens(\R^n)\vert_X$
é munida da (restrição da) medida de Lebesgue $\leb:\Lebmens(\R^n)\to[0,+\infty]$.
Quando for necessário enfatizar essa convenção, diremos também que $f$ é {\em Lebesgue
quase integrável\/}\index[indice]{quase integravel@quase integrável!Lebesgue}\index[indice]{Lebesgue!quase integravel@quase integrável} ou
{\em Lebesgue integrável}\index[indice]{integravel@integrável!Lebesgue}\index[indice]{Lebesgue!integravel@integrável},
dependendo do caso.
\end{convention}

\begin{defin}
Se $X\in\Lebmens(\R^n)$ é um subconjunto Lebesgue mensurável de $\R^n$ e se
$f:X\to\overline\R$ é uma função quase integrável então a integral de $f$ com respeito
à (restrição à $\Lebmens(\R^n)\vert_X$) da medida de Lebesgue $\leb$ será chamada a
{\em integral de Lebesgue\/}\index[indice]{integral!de Lebesgue}\index[indice]{Lebesgue!integral de}
de $f$ e será denotada (seguindo as notações anteriormente introduzidas) por $\int_Xf\,\dd\leb$ ou
por $\int_Xf(x)\,\dd\leb(x)$.
\end{defin}

\begin{notation}
Seja $f:I\to\overline\R$ uma função definida num intervalo $I\subset\R$. Dados
$a,b\in I$ com $a\le b$ então, se $f$ for quase integrável no intervalo $[a,b]$, denotamos
por:\index[simbolos]{$\int_a^bf\,\dd\leb$}\index[simbolos]{$\int_a^bf(x)\,\dd\leb(x)$}
\[\int_a^bf\,\dd\leb=\int_a^bf(x)\,\dd\leb(x)\]
a integral de Lebesgue de $f\vert_{[a,b]}$. Se $b<a$ e se $f$ é quase integrável
em $[b,a]$ então escrevemos:
\[\int_a^bf\,\dd\leb=\int_a^bf(x)\,\dd\leb(x)\stackrel{\text{def}}=-\int_b^af.\]
Se $a\in I$, $I$ é ilimitado à direita e $f$ é quase integrável em $\left[a,+\infty\right[$
então denotamos por:\index[simbolos]{$\int_a^{+\infty}f\,\dd\leb$}\index[simbolos]{$\int_a^{+\infty}f(x)\,\dd\leb(x)$}
\[\int_a^{+\infty}f\,\dd\leb=\int_a^{+\infty}f(x)\,\dd\leb(x)\]
a integral de Lebesgue de $f\vert_{\left[a,+\infty\right[}$; escrevemos também:
\[\int_{+\infty}^af\,\dd\leb=\int_{+\infty}^af(x)\,\dd\leb(x)\stackrel{\text{def}}=
-\int_a^{+\infty}f\,\dd\leb.\]
Similarmente, se $a\in I$, $I$ é ilimitado à esquerda e $f$ é quase integrável em $\left]-\infty,a\right]$
então denotamos por:
\[\int_{-\infty}^af\,\dd\leb=\int_{-\infty}^af(x)\,\dd\leb(x)\]
a integral de Lebesgue de $f\vert_{\left]-\infty,a\right]}$; escrevemos também:
\[\int_a^{-\infty}f\,\dd\leb=\int_a^{-\infty}f(x)\,\dd\leb(x)\stackrel{\text{def}}=
-\int_{-\infty}^af\,\dd\leb.\]
\end{notation}
Claramente a restrição de $f$ ao intervalo degenerado $[a,a]=\{a\}$ é uma função
simples integrável e:
\[\int_a^af\,\dd\leb=f^+(a)\leb\big(\{a\}\big)-f^-(a)\leb\big(\{a\}\big)=0.\]

\begin{lem}\label{thm:fYfchiYgeral}
Seja $f:X\to\overline\R$ uma função e seja $Y\in\mathcal A$. Então
$f\vert_Y$ é quase integrável se e somente se $f\chilow Y$ é quase integrável; nesse caso:
\[\int_Yf\,\dd\mu=\int_Xf\chilow Y\,\dd\mu.\]
\end{lem}
\begin{proof}
Pelo Lema~\ref{thm:fYfchiY}, temos que $f\vert_Y$ é mensurável se e somente se
$f\chilow Y$ é mensurável. Além do mais, temos:
\begin{gather*}
(f\vert_Y)^+=f^+\vert_Y,\quad(f\vert_Y)^-=f^-\vert_Y,\\
(f\chilow Y)^+=f^+\chilow Y,\quad(f\chilow Y)^-=f^-\chilow Y.
\end{gather*}
A conclusão segue então das igualdades acima e do Lema~\ref{thm:vertYchiYnneg}.
\end{proof}

\begin{lem}\label{thm:fmaismenosdecompminima}
Sejam $f_1:X\to[0,+\infty]$, $f_2:X\to[0,+\infty]$ funções mensuráveis não negativas tais que
a diferença $f=f_1-f_2$ esteja bem-definida (i.e., não existe $x\in X$ com $f_1(x)=f_2(x)=+\infty$).
Então existe uma função mensurável não negativa $h:X\to[0,+\infty]$ tal que
$f_1=f^++h$ e $f_2=f^-+h$.
\end{lem}
\begin{proof}
Observe em primeiro lugar que $f^+\le f_1$. De fato, se $f(x)\ge0$ então
$f^+(x)=f(x)=f_1(x)-f_2(x)\le f_1(x)$ e se $f(x)<0$ então $f^+(x)=0\le f_1(x)$. Definimos $h$ fazendo:
\[h(x)=\begin{cases}
f_1(x)-f^+(x),&\text{se $x\in f^{-1}(\R)$},\\
\hfil f_2(x),&\text{se $x\in f^{-1}(+\infty)$},\\
\hfil f_1(x),&\text{se $x\in f^{-1}(-\infty)$}.
\end{cases}\]
Claramente $h$ é não negativa; a mensurabilidade de $h$ segue do Lema~\ref{thm:cobremensf}
e da Proposição~\ref{thm:somaprodmensRest}. Verifiquemos que $f_1=f^++h$ e $f_2=f^-+h$.
Para $x\in f^{-1}(\R)$, temos:
\begin{gather*}
f^+(x)+h(x)=f^+(x)+f_1(x)-f^+(x)=f_1(x),\\
f^-(x)+h(x)=f^-(x)+f_1(x)-f^+(x)=f_1(x)-f(x)=f_2(x).
\end{gather*}
Se $x\in f^{-1}(+\infty)$ então:
\[f^+(x)+h(x)=+\infty=f_1(x),\quad f^-(x)+h(x)=h(x)=f_2(x);\]
finalmente, se $x\in f^{-1}(-\infty)$:
\[f^+(x)+h(x)=h(x)=f_1(x),\quad f^-(x)+h(x)=+\infty=f_2(x).\qedhere\]
\end{proof}

\begin{prop}\label{thm:propintlinear}
Sejam $f:X\to\overline\R$, $g:X\to\overline\R$ funções quase integráveis e seja $c\in\R$.
\begin{itemize}
\item[(a)] Se as somas $\int_Xf\,\dd\mu+\int_Xg\,\dd\mu$ e $f+g$ estiverem bem-definidas então
a função $f+g$ é quase integrável e $\int_Xf+g\,\dd\mu=\int_Xf\,\dd\mu+\int_Xg\,\dd\mu$.
\item[(b)] A função $cf$ é quase integrável e $\int_Xcf\,\dd\mu=c\int_Xf\,\dd\mu$.
\end{itemize}
\end{prop}
\begin{proof}
Temos:
\[f+g=(f^+-f^-)+(g^+-g^-)=(f^++g^+)-(f^-+g^-);\]
pelo Lema~\ref{thm:fmaismenosdecompminima} existe uma função mensurável $h:X\to[0,+\infty]$
tal que:
\[f^++g^+=(f+g)^++h,\quad f^-+g^-=(f+g)^-+h.\]
O Lema~\ref{thm:intlinearnneg} nos dá:
\begin{gather}
\label{eq:fg1}\int_Xf^+\,\dd\mu+\int_Xg^+\,\dd\mu=\int_X(f+g)^+\,\dd\mu+\int_Xh\,\dd\mu,\\
\label{eq:fg2}\int_Xf^-\,\dd\mu+\int_Xg^-\,\dd\mu=\int_X(f+g)^-\,\dd\mu+\int_Xh\,\dd\mu.
\end{gather}
Por definição temos:
\[\int_Xf\,\dd\mu=\int_Xf^+\,\dd\mu-\int_Xf^-\,\dd\mu,\quad
\int_Xg\,\dd\mu=\int_Xg^+\,\dd\mu-\int_Xg^-\,\dd\mu.\]
A quase integrabilidade das funções $f$ e $g$ juntamente com o fato que a soma
$\int_Xf\,\dd\mu+\int_Xg\,\dd\mu$ está bem definida
implicam que o lado esquerdo de pelo menos uma das igualdades \eqref{eq:fg1} e \eqref{eq:fg2}
é finito. Isso implica que a integral de $h$ é finita e que pelo menos uma das integrais
$\int_X(f+g)^+\,\dd\mu$, $\int_X(f+g)^-\,\dd\mu$ é finita, i.e., $f+g$ é quase integrável.
A demonstração do item (a) é obtida então subtraindo a igualdade \eqref{eq:fg2} da igualdade \eqref{eq:fg1}.

Para demonstrar o item (b), consideramos primeiramente o caso que $c\ge0$. Nesse caso,
usando o Lema~\ref{thm:intlinearnneg}, temos:
\begin{gather*}
\int_X(cf)^+\,\dd\mu=\int_Xcf^+\,\dd\mu=c\int_Xf^+\,\dd\mu,\\
\int_X(cf)^-\,\dd\mu=\int_Xcf^-\,\dd\mu=c\int_Xf^-\,\dd\mu.
\end{gather*}
Isso mostra que $cf$ é quase integrável e $\int_Xcf\,\dd\mu=c\int_Xf\,\dd\mu$. Se $c<0$ temos:
\begin{gather*}
\int_X(cf)^+\,\dd\mu=\int_X(-c)f^-\,\dd\mu=(-c)\int_Xf^-\,\dd\mu,\\
\int_X(cf)^-\,\dd\mu=\int_X(-c)f^+\,\dd\mu=(-c)\int_Xf^+\,\dd\mu,
\end{gather*}
o que completa a demonstração do item (b).
\end{proof}

\begin{lem}
Sejam $f:X\to\overline\R$, $g:X\to\overline\R$ funções quase integráveis. Se $f\le g$
então $\int_Xf\,\dd\mu\le\int_Xg\,\dd\mu$.
\end{lem}
\begin{proof}
Verifica-se facilmente que $f^+\le g^+$ e $f^-\ge g^-$, donde, pelo Lema~\ref{thm:intmensnnegcresce}:
\[\int_Xf^+\,\dd\mu\le\int_Xg^+\,\dd\mu,\quad\int_Xf^-\,\dd\mu\ge\int_Xg^-\,\dd\mu.\]
A conclusão é obtida subtraindo as duas desigualdades acima.
\end{proof}

\begin{lem}\label{thm:pedacosXint}
Dada uma função $f:X\to\overline\R$, temos:
\begin{itemize}
\item[(a)] se $f$ é quase integrável então $f\vert_Y$ também é quase integrável
para todo $Y\in\mathcal A$;
\item[(b)] se $X_1,\ldots,X_k\in\mathcal A$ são conjuntos dois a dois disjuntos
tais que $X=\bigcup_{i=1}^kX_i$, $f\vert_{X_i}$ é quase integrável para $i=1,\ldots,k$
e tais que a soma:
\begin{equation}\label{eq:somaintXi}
\int_{X_1}f\,\dd\mu+\int_{X_2}f\,\dd\mu+\cdots+\int_{X_k}f\,\dd\mu
\end{equation}
está bem definida então $f$ é quase integrável e $\int_Xf\,\dd\mu$ é igual à soma \eqref{eq:somaintXi}.
\end{itemize}
\end{lem}
\begin{proof}
Pelos Corolário~\ref{thm:cornoYdiminui} temos:
\[\int_Yf^+\,\dd\mu\le\int_Xf^+\,\dd\mu,\quad
\int_Yf^-\,\dd\mu\le\int_Xf^-\,\dd\mu,\]
o que prova o item (a). Passemos à prova do item (b). Temos:
\[f=f\chilow{X_1}+f\chilow{X_2}+\cdots+f\chilow{X_k}.\]
Pelo Lema~\ref{thm:fYfchiYgeral}, as funções $f\chilow{X_i}$ são quase integráveis e:
\[\int_{X_i}f\,\dd\mu=\int_Xf\chilow{X_i}\,\dd\mu,\]
para $i=1,\ldots,k$. A conclusão segue da Proposição~\ref{thm:propintlinear}.
\end{proof}

\begin{lem}\label{thm:muXzero}
Se $\mu(X)=0$ então $\int_Xf\,\dd\mu=0$ para toda função mensurável $f:X\to\overline\R$.
\end{lem}
\begin{proof}
Se $\phi:X\to[0,+\infty]$ é uma função simples mensurável então $\int_X\phi\,\dd\mu=0$,
já que $\mu\big(\phi^{-1}(c)\big)=0$, para todo $c\in\Img(\phi)$. Daí, se $f$ é não negativa
então $\int_Xf\,\dd\mu=0$, já que $\int_X\phi\,\dd\mu=0$ para toda função simples mensurável
não negativa $\phi\le f$. Finalmente, se $f:X\to\overline\R$ é uma função mensurável arbitrária
então $\int_Xf^+\,\dd\mu=\int_Xf^-\,\dd\mu=0$ e portanto $\int_Xf\,\dd\mu=0$.
\end{proof}

\begin{cor}\label{thm:corXlinhazero}
Se $X'\in\mathcal A$ é tal que $\mu(X\setminus X')=0$ então uma função mensurável
$f:X\to\overline\R$ é quase integrável se e somente se $f\vert_{X'}$ é quase integrável
e nesse caso $\int_Xf\,\dd\mu=\int_{X'}f\,\dd\mu$.
\end{cor}
\begin{proof}
Pelo Lema~\ref{thm:muXzero} temos $\int_{X\setminus X'}f\,\dd\mu=0$. A conclusão
segue do Lema~\ref{thm:pedacosXint}, já que:
\[\int_Xf\,\dd\mu=\int_{X'}f\,\dd\mu+\int_{X\setminus X'}f\,\dd\mu.\qedhere\]
\end{proof}

A seguinte terminologia é extremamente conveniente:
\begin{defin}
Dizemos que uma propriedade $\mathbb P$ referente a pontos do espaço de medida $X$
é válida {\em quase sempre\/}\index[indice]{quase sempre} (ou em {\em quase todo ponto\/} de $X$) se existe um conjunto
$X'\in\mathcal A$ tal que $\mu(X\setminus X')=0$ e tal que a propriedade $\mathbb P$
é válida em todos os pontos de $X'$. Dizemos também que a propriedade $\mathbb P$
é satisfeita \qs\index[indice]{qs@\qs} (ou $\mu$-\qs\index[indice]{mu qs@$\mu$-\qs}).
\end{defin}

\begin{cor}\label{thm:fquaseiggintr}
Sejam $f:X\to\overline\R$, $g:X\to\overline\R$ funções mensuráveis.
Se $f=g$ quase sempre então $f$ é quase integrável se e somente se $g$ é quase integrável
e, nesse caso, $\int_Xf\,\dd\mu=\int_Xg\,\dd\mu$.
\end{cor}
\begin{proof}
Por hipótese existe $X'\in\mathcal A$ tal que $\mu(X\setminus X')=0$ e $f\vert_{X'}=g\vert_{X'}$.
A conclusão segue do Corolário~\ref{thm:corXlinhazero}, já que:
\[\int_Xf\,\dd\mu=\int_{X'}f\,\dd\mu=\int_{X'}g\,\dd\mu=\int_Xg\,\dd\mu.\qedhere\]
\end{proof}

\end{section}

\begin{section}{Teoremas de Convergência}

No que segue, $(X,\mathcal A,\mu)$ denota sempre um espaço de medida.

\begin{teo}[da convergência monotônica]\label{thm:teomonotonicageral}
\index[indice]{teorema!da convergencia monotonica@da convergência monotô-\hfil\break nica}\index[indice]{convergencia monotonica@convergência monotônica!teorema da}%
Seja $(f_n)_{n\ge1}$ uma se\-qüên\-cia de funções mensuráveis $f_n:X\to\overline\R$ e seja $f:X\to\overline\R$ uma função mensurável.
Suponha que $f_1$ é quase integrável. Então:
\begin{itemize}
\item[(a)] se $\int_Xf_1\,\dd\mu>-\infty$ e $f_n\nearrow f$ \qs\ então $f$ e $f_n$ são quase integráveis para todo $n\ge1$
e $\lim_{n\to\infty}\int_Xf_n\,\dd\mu=\int_Xf\,\dd\mu$;
\item[(b)] se $\int_Xf_1\,\dd\mu<+\infty$ e $f_n\searrow f$ \qs\ então $f$ e $f_n$ são quase integráveis para todo $n\ge1$
e $\lim_{n\to\infty}\int_Xf_n\,\dd\mu=\int_Xf\,\dd\mu$.
\end{itemize}
\end{teo}
\begin{proof}
É suficiente provar o item (a), já que o item (b) segue do item (a) trocando $f_n$ por $-f_n$ e $f$ por $-f$.
Em primeiro lugar, como $\int_Xf_1\,\dd\mu>-\infty$, segue do resultado do Exercício~\ref{exe:finitaqs}
que $f_1>-\infty$ quase sempre; existe portanto um subconjunto mensurável $X'$ de $X$ com complementar de medida
nula tal que $f_1(x)>-\infty$ e $f_n(x)\nearrow f(x)$, para todo $x\in X'$. Em vista do Corolário~\ref{thm:corXlinhazero},
é suficiente mostrar a tese do teorema para as restrições a $X'$ das funções em questão. Para cada $n\ge1$, defina
$g_n:X'\to[0,+\infty]$ fazendo $g_n(x)=f_n(x)-f_1(x)$, se $f_1(x)<+\infty$ e $g_n(x)=0$, se $f_1(x)=+\infty$;
daí $g_n$ é mensurável e $f_n=g_n+f_1$. De modo análogo, definimos $g:X'\to[0,+\infty]$ mensurável com
$f=g+f_1$. Daí $g_n\nearrow g$ e portanto o Teorema~\ref{thm:monotonicanneg} nos dá:
\begin{equation}\label{eq:gngmonotnneg}
\lim_{n\to\infty}\int_{X'}g_n\,\dd\mu=\int_{X'}g\,\dd\mu.
\end{equation}
Note que como $\int_{X'}f_1\,\dd\mu>-\infty$ e $\int_{X'}g_n\,\dd\mu\ge0$, o item (a) da Proposição~\ref{thm:propintlinear}
nos diz que $f_n=g_n+f_1$ é quase integrável e:
\begin{equation}\label{eq:fngnmaisf1}
\int_{X'}f_n\,\dd\mu=\int_{X'}g_n\,\dd\mu+\int_{X'}f_1\,\dd\mu;
\end{equation}
similarmente, $f$ é quase integrável e $\int_{X'}f\,\dd\mu=\int_{X'}g\,\dd\mu+\int_{X'}f_1\,\dd\mu$.
A conclusão é obtida agora fazendo $n\to\infty$ em \eqref{eq:fngnmaisf1} e usando \eqref{eq:gngmonotnneg}.
\end{proof}

\begin{prop}[Lema de Fatou]\label{thm:lemaFatou}
\index[indice]{lema!de Fatou}\index[indice]{Fatou!lema de}%
Seja $(f_n)_{n\ge1}$ uma seqüência de funções mensuráveis $f_n:X\to\overline\R$. Então:
\begin{itemize}
\item[(a)] se existe uma função quase integrável $\phi:X\to\overline\R$ tal que $f_n\ge\phi$ \qs\ para
todo $n\ge1$ e $\int_X\phi\,\dd\mu>-\infty$
então $f_n$ é quase integrável para todo $n\ge1$, $\liminf_{n\to\infty}f_n$
é quase integrável e:
\[\int_X\liminf_{n\to\infty}f_n\,\dd\mu\le\liminf_{n\to\infty}\int_Xf_n\,\dd\mu;\]
\item[(b)] se existe uma função quase integrável $\phi:X\to\overline\R$ tal que $f_n\le\phi$ \qs\  para
todo $n\ge1$ e $\int_X\phi\,\dd\mu<+\infty$
então $f_n$ é quase integrável para todo $n\ge1$, $\limsup_{n\to\infty}f_n$
é quase integrável e:
\[\limsup_{n\to\infty}\int_Xf_n\,\dd\mu\le\int_X\limsup_{n\to\infty}f_n\,\dd\mu.\]
\end{itemize}
\end{prop}
\begin{proof}
É suficiente mostrar o item (a), já que o item (b) segue do item (a) trocando $f_n$ por $-f_n$ e $\phi$ por $-\phi$.
Em primeiro lugar, a quase integrabilidade das funções $f_n$ segue do resultado do Exercício~\ref{exe:implicaquaseint}.
Para cada $n\ge1$ seja $g_n=\inf_{k\ge n}f_k$. Daí $g_n\ge\phi$ \qs, de modo que $g_n$ é quase integrável e $\int_Xg_n\,\dd\mu>-\infty$;
além do mais, $g_n\le f_k$ para todo $k\ge n$ e portanto:
\[\int_Xg_n\,\dd\mu\le\inf_{k\ge n}\int_Xf_k\,\dd\mu.\]
Claramente $g_n\nearrow(\liminf_{k\to\infty}f_k)$ e portanto a conclusão segue do item (a)
do Teorema~\ref{thm:teomonotonicageral}, fazendo $n\to\infty$ na desigualdade acima.
\end{proof}

\begin{notation}
Se $(f_n)_{n\ge1}$ é uma seqüência de funções $f_n:X\to\overline\R$ e $f:X\to\overline\R$ é uma função então escrevemos
$f_n\to f$\index[simbolos]{$f_n\to f$} quando $(f_n)_{n\ge1}$ convergir para $f$
{\em pontualmente},\index[indice]{convergencia@convergência!pontual}%
\index[indice]{pontual!convergencia@convergência}\index[indice]{sequencia@seqüência!pontualmente convergente} i.e.,
$\lim_{n\to\infty}f_n(x)=f(x)$ para todo $x\in X$. Se $(X,\mathcal A,\mu)$ é um espaço de medida, escrevemos
$f_n\to f$ \qs\ \index[simbolos]{$f_n\to f$ \qs}quando a seqüência $(f_n)_{n\ge1}$ converge para $f$
{\em pontualmente quase sempre},\index[indice]{convergencia@convergência!pontual!quase sempre@quase sempre (\qs)}%
\index[indice]{pontual!convergencia@convergência!quase sempre@quase sempre (\qs)}\index[indice]{sequencia@seqüência!pontualmente convergente!quase sempre}
i.e., se existe $X'\in\mathcal A$ tal que $\mu(X\setminus X')=0$ e tal que
$\lim_{n\to\infty}f_n(x)=f(x)$ para todo $x\in X'$.
\end{notation}

\begin{teo}[da convergência dominada]\label{thm:convdominada}
\index[indice]{teorema!da convergencia dominada@da convergência dominada}\index[indice]{convergencia dominada@convergência dominada!teorema da}%
Seja $(f_n)_{n\ge1}$ uma se\-qüên\-cia de funções mensuráveis $f_n:X\to\overline\R$ tal que $f_n\to f$ \qs, onde $f:X\to\overline\R$
é uma função mensurável. Se existe uma função integrável $\phi:X\to[0,+\infty]$ tal que $\vert f_n\vert\le\phi$ \qs\ para
todo $n\ge1$ então $f_n$ é integrável para todo $n\ge1$, $f$ é integrável e:
\[\lim_{n\to\infty}\int_Xf_n\,\dd\mu=\int_Xf\,\dd\mu.\]
\end{teo}
\begin{proof}
A integrabilidade das funções $f_n$, $f$ segue das desigualdades $\vert f_n\vert\le\phi$ \qs\ e $\vert f\vert\le\phi$ \qs\ e
do resultado do Exercício~\ref{exe:implicaquaseint}. Como $-\phi\le f_n\le\phi$ \qs\ para todo $n\ge1$
e $\int_X\phi\,\dd\mu\in\R$, estamos dentro das hipóteses de ambos os itens da Proposição~\ref{thm:lemaFatou} e portanto:
\begin{multline*}
\int_Xf\,\dd\mu=\int_X\liminf_{n\to\infty}f_n\,\dd\mu\le\liminf_{n\to\infty}\int_Xf_n\,\dd\mu
\le\limsup_{n\to\infty}\int_Xf_n\,\dd\mu\\
\le\int_X\limsup_{n\to\infty}f_n\,\dd\mu=\int_Xf\,\dd\mu.
\end{multline*}
Logo $\lim_{n\to\infty}\int_Xf_n\,\dd\mu=\int_Xf\,\dd\mu$.
\end{proof}

\begin{prop}\label{thm:proplimiteintegral}
Sejam $Y$ um subconjunto de $\R^n$, $y_0\in\R^n$ um ponto de acumulação de $Y$ e
$f:X\times Y\to\R$ uma função tal que:
\begin{itemize}
\item para todo $y\in Y$, a função $X\ni x\mapsto f(x,y)\in\R$ é integrável;
\item para todo $x\in X$ o limite $\lim_{y\to y_0}f(x,y)$ existe em $\R$;
\item existe uma função integrável $\phi:X\to\R$ e uma vizinhança $V$ de $y_0$ em $\R^n$
tal que $\vert f(x,y)\vert\le\phi(x)$, para todo $x\in X$ e todo $y\in V\cap Y$ com $y\ne y_0$.
\end{itemize}
Então, a função $X\ni x\mapsto\lim_{y\to y_0}f(x,y)\in\R$ é integrável, o limite
$\lim_{y\to y_0}\int_Xf(x,y)\,\dd\mu(x)$ existe e:
\[\lim_{y\to y_0}\int_Xf(x,y)\,\dd\mu(x)=\int_X\lim_{y\to y_0}f(x,y)\,\dd\mu(x).\]
\end{prop}
\begin{proof}
Considere a aplicação $g:Y\to\R$ definida por:
\[g(y)=\int_Xf(x,y)\,\dd\mu(x),\]
para todo $y\in Y$ e a aplicação $h:X\to\R$ definida por:
\[h(x)=\lim_{y\to y_0}f(x,y),\]
para todo $x\in X$. Devemos mostrar que $h$ é integrável e que o limite $\lim_{y\to y_0}g(y)$
existe e é igual à integral de $h$. Seja $(y_n)_{n\ge1}$ uma seqüência em $Y$ com $y_n\ne y_0$
para todo $n\ge1$ e $\lim_{n\to\infty}y_n=y_0$.
Para cada $n\ge1$, considere a função $f_n:X\to\R$ definida por $f_n(x)=f(x,y_n)$, para todo $x\in X$.
Temos que $f_n$ é integrável, para todo $n\ge1$ e que $f_n\to h$. Para $n$ suficientemente
grande temos $y_n\in V$ e portanto $\vert f_n\vert\le\phi$. Segue do Teorema~\ref{thm:convdominada}
que $h$ é integrável e que:
\[\int_Xh\,\dd\mu=\lim_{n\to\infty}\int_Xf_n\,\dd\mu=\lim_{n\to\infty}g(y_n).\]
Como $(y_n)_{n\ge1}$ é uma seqüência arbitrária em $Y\setminus\{y_0\}$ convergindo
para $y_0$, segue que $\lim_{y\to y_0}g(y)=\int_Xh\,\dd\mu$.
\end{proof}

\begin{cor}
Seja $Y$ um subconjunto de $\R^n$, $y_0$ um ponto de $Y$ e $f:X\times Y\to\R$ uma função tal que:
\begin{itemize}
\item para todo $y\in Y$, a função $X\ni x\mapsto f(x,y)\in\R$ é integrável;
\item para todo $x\in X$, a função $Y\ni y\mapsto f(x,y)\in\R$ é contínua no ponto $y_0$;
\item existe uma função integrável $\phi:X\to\R$ e uma vizinhança $V$ de $y_0$ em $\R^n$
tal que $\vert f(x,y)\vert\le\phi(x)$, para todo $x\in X$ e todo $y\in V\cap Y$ com $y\ne y_0$.
\end{itemize}
Então, a função $Y\ni y\mapsto\int_Xf(x,y)\,\dd\mu(x)\in\R$ é contínua no ponto $y_0$.
\end{cor}
\begin{proof}
Se $y_0$ é um ponto isolado de $Y$ então não há nada para ser mostrado, já que toda função é contínua
em pontos isolados de seu domínio. Se $y_0$ é um ponto de acumulação de $Y$, a Proposição~\ref{thm:proplimiteintegral}
nos dá:
\[\lim_{y\to y_0}\int_Xf(x,y)\,\dd\mu(x)=\int_X\lim_{y\to y_0}f(x,y)\,\dd\mu(x)
=\int_Xf(x,y_0)\,\dd\mu(x),\]
o que completa a demonstração.
\end{proof}

\begin{prop}
Sejam $I\subset\R$ um intervalo com mais de um ponto, $y_0$ um ponto de $I$ e $f:X\times I\to\R$
uma função tal que:
\begin{itemize}
\item para todo $y\in I$, a função $X\ni x\mapsto f(x,y)\in\R$ é integrável;
\item para todo $x\in X$, a função $I\ni y\mapsto f(x,y)\in\R$ é derivável;
\item existe uma função integrável $\phi:X\to\R$ e $\varepsilon>0$ tal que:
\[\Big\vert\frac{\partial f}{\partial y}(x,y)\Big\vert\le\phi(x),\]
para todo $x\in X$ e todo $y\in I\cap\left]y_0-\varepsilon,y_0+\varepsilon\right[$ com $y\ne y_0$.
\end{itemize}
Então a função $I\ni y\mapsto\int_Xf(x,y)\,\dd\mu(x)\in\R$ é derivável no ponto $y_0$,
a função $X\ni x\mapsto\frac{\partial f}{\partial y}(x,y_0)\in\R$ é integrável e:
\[\frac{\dd}{\dd y}\Big\vert_{y=y_0}\int_Xf(x,y)\,\dd\mu(x)=\int_X\frac{\partial f}{\partial y}(x,y_0)\,\dd\mu(x).\]
\end{prop}
\begin{proof}
Considere a função $g:I\to\R$ definida por:
\[g(x,y)=\int_Xf(x,y)\,\dd\mu(x),\]
para todo $y\in I$. Dado $h\ne0$ com $y_0+h\in I$ então:
\begin{equation}\label{eq:aquifazhtozero}
\frac{g(y_0+h)-g(y_0)}h=\int_X\frac{f(x,y_0+h)-f(x,y_0)}h\,\dd\mu(x).
\end{equation}
Obviamente:
\begin{equation}\label{eq:htozerointegrando}
\lim_{h\to0}\frac{f(x,y_0+h)-f(x,y_0)}h=\frac{\partial f}{\partial y}(x,y_0),
\end{equation}
para todo $x\in X$. Se $h\ne0$, $y_0+h\in I$ e $\vert h\vert\le\varepsilon$ então o Teorema
do Valor Médio nos dá:
\begin{equation}\label{eq:dominacaoh}
\Big\vert\frac{f(x,y_0+h)-f(x,y_0)}h\Big\vert=\Big\vert\frac{\partial f}{\partial y}(x,y_0+\theta h)\Big\vert
\le\phi(x),
\end{equation}
onde $\theta\in\left]0,1\right[$. A conclusão segue da Proposição~\ref{thm:proplimiteintegral}
e de \eqref{eq:htozerointegrando} e \eqref{eq:dominacaoh}, fazendo $h\to0$ em \eqref{eq:aquifazhtozero}.
\end{proof}

\end{section}

\begin{section}{Riemann x Lebesgue}

No que segue usaremos sistematicamente a terminologia e notação introduzidas nas Definições~\ref{thm:defbloco}
e \ref{thm:defcoisasbloco}. Introduzimos mais alguma terminologia sobre partições e blocos.
\begin{defin}
Seja $B$ um bloco retangular $n$-dimensional tal que $\vert B\vert>0$ e seja $P=(P_1,\ldots,P_n)$ uma partição
do bloco $B$. Uma partição $Q=(Q_1,\ldots,Q_n)$ de $B$ é dita um
{\em refinamento\/}\index[indice]{refinamento!de uma particao@de uma partição}\index[indice]{particao@partição!refinamento de}
de $P$ se $Q_i\supset P_i$, para $i=1,\ldots,n$.
A {\em norma\/}\index[indice]{norma!de uma particao@de uma partição}\index[indice]{particao@partição!norma de}
da partição $P$, denotada por $\Vert P\Vert$\index[simbolos]{$\Vert P\Vert$},
é definida como o máximo dos diâmetros dos sub-blocos de $B$ determinados por $P$.
\end{defin}
Claramente se uma partição $Q$ refina uma partição $P$ então todo sub-bloco de $B$ determinado
por $Q$ está contido em algum sub-bloco de $B$ determinado por $P$.

No que segue, consideramos fixado um bloco retangular $n$-dimensional $B$ com $\vert B\vert>0$
e uma função limitada $f:B\to\R$.
\begin{defin}
Se $P$ é uma partição de $B$ então a
{\em soma inferior de Riemann\/}\index[indice]{soma!inferior de Riemann}\index[indice]{Riemann!soma inferior de}
de $f$ com respeito
a $P$ é definida por:\index[simbolos]{$s(f;P)$}
\[s(f;P)=\sum_{\mathfrak b\in\overline P}\inf f(\mathfrak b)\,\vert\mathfrak b\vert,\]
e a {\em soma superior de Riemann\/}\index[indice]{soma!superior de Riemann}\index[indice]{Riemann!soma superior de}
de $f$ com respeito a $P$ é definida por:\index[simbolos]{$S(f;P)$}
\[S(f;P)=\sum_{\mathfrak b\in\overline P}\sup f(\mathfrak b)\,\vert\mathfrak b\vert.\]
\end{defin}

Obviamente:
\begin{equation}\label{eq:sleSP}
s(f;P)\le S(f;P),
\end{equation}
para toda partição $P$ de $B$.

Nós consideramos as seguintes funções $m_P:B\to\R$, $M_P:B\to\R$ associadas a uma partição
$P$ de $B$:
\[m_P=\sum_{\mathfrak b\in\overline P}\inf f(\mathfrak b)\,\chilow{\Int(\mathfrak b)},\quad
M_P=\sum_{\mathfrak b\in\overline P}\sup f(\mathfrak b)\,\chilow{\Int(\mathfrak b)}.\]
Mais explicitamente, dado $x\in B$, se $x$ pertence ao interior de algum sub-bloco
$\mathfrak b$ de $B$ determinado por $P$ então o valor da função $m_P$ (resp., da função $M_P$)
no ponto $x$ é igual ao ínfimo (resp., o supremo) de $f$ no bloco $\mathfrak b$;
se $x$ pertence à fronteira de algum sub-bloco de $B$ determinado por $P$ então $m_P(x)=M_P(x)=0$.
Obviamente $m_P$ e $M_P$ são funções simples Lebesgue integráveis e:
\begin{equation}\label{eq:intmPMP}
\int_Bm_P\,\dd\leb=s(f;P),\quad\int_BM_P\,\dd\leb=S(f;P),
\end{equation}
já que $\leb\big(\Int(\mathfrak b)\big)=\leb(\mathfrak b)=\vert\mathfrak b\vert$,
para todo $\mathfrak b\in\overline P$ (vide Corolário~\ref{thm:frontbloconula}). Temos:
\begin{multline}\label{eq:mPMP}
\inf f(B)\le m_P(x)\le f(x)\le M_P(x)\le\sup f(B),\\
\text{para todo}\ x\in\bigcup_{\mathfrak b\in\overline P}\Int(\mathfrak b);
\end{multline}
como a união das fronteiras dos blocos $\mathfrak b\in\overline P$ tem medida nula,
segue que as desigualdades em \eqref{eq:mPMP} valem para quase todo $x\in B$. Se
$Q$ é uma partição de $B$ que refina $P$ então afirmamos que:
\begin{equation}\label{eq:mPmQMPMQ}
m_P(x)\le m_Q(x),\quad M_Q(x)\le M_P(x),\quad
\text{para todo}\ x\in\bigcup_{\mathfrak b\in\overline Q}\Int(\mathfrak b);
\end{equation}
de fato, se $x\in\Int(\mathfrak b)$, para algum bloco $\mathfrak b\in\overline Q$ então
$\mathfrak b$ está contido em algum bloco $\mathfrak b'\in\overline P$, donde
$\Int(\mathfrak b)\subset\Int(\mathfrak b')$ e portanto:
\begin{gather*}
m_P(x)=\inf f(\mathfrak b')\le\inf f(\mathfrak b)=m_Q(x),\\
M_Q(x)=\sup f(\mathfrak b)\le\sup f(\mathfrak b')=M_P(x).
\end{gather*}

\begin{lem}\label{thm:sSrefina}
Dadas partições $P$ e $Q$ de $B$, se $Q$ refina $P$ então:
\[s(f;P)\le s(f;Q),\quad S(f;Q)\le S(f;P).\]
\end{lem}
\begin{proof}
Note que as desigualdades em \eqref{eq:mPmQMPMQ} valem para quase todo $x\in B$. Basta
então usar integração e as igualdades \eqref{eq:intmPMP}.
\end{proof}

\begin{cor}\label{thm:corsfPSfQ}
Para quaisquer partições $P$ e $Q$ de $B$ temos:
\[s(f;P)\le S(f;Q).\]
\end{cor}
\begin{proof}
Se $P=(P_1,\ldots,P_n)$ e $Q=(Q_1,\ldots,Q_n)$, denotamos por $P\cup Q$ a partição de $B$
dada por $P\cup Q=(P_1\cup Q_1,\ldots,P_n\cup Q_n)$; daí $P\cup Q$ refina tanto $P$ como $Q$.
Usando o Lema~\ref{thm:sSrefina} e a desigualdade \eqref{eq:sleSP} obtemos:
\[s(f;P)\le s(f;P\cup Q)\le S(f;P\cup Q)\le S(f;Q).\qedhere\]
\end{proof}

\begin{defin}\label{thm:defintRiemann}
A {\em integral inferior de Riemann\/}\index[indice]{integral!inferior de Riemann}\index[indice]{Riemann!integral inferior de}
e a {\em integral superior de Riemann\/}\index[indice]{integral!superior de Riemann}\index[indice]{Riemann!integral superior de}
de uma função
limitada $f:B\to\R$ são definidas respectivamente por:\index[simbolos]{$\intinf f$}\index[simbolos]{$\intsup f$}
\begin{gather*}
\intinfd f=\sup\big\{s(f;P):\text{$P$ partição de $B$}\big\},\\
\intsupd f=\inf\big\{S(f;P):\text{$P$ partição de $B$}\big\}.
\end{gather*}
Quando a integral inferior e a integral superior de $f$ coincidem dizemos que $f$ é
{\em Riemann integrável\/}\index[indice]{Riemann!integravel@integrável}
e nesse caso a {\em integral de Riemann\/}\index[indice]{integral!de Riemann}\index[indice]{Riemann!integral de}
de $f$ é definida por:\index[simbolos]{$\intR f$}
\[\intRd f=\intinfd f=\intsupd f.\]
\end{defin}
Note que o Corolário~\ref{thm:corsfPSfQ} implica que:
\[\intinfd f\le\intsupd f.\]

Vamos agora determinar condições necessárias e suficientes para que uma função $f$
seja Riemann integrável e vamos comparar a integral de Riemann de $f$ com a integral
de Lebesgue de $f$.

Consideraremos as funções $m:B\to\R$, $M:B\to\R$ definidas por:
\[m(x)=\sup_{\delta>0}\!\inf_{\substack{y\in B\\d(y,x)<\delta}}\!\!f(y),\quad
M(x)=\inf_{\delta>0}\!\sup_{\substack{y\in B\\d(y,x)<\delta}}\!\!f(y),\]
para todo $x\in B$. Claramente:
\begin{equation}\label{eq:mfM}
\inf f(B)\le m(x)\le f(x)\le M(x)\le\sup f(B),
\end{equation}
para todo $x\in B$.

Temos o seguinte:
\begin{lem}\label{thm:migualMcontinua}
Dado $x\in B$ então $m(x)=M(x)$ se e somente se $f$ é contínua no ponto $x$.
\end{lem}
\begin{proof}
Suponha que $f$ é contínua no ponto $x$. Dado $\varepsilon>0$ então existe $\delta>0$
tal que $f(x)-\varepsilon<f(y)<f(x)+\varepsilon$, para todo $y\in B$
com $d(y,x)<\delta$. Daí:
\[\inf_{\substack{y\in B\\d(y,x)<\delta}}\!\!f(y)\ge f(x)-\varepsilon,\quad
\sup_{\substack{y\in B\\d(y,x)<\delta}}\!\!f(y)\le f(x)+\varepsilon,\]
e portanto:
\[f(x)-\varepsilon\le m(x)\le M(x)\le f(x)+\varepsilon.\]
Como $\varepsilon>0$ é arbitrário, segue que $m(x)=M(x)$. Reciprocamente, suponha que $m(x)=M(x)$;
daí, por \eqref{eq:mfM}, temos $m(x)=f(x)=M(x)$. Portanto, para todo $\varepsilon>0$, existem $\delta_1,\delta_2>0$
tais que:
\[\inf_{\substack{y\in B\\d(y,x)<\delta_1}}\!\!f(y)>f(x)-\varepsilon,\quad
\sup_{\substack{y\in B\\d(y,x)<\delta_2}}\!\!f(y)<f(x)+\varepsilon.\]
Tome $\delta=\min\{\delta_1,\delta_2\}>0$; daí, para todo $y\in B$ com $d(y,x)<\delta$,
temos:
\[f(x)-\varepsilon<f(y)<f(x)+\varepsilon,\]
o que prova que $f$ é contínua no ponto $x$.
\end{proof}

Se $P$ é uma partição de $B$, observamos que:
\begin{equation}\label{eq:mmPMMP}
m_P(x)\le m(x),\quad M(x)\le M_P(x),\quad
\text{para todo}\ x\in\bigcup_{\mathfrak b\in\overline P}\Int(\mathfrak b);
\end{equation}
de fato, basta observar que se $x$ pertence ao interior de um bloco $\mathfrak b\in\overline P$
então existe $\delta>0$ tal que a bola de centro $x$ e raio $\delta$ está contida em $\mathfrak b$
e portanto:
\begin{gather*}
m_P(x)=\inf_{y\in\mathfrak b}f(y)\le\!\!\!\inf_{\substack{y\in B\\d(y,x)<\delta}}\!\!f(y)\le m(x),\\
M(x)\le\!\!\!\sup_{\substack{y\in B\\d(y,x)<\delta}}\!\!f(y)\le\sup_{y\in\mathfrak b}f(y)=M_P(x).
\end{gather*}

Além do mais, temos o seguinte:
\begin{lem}\label{thm:PktomM}
Se $(P_k)_{k\ge1}$ é uma seqüência de partições do bloco retangular $B$ tal que $\Vert P_k\Vert\to0$
então $m_{P_k}\to m$ \qs\ e $M_{P_k}\to M$ \qs.
\end{lem}
\begin{proof}
Seja $A$ a união das fronteiras de todos os sub-blocos de $B$ determinados por todas as partições
$P_k$; como a quantidade de blocos em questão é enumerável, temos que $A$ tem medida nula.
Seja $x\in B$, $x\not\in A$; vamos mostrar que $m_{P_k}(x)\to m(x)$ e $M_{P_k}(x)\to M(x)$.
Seja dado $\varepsilon>0$. Temos que existem $\delta_1,\delta_2>0$ tais que:
\[\inf_{\substack{y\in B\\d(y,x)<\delta_1}}\!\!f(y)>m(x)-\varepsilon,\quad
\sup_{\substack{y\in B\\d(y,x)<\delta_2}}\!\!f(y)<M(x)+\varepsilon.\]
Seja $k_0$ tal que $\Vert P_k\Vert<\min\{\delta_1,\delta_2\}$, para todo $k\ge k_0$.
Vamos mostrar que:
\begin{equation}\label{eq:mPkmMPkM}
m_{P_k}(x)>m(x)-\varepsilon,\quad M_{P_k}(x)<M(x)+\varepsilon,
\end{equation}
para todo $k\ge k_0$. Fixado $k\ge k_0$, seja $\mathfrak b\in\overline{P_k}$ tal que
$x$ pertence ao interior de $\mathfrak b$. Como o diâmetro de $\mathfrak b$
é menor que $\min\{\delta_1,\delta_2\}$, temos que $\mathfrak b$ está contido na bola
de centro $x$ e raio $\delta_1$ e na bola de centro $x$ e raio $\delta_2$, de modo que:
\begin{gather*}
m_{P_k}(x)=\inf_{y\in\mathfrak b}f(y)\ge\!\!\!\inf_{\substack{y\in B\\d(y,x)<\delta_1}}\!\!f(y)>m(x)-\varepsilon,\\
M_{P_k}(x)=\sup_{y\in\mathfrak b}f(y)\le\!\!\!\sup_{\substack{y\in B\\d(y,x)<\delta_2}}\!\!f(y)<M(x)+\varepsilon,
\end{gather*}
provando \eqref{eq:mPkmMPkM}.
Usando \eqref{eq:mmPMMP} e \eqref{eq:mPkmMPkM} concluímos agora que:
\[m(x)-\varepsilon<m_{P_k}(x)\le m(x),\quad M(x)\le M_{P_k}(x)<M(x)+\varepsilon,\]
o que completa a demonstração.
\end{proof}

\begin{cor}\label{thm:corintmM}
As funções $m$ e $M$ são Lebesgue integráveis e:
\[\int_Bm\,\dd\leb=\intinfd f,\quad\int_BM\,\dd\leb=\intsupd f.\]
\end{cor}
\begin{proof}
Segue do Lema~\ref{thm:PktomM} e do resultado do item (c) do Exercício~\ref{exe:mensqs}
que as funções $m$ e $M$ são mensuráveis. Seja agora $(P_k)_{k\ge1}$ uma seqüência de partições
de $B$ tal que:
\begin{equation}\label{eq:sfPktointinf}
\lim_{k\to\infty}s(f;P_k)=\intinfd f.
\end{equation}
Podemos refinar cada partição $P_k$ de modo que $\Vert P_k\Vert\to0$; o Lema~\ref{thm:sSrefina}
garante que a condição \eqref{eq:sfPktointinf} continua satisfeita. Como o bloco $B$
tem medida finita, qualquer função constante finita definida em $B$ é integrável; logo,
as desigualdades em \eqref{eq:mPMP} implicam que a seqüência de funções $(m_{P_k})_{k\ge1}$
satisfaz as hipótese do Teorema da Convergência Dominada. Usando o Lema~\ref{thm:PktomM}
e as identidades \eqref{eq:intmPMP} obtemos então:
\[\lim_{k\to\infty}s(f;P_k)=\lim_{k\to\infty}\int_Bm_{P_k}\,\dd\leb=\int_Bm\,\dd\leb.\]
De modo totalmente análogo, mostra-se que a integral de Lebesgue de $M$ é igual à integral superior
de Riemann de $f$.
\end{proof}

Estamos em condições agora de provar o resultado principal desta seção.
\begin{prop}\label{thm:propRiemannLebesgue}
Seja $B$ um bloco retangular $n$-dimensional com $\vert B\vert>0$ e seja $f:B\to\R$ uma função
limitada. Então:
\begin{itemize}
\item[(a)] $f$ é Riemann integrável se e somente se o conjunto das descontinuidades de $f$
tem medida nula;
\item[(b)] se $f$ é Riemann integrável então $f$ é Lebesgue integrável e:
\[\int_Bf\,\dd\leb=\intRd f.\]
\end{itemize}
\end{prop}
\begin{proof}
Em vista do Corolário~\ref{thm:corintmM}, $f$ é Riemann integrável se e somente se:
\[\int_Bm\,\dd\leb=\int_BM\,\dd\leb.\]
Como $m\le M$, o resultado do Exercício~\ref{exe:flegintigualqs} implica que
$f$ é Riemann integrável se e somente se $M=m$ quase sempre. O item (a) segue portanto
do Lema~\ref{thm:migualMcontinua}. Passemos à demonstraçao do item (b). Suponha que $f$
é Riemann integrável. Então $M=m$ quase sempre e de \eqref{eq:mfM} segue que
$m=f=M$ quase sempre. O resultado do item (b) do Exercício~\ref{exe:mensqs} implica
então que $f$ é mensurável; além do mais:
\[\int_B f\,\dd\leb=\int_B m\,\dd\leb=\intinfd f=\intRd f.\qedhere\]
\end{proof}

\begin{subsection}{A integral imprópria de Riemann}

Na Definição~\ref{thm:defintRiemann} introduzimos a noção de integral de Riemann
para funções limitadas definidas em blocos retangulares. A noção de integral de Riemann
pode ser estendida para contextos mais gerais, envolvendo funções não limitadas definidas
em domínios não limitados. Tais extensões são normalmente conhecidas como {\em integrais
impróprias de Riemann\/} e são definidas através de limites de integrais próprias (i.e.,
integrais de funções limitadas em conjuntos limitados).

\begin{notation}
Seja $[a,b]\subset\R$ um intervalo com $a<b$. Se $f$ é uma função a valores reais definida
num conjunto que contém $[a,b]$ e se $f\vert_{[a,b]}$ é limitada e Riemann integrável então a integral
de Riemann de $f\vert_{[a,b]}$ será denotada por:\index[simbolos]{$\intR_a^bf$}
\[\intRd_a^bf.\]
\end{notation}

\begin{defin}
Seja $f:\left[a,+\infty\right[\to\R$ uma função tal que para todo $u\in\left]a,+\infty\right[$,
a restrição de $f$ ao intervalo $[a,u]$ é limitada e Riemann integrável. A {\em integral
imprópria de Riemann\/}\index[indice]{integral!impropria de Riemann@imprópria de Riemann}\index[indice]{Riemann!integral impropria de@integral imprópria de}
de $f$ é definida por:
\[\intRd_a^{+\infty}f=\lim_{u\to\infty}\intRd_a^uf,\]
desde que o limite acima exista em $\overline\R$. Quando esse limite é finito, dizemos que
a integral imprópria de $f$ é {\em convergente}\index[indice]{integral impropria@integral imprópria!convergente}.
\end{defin}

\begin{prop}\label{thm:RiemannimpLeb}
Seja $f:\left[a,+\infty\right[\to\R$ uma função tal que para todo $u\in\left]a,+\infty\right[$,
a restrição de $f$ ao intervalo $[a,u]$ é limitada e Riemann integrável. Então $f$ é mensurável.
Além do mais, se $f$ é Lebesgue quase integrável então a integral
imprópria de Riemann de $f$ existe em $\overline\R$ e:
\begin{equation}\label{eq:intimpRigualLeb}
\intRd_a^{+\infty}f=\int_a^{+\infty}f\,\dd\leb.
\end{equation}
\end{prop}
\begin{proof}
Seja $(u_n)_{n\ge1}$ uma seqüência arbitrária em $\left]a,+\infty\right[$ tal que
$u_n\to+\infty$. Pela Proposição~\ref{thm:propRiemannLebesgue}, a restrição de $f$
ao intervalo $[a,u_n]$ é Lebesgue integrável e:
\begin{equation}\label{eq:intpropRigualLeb}
\int_a^{u_n}f\,\dd\leb=\intRd_a^{u_n}f,
\end{equation}
para todo $n\ge1$. Obviamente:
\[\lim_{n\to\infty}f\chilow{[a,u_n]}=f;\]
como $f\chilow{[a,u_n]}$ é mensurável para todo $n\ge1$,
concluímos que $f$ é mensurável. Em vista de \eqref{eq:intpropRigualLeb}, para mostrar
\eqref{eq:intimpRigualLeb}, é suficiente mostrar que:
\begin{equation}\label{eq:limaun}
\lim_{n\to\infty}\int_a^{u_n}f\,\dd\leb=\int_a^{+\infty}f\,\dd\leb,
\end{equation}
para toda seqüência $(u_n)_{n\ge1}$ em $\left]a,+\infty\right[$ com $u_n\to+\infty$.
Verifiquemos \eqref{eq:limaun} primeiramente no caso em que $f$ é não negativa. Pelo Lema
de Fatou, temos:
\begin{multline*}
\int_a^{+\infty}f\,\dd\leb=
\int_a^{+\infty}\liminf_{n\to\infty}f\chilow{[a,u_n]}\,\dd\leb\le
\liminf_{n\to\infty}\int_a^{+\infty}f\chilow{[a,u_n]}\,\dd\leb
\\=\liminf_{n\to\infty}\int_a^{u_n}f\,\dd\leb.
\end{multline*}
Por outro lado, $\int_a^{u_n}f\,\dd\leb\le\int_a^{+\infty}f\,\dd\leb$
para todo $n\ge1$, donde:
\[\int_a^{+\infty}f\,\dd\leb\le\liminf_{n\to\infty}\int_a^{u_n}f\,\dd\leb\le
\limsup_{n\to\infty}\int_a^{u_n}f\,\dd\leb\le\int_a^{+\infty}f\,\dd\leb,\]
provando \eqref{eq:limaun} no caso $f\ge0$. Em geral, se $f:\left[a,+\infty\right[\to\R$ é uma função quase integrável
qualquer então \eqref{eq:limaun} é satisfeita para $f^+$ e $f^-$, ou seja:
\[\lim_{n\to\infty}\int_a^{u_n}f^+\,\dd\leb=\int_a^{+\infty}f^+\,\dd\leb,\quad
\lim_{n\to\infty}\int_a^{u_n}f^-\,\dd\leb=\int_a^{+\infty}f^-\,\dd\leb;\]
a conclusão é obtida subtraindo as duas igualdades acima.
\end{proof}

Resultados análogos aos da Proposição~\ref{thm:RiemannimpLeb} podem ser mostrados para outros tipos
de integrais impróprias de Riemann (por exemplo, integrais de funções ilimitadas
em intervalos limitados). O passo central da demonstração de tais resultados
é dado pelo resultado do Exercício~\ref{exe:conjuntosAk}. Note, por exemplo,
que o resultado desse exercício pode ser usado para justificar a igualdade
\eqref{eq:limaun} na demonstração da Proposição~\ref{thm:RiemannimpLeb}.

\begin{example}\label{exa:senxx}
É possível que uma função $f:\left[a,+\infty\right[\to\R$ admita uma integral imprópria
de Riemann convergente mas não seja Lebesgue quase integrável.
Considere a função $f:\left[0,+\infty\right[\to\R$ definida por:
\[f(x)=\frac{\sen\,x}x,\]
para $x>0$ e $f(0)=1$. Temos que $f$ é contínua e portanto $f\vert_{[0,u]}$ é limitada e Riemann
integrável para todo $u\in\left]0,+\infty\right[$. Temos que $f$ se anula nos pontos $k\pi$,
com $k$ inteiro positivo, $f$ é positiva nos intervalos da forma $\left]k\pi,(k+1)\pi\right[$
com $k$ inteiro positivo par e $f$ é negativa nos intervalos da forma $\left]k\pi,(k+1)\pi\right[$
com $k$ inteiro positivo ímpar. Para cada inteiro $k\ge0$, seja:
\[a_k=\int_{k\pi}^{(k+1)\pi}\vert f\vert\,\dd\leb\ge0.\]
Em vista do resultado do Exercício~\ref{exe:propintegralmedida} temos:
\begin{equation}\label{eq:kparimpar}
\int_0^{+\infty}f^+\,\dd\leb=\sum_{\substack{k=0\\\text{$k$ par}}}^\infty a_k,\quad
\int_0^{+\infty}f^-\,\dd\leb=\sum_{\substack{k=1\\\text{$k$ ímpar}}}^\infty a_k.
\end{equation}
Além do mais:
\[\int_0^{n\pi}f\,\dd\leb=\sum_{k=0}^{n-1}(-1)^ka_k,\]
e portanto:
\[\lim_{n\to\infty}\int_0^{n\pi}f\,\dd\leb=\lim_{n\to\infty}\sum_{k=0}^{n-1}(-1)^ka_k=
\sum_{k=0}^\infty(-1)^ka_k.\]
Façamos algumas estimativas sobre os números $a_k$. Para $x\in[k\pi,(k+1)\pi]$, temos
$\big\vert\frac{\sen\,x}x\big\vert\le\frac1{k\pi}$ e portanto:
\[a_k\le\frac1{k\pi}\big((k+1)\pi-k\pi\big)=\frac1k,\]
para todo $k\ge1$. Segue que $a_k\to0$. Vamos mostrar que a seqüência $(a_k)_{k\ge0}$
é decrescente. Temos:
\begin{multline*}
a_{k+1}=\int_{(k+1)\pi}^{(k+2)\pi}\Big\vert\frac{\sen\,x}x\Big\vert\,\dd\leb(x)
=\int_{k\pi}^{(k+1)\pi}\Big\vert\frac{\sen(x+\pi)}{x+\pi}\Big\vert\,\dd\leb(x)\\
=\int_{k\pi}^{(k+1)\pi}\Big\vert\frac{\sen\,x}{x+\pi}\Big\vert\,\dd\leb(x)\le
\int_{k\pi}^{(k+1)\pi}\Big\vert\frac{\sen\,x}x\Big\vert\,\dd\leb(x)=a_k;
\end{multline*}
a segunda igualdade acima pode ser justificada fazendo a mudança de variável
$y=x-\pi$ na integral de Riemann $\intR_{(k+1)\pi}^{(k+2)\pi}\big\vert\frac{\sen\,x}x\big\vert\,\dd x$
ou utilizando o resultado do Exercício~\ref{exe:measurepreserving} e o fato que a função
$x\mapsto x+\pi$ preserva medida (veja Lema~\ref{thm:extmeastransinv} e Definição~\ref{thm:measurepres}).
Como a seqüência $(a_k)_{k\ge0}$ é decrescente e tende a zero, segue do critério de Dirichlet
(ou critério da série alternada) que a série $\sum_{k=0}^\infty(-1)^ka_k$ converge; defina:
\[\sum_{k=0}^\infty(-1)^ka_k=L\in\R.\]
Vamos mostrar agora que:
\begin{equation}\label{eq:limuintzerou}
\lim_{u\to+\infty}\int_0^uf\,\dd\leb=L.
\end{equation}
Dado $\varepsilon>0$, temos que existe $n_0$ tal que:
\[\Big\vert L-\sum_{k=0}^n(-1)^ka_k\Big\vert<\frac\varepsilon2,\]
para todo $n\ge n_0$. Podemos escolher $n_0$ também de modo que:
\[a_n<\frac\varepsilon2,\]
para todo $n\ge n_0$.
Dado $u\in\R$, $u\ge n_0\pi$, seja $n\ge n_0$ o maior inteiro
tal que $n\pi\le u$; daí $n\pi\le u<(n+1)\pi$ e:
\[\int_0^uf\,\dd\leb=\int_0^{(n+1)\pi}f\,\dd\leb-\int_u^{(n+1)\pi}f\,\dd\leb=
\sum_{k=0}^n(-1)^ka_k-\int_u^{(n+1)\pi}f\,\dd\leb.\]
Daí:
\begin{multline*}
\Big\vert L-\int_0^uf\,\dd\leb\Big\vert\le\Big\vert L-\sum_{k=0}^n(-1)^ka_k\Big\vert
+\Big\vert\int_u^{(n+1)\pi}f\,\dd\leb\Big\vert\\
\le\Big\vert L-\sum_{k=0}^n(-1)^ka_k\Big\vert+a_n<\varepsilon,
\end{multline*}
para todo $u\ge n_0\pi$. Isso prova \eqref{eq:limuintzerou}. Concluímos então que:
\[\intRd_0^{+\infty}f=L\in\R.\]
Vamos agora mostrar que $f$ não é Lebesgue quase integrável.
Para isso, fazemos uma estimativa inferior para os números $a_k$. Dado um inteiro
$k\ge0$ então, para $k\pi+\frac\pi4\le x\le(k+1)\pi-\frac\pi4$ temos:
\[\vert\sen\,x\vert\ge\frac{\sqrt2}2,\quad
\Big\vert\frac{\sen\,x}x\Big\vert\ge\frac{\sqrt2}2\,\frac1{(k+1)\pi},\]
e portanto:
\[a_k=\int_{k\pi}^{(k+1)\pi}\vert f\vert\,\dd\leb
\ge\int_{k\pi+\frac\pi4}^{(k+1)\pi-\frac\pi4}\Big\vert\frac{\sen\,x}x\Big\vert\,\dd\leb(x)
\ge\frac{\sqrt2}2\,\frac1{(k+1)\pi}\,\frac\pi2.\]
Segue que as séries em \eqref{eq:kparimpar} são divergentes e portanto:
\[\int_0^{+\infty}f^+\,\dd\leb=+\infty=\int_0^{+\infty}f^-\,\dd\leb.\]
Logo $f$ não é Lebesgue quase integrável.
\end{example}
No Exercício~\ref{exe:senxx} pedimos ao leitor para computar explicitamente
o valor da integral imprópria de Riemann $\intR_0^{+\infty}f$ da função
$f$ do Exemplo~\ref{exa:senxx}.

\end{subsection}

\end{section}

\begin{section}{Mais sobre Convergência de Seqüências de Funções}

Recorde que, dado um conjunto $X$, uma seqüência $(f_n)_{n\ge1}$ de funções $f_n:X\to\overline\R$ e uma função $f:X\to\overline\R$,
dizemos que $(f_n)_{n\ge1}$ {\em converge pontualmente\/}\index[indice]{convergencia@convergência!pontual}%
\index[indice]{pontual!convergencia@convergência}\index[indice]{sequencia@seqüência!pontualmente convergente} para
$f$ e escrevemos $f_n\to f$ quando $\lim_{n\to\infty}f_n(x)=f(x)$,
para todo $x\in X$. Se as funções $f_n$ e $f$ tomam valores em $\R$, isso significa que para todo $x\in X$
e todo $\varepsilon>0$, existe $n_0\ge1$ (possivelmente dependendo de $x$) tal que
$\vert f_n(x)-f(x)\vert<\varepsilon$, para todo $n\ge n_0$; dizemos que $(f_n)_{n\ge1}$
{\em converge uniformemente\/}\index[indice]{convergencia@convergência!uniforme}\index[indice]{uniforme!convergencia@convergência}%
\index[indice]{sequencia@seqüência!uniformemente convergente@uniformemente conver-\hfil\break gente} para
$f$ e escrevemos $f_n\To uf$,\index[simbolos]{$f_n\To uf$} se para todo $\varepsilon>0$ existe
$n_0\ge1$ tal que $\vert f_n(x)-f(x)\vert<\varepsilon$, para todo $n\ge n_0$ e todo $x\in X$.
Alternativamente, temos que $(f_n)_{n\ge1}$ converge uniformemente para $f$ se:
\[\lim_{n\to\infty}\sup_{x\in X}\big\vert f_n(x)-f(x)\big\vert=0.\]
Obviamente convergência uniforme implica em convergência pontual.

Em teoria da medida, estamos em geral mais interessados em conceitos que desprezem tudo aquilo que ocorre em conjuntos de medida nula.
Recorde que se $(X,\mathcal A,\mu)$ é um espaço de medida e se $(f_n)_{n\ge1}$ é uma seqüência de funções
$f_n:X\to\overline\R$, então dizemos que $(f_n)_{n\ge1}$ converge para $f:X\to\overline\R$ {\em pontualmente quase
sempre\/}\index[indice]{convergencia@convergência!pontual!quase sempre@quase sempre (\qs)}%
\index[indice]{pontual!convergencia@convergência!quase sempre@quase sempre (\qs)}\index[indice]{sequencia@seqüência!pontualmente convergente!quase sempre}
e escrevemos $f_n\to f$ \qs\ quando existe $X'\in\mathcal A$ tal que $\mu(X\setminus X')=0$ e $\lim_{n\to\infty}f_n(x)=f(x)$,
para todo $x\in X'$. Precisamos agora de uma versão da noção de convergência uniforme que ignore conjuntos de medida
nula. Temos a seguinte:

\begin{defin}
Sejam $(X,\mathcal A,\mu)$ um espaço de medida, $(f_n)_{n\ge1}$ uma seqüência de funções $f_n:X\to\R$
e $f:X\to\R$ uma função. Dizemos que $(f_n)_{n\ge1}$ converge para $f$
{\em uniformemente quase sempre\/}\index[indice]{convergencia@convergência!uniforme!quase sempre@quase sempre (\qs)}%
\index[indice]{uniforme!convergencia@convergência!quase sempre@quase sempre (\qs)}\index[indice]{sequencia@seqüência!uniformemente convergente!quase sempre}
e escrevemos $f_n\To uf$ \qs\ \index[simbolos]{$f_n\To uf$ \qs}se existe $X'\in\mathcal A$ tal que $\mu(X\setminus X')=0$ e tal que
$f_n\vert_{X'}\To uf\vert_{X'}$.
\end{defin}
Evidentemente, convergência uniforme quase sempre implica em convergência pontual quase sempre.

\begin{example}\label{exa:unifqs}
Seja $(A_n)_{n\ge1}$ uma seqüência de subconjuntos de $\R$ de medida nula e seja $(f_n)_{n\ge1}$ a seqüência de funções
$f_n:\R\to\R$ definida por $f_n=\chilow{A_n}$, para todo $n\ge1$. Temos que $(f_n)_{n\ge1}$ converge
uniformemente quase sempre para a função nula. De fato, tomando $A=\bigcup_{n=1}^\infty A_n$ então $A$ tem medida
nula e todas as funções $f_n$ são identicamente nulas no complementar de $A$.
\end{example}

\begin{example}\label{exa:naounifqs}
Seja $(f_n)_{n\ge1}$ a seqüência de funções $f_n:[0,1]\to\R$ definida por $f_n(x)=x^n$, para todo $n\ge1$ e todo
$x\in[0,1]$. Temos que $(f_n)_{n\ge1}$ converge pontualmente para a função $f=\chilow{\{1\}}$. Vamos determinar
para quais subconjuntos $S$ de $[0,1]$ tem-se $f_n\vert_S\To uf\vert_S$. Temos:
\[\big\vert f_n(x)-f(x)\big\vert=\begin{cases}
0,&\text{se $x=1$},\\
x^n,&\text{se $x\in\left[0,1\right[$}.
\end{cases}\]
Daí, se $S\subset[0,1]$ não está contido em $\{1\}$, temos:
\[\sup_{x\in S}\big\vert f_n(x)-f(x)\big\vert=\sup_{x\in S\setminus\{1\}}x^n
=\big[\sup\big(S\setminus\{1\}\big)\big]^n.\]
Logo $f_n\vert_S\To uf\vert_S$ se e somente se $1$ não é um ponto de acumulação do conjunto $S$.
Concluímos que não é o caso que $(f_n)_{n\ge1}$ converge uniformemente quase sempre para $f$; de fato,
se $S\subset[0,1]$ é tal que $f_n\vert_S\To uf\vert_S$ então $[0,1]\setminus S$ contém um intervalo da forma
$\left]1-\varepsilon,1\right[$, $\varepsilon>0$, e em particular o conjunto $[0,1]\setminus S$ não tem medida nula.
Note, no entanto, que para todo $\varepsilon>0$ a seqüência $(f_n)_{n\ge1}$ converge uniformemente para $f$
em $[0,1-\varepsilon]$.
\end{example}

Os Exemplos~\ref{exa:unifqs} e \ref{exa:naounifqs} ilustram que a definição de convergência uniforme quase sempre não
é tão interessante. A seguinte definição é mais interessante.
\begin{defin}
Sejam $(X,\mathcal A,\mu)$ um espaço de medida, $(f_n)_{n\ge1}$ uma seqüência de funções $f_n:X\to\R$
e $f:X\to\R$ uma função. Dizemos que $(f_n)_{n\ge1}$ converge para $f$
{\em quase uniformemente\/}\index[indice]{convergencia@convergência!quase uniforme}%
\index[indice]{quase uniforme!convergencia@convergência}\index[indice]{sequencia@seqüência!quase uniformemente convergente} e escrevemos
$f_n\To{qu}f$\index[simbolos]{$f_n\To{qu}f$} se para todo $\varepsilon>0$ existe $A\in\mathcal A$ tal que $\mu(A)<\varepsilon$
e $f_n\vert_{A^\compl}\To uf\vert_{A^\compl}$.
\end{defin}
Evidentemente, convergência uniforme quase sempre implica em convergência quase uniforme.

\begin{example}
A seqüência $(f_n)_{n\ge1}$ do Exemplo~\ref{exa:naounifqs} converge quase uniformemente para $f$, mas não converge
uniformemente quase sempre.
\end{example}

\begin{lem}\label{thm:Egorofacil}
Sejam $(X,\mathcal A,\mu)$ um espaço de medida, $(f_n)_{n\ge1}$ uma se\-qüên\-cia de funções $f_n:X\to\R$
e $f:X\to\R$ uma função. Se $f_n\To{qu}f$ então $f_n\to f$ pontualmente quase sempre.
\end{lem}
\begin{proof}
Para todo $k\ge1$, existe $A_k\subset X$ mensurável com $\mu(A_k)<\frac1k$ tal que
$(f_n)_{n\ge1}$ converge uniformemente para $f$ em $A_k^\compl$. Daí $\lim_{n\to\infty}f_n(x)=f(x)$ para
todo $x\in\bigcup_{k=1}^\infty A_k^\compl$; mas:
\[\bigcup_{k=1}^\infty A_k^\compl=\Big(\bigcap_{k=1}^\infty A_k\Big)^\compl\]
e obviamente $\mu\big(\bigcap_{k=1}^\infty A_k\big)=0$. Logo $(f_n)_{n\ge1}$ converge para $f$ pontualmente quase sempre.
\end{proof}

Para espaços de medida finita, temos o surpreendente fato de que a recíproca do Lema~\ref{thm:Egorofacil}
é válida.
\begin{teo}[Egoroff]\index[indice]{Egoroff!teorema de}\index[indice]{teorema!de Egoroff}
\label{thm:Egoroff}
Seja $(X,\mathcal A,\mu)$ um espaço de medida e sejam $(f_n)_{n\ge1}$ uma seqüência
de funções mensuráveis $f_n:X\to\R$ e $f:X\to\R$ uma função mensurável. Se $\mu(X)<+\infty$
e $f_n\to f$ pontualmente quase sempre então $f_n\To{qu}f$.
\end{teo}
\begin{proof}
Se $x\in X$ é tal que $\lim_{n\to\infty}f_n(x)=f(x)$
então para todo $k\ge1$ existe $n_0\ge1$ tal que $\vert f_n(x)-f(x)\vert<\frac1k$, para todo $n\ge n_0$; em outras
palavras, para todo $k\ge1$, $x$ pertence ao conjunto:
\begin{equation}\label{eq:ondeconvergek}
\bigcup_{n_0=1}^\infty\bigcap_{n=n_0}^{\infty}\big\{y\in X:\big\vert f_n(y)-f(y)\big\vert<\tfrac1k\big\}.
\end{equation}
Como $f_n\to f$ \qs, vemos que o complementar de \eqref{eq:ondeconvergek} tem medida nula,
para todo $k\ge1$; esse complementar é igual a:
\[\bigcap_{n_0=1}^\infty\bigcup_{n=n_0}^\infty\big\{y\in X:\big\vert f_n(y)-f(y)\big\vert\ge\tfrac1k\big\}.\]
Temos que a seqüência de conjuntos $\bigcup_{n=n_0}^\infty\big\{y\in X\!:\!\vert f_n(y)-f(y)\vert\ge\frac1k\big\}$
(indexada em $n_0$) é decrescente e portanto, como $\mu(X)<+\infty$, temos (Lema~\ref{thm:setlimits}):
\begin{equation}\label{eq:cadakn0toinfty}
\lim_{n_0\to\infty}\mu\Big(\bigcup_{n=n_0}^\infty\big\{y\in X:\big\vert f_n(y)-f(y)\big\vert\ge\tfrac1k\big\}\Big)=0.
\end{equation}
Seja $\varepsilon>0$ fixado. De \eqref{eq:cadakn0toinfty}, segue que para cada $k\ge1$ podemos encontrar
$n_k\ge1$ tal que $\mu(A_k)<\frac\varepsilon{2^k}$, onde:
\[A_k=\bigcup_{n=n_k}^\infty\big\{y\in X:\big\vert f_n(y)-f(y)\big\vert\ge\tfrac1k\big\}.\]
Seja $A=\bigcup_{k=1}^\infty A_k$; temos $\mu(A)\le\sum_{k=1}^\infty\mu(A_k)<\varepsilon$.
Afirmamos que $(f_n)_{n\ge1}$ converge uniformemente para $f$ em $A^\compl$. De fato, para todo $k\ge1$
e todo $x\in A^\compl$ temos que $x\not\in A_k$, o que significa que $\vert f_n(x)-f(x)\vert<\frac 1k$,
para todo $n\ge n_k$. Isso completa a demonstração.
\end{proof}

\begin{example}
Seja $(f_n)_{n\ge1}$ a seqüência de funções $f_n:\R\to\R$ definida por:
\[f_n(x)=\frac xn,\]
para todo $x\in\R$ e todo $n\ge1$.
Temos que $(f_n)_{n\ge1}$ converge pontualmente para a função nula. Dado $S\subset\R$, então:
\[\sup_{x\in S}\big\vert f_n(x)\big\vert=\frac1n\sup_{x\in S}\vert x\vert,\]
donde $(f_n)_{n\ge1}$ converge uniformemente para a função nula em $S$ se e somente se o conjunto $S$ é limitado.
Mas se $A\subset\R$ tem medida finita então $S=A^\compl$ não pode ser limitado, pois se $S$ é limitado então
$A=S^\compl$ contém um intervalo ilimitado. Logo não é o caso que $f_n\To{qu}f$, embora $f_n\to f$ pontualmente.
Note que não temos uma contradição com o Teorema~\ref{thm:Egoroff}, já que $\R$ não tem medida finita.
\end{example}

\begin{defin}
Sejam $(X,\mathcal A,\mu)$ um espaço de medida, $(f_n)_{n\ge1}$ uma seqüência de funções mensuráveis $f_n:X\to\R$
e $f:X\to\R$ uma função mensurável. Dizemos que $(f_n)_{n\ge1}$ {\em converge para $f$ em medida\/}\index[indice]{convergencia@convergência!em medida}%
\index[indice]{medida!convergencia em@convergência em}\index[indice]{sequencia@seqüência!convergente em medida}
e escrevemos $f_n\To\mu f$\index[simbolos]{$f_n\To\mu f$} se para todo $\varepsilon>0$ temos:
\begin{equation}\label{eq:defconvmedida}
\lim_{n\to\infty}\mu\Big(\big\{x\in X:\big\vert f_n(x)-f(x)\big\vert\ge\varepsilon\big\}\Big)=0.
\end{equation}
\end{defin}

\begin{lem}\label{thm:convquimplmu}
Sejam $(X,\mathcal A,\mu)$ um espaço de medida, $(f_n)_{n\ge1}$ uma se\-qüên\-cia de funções mensuráveis $f_n:X\to\R$
e $f:X\to\R$ uma função mensurável. Se $f_n\To{qu}f$ então $f_n\To\mu f$.
\end{lem}
\begin{proof}
Seja $\varepsilon>0$ dado e provemos \eqref{eq:defconvmedida}. Como $f_n\To{qu}f$, para todo $\eta>0$ dado,
existe um conjunto mensurável $A$ com $\mu(A)<\eta$ tal que $(f_n)_{n\ge1}$ converge uniformemente para $f$
em $A^\compl$; daí, existe $n_0\ge1$ tal que $\vert f_n(x)-f(x)\vert<\varepsilon$, para todo $x\in A^\compl$
e todo $n\ge n_0$. Temos então:
\[\big\{x\in X:\big\vert f_n(x)-f(x)\big\vert\ge\varepsilon\big\}\subset A,\]
para todo $n\ge n_0$, donde:
\[\mu\Big(\big\{x\in X:\big\vert f_n(x)-f(x)\big\vert\ge\varepsilon\big\}\Big)<\eta,\]
para todo $n\ge n_0$. Isso completa a demonstração.
\end{proof}

\begin{example}
A recíproca do Lema~\ref{thm:convquimplmu} não é verdadeira; na verdade, convergência em medida não implica
sequer convergência pontual quase sempre. De fato, seja $(I_n)_{n\ge1}$ uma seqüência de intervalos contidos
em $[0,1]$ de modo que:
\begin{itemize}
\item $\lim_{n\to\infty}\leb(I_n)=0$;
\item para todo $x\in[0,1]$, existem infinitos índices $n$ com $x\in I_n$ e infinitos índices $n$ com $x\not\in I_n$.
\end{itemize}
Por exemplo, uma possível seqüência $(I_n)_{n\ge1}$ é:
\begin{multline*}
[0,1],\big[0,\tfrac12\big],\big[\tfrac12,1\big],\big[0,\tfrac13\big],\big[\tfrac13,\tfrac23\big],
\big[\tfrac23,1\big],\ldots,\\
\big[0,\tfrac1k\big],\big[\tfrac1k,\tfrac2k\big],\big[\tfrac2k,\tfrac3k\big],\ldots,\big[\tfrac ik,\tfrac{i+1}k\big],\ldots,
\big[\tfrac{k-1}k,1\big],\ldots
\end{multline*}
Seja $(f_n)_{n\ge1}$ a seqüência de funções $f_n:[0,1]\to\R$ definida por $f_n=\chilow{I_n}$, para todo $n\ge1$.
Afirmamos que $(f_n)_{n\ge1}$ converge em medida para a função nula. De fato, fixado $\varepsilon>0$ então:
\[\big\{x\in[0,1]:\big\vert f_n(x)\big\vert\ge\varepsilon\big\}\subset I_n,\]
e $\lim_{n\to\infty}\leb(I_n)=0$. Logo $f_n\To\mu0$. No entanto, para todo $x\in[0,1]$, temos que a seqüência
$\big(f_n(x)\big)_{n\ge1}$ possui uma subseqüência constante e igual a zero e uma subseqüência constante e igual a $1$;
logo $\big(f_n(x)\big)_{n\ge1}$ não converge para {\em nenhum\/} ponto $x\in[0,1]$.
\end{example}

A recíproca do Lema~\ref{thm:convquimplmu} não vale, mas temos o seguinte:
\begin{lem}\label{thm:medsubseqqu}
Sejam $(X,\mathcal A,\mu)$ um espaço de medida, $(f_n)_{n\ge1}$ uma se\-qüên\-cia de funções mensuráveis $f_n:X\to\R$
e $f:X\to\R$ uma função mensurável. Se $f_n\To\mu f$ então existe uma subseqüência $(f_{n_k})_{k\ge1}$
de $(f_n)_{n\ge1}$ tal que $f_{n_k}\To{qu}f$; em particular, pelo Lema~\ref{thm:Egorofacil}, $f_{n_k}\to f$ \qs.
\end{lem}
\begin{proof}
Vamos contruir indutivamente uma seqüência de ín\-di\-ces $(n_k)_{k\ge1}$ tal que $n_1<n_2<\cdots$ e tal que:
\[\mu\Big(\big\{x\in X:\big\vert f_{n_k}(x)-f(x)\big\vert\ge\tfrac1k\big\}\Big)<\frac1{2^k},\]
para todo $k\ge1$. Como $f_n\To\mu f$, podemos escolher $n_1\ge1$ tal que:
\[\mu\Big(\big\{x\in X:\big\vert f_{n_1}(x)-f(x)\big\vert\ge1\big\}\Big)<\frac12;\]
supondo $n_k$ já definido, podemos escolher $n_{k+1}>n_k$ tal que:
\[\mu\Big(\big\{x\in X:\big\vert f_{n_{k+1}}(x)-f(x)\big\vert\ge\tfrac1{k+1}\big\}\Big)<\frac1{2^{k+1}}.\]
Obtemos assim a seqüência $(n_k)_{k\ge1}$ com as propriedades desejadas. Vamos mostrar que $f_{n_k}\To{qu}f$. Dado $\varepsilon>0$,
devemos encontrar um conjunto mensurável $A$ de medida menor que $\varepsilon$, de modo que $(f_{n_k})_{k\ge1}$
converge uniformemente para $f$ em $A^\compl$. Seja $t\ge1$ de modo que:
\[\sum_{k=t}^\infty\frac1{2^k}=\frac1{2^{t-1}}\le\varepsilon,\]
e tome:
\[A=\bigcup_{k=t}^\infty\big\{x\in X:\big\vert f_{n_k}(x)-f(x)\big\vert\ge\tfrac1k\big\};\]
daí $\mu(A)<\sum_{k=t}^\infty\frac1{2^k}\le\varepsilon$. Para $x\in A^\compl$ e $k\ge t$, temos
$\vert f_{n_k}(x)-f(x)\vert<\frac1k$ e portanto:
\[\sup_{x\in A^\compl}\big\vert f_{n_k}(x)-f(x)\big\vert\le\frac1k,\]
para todo $k\ge t$. Segue então que $(f_{n_k})_{k\ge1}$ converge uniformemente para $f$ em $A^\compl$.
\end{proof}

A cada uma das noções de convergência que consideramos até agora está associada uma correspondente noção de seqüência
de Cauchy. Enunciamos a seguinte:
\begin{defin}
Seja $X$ um conjunto e $(f_n)_{n\ge1}$ uma seqüência de funções $f_n:X\to\R$. Dizemos que:
\begin{itemize}
\item a seqüência $(f_n)_{n\ge1}$ é {\em pontualmente de Cauchy\/}\index[indice]{sequencia@seqüência!pontualmente de Cauchy}
se para todo $x\in X$ a seqüência $\big(f_n(x)\big)_{n\ge1}$ é de Cauchy em $\R$;
\item a seqüência $(f_n)_{n\ge1}$ é {\em uniformemente de Cauchy\/}\index[indice]{sequencia@seqüência!uniformemente de Cauchy}
se para todo $\varepsilon>0$ existe $n_0\ge1$
tal que $\vert f_n(x)-f_m(x)\vert<\varepsilon$, para todos $n,m\ge n_0$ e todo $x\in X$.
\end{itemize}
Se $(X,\mathcal A,\mu)$ é um espaço de medida, dizemos que:
\begin{itemize}
\item a seqüência $(f_n)_{n\ge1}$ é {\em pontualmente de Cauchy quase sempre\/}\index[indice]{sequencia@seqüência!pontualmente de Cauchy!quase sempre}
se para quase todo $x\in X$ a seqüência
$\big(f_n(x)\big)_{n\ge1}$ é de Cauchy em $\R$, i.e., se existe $X'\in\mathcal A$ com $\mu(X\setminus X')=0$ tal que
$\big(f_n(x)\big)_{n\ge1}$ é de Cauchy em $\R$ para todo $x\in X'$;
\item a seqüência $(f_n)_{n\ge1}$ é {\em uniformemente de Cauchy quase sempre\/}\index[indice]{sequencia@seqüência!uniformemente de Cauchy!quase sempre}
se existe $X'\in\mathcal A$ tal que
$\mu(X\setminus X')=0$ e tal que a seqüência $(f_n\vert_{X'})_{n\ge1}$ é uniformemente de Cauchy;
\item a seqüência $(f_n)_{n\ge1}$ é {\em quase uniformemente de Cauchy\/}\index[indice]{sequencia@seqüência!quase uniformemente de Cauchy}
se para todo $\varepsilon>0$ existe
$A\in\mathcal A$ com $\mu(A)<\varepsilon$ de modo que a seqüência $(f_n\vert_{A^\compl})_{n\ge1}$ é uniformemente
de Cauchy.
\end{itemize}
Supondo também que as funções $f_n$ são todas mensuráveis, dizemos que a seqüência $(f_n)_{n\ge1}$ é
{\em de Cauchy em medida\/}\index[indice]{sequencia@seqüência!de Cauchy em medida}
se para todo $\varepsilon>0$ e todo $\eta>0$, existe $n_0\ge1$ tal que:
\[\mu\Big(\big\{x\in X:\big\vert f_n(x)-f_m(x)\big\vert\ge\varepsilon\big\}\Big)<\eta,\]
para todos $n,m\ge n_0$.
\end{defin}
Evidentemente, toda seqüência uniformemente de Cauchy (resp., quase sempre) é pontualmente de Cauchy (resp., quase sempre)
e toda seqüência pontualmente convergente (resp., quase sempre) é pontualmente de Cauchy (resp., quase sempre).
Além do mais, se $(f_n)_{n\ge1}$ é uma seqüência pontualmente de Cauchy então existe uma (única) função $f$
tal que $f_n\to f$ pontualmente. Outras propriedades simples dos vários tipos de seqüências de Cauchy
definidos acima são exploradas nos Exercícios~\ref{exe:Cauchy1}, \ref{exe:Cauchy2}, \ref{exe:Cauchy3},
\ref{exe:Cauchy4}, \ref{exe:Cauchy5}, \ref{exe:Cauchy6}.

Temos a seguinte versão do Lema~\ref{thm:medsubseqqu} para seqüências de Cauchy.
\begin{lem}\label{thm:CauchyMedida}
Sejam $(X,\mathcal A,\mu)$ um espaço de medida e $(f_n)_{n\ge1}$ uma se\-qüên\-cia de funções mensuráveis $f_n:X\to\R$
que seja de Cauchy em medida. Então existe uma subseqüência $(f_{n_k})_{k\ge1}$ de $(f_n)_{n\ge1}$ que é
quase uniformemente de Cauchy; em particular, pelo resultado dos Exercícios~\ref{exe:Cauchy1},
\ref{exe:Cauchy2} e \ref{exe:Cauchy3}, $(f_{n_k})_{k\ge1}$ converge
quase uniformemente (e também converge pontualmente quase sempre) para uma função mensurável
$f:X\to\R$.
\end{lem}
\begin{proof}
Vamos contruir indutivamente uma seqüência de ín\-di\-ces $(n_k)_{k\ge1}$ tal que $n_1<n_2<\cdots$ e tal que:
\begin{equation}\label{eq:fnkfnkplus1}
\mu\Big(\big\{x\in X:\big\vert f_{n_k}(x)-f_{n_{k+1}}(x)\big\vert\ge\tfrac1{2^k}\big\}\Big)<\frac1{2^k},
\end{equation}
para todo $k\ge1$. Como $(f_n)_{n\ge1}$ é de Cauchy em medida, podemos escolher $n_1\ge1$ tal que:
\[\mu\Big(\big\{x\in X:\big\vert f_n(x)-f_m(x)\big\vert\ge\tfrac12\big\}\Big)<\frac12,\]
para todos $n,m\ge n_1$. Supondo $n_k$ já definido, escolhemos $n_{k+1}>n_k$ tal que:
\[\mu\Big(\big\{x\in X:\big\vert f_n(x)-f_m(x)\big\vert\ge\tfrac1{2^{k+1}}\big\}\Big)<\frac1{2^{k+1}},\]
para todos $n,m\ge n_{k+1}$. É fácil ver que a seqüência $(n_k)_{n\ge1}$ assim construída satisfaz
\eqref{eq:fnkfnkplus1}, para todo $k\ge1$. Vamos mostrar que a seqüência $(f_{n_k})_{k\ge1}$ é quase uniformemente
de Cauchy. Seja dado $\varepsilon>0$; escolha $t\ge1$ com:
\[\sum_{k=t}^\infty\frac1{2^k}=\frac1{2^{t-1}}\le\varepsilon\]
e defina:
\[A=\bigcup_{k=t}^\infty\big\{x\in X:\big\vert f_{n_k}(x)-f_{n_{k+1}}(x)\big\vert\ge\tfrac1{2^k}\big\}.\]
Claramente, $\mu(A)<\sum_{k=t}^\infty\frac1{2^k}\le\varepsilon$. Vamos mostrar que a seqüência
$(f_{n_k})_{k\ge1}$ é uniformemente de Cauchy em $A^\compl$. Se $x\in A^\compl$ então:
\[\big\vert f_{n_k}(x)-f_{n_{k+1}}(x)\big\vert<\frac1{2^k},\]
para todo $k\ge t$. Daí, se $l\ge k\ge t$, temos:
\[\big\vert f_{n_k}(x)-f_{n_l}(x)\big\vert\le\sum_{i=k}^{l-1}\big\vert f_{n_i}(x)-f_{n_{i+1}}(x)\big\vert
\le\sum_{i=k}^{l-1}\frac1{2^i}<\sum_{i=k}^\infty\frac1{2^i}=\frac1{2^{k-1}},\]
para todo $x\in A^\compl$. Conclui-se então que a seqüência $(f_{n_k})_{k\ge1}$ é uniformemente de Cauchy em $A^\compl$;
de fato, dado $\eta>0$, escolhemos $k_0\ge t$ com $\frac1{2^{k_0-1}}\le\eta$ e daí:
\[\big\vert f_{n_k}(x)-f_{n_l}(x)\big\vert<\frac1{2^{k_0-1}}\le\eta,\]
para todo $x\in A^\compl$ e todos $k,l\ge k_0$.
\end{proof}

\begin{cor}
Toda seqüência de Cauchy em medida de funções mensuráveis converge em medida para alguma função mensurável.
\end{cor}
\begin{proof}
Se $(f_n)_{n\ge1}$ é de Cauchy em medida então, pelo Lema~\ref{thm:CauchyMedida}, existe uma subseqüência
$(f_{n_k})_{k\ge1}$ que converge quase uniformemente para uma função mensurável $f$. Mas, pelo Lema~\ref{thm:convquimplmu},
isso implica que $(f_{n_k})_{k\ge1}$ converge em medida para $f$. Segue então do resultado do Exercício~\ref{exe:Cauchysubseq}
que $(f_n)_{n\ge1}$ converge em medida para $f$.
\end{proof}

\end{section}

\begin{section}[O Teorema de Fubini em $\R^n$]{O Teorema de Fubini em ${\R^n}$}
\label{sec:Fubini}

Ao longo desta seção consideramos fixados inteiros positivos $m$ e $n$ e identificamos
$\R^{m+n}$ com o produto $\R^m\times\R^n$ através da aplicação:
\begin{equation}\label{eq:identRmRnRmn}
\R^m\times\R^n\ni(x,y)\longmapsto(x_1,\ldots,x_m,y_1,\ldots,y_n)\in\R^{m+n}.
\end{equation}
Dado um subconjunto $A$ de $\R^{m+n}$
e dado $x\in\R^m$ denotamos por $A_x$ a {\em fatia vertical de $A$ correspondente à
abscissa $x$\/}\index[indice]{fatia!vertical} definida por:
\[A_x=\big\{y\in\R^n:(x,y)\in A\big\}.\]
Se $i_x:\R^n\to\R^{m+n}$ denota a função $i_x(y)=(x,y)$ então obviamente:\index[simbolos]{$i_x$}\index[simbolos]{$A_x$}
\begin{equation}\label{eq:Axix}
A_x=i_x^{-1}(A),
\end{equation}
para todo $x\in\R^m$. Temos portanto o seguinte:
\begin{lem}\label{thm:fatiasBorel}
Se $A$ é um Boreleano de $\R^{m+n}$ então $A_x$ é um Boreleano de $\R^n$ para todo $x\in\R^m$.
\end{lem}
\begin{proof}
A função $i_x$ é contínua e portanto Borel mensurável (veja Lema~\ref{thm:contmens}).
A conclusão segue de \eqref{eq:Axix}.
\end{proof}

Segue do Lema~\ref{thm:fatiasBorel} que se $A$ é um Boreleano de $\R^{m+n}$ então
faz sentido considerar a medida de Lebesgue $\leb(A_x)$ da fatia $A_x$, para cada
$x\in\R^m$.
\begin{lem}\label{thm:integralfatias}
Se $A$ é um Boreleano de $\R^{m+n}$ então a função:
\begin{equation}\label{eq:funcaomedidafatia}
\R^m\ni x\longmapsto\leb(A_x)\in[0,+\infty]
\end{equation}
é mensurável e vale a igualdade:
\begin{equation}\label{eq:integralfatias}
\int_{\R^m}\leb(A_x)\,\dd\leb(x)=\leb(A).
\end{equation}
\end{lem}

Note que usamos a notação $\leb$ indistintamente para a medida de Lebesgue de
$\R^m$, $\R^n$ e $\R^{m+n}$;
mais especificamente, em \eqref{eq:funcaomedidafatia} usamos a medida de Lebesgue de $\R^n$,
a integral do lado esquerdo da igualdade em \eqref{eq:integralfatias} é feita com respeito à
medida de Lebesgue de $\R^m$ e no lado direito da igualdade em \eqref{eq:integralfatias}
usamos a medida de Lebesgue de $\R^{m+n}$.

\begin{proof}[Demonstração do Lema~\ref{thm:integralfatias}]
Denote por $\mathcal C$ a coleção de todos os Boreleanos $A$ de $\R^{m+n}$ para os quais
a função \eqref{eq:funcaomedidafatia} é mensurável e a igualdade \eqref{eq:integralfatias}
é satisfeita. A idéia da prova é mostrar várias propriedades da coleção $\mathcal C$
até que finalmente concluímos que ela coincide com a classe de todos os Boreleanos de $\R^{m+n}$.

\begin{stepindent}
\item\label{itm:fatia1}
{\em Os blocos retangulares $(m+n)$-dimensionais pertencem a $\mathcal C$}.

Se $A$ é um bloco retangular $(m+n)$-dimensional então podemos escrever $A=A_1\times A_2$,
onde $A_1$ e $A_2$ são respectivamente um bloco retangular $m$-dimensional e um bloco
retangular $n$-dimensional. Para todo $x\in\R^m$, temos:
\[A_x=\begin{cases}
A_2,&\text{se $x\in A_1$},\\
\hfil\emptyset,&\text{se $x\not\in A_1$},
\end{cases}\]
e portanto:
\[\leb(A_x)=\vert A_2\vert\,\chilow{A_1}(x),\]
para todo $x\in\R^m$. Segue que \eqref{eq:funcaomedidafatia} é uma função
simples mensurável cuja integral é igual a $\vert A_2\vert\,\vert A_1\vert=\vert A\vert$.

\item\label{itm:fatia2}
{\em Se $A,B\in\mathcal C$ e $A$ e $B$ são disjuntos então $A\cup B\in\mathcal C$}.

Segue de \eqref{eq:Axix} que $(A\cup B)_x=A_x\cup B_x$ e que $A_x$ e $B_x$ são disjuntos
para todo $x\in\R^m$; logo:
\[\leb\big((A\cup B)_x\big)=\leb(A_x)+\leb(B_x),\]
para todo $x\in\R^m$. Segue que a função $x\mapsto\leb\big((A\cup B)_x\big)$ é mensurável,
sendo uma soma de funções mensuráveis; sua integral é dada por:
\begin{multline*}
\int_{\R^m}\leb\big((A\cup B)_x\big)\,\dd\leb(x)=\int_{\R^m}\leb(A_x)\,\dd\leb(x)+
\int_{\R^m}\leb(B_x)\,\dd\leb(x)\\=\leb(A)+\leb(B)=\leb(A\cup B).
\end{multline*}

\item\label{itm:fatia3}
{\em Se $A,B\in\mathcal C$, $B\subset A$ e $B$ é limitado então $A\setminus B\in\mathcal C$}.

Como $B$ é limitado então $\leb(B)<+\infty$ e $\leb(B_x)<+\infty$, para todo $x\in\R^m$.
Segue de \eqref{eq:Axix} que $B_x\subset A_x$ e $(A\setminus B)_x=A_x\setminus B_x$,
para todo $x\in\R^m$; logo:
\[\leb\big((A\setminus B)_x\big)=\leb(A_x)-\leb(B_x),\]
para todo $x\in\R^m$, provando que a função $x\mapsto\leb\big((A\setminus B)_x\big)$
é mensurável. Além do mais:
\begin{multline*}
\int_{\R^m}\leb\big((A\setminus B)_x\big)\,\dd\leb(x)=\int_{\R^m}\leb(A_x)\,\dd\leb(x)-
\int_{\R^m}\leb(B_x)\,\dd\leb(x)\\=\leb(A)-\leb(B)=\leb(A\setminus B).
\end{multline*}

\item\label{itm:fatia4}
{\em Se $(A^k)_{k\ge1}$ é uma seqüência de elementos de $\mathcal C$ e se $A^k\nearrow A$ então
$A\in\mathcal C$}.

Segue de \eqref{eq:Axix} que $A^k_x\nearrow A_x$, para todo $x\in\R^m$; logo, pelo Lema~\ref{thm:setlimits}:
\[\leb(A_x)=\lim_{k\to\infty}\leb(A^k_x),\]
para todo $x\in\R^m$. Segue que a função $x\mapsto\leb(A_x)$ é mensurável, sendo um limite
de funções mensuráveis. Pelo Teorema da Convergência Monotônica, temos:
\[\int_{\R^m}\leb(A_x)\,\dd\leb(x)=\lim_{k\to\infty}\int_{\R^m}\leb(A^k_x)\,\dd\leb(x)
=\lim_{k\to\infty}\leb(A^k)=\leb(A).\]

\item\label{itm:fatia5}
{\em Se $(A^k)_{k\ge1}$ é uma seqüência de elementos de $\mathcal C$, $A^1$ é limitado
e $A^k\searrow A$ então $A\in\mathcal C$}.

Como $A^1$ é limitado, temos $\leb(A^k)<+\infty$ e $\leb(A^k_x)<+\infty$, para todos
$k\ge1$ e $x\in\R^m$. Essa observação permite demonstrar o passo~\ref{itm:fatia5} de forma
análoga à demonstração do passo~\ref{itm:fatia4}.

\item\label{itm:fatia6}
{\em Se $A,B\in\mathcal C$, $A\cap B\in\mathcal C$ e $A\cap B$ é limitado então
$A\cup B\in\mathcal C$}.

Segue dos passos~\ref{itm:fatia2} e \ref{itm:fatia3}, observando que:
\[A\cup B=(A\setminus B)\cup B=\big(A\setminus(A\cap B)\big)\cup B,\]
sendo que os conjuntos $A\setminus(A\cap B)$ e $B$ são disjuntos.

\item\label{itm:fatia7}
{\em Se $B_1$, \dots, $B_k$ são blocos retangulares $(m+n)$-dimensionais então
$\bigcup_{i=1}^kB_i\in\mathcal C$}.

Usamos indução em $k$. O caso $k=1$ segue do passo~\ref{itm:fatia1}. Suponha que a união
de qualquer coleção de $k$ blocos retangulares $(m+n)$-dimensionais pertence a $\mathcal C$
e sejam dados blocos retangulares $(m+n)$-dimensionais $B_1$, \dots, $B_{k+1}$.
Como qualquer subconjunto de uma união finita de blocos retangulares é sempre um conjunto
limitado, em virtude do passo~\ref{itm:fatia6}, para mostrar que
$\bigcup_{i=1}^{k+1}B_i=\big(\bigcup_{i=1}^kB_i\big)\cup B_{k+1}$ está em $\mathcal C$ é suficiente
mostrar que $\big(\bigcup_{i=1}^kB_i\big)\cap B_{k+1}$ está em $\mathcal C$. Mas:
\[\Big(\bigcup_{i=1}^kB_i\Big)\cap B_{k+1}=\bigcup_{i=1}^k(B_i\cap B_{k+1}),\]
sendo que $B_i\cap B_{k+1}$ é um bloco retangular $(m+n)$-dimensional para $i=1,\ldots,k$.
Segue da hipótese de indução que $\big(\bigcup_{i=1}^kB_i\big)\cap B_{k+1}\in\mathcal C$.

\item\label{itm:fatia8}
{\em Todo subconjunto aberto de $\R^{m+n}$ pertence a $\mathcal C$}.

Se $U\subset\R^{m+n}$ é aberto então o Lema~\ref{thm:abertocubos} nos permite escrever
$U=\bigcup_{i=1}^\infty B_i$, onde cada $B_i$ é um bloco retangular $(m+n)$-dimensional.
Definindo $A_k=\bigcup_{i=1}^kB_i$ então $A_k\in\mathcal C$, pelo passo~\ref{itm:fatia7}
e $A_k\nearrow U$. A conclusão segue do passo~\ref{itm:fatia4}.

\item\label{itm:fatia9}
{\em Todo subconjunto de $\R^{m+n}$ de tipo $G_\delta$ está em $\mathcal C$}.

Seja $Z\subset\R^{m+n}$ um $G_\delta$. Assumimos inicialmente que $Z$ é limitado.
Seja $(U_k)_{k\ge1}$ uma seqüência de abertos de $\R^{m+n}$ com $Z=\bigcap_{k=1}^\infty U_k$
e seja $U_0$ um aberto limitado de $\R^{m+n}$ que contém $Z$. Definindo:
\[A_k=\bigcap_{i=0}^kU_i,\]
então $A_k$ é um aberto limitado para todo $k\ge1$ e $A_k\searrow Z$. Segue dos passos~\ref{itm:fatia5}
e \ref{itm:fatia8} que $Z\in\mathcal C$.

Seja agora $Z\subset\R^{m+n}$ um $G_\delta$ arbitrário. Temos que
\[Z_k=Z\cap\left]-k,k\right[^{\,m+n}\]
é um $G_\delta$ limitado para todo $k\ge1$ e portanto $Z_k\in\mathcal C$,
pelo que mostramos acima. A conclusão segue do passo~\ref{itm:fatia4}, já que $Z_k\nearrow Z$.

\item\label{itm:fatia10}
{\em A coleção $\mathcal C$ coincide com a coleção de todos os subconjuntos Boreleanos de $\R^{m+n}$}.

Seja $A\subset\R^{m+n}$ um Boreleano. Pelo Lema~\ref{thm:Gdeltazero} existe um subconjunto
$Z$ de $\R^{m+n}$ de tipo $G_\delta$ com $A\subset Z$ e $\leb(Z\setminus A)=0$.
Pelo Lema~\ref{thm:aproxGdelta}, existe um subconjunto $E$ de $\R^{m+n}$ de tipo $G_\delta$
com $Z\setminus A\subset E$ e $\leb(E)=\leb(Z\setminus A)=0$. O passo~\ref{itm:fatia9}
nos garante que $E$ e $Z$ estão em $\mathcal C$. Logo:
\[\int_{\R^m}\leb(E_x)\,\dd\leb(x)=\leb(E)=0;\]
como $\leb(E_x)\ge0$, para todo $x$, o resultado do Exercício~\ref{exe:intzerofzeroqs} implica que
$\leb(E_x)=0$ para quase todo $x\in\R^{m+n}$. Como $(Z\setminus A)_x\subset E_x$, para todo $x\in\R^m$,
segue que $\leb\big((Z\setminus A)_x\big)=0$ para quase todo $x\in\R^m$. Temos então:
\[\leb(Z_x)=\leb(A_x)+\leb\big((Z\setminus A)_x\big)=\leb(A_x),\]
para quase todo $x\in\R^m$, já que $Z_x$ é união disjunta de $A_x$ e $(Z\setminus A)_x$, para
todo $x$. Vemos então que a funções $x\mapsto\leb(Z_x)$ e $x\mapsto\leb(A_x)$ são iguais
quase sempre, o que implica que $x\mapsto\leb(A_x)$ é uma função mensurável
pelo resultado do item (b) do Exercício~\ref{exe:mensqs}. Além do mais:
\[\int_{\R^m}\leb(A_x)\,\dd\leb(x)=\int_{\R^m}\leb(Z_x)\,\dd\leb(x)=\leb(Z)=\leb(A),\]
provando que $A\in\mathcal C$. Isso completa a demonstração.\qedhere
\end{stepindent}
\end{proof}

Se $A$ é um subconjunto mensurável de $\R^{m+n}$ então não é verdade em geral que
as fatias verticais $A_x$ são mensuráveis para todo $x\in\R^m$; por exemplo, se $B$ é um subconjunto
não mensurável de $\R^n$ então $A=\{0\}\times B$ é um subconjunto mensurável de $\R^{m+n}$
(com medida exterior nula), mas a fatia $A_0=B$ não é mensurável. No entanto, mostraremos
abaixo que se $A$ é mensurável então {\em quase todas\/} as fatias $A_x$ de $A$ são mensuráveis.
Faz sentido também então considerar a integral em \eqref{eq:integralfatias}, tendo em mente
a seguinte convenção: se $X$ é um subconjunto mensurável de $\R^n$ e se $f(x)$ é uma expressão
que faz sentido apenas {\em para quase todo\/} $x\in X$ então escrevemos $\int_Xf(x)\,\dd\leb(x)$,
entendendo que {\em valores arbitrários\/} de $\overline\R$ podem ser atribuídos à expressão
$f(x)$ no conjunto de medida nula no qual ela não está definida. Em vista do resultado
do Exercício~\ref{exe:mensqs} e do Corolário~\ref{thm:corXlinhazero}, essa convenção define
o símbolo $\int_Xf(x)\,\dd\leb(x)$ de forma inequívoca.
\begin{prop}\label{thm:quasetodafatia}
Se $A$ é um subconjunto mensurável de $\R^{m+n}$ então para quase todo $x\in\R^m$ a fatia
vertical $A_x$ é um subconjunto mensurável de $\R^n$, a função $x\mapsto\leb(A_x)$ é mensurável
e a medida de $A$ é dada pela igualdade \eqref{eq:integralfatias}.
\end{prop}
\begin{proof}
Basta repetir os argumentos da demonstração do passo~\ref{itm:fatia10} do Lema~\ref{thm:integralfatias};
a única diferença é que não sabemos {\it a priori\/} que as fatias de $A$ são mensuráveis.
Mas sabemos que $E_x$ tem medida nula para quase todo $x\in\R^m$ e portanto $(Z\setminus A)_x$ é
mensurável e tem medida nula para quase todo $x\in\R^m$; como:
\[A_x=Z_x\setminus(Z\setminus A)_x,\]
segue que também $A_x$ é mensurável para quase todo $x\in\R^m$.
\end{proof}

Observamos que se $X$ é um subconjunto mensurável de $\R^m$ e se $Y$ é um subconjunto
mensurável de $\R^n$ então $X\times Y$ é um subconjunto mensurável de $\R^{m+n}$
(veja Exercício~\ref{exe:medLebproduto}).
\begin{teo}[Fubini--Tonelli]\index[indice]{teorema!de Fubini--Tonelli}\index[indice]{Fubini!teorema de}\index[indice]{Tonelli!teorema de}
\label{thm:Fubini}
Sejam $X\subset\R^m$, $Y\subset\R^n$ conjuntos mensuráveis e $f:X\times Y\to\overline\R$
uma função quase integrável. Então:
\begin{itemize}
\item para quase todo $x\in X$, a função $Y\ni y\mapsto f(x,y)\in\overline\R$ é
quase integrável;
\item a função $X\ni x\mapsto\int_Yf(x,y)\,\dd\leb(y)\in\overline\R$ é
quase integrável;
\item vale a igualdade:
\[\int_X\Big(\int_Yf(x,y)\,\dd\leb(y)\Big)\,\dd\leb(x)=\int_{X\times Y}f(x,y)\,\dd\leb(x,y).\]
\end{itemize}
\end{teo}
\begin{proof}
Dividimos a demonstração em itens.
\begin{bulletindent}
\item {\em O teorema vale se $f$ é simples, mensurável e não negativa}.

Podemos escrever $f=\sum_{i=1}^kc_i\chilow{A^i}$, com $c_i\in[0,+\infty]$ e $A^i$
um subconjunto mensurável de $X\times Y$, para $i=1,\ldots,k$. Note que,
se $x\in X$, temos:
\begin{equation}\label{eq:fxysimples}
f(x,y)=\sum_{i=1}^kc_i\chilow{A^i_x}(y),
\end{equation}
para todo $y\in Y$. Pela Proposição~\ref{thm:quasetodafatia}, existe para cada $i=1,\ldots,k$
um conjunto de medida nula $N_i\subset\R^m$ tal que $A^i_x$ é mensurável para todo
$x\in\R^m\setminus N_i$. Daí $N=\bigcup_{i=1}^kN_i$ tem medida nula e segue de
\eqref{eq:fxysimples} que para $x\in\R^m\setminus N$, a função $y\mapsto f(x,y)$ é mensurável
e sua integral é dada por:
\[\int_Yf(x,y)\,\dd\leb(y)=\int_Y\sum_{i=1}^kc_i\chilow{A^i_x}(y)\,\dd\leb(y)
=\sum_{i=1}^kc_i\leb(A^i_x).\]
Logo:
\begin{multline*}
\int_X\Big(\int_Yf(x,y)\,\dd\leb(y)\Big)\,\dd\leb(x)=\int_{\R^m}\sum_{i=1}^kc_i\leb(A^i_x)\,\dd\leb(x)
=\sum_{i=1}^kc_i\leb(A^i)\\
=\int_{X\times Y}f(x,y)\,\dd\leb(x,y).
\end{multline*}

\item {\em O teorema vale se $f$ é mensurável e não negativa}.

Seja $(f_k)_{k\ge1}$ uma seqüências de funções $f_k:X\times Y\to[0,+\infty]$
simples e mensuráveis com $f_k\nearrow f$. Seja $N_k\subset\R^m$ um conjunto de medida
nula tal que a função $y\mapsto f_k(x,y)$ é mensurável para todo $x\in X\setminus N_k$.
Daí $N=\bigcup_{k=1}^\infty N_k$ tem medida nula e a função:
\[Y\ni y\longmapsto f(x,y)=\lim_{k\to\infty}f_k(x,y)\in[0,+\infty]\]
é mensurável para todo $x\in X\setminus N$. Pelo Teorema da Convergência Monotônica,
temos:
\[\int_Yf(x,y)\,\dd\leb(y)=\lim_{k\to\infty}\int_Yf_k(x,y)\,\dd\leb(y),\]
para todo $x\in X\setminus N$. Logo a função $x\mapsto\int_Yf(x,y)\,\dd\leb(y)$
é mensurável e, usando novamente o Teorema da Convergência Monotônica, obtemos:
\begin{multline*}
\int_X\Big(\int_Yf(x,y)\,\dd\leb(y)\Big)\,\dd\leb(x)=\lim_{k\to\infty}
\int_X\Big(\int_Yf_k(x,y)\,\dd\leb(y)\Big)\,\dd\leb(x)\\
=\lim_{k\to\infty}\int_{X\times Y}f_k(x,y)\,\dd\leb(x,y)=\int_{X\times Y}f(x,y)\,\dd\leb(x,y).
\end{multline*}

\item {\em O teorema vale se $f$ é quase integrável}.

Como $f^+$ e $f^-$ são funções mensuráveis não negativas, temos:
\begin{gather}
\int_X\Big(\int_Yf^+(x,y)\,\dd\leb(y)\Big)\,\dd\leb(x)=\int_{X\times Y}f^+(x,y)\,\dd\leb(x,y),\label{eq:Fubinif+}\\
\int_X\Big(\int_Yf^-(x,y)\,\dd\leb(y)\Big)\,\dd\leb(x)=\int_{X\times Y}f^-(x,y)\,\dd\leb(x,y).\label{eq:Fubinif-}
\end{gather}
Como $f$ é quase integrável, temos que $f^+$ é integrável ou $f^-$ é integrável;
para fixar as idéias, vamos supor que $\int_{X\times Y}f^-\,\dd\leb<+\infty$.
Tendo em mente o resultado do Exercício~\ref{exe:finitaqs}, segue de \eqref{eq:Fubinif-} que:
\[\int_Yf^-(x,y)\,\dd\leb(y)<+\infty,\]
para quase todo $x\in X$. Segue que a função $y\mapsto f(x,y)$ é quase integrável
para quase todo $x\in X$; além do mais, de \eqref{eq:Fubinif+} e \eqref{eq:Fubinif-} vem:
\begin{multline*}
\begin{aligned}\int_X\Big(\int_Yf(x,y)\,\dd\leb(y)\Big)\,\dd\leb(x)&=
\int_X\Big(\int_Yf^+(x,y)\,\dd\leb(y)\Big)\,\dd\leb(x)\\
&-\int_X\Big(\int_Yf^-(x,y)\,\dd\leb(y)\Big)\,\dd\leb(x)
\end{aligned}\\
=\int_{X\times Y}f^+(x,y)\,\dd\leb(x,y)-\int_{X\times Y}f^-(x,y)\,\dd\leb(x,y)\\
=\int_{X\times Y}f(x,y)\,\dd\leb(x,y).\qedhere
\end{multline*}
\end{bulletindent}
\end{proof}

Seja $\sigma:\{1,\ldots,m+n\}\to\{1,\ldots,m+n\}$ uma aplicação bijetora (i.e., uma
{\em permutação\/}\index[indice]{permutacao@permutação} de $m+n$ elementos) e considere
o isomorfismo linear $\widehat\sigma$ de $\R^{m+n}$ definido por:
\[\widehat\sigma(z_1,\ldots,z_{m+n})=(z_{\sigma(1)},\ldots,z_{\sigma(m+n)}),\]
para todo $(z_1,\ldots,z_{m+n})\in\R^{m+n}$.
Segue do resultado do Exercício~\ref{exe:permutacao} que $\widehat\sigma$ {\em preserva medida},
i.e., $\leb\big(\widehat\sigma^{-1}(A)\big)=\leb(A)$, para todo subconjunto mensurável
$A$ de $\R^{m+n}$ (veja Definição~\ref{thm:measurepres}). Pelo resultado do Exercício~\ref{exe:measurepreserving},
uma função $f:\R^{m+n}\to\overline\R$ é quase integrável se e somente se
$f\circ\widehat\sigma$ é quase integrável e, nesse caso, as integrais de $f$
e $f\circ\widehat\sigma$ coincidem. Em vista dessas observações, temos o seguinte:
\begin{cor}\label{thm:Fubinireverso}
Sejam $X\subset\R^m$, $Y\subset\R^n$ conjuntos mensuráveis e $f:X\times Y\to\overline\R$
uma função quase integrável. Então:
\begin{itemize}
\item para quase todo $y\in Y$, a função $X\ni x\mapsto f(x,y)\in\overline\R$ é
quase integrável;
\item a função $y\mapsto\int_Xf(x,y)\,\dd\leb(x)\in\overline\R$ é
quase integrável;
\item vale a igualdade:
\begin{align*}
\int_Y\Big(\int_Xf(x,y)\,\dd\leb(x)\Big)\,\dd\leb(y)&=\int_{X\times Y}f(x,y)\,\dd\leb(x,y)\\
&=\int_X\Big(\int_Yf(x,y)\,\dd\leb(y)\Big)\,\dd\leb(x).
\end{align*}
\end{itemize}
\end{cor}
\begin{proof}
Considere a permutação $\sigma$ de $m+n$ elementos dada por:
\[\sigma(i)=\begin{cases}
n+i,&\text{se $1\le i\le m$},\\
i-m,&\text{se $m+1\le i\le m+n$},
\end{cases}\]
de modo que:
\[\widehat\sigma(y_1,\ldots,y_n,x_1,\ldots,x_m)=(x_1,\ldots,x_m,y_1,\ldots,y_n),\]
para todos $x\in\R^m$, $y\in\R^n$. Temos que:
\[\widehat\sigma^{-1}(X\times Y)=Y\times X\subset\R^n\times\R^m\cong\R^{m+n}.\]
Em vista das observações que precedem o enunciado
do corolário, temos que $f\circ\widehat\sigma\vert_{Y\times X}:Y\times X\to\overline\R$ é quase integrável
e tem a mesma integral que $f$. A conclusão é obtida aplicando o Teorema~\ref{thm:Fubini}
à função $f\circ\widehat\sigma\vert_{Y\times X}$, trocando os papéis de $m$ e $n$.
\end{proof}

É possível que uma função mensurável $f:X\times Y\to\overline\R$ seja tal que as {\em
integrais iteradas\/}\index[indice]{integrais iteradas} $\int_X\big(\int_Yf(x,y)\,\dd\leb(y)\big)\,\dd\leb(x)$
e $\int_Y\big(\int_Xf(x,y)\,\dd\leb(x)\big)\,\dd\leb(y)$ sejam ambas bem-definidas,
porém distintas; em vista do Corolário~\ref{thm:Fubinireverso}, isso somente é possível
quando a função $f$ não é quase integrável.
\begin{example}
Seja $(a_{ij})_{i,j\ge1}$ uma seqüência dupla de números reais tal que as séries:
\begin{gather}
\sum_{j=1}^\infty a_{ij},\quad i=1,2,\ldots,\qquad
\sum_{i=1}^\infty a_{ij},\quad j=1,2,\ldots,\label{eq:series1}\\
\sum_{i=1}^\infty\Big(\sum_{j=1}^\infty a_{ij}\Big),\qquad
\sum_{j=1}^\infty\Big(\sum_{i=1}^\infty a_{ij}\Big),\label{eq:series2}
\end{gather}
são todas absolutamente convergentes, mas:
\[\sum_{i=1}^\infty\Big(\sum_{j=1}^\infty a_{ij}\Big)\ne\sum_{j=1}^\infty\Big(\sum_{i=1}^\infty a_{ij}\Big).\]
Tome, por exemplo:
\[a_{ij}=\begin{cases}
1,&\text{se $i=j$},\\
-1,&\text{se $i+1=j$},\\
0,&\text{caso contrário},
\end{cases}\]
de modo que todas as séries em \eqref{eq:series1} e \eqref{eq:series2}
têm apenas um número finito de termos não nulos e:
\[\sum_{i=1}^\infty\Big(\sum_{j=1}^\infty a_{ij}\Big)=0,\quad
\sum_{j=1}^\infty\Big(\sum_{i=1}^\infty a_{ij}\Big)=1.\]
Considere a função $f:\left[0,+\infty\right[\times\left[0,+\infty\right[\to\R$
definida por:
\[f=\sum_{i,j=1}^\infty a_{ij}\,\chilow{\left[i-1,i\right[\times\left[j-1,j\right[},\]
ou seja, a restrição de $f$ ao retângulo $\left[i-1,i\right[\times\left[j-1,j\right[$
é igual a $a_{ij}$, para todos $i,j\ge1$. Fixado $x\in\left[0,+\infty\right[$ então:
\[f(x,y)=\sum_{j=1}^\infty a_{ij}\,\chilow{\left[j-1,j\right[}(y),\]
para todo $y\in\left[0,+\infty\right[$, onde $i\ge1$ é tal que $x\in\left[i-1,i\right[$.
Como a série $\sum_{j=1}^\infty a_{ij}$ é absolutamente convergente, segue do
resultado do Exercício~\ref{exe:intseries} que a função $y\mapsto f(x,y)$ é integrável e:
\[\int_0^{+\infty}f(x,y)\,\dd\leb(y)=\sum_{j=1}^\infty a_{ij};\]
daí:
\[\int_0^{+\infty}f(x,y)\,\dd\leb(y)=\sum_{i=1}^\infty\Big(\sum_{j=1}^\infty a_{ij}\Big)
\chilow{\left[i-1,i\right[}(x),\]
para todo $x\in\left[0,+\infty\right[$. Como a série $\sum_{i=1}^\infty\big(\sum_{j=1}^\infty a_{ij}\big)$
é absolutamente convergente, usando novamente o resultado do Exercício~\ref{exe:intseries}, concluímos que
a função $x\mapsto\int_0^{+\infty}f(x,y)\,\dd\leb(y)$ é integrável e:
\[\int_0^{+\infty}\Big(\int_0^{+\infty}f(x,y)\,\dd\leb(y)\Big)\,\dd\leb(x)=
\sum_{i=1}^\infty\Big(\sum_{j=1}^\infty a_{ij}\Big).\]
De modo análogo, mostra-se que:
\[\int_0^{+\infty}\Big(\int_0^{+\infty}f(x,y)\,\dd\leb(x)\Big)\,\dd\leb(y)=
\sum_{j=1}^\infty\Big(\sum_{i=1}^\infty a_{ij}\Big),\]
e portanto:
\[\int_0^{+\infty}\Big(\int_0^{+\infty}f(x,y)\,\dd\leb(y)\Big)\,\dd\leb(x)\ne
\int_0^{+\infty}\Big(\int_0^{+\infty}f(x,y)\,\dd\leb(x)\Big)\,\dd\leb(y).\]
\end{example}

\end{section}

\section*{Exercícios para o Capítulo~\ref{CHP:INTEGRAL}}

\subsection*{Funções Mensuráveis}

\begin{exercise}\label{exe:constmens}
Sejam $(X,\mathcal A)$, $(X',\mathcal A')$ espaços mensuráveis arbitrários. Mostre que
toda função constante $f:X\to X'$ é mensurável.
\end{exercise}

\begin{exercise}\label{exe:AvertY}
Sejam $X$ um conjunto, $\mathcal A$ uma $\sigma$-álgebra de partes de $X$ e $Y\subset X$
um subconjunto. Mostre que $\mathcal A\vert_Y$ é uma $\sigma$-álgebra de partes de $Y$.
\end{exercise}

\begin{exercise}\label{exe:gerarestrita}
Sejam $X$ um conjunto e $Y\subset X$ um subconjunto. Se $\mathcal C$ é um conjunto de geradores
para uma $\sigma$-álgebra $\mathcal A$ de partes de $X$, mostre que o conjunto:
\[\mathcal C\vert_Y=\big\{E\cap Y:E\in\mathcal C\big\}\]
é um conjunto de geradores para a $\sigma$-álgebra $\mathcal A\vert_Y$ de partes de $Y$; em símbolos:
\[\sigma[\mathcal C]\vert_Y=\sigma[\mathcal C\vert_Y].\]
\end{exercise}

\begin{exercise}\label{exe:BorelbarRrestrR}
Mostre que $\Borel(\overline\R)\vert_{\R}=\Borel(\R)$.
\end{exercise}

\begin{exercise}\label{exe:geradoresbarR}
Mostre que os intervalos $[-\infty,c]$, $c\in\R$, constituem um conjunto de geradores
para a $\sigma$-álgebra de Borel de $\overline\R$.
\end{exercise}

\begin{exercise}\label{exe:figualg}
Seja $(X,\mathcal A)$ um espaço mensurável e sejam $f:X\to\overline\R$, $g:X\to\overline\R$
funções mensuráveis. Mostre que o conjunto:
\[\big\{x\in X:f(x)=g(x)\big\}\]
é mensurável.
\end{exercise}

\begin{exercise}\label{exe:funcaomaluca}
Mostre que a função $f:\R^2\to\R$ definida por:
\[f(x,y)=\begin{cases}
\hfil\cos\frac xy,&\text{se $y\ge1$},\\[3pt]
\sum_{n=1}^\infty\frac{y^n}{n^2},&\text{se $-1<y<1$},\\
\chilow{\mathbb Q}(x+y),&\text{se $y\le-1$},
\end{cases}\]
é Borel mensurável.
\end{exercise}

\begin{exercise}\label{exe:mensqs}
Sejam $X\subset\R^n$ um subconjunto mensurável e $(X',\mathcal A')$ um espaço mensurável.
Dada uma função $f:X\to X'$, mostre que:
\begin{itemize}
\item[(a)] se existe $X_1\subset X$ tal que $X\setminus X_1$ tem medida nula e
tal que $f\vert_{X_1}$ é mensurável então $f$ é mensurável;
\item[(b)] se $f$ é mensurável e se $g:X\to X'$ é igual a $f$ quase sempre então $g$
também é mensurável;
\item[(c)] se $(f_k)_{k\ge1}$ é uma seqüência de funções mensuráveis $f_k:X\to\overline\R$
e se $f_k\to g$ \qs\ então $g:X\to\overline\R$ também é mensurável.
\end{itemize}
\end{exercise}

\begin{exercise}\label{thm:projLebmens}
Denote por $\pi:\R^{m+n}\to\R^m$ a projeção nas $m$ primeiras coordenadas. Mostre que
a função:
\[\pi:\big(\R^{m+n},\Lebmens(\R^{m+n})\big)\longrightarrow\big(\R^m,\Lebmens(\R^m)\big),\]
é mensurável (note que não estamos seguindo a convenção~\ref{cnv:menscontrRn}).
\end{exercise}

\begin{exercise}\label{thm:graficomens}
Seja $f:X\to\R^n$ uma função definida num subconjunto $X$ de $\R^m$. Recorde que o
{\em gráfico\/}\index[indice]{grafico@gráfico}\index[indice]{funcao@função!grafico de@gráfico de}
de $f$ é o conjunto:\index[simbolos]{$\Gr(f)$}
\begin{equation}\label{eq:defGrf}
\Gr(f)=\big\{\big(x,f(x)\big):x\in X\big\}\subset\R^{m+n}.
\end{equation}
Mostre que:
\begin{itemize}
\item se $X$ é Boreleano e $f$ é Borel mensurável então $\Gr(f)$ é Boreleano;
\item se $X$ é mensurável e $f$ é mensurável então $\Gr(f)$ é mensurável.
\end{itemize}
\end{exercise}

\subsection*{Definição da Integral}

\begin{exercise}\label{exe:moduloint}
Sejam $(X,\mathcal A,\mu)$ um espaço de medida e $f:X\to\overline\R$ uma função mensurável.
Mostre que:
\begin{itemize}
\item[(a)] $f$ é integrável se e somente se $\vert f\vert$ é integrável;
\item[(b)] se $f$ é quase integrável então:
\[\Big\vert\int_Xf\,\dd\mu\Big\vert\le\int_X\vert f\vert\,\dd\mu.\]
\end{itemize}
\end{exercise}

\begin{exercise}\label{exe:intserienneg}
Seja $(X,\mathcal A,\mu)$ um espaço de medida e seja $(f_k)_{k\ge1}$ uma seqüência de funções
mensuráveis $f_k:X\to[0,+\infty]$. Se $f(x)=\sum_{k=1}^\infty f_k(x)$, mostre que:
\[\int_X f\,\dd\mu=\sum_{k=1}^\infty\int_X f_k\,\dd\mu.\]
\end{exercise}

\begin{exercise}\label{exe:integralindef}
Seja $(X,\mathcal A,\mu)$ um espaço de medida. Dada uma função mensurável $f:X\to[0,+\infty]$,
mostre que a aplicação $\nu_f:\mathcal A\to[0,+\infty]$ definida por:
\[\nu_f(E)=\int_Ef\,\dd\mu,\quad E\in\mathcal A,\]
é uma medida (a medida $\nu_f$ é chamada a {\em integral indefinida\/}\index[indice]{integral!indefinida}
de $f$ e é denotada por $\nu_f=\int f\,\dd\mu$\index[simbolos]{$\int f\,\dd\mu$}).
\end{exercise}

\begin{exercise}\label{exe:propintegralmedida}
Sejam $(X,\mathcal A,\mu)$ um espaço de medida e $f:X\to\overline\R$ uma função
quase integrável. Mostre que:
\begin{itemize}
\item[(a)] se $(A_k)_{k\ge1}$ é uma seqüência de conjuntos mensuráveis dois a dois disjuntos
e se $A=\bigcup_{k=1}^\infty A_k$ então:
\[\int_Af\,\dd\mu=\sum_{k=1}^\infty\int_{A_k}f\,\dd\mu\stackrel{\text{def}}=
\lim_{r\to\infty}\sum_{k=1}^r\int_{A_k}f\,\dd\mu;\]
\item[(b)] se $(A_k)_{k\ge1}$ é uma seqüência de conjuntos mensuráveis e $A_k\nearrow A$ então:
\begin{equation}\label{eq:integrallimite}
\int_Af\,\dd\mu=\lim_{k\to\infty}\int_{A_k}f\,\dd\mu;
\end{equation}
\item[(c)] se $(A_k)_{k\ge1}$ é uma seqüência de conjuntos mensuráveis, $A_k\searrow A$
e se $f\vert_{A_1}$ é integrável então vale a igualdade \eqref{eq:integrallimite}.
\end{itemize}
\end{exercise}

\begin{exdefin}\label{thm:measurepres}
Sejam $(X,\mathcal A,\mu)$ e $(X',\mathcal A',\mu')$ espaços de medida. Dizemos que uma função
$\phi:X\to X'$ {\em preserva medida\/}\index[indice]{funcao@função!que preserva medida}\index[indice]{preserva medida!funcao que@função que}
se $\phi$ é mensurável e se $\mu\big(\phi^{-1}(A)\big)=\mu'(A)$, para todo $A\in\mathcal A'$.
\end{exdefin}

\begin{exercise}\label{exe:phistarmu}
Sejam $(X,\mathcal A,\mu)$ um espaço de medida, $(X',\mathcal A')$ um espaço mensurável e $\phi:X\to X'$ uma aplicação
mensurável. Mostre que:
\begin{itemize}
\item[(a)] a aplicação $(\phi_*\mu):\mathcal A'\to[0,+\infty]$ definida por:\index[simbolos]{$\phi_*\mu$}
\[(\phi_*\mu)(A)=\mu\big(\phi^{-1}(A)\big),\]
para todo $A\in\mathcal A'$, é uma medida em $\mathcal A'$;
\item[(b)] se $\mu'$ é uma medida em $\mathcal A'$ então $\phi:(X,\mathcal A,\mu)\to(X',\mathcal A',\mu')$ preserva
medida se e somente se $\phi_*\mu=\mu'$.
\end{itemize}
Dizemos que $\phi_*\mu$ é a {\em imagem\/}\index[indice]{imagem!de uma medida}\index[indice]{medida!imagem de} da medida $\mu$
pela aplicação mensurável $\phi$.
\end{exercise}

\begin{exercise}\label{exe:measurepreserving}
Sejam $(X,\mathcal A,\mu)$ e $(X',\mathcal A',\mu')$ espaços de medida e seja $\phi:X\to X'$
uma função que preserva medida. Dada uma função mensurável $f:X'\to\overline\R$, mostre
que $f$ é quase integrável se e somente se $f\circ\phi$ é quase integrável e, nesse caso:
\[\int_{X'}f\,\dd\mu'=\int_Xf\circ\phi\,\dd\mu.\]
\end{exercise}

\begin{exercise}\label{exe:sigmaalgmenor}
Sejam $(X,\mathcal A,\mu)$ um espaço de medida, $\mathcal A'$ uma $\sigma$-álgebra de partes de $X$ contida
em $\mathcal A$ e $\mu'$ a restrição de $\mu$ a $\mathcal A'$. Dada uma função mensurável
$f:(X,\mathcal A')\to\overline\R$, mostre que $f$ é quase integrável com respeito a $\mu$ se e somente se
$f$ é quase integrável com respeito a $\mu'$ e, nesse caso $\int_Xf\,\dd\mu=\int_Xf\,\dd\mu'$.
\end{exercise}

\begin{exdefin}\label{thm:defmedcont}
Seja $X$ um conjunto. A aplicação $\mu:\wp(X)\to[0,+\infty]$ definida por:
\[\mu(E)=\text{número de elementos do conjunto $E$},\quad E\subset X,\]
é chamada a {\em medida de contagem}\index[indice]{medida!de contagem}.
\end{exdefin}

\begin{exercise}
Seja $X$ o conjunto dos números inteiros positivos e seja $\mu:\wp(X)\to[0,+\infty]$ a medida
de contagem. Mostre que:
\begin{itemize}
\item dada uma função $f:X\to[0,+\infty]$ então:
\begin{equation}\label{eq:intXcontagem}
\int_Xf\,\dd\mu=\sum_{n=1}^\infty f(n);
\end{equation}
\item uma função $f:X\to\overline\R$ é integrável se e somente se
a série $\sum_{n=1}^\infty f(n)$ é absolutamente convergente
e nesse caso vale a identidade \eqref{eq:intXcontagem}.
\end{itemize}
\end{exercise}

\begin{exercise}\label{exe:finitaqs}
Sejam $(X,\mathcal A,\mu)$ um espaço de medida e $f:X\to\overline\R$ uma função quase integrável.
Mostre que:
\begin{itemize}
\item se $\int_Xf\,\dd\mu<+\infty$ então $f(x)<+\infty$ para quase todo $x\in X$;
\item se $\int_Xf\,\dd\mu>-\infty$ então $f(x)>-\infty$ para quase todo $x\in X$;
\item se $f$ é integrável então $f(x)\in\R$ para quase todo $x\in X$.
\end{itemize}
\end{exercise}

\begin{exercise}\label{exe:implicaquaseint}
Sejam $(X,\mathcal A,\mu)$ um espaço de medida e $f:X\to\overline\R$, $g:X\to\overline\R$ funções mensuráveis,
com $g$ quase integrável. Mostre que:
\begin{itemize}
\item se $\int_Xg\,\dd\mu>-\infty$ e $f\ge g$ \qs\ então $f$ é quase integrável e $\int_Xf\,\dd\mu>-\infty$;
\item se $\int_Xg\,\dd\mu<+\infty$ e $f\le g$ \qs\ então $f$ é quase integrável e $\int_Xf\,\dd\mu<+\infty$;
\item se $g$ é integrável e $\vert f\vert\le g$ \qs\ então $f$ é integrável.
\end{itemize}
\end{exercise}

\begin{exercise}\label{exe:intzerofzeroqs}
Seja $(X,\mathcal A,\mu)$ um espaço de medida. Dada uma função mensurável
$f:X\to[0,+\infty]$, mostre que $\int_Xf\,\dd\mu=0$ se e somente se $f=0$ quase sempre.
\end{exercise}

\begin{exercise}\label{exe:flegintigualqs}
Seja $(X,\mathcal A,\mu)$ um espaço de medida. Dadas funções integráveis
$f:X\to\overline\R$, $g:X\to\overline\R$ tais que $f\le g$ e:
\[\int_Xf\,\dd\mu=\int_Xg\,\dd\mu,\]
mostre que $f=g$ quase sempre.
\end{exercise}

\begin{exercise}
Sejam $(X,\mathcal A,\mu)$ um espaço de medida e $f:X\to\overline\R$ uma função
integrável. Mostre que para todo $\varepsilon>0$ existe um $\delta>0$ tal que para
todo conjunto mensurável $A\in\mathcal A$ com $\mu(A)<\delta$ temos:
\[\Big\vert\int_Af\,\dd\mu\Big\vert<\varepsilon.\]
\end{exercise}

\begin{exercise}
Seja $f:I\to\overline\R$ uma função integrável definida num intervalo $I\subset\R$.
Fixado $t_0\in I$, considere a função $F:I\to\R$ definida por:
\[F(t)=\int_{t_0}^tf\,\dd\leb,\]
para todo $t\in I$. Mostre que:
\begin{itemize}
\item[(a)] $F$ é contínua;
\item[(b)] dado $\varepsilon>0$, existe $\delta>0$ tal que dados $n\ge1$ e
intervalos abertos dois a dois disjuntos $\left]x_i,y_i\right[\subset I$, $i=1,\ldots,n$,
então:
\[\sum_{i=1}^ny_i-x_i<\delta\Longrightarrow\sum_{i=1}^n\vert F(y_i)-F(x_i)\vert<\varepsilon;\]
\item[(c)] se $f$ é limitada então $F$ é Lipschitziana com constante de Lipschitz igual a
$\sup_{t\in I}\vert f(t)\vert$;
\item[(d)] ({\em teorema fundamental do cálculo})\index[indice]{teorema!fundamental do calculo@fundamental do cálculo}
se $f$ é contínua num ponto $t\in I$ então $F$ é derivável no ponto $t$
e $F'(t)=f(t)$;
\item[(e)] se $f$ é contínua e $G:I\to\R$ é uma primitiva qualquer de $f$ (i.e., $G'=f$)
então:
\[\int_a^bf\,\dd\leb=G(b)-G(a),\]
para todos $a,b\in I$.
\end{itemize}
\end{exercise}

\begin{exercise}
({\em integração por partes})\index[indice]{integracao por partes@integração por partes}
Se $f:[a,b]\to\R$, $g:[a,b]\to\R$ são funções de classe $C^1$, mostre que:
\[\int_a^bf(x)g'(x)\,\dd\leb(x)=f(b)g(b)-f(a)g(a)-\int_a^bf'(x)g(x)\,\dd\leb(x).\]
\end{exercise}

\subsection*{Teoremas de Convergência}

\begin{exercise}\label{exe:intconvunif}
Sejam $(X,\mathcal A,\mu)$ um espaço de medida e $(f_k)_{k\ge1}$ uma seqüência
de funções integráveis (resp., quase integráveis) $f_k:X\to\R$. Suponha que $(f_k)_{k\ge1}$
{\em converge uniformemente\/}\index[indice]{convergencia@convergência!uniforme}\index[indice]{uniforme!convergencia@convergência}%
\index[indice]{sequencia@seqüência!uniformemente convergente@uniformemente conver-\hfil\break gente} para uma função $f:X\to\R$, i.e.,
para todo $\varepsilon>0$ existe $k_0\ge1$ tal que
$\vert f_k(x)-f(x)\vert<\varepsilon$, para todo $x\in X$ e todo $k\ge k_0$. Se $\mu(X)<+\infty$, mostre que $f$ também é
integrável (resp., quase integrável) e que:
\[\int_Xf\,\dd\mu=\lim_{k\to\infty}\int_Xf_k\,\dd\mu.\]
\end{exercise}

\begin{exercise}\label{exe:intseries}
Sejam $(X,\mathcal A,\mu)$ um espaço de medida e $(f_k)_{k\ge1}$ uma seqüência
de funções integráveis $f_k:X\to\R$ tal que:
\[\sum_{k=1}^\infty\int_X\vert f_k\vert\,\dd\mu<+\infty.\]
Mostre que:
\begin{itemize}
\item a série $\sum_{k=1}^\infty f_k(x)$ é absolutamente convergente para quase todo $x\in X$;
\item se $f:X\to\R$ é uma função mensurável tal que $f=\sum_{k=1}^\infty f_k$ \qs\ então
$f$ é integrável e:
\[\int_Xf\,\dd\mu=\sum_{k=1}^\infty\int_Xf_k\,\dd\mu\in\R.\]
\end{itemize}
\end{exercise}

\begin{exercise}
Sejam $(X,\mathcal A,\mu)$ um espaço de medida e $f:X\to\R$ uma função integrável.
Mostre que para todo $\varepsilon>0$ existe uma função simples integrável $\phi:X\to\R$
tal que:
\[\int_X\vert f-\phi\vert\,\dd\mu<\varepsilon.\]
\end{exercise}

\begin{exercise}\label{exe:conjuntosAk}
Sejam $(X,\mathcal A,\mu)$ um espaço de medida, $(A_k)_{k\ge1}$ uma seqüência de subconjuntos
mensuráveis de $X$ e $f:X\to\overline\R$ uma função quase integrável. Assuma que para todo
$x\in X$ o conjunto:
\[\big\{k\ge1:x\not\in A_k\big\}\]
é finito. Mostre que:
\[\int_Xf\,\dd\mu=\lim_{k\to\infty}\int_{A_k}f\,\dd\mu.\]
\end{exercise}

\begin{exercise}
Seja $f:\R\to\R$ uma função integrável. Mostre que as funções:
\[g_1(t)=\int_{\R}f(x)\cos(tx)\,\dd\leb(x),\quad
g_2(t)=\int_{\R}f(x)\sen(tx)\,\dd\leb(x),\]
são contínuas e que:
\[\lim_{t\to\pm\infty}g_1(t)=0,\quad\lim_{t\to\pm\infty}g_2(t)=0.\]
\end{exercise}

\begin{exercise}
Considere a função $\phi:\R\to\R$ definida por:
\[\phi(t)=\int_{\R}e^{-x^2}\cos(tx)\,\dd\leb(x),\]
para todo $t\in\R$.
\begin{itemize}
\item[(a)] Mostre que $\phi$ é derivável e que:
\[\phi'(t)=-\frac t2\,\phi(t),\]
para todo $t\in\R$.
\item[(b)] Mostre que
$\phi(t)=ce^{-\frac{t^{\hbox to0pt{$\scriptscriptstyle2$\hskip 0pt minus 1fil}}}4}$,
para todo $t\in\R$, onde:
\begin{equation}\label{eq:cintegralGauss}
c=\int_{\R}e^{-x^2}\,\dd\leb(x).
\end{equation}
\end{itemize}
\end{exercise}

No Exercício~\ref{exe:integralGauss} pediremos ao leitor para calcular explicitamente a integral
em \eqref{eq:cintegralGauss}.

\begin{exercise}\label{exe:senxx}
Considere a função $\phi:\left]0,+\infty\right[\to\R$ definida por:
\[\phi(t)=\int_0^{+\infty}e^{-tx}\,\frac{\sen\,x}x\,\dd\leb(x),\]
para todo $t>0$.
\begin{itemize}
\item[(a)] Mostre que $\phi$ é derivável e que $\phi'(t)=-\frac1{1+t^2}$, para todo $t>0$.
\item[(b)] Mostre que $\lim_{t\to+\infty}\phi(t)=0$.
\item[(c)] Conclua que $\phi(t)=\frac\pi2-\arctan t$, para todo $t>0$.
\item[(d)] Usando integração por partes, verifique que:
\[\phi(t)=\int_0^1e^{-tx}\,\frac{\sen\,x}x\,\dd\leb(x)+
e^{-t}\cos1-\int_1^{+\infty}\cos x\;e^{-tx}\,\frac{1+tx}{x^2}\,\dd\leb(x),\]
para todo $t>0$.
\item[(e)] Mostre que:
\[\lim_{t\to0}\phi(t)=\intRd_0^{+\infty}f=\frac\pi2,\]
onde $f:\left[0,+\infty\right[\to\R$ é definida por $f(x)=\frac{\sen\,x}x$, para $x>0$
e $f(0)=1$.
\end{itemize}
\end{exercise}

\subsection*{Mais sobre Convergência de Seqüências de Funções}

\begin{exercise}
Sejam $(X,\mathcal A,\mu)$ um espaço de medida, $(f_n)_{n\ge1}$ uma seqüência de funções mensuráveis $f_n:X\to\R$
e $f:X\to\R$ uma função mensurável. Se $(f_n)_{n\ge1}$ converge em medida para $f$, mostre que toda subseqüência
de $(f_n)_{n\ge1}$ também converge em medida para $f$.
\end{exercise}

\begin{exercise}\label{exe:Cauchy1}
Sejam $(X,\mathcal A,\mu)$ um espaço de medida e $(f_n)_{n\ge1}$ uma seqüência pontualmente de Cauchy quase sempre.
Mostre que existe uma função $f:X\to\R$ tal que $f_n\to f$ pontualmente \qs; se as funções $f_n$ são todas mensuráveis, mostre
que podemos escolher a função $f$ também mensurável.
\end{exercise}

\begin{exercise}\label{exe:Cauchy2}
Mostre que toda seqüência uniformemente de Cauchy quase sempre é quase uniformemente de Cauchy e que toda
seqüência quase uniformemente de Cauchy é pontualmente de Cauchy quase sempre.
\end{exercise}

\begin{exercise}\label{exe:Cauchy3}
Se $X$ é um conjunto, $(f_n)_{n\ge1}$ é uma seqüência uniformemente de Cauchy de funções $f_n:X\to\R$
e $f:X\to\R$ é uma função tal que $f_n\to f$ pontualmente, mostre que $f_n\To uf$.
Se $(X,\mathcal A,\mu)$ é um espaço de medida, $(f_n)_{n\ge1}$ é uniformemente de Cauchy quase sempre
(resp., quase uniformemente de Cauchy) e $f_n\to f$ pontualmente \qs, mostre que
$f_n\To uf$ \qs\ (resp., que $f_n\To{qu}f$).
\end{exercise}

\begin{exercise}\label{exe:Cauchy4}
Mostre que:
\begin{itemize}
\item toda seqüência uniformemente convergente (resp., quase sempre) é uniformemente de Cauchy (resp., quase sempre);
\item toda seqüência quase uniformemente convergente é quase uniformemente de Cauchy;
\item toda seqüência de funções mensuráveis que é convergente em medida é de Cauchy em medida.
\end{itemize}
\end{exercise}

\begin{exercise}\label{exe:Cauchy5}
Seja $(X,\mathcal A,\mu)$ um espaço de medida com $\mu(X)<+\infty$ e seja $(f_n)_{n\ge1}$ uma seqüência
de funções mensuráveis $f_n:X\to\R$. Mostre que se $(f_n)_{n\ge1}$ é pontualmente de Cauchy quase sempre
então $(f_n)_{n\ge1}$ é quase uniformemente de Cauchy.
\end{exercise}

\begin{exercise}\label{exe:Cauchy6}
Mostre que se uma seqüência de funções mensuráveis é quase uniformemente de Cauchy então ela é de Cauchy em medida.
\end{exercise}

\begin{exercise}\label{exe:Cauchysubseq}
Sejam $(X,\mathcal A,\mu)$ um espaço de medida, $(f_n)_{n\ge1}$ uma seqüência de funções mensuráveis $f_n:X\to\R$
que é de Cauchy em medida e $(f_{n_k})_{k\ge1}$ uma subseqüência de $(f_n)_{n\ge1}$ que converge em medida
para uma função mensurável $f:X\to\R$. Mostre que $(f_n)_{n\ge1}$ também converge em medida para $f$.
\end{exercise}

\begin{exercise}
Sejam $(X,\mathcal A,\mu)$ um espaço de medida, $(f_n)_{n\ge1}$ uma seqüência de funções mensuráveis $f_n:X\to\R$
e $f:X\to\R$, $g:X\to\R$ funções mensuráveis. Se $f_n\To\mu f$ e $f_n\To\mu g$, mostre que $f(x)=g(x)$ para quase
todo $x\in X$.
\end{exercise}

\subsection*{O Teorema de Fubini em ${\R^n}$}

\begin{exercise}
Seja $f:X\to\R^n$ uma função definida num subconjunto $X$ de $\R^m$.
Mostre que se o gráfico de $f$ (recorde \eqref{eq:defGrf}) é mensurável então
$\leb\big(\Gr(f)\big)=0$.
\end{exercise}

\begin{exercise}
Sejam $X\subset\R^m$, $Y\subset\R^n$ conjuntos mensuráveis e $f:X\to\overline\R$,
$g:Y\to\overline\R$ funções integráveis. Mostre que a função:
\[X\times Y\ni(x,y)\longmapsto f(x)g(y)\in\overline\R\]
é integrável e que sua integral é dada por:
\[\int_{X\times Y}f(x)g(y)\,\dd\leb(x,y)=\Big(\int_Xf\,\dd\leb\Big)\Big(\int_Yg\,\dd\leb\Big).\]
\end{exercise}

\begin{exercise}
Seja $\Delta_n$ o {\em simplexo padrão $n$-dimensional\/}\index[indice]{simplexo!padrao@padrão}
definido por:\index[simbolos]{$\Delta_n$}
\[\Delta_n=\Big\{(x_1,\ldots,x_n)\in\left[0,+\infty\right[^n:\sum_{i=1}^nx_i\le1\Big\}.\]
\begin{itemize}
\item[(a)] Mostre que $\Delta_n$ é mensurável para todo $n\ge1$.
\item[(b)] Se $a_n=\leb(\Delta_n)$, mostre que:
\[a_n=a_{n-1}\int_0^1(1-t)^{n-1}\,\dd\leb(t),\]
para todo $n\ge1$.
\item[(c)] Determine $\leb(\Delta_n)$.
\end{itemize}
\end{exercise}

\end{chapter}

\begin{chapter}{O Teorema de Mudança de Variáveis para Integrais de Lebesgue}
\label{CHP:MUDVAR}

\begin{section}[Aplicações Lipschitzianas]{O Efeito de Aplicações Lipschitzianas sobre a Medida de Lebesgue}

\begin{notation}\label{not:normainfinito}
Dado $x\in\R^n$, escrevemos:\index[simbolos]{$\Vert x\Vert_\infty$}
\[\Vert x\Vert_\infty=\max\big\{\vert x_i\vert:i=1,\ldots,n\big\},\]
e para $x,y\in\R^n$, escrevemos:\index[simbolos]{$d_\infty(x,y)$}
\[d_\infty(x,y)=\Vert x-y\Vert_\infty=\max\big\{\vert x_i-y_i\vert:i=1,\ldots,n\big\}.\]
\end{notation}
Claramente se $B$ é um cubo $n$-dimensional com aresta $a$ (veja Definição~\ref{thm:defcubo})
então $d_\infty(x,y)\le a$, para todos $x,y\in B$. Provamos agora a seguinte recíproca
para essa afirmação:

\begin{lem}\label{thm:lemadentrodocubo}
Sejam $A\subset\R^n$ e $a\ge0$ tais que $d_\infty(x,y)\le a$, para todos $x,y\in A$. Então
$A$ está contido em um cubo $n$-dimensional de aresta $a$; em particular:
\[\leb^*(A)\le a^n.\]
\end{lem}
\begin{proof}
Se $A$ é vazio, não há nada para se mostrar. Senão, seja $\pi_i:\R^n\to\R$ a projeção sobre
a $i$-ésima coordenada e considere o conjunto $A_i=\pi_i(A)$.
Temos $\vert t-s\vert\le a$, para todos $t,s\in A_i$ e portanto $\sup A_i-\inf A_i\le a$;
se $a_i=\inf A_i$, segue que:
\[A_i\subset[a_i,a_i+a]\]
e portanto:
\[A\subset\prod_{i=1}^nA_i\subset\prod_{i=1}^n[a_i,a_i+a].\qedhere\]
\end{proof}

\begin{defin}
Seja $\phi:X\to\R^n$ uma função definida num subconjunto $X$ de $\R^m$. Dizemos que $\phi$ é
{\em Lipschitziana\/}\index[indice]{Lipschitziana!funcao@função}\index[indice]{funcao@função!Lipschitziana}
se existe uma constante $k\ge0$ tal que:
\[d_\infty\big(\phi(x),\phi(y)\big)\le k\,d_\infty(x,y),\]
para todos $x,y\in X$. A constante $k$ é dita uma
{\em constante de Lipschitz\/}\index[indice]{Lipschitz!constante de}\index[indice]{constante!de Lipschitz}
para a função $\phi$.
\end{defin}
Claramente toda função Lipschitziana é (uniformemente) contínua.

\begin{lem}\label{thm:medextcubos}
Seja $A$ um subconjunto de $\R^n$. Dado $\varepsilon>0$, existe um conjunto enumerável
$\mathcal R$ de cubos $n$-dimensionais tal que $A\subset\bigcup_{B\in\mathcal R}B$
e $\sum_{B\in\mathcal R}\vert B\vert\le\leb^*(A)+\varepsilon$.
\end{lem}
\begin{proof}
Pelo Lema~\ref{thm:aproxaberto} existe um aberto $U$ em $\R^n$ contendo $A$ tal que
$\leb(U)\le\leb^*(A)+\varepsilon$ e pelo Lema~\ref{thm:abertocubos} existe um conjunto
enumerável $\mathcal R$ de cubos $n$-dimensionais com interiores dois a dois disjuntos
tal que $U=\bigcup_{B\in\mathcal R}B$. Daí:
\[\sum_{B\in\mathcal R}\vert B\vert=\leb(U)\le\leb^*(A)+\varepsilon.\qedhere\]
\end{proof}

\begin{prop}\label{thm:propefeitoLips}
Seja $\phi:X\to\R^n$ uma função Lipschitziana com constante de Lipschitz $k\ge0$, onde
$X$ é um subconjunto de $\R^n$. Então, para todo subconjunto $A$ de $X$, temos:
\[\leb^*\big(\phi(A)\big)\le k^n\leb^*(A).\]
\end{prop}
\begin{proof}
Dado $\varepsilon>0$ então, pelo Lema~\ref{thm:medextcubos} existe um conjunto enumerável
$\mathcal R$ de cubos $n$-dimensionais tal que $A\subset\bigcup_{B\in\mathcal R}B$
e:
\begin{equation}\label{eq:provaLips3}
\sum_{B\in\mathcal R}\vert B\vert\le\leb^*(A)+\varepsilon.
\end{equation}
Daí $\phi(A)\subset\bigcup_{B\in\mathcal R}\phi(B\cap X)$ e portanto:
\begin{equation}\label{eq:provaLips1}
\leb^*\big(\phi(A)\big)\le\sum_{B\in\mathcal R}\leb^*\big(\phi(B\cap X)\big).
\end{equation}
Fixado um cubo $B\in\mathcal R$ então, se $a$ denota a aresta de $B$, temos:
\[d_\infty\big(\phi(x),\phi(y)\big)\le k\,d_\infty(x,y)\le ka,\]
para todos $x,y\in B\cap X$. Segue do Lema~\ref{thm:lemadentrodocubo} que:
\begin{equation}\label{eq:provaLips2}
\leb^*\big(\phi(B\cap X)\big)\le(ka)^n=k^n\vert B\vert.
\end{equation}
De \eqref{eq:provaLips3}, \eqref{eq:provaLips1} e \eqref{eq:provaLips2} vem:
\[\leb^*\big(\phi(A)\big)\le k^n\sum_{B\in\mathcal R}\vert B\vert\le k^n\big(\leb^*(A)+\varepsilon\big).\]
A conclusão segue fazendo $\varepsilon\to0$.
\end{proof}

\begin{cor}\label{thm:corLipsnula}
Se $\phi:X\to\R^n$ é uma função Lipschitziana definida num subconjunto $X$ de $\R^n$ então $\phi$ leva
subconjuntos de $X$ de medida nula em subconjuntos de medida nula de $\R^n$.\qed
\end{cor}

\begin{rem}\label{thm:linearLips}
Recorde que toda aplicação linear $T:\R^m\to\R^n$ é Lipschitziana. Mais explicitamente, se a
{\em norma\/}\index[indice]{norma!de uma aplicacao linear@de uma aplicação linear}
da aplicação linear $T$ é definida por:
\begin{equation}\label{eq:defnormaoperador}
\Vert T\Vert=\sup_{\Vert x\Vert_\infty\le1}\Vert T(x)\Vert_\infty,
\end{equation}
então:
\[\Vert T(x)\Vert_\infty\le\Vert T\Vert\Vert x\Vert_\infty,\]
para todo $x\in\R^m$, donde segue facilmente que $\Vert T\Vert$ é uma constante de Lipschitz para $T$.
A finitude do supremo em \eqref{eq:defnormaoperador} segue, por exemplo, do fato que a aplicação $x\mapsto\Vert T(x)\Vert_\infty$
é contínua e a bola $\big\{x:\Vert x\Vert_\infty\le1\big\}$ é compacta.
\end{rem}

\begin{cor}\label{thm:corlinearnula}
Uma aplicação linear de $\R^n$ em $\R^n$ leva subconjuntos de medida nula de $\R^n$ em subconjuntos de medida
nula de $\R^n$.
\end{cor}
\begin{proof}
Segue do Corolário~\ref{thm:corLipsnula} e da Observação~\ref{thm:linearLips}.
\end{proof}

\begin{cor}\label{thm:corsubespaconula}
Todo subespaço vetorial próprio de $\R^n$ tem medida nula.
\end{cor}
\begin{proof}
Se $V$ é um subespaço vetorial próprio de $\R^n$ então existe uma aplicação linear $T:\R^n\to\R^n$
tal que $T\big(\R^{n-1}\times\{0\}\big)=V$; de fato, podemos escolher uma aplicação linear $T$ que
leva os $n-1$ primeiros vetores da base canônica de $\R^n$ sobre uma base qualquer de $V$ (note que $\Dim(V)\le n-1$\index[simbolos]{$\Dim(V)$}).
A conclusão segue do Corolário~\ref{thm:hiperplanonula} e do Corolário~\ref{thm:corlinearnula}.
\end{proof}

\begin{defin}
Uma função $\phi:X\to\R^n$ definida num subconjunto $X$ de $\R^m$ é dita
{\em localmente Lipschitziana}\index[indice]{localmente!Lipschitziana}\index[indice]{funcao@função!localmente Lipschitziana}\index[indice]{Lipschitziana!localmente}
se todo $x\in X$ possui uma vizinhança $V$ em $\R^m$ tal que a função $\phi\vert_{V\cap X}$ é Lipschitziana.
\end{defin}

\begin{prop}\label{thm:proplocLips}
Se $\phi:X\to\R^n$ é uma função localmente Lipschitziana definida num subconjunto $X$ de $\R^n$ então $\phi$ leva
subconjuntos de $X$ de medida nula em subconjuntos de medida nula de $\R^n$.
\end{prop}
\begin{proof}
Para cada $x\in X$ seja $U_x$ um aberto em $\R^n$ contendo $x$ tal que a restrição de $\phi$ a $U_x\cap X$ seja
Lipschitziana. A cobertura aberta $X\subset\bigcup_{x\in X}U_x$ possui uma subcobertura enumerável
$X\subset\bigcup_{i=1}^\infty U_{x_i}$. Agora, dado qualquer subconjunto $A$ de $X$ com $\leb(A)=0$, segue
do Corolário~\ref{thm:corLipsnula} que:
\[\leb\big(\phi(U_{x_i}\cap A)\big)=0,\]
para todo $i$. A conclusão é obtida agora da igualdade:
\[\phi(A)=\bigcup_{i=1}^\infty\phi(U_{x_i}\cap A).\qedhere\]
\end{proof}

\begin{prop}\label{thm:propmenslocLips}
Seja $\phi:X\to\R^n$ uma função localmente Lipschitziana definida num subconjunto
$X$ de $\R^n$. Então, para todo subconjunto mensurável $A$ de $\R^n$ contido em $X$, temos que
$\phi(A)$ é mensurável.
\end{prop}
\begin{proof}
Como $A$ é mensurável, pelo Corolário~\ref{thm:innerFsigma}, existe um subconjunto $W$ de $\R^n$ de tipo $F_\sigma$
com $W\subset A$ e $\leb(A\setminus W)=0$; temos então que $A=W\cup N$, onde $W$ é um $F_\sigma$
e $N=A\setminus W$ tem medida nula. Como $\phi$ é localmente Lipschitziana então $\phi$
é localmente contínua e portanto contínua; daí $\phi$ leva compactos em compactos. Como
$W$ é uma união enumerável de fechados e todo fechado é uma união enumerável de compactos,
segue que $W$ é uma união enumerável de compactos; portanto também $\phi(W)$ é uma união enumerável
de compactos. Temos então:
\[\phi(A)=\phi(W)\cup\phi(N),\]
onde $\phi(W)$ é um $F_\sigma$ e $\phi(N)$ (é mensurável e) tem medida nula,
pela Proposição~\ref{thm:proplocLips}.
\end{proof}

\begin{cor}\label{thm:linearmenstomens}
Se $T:\R^n\to\R^n$ é uma aplicação linear então $T$ leva subconjuntos mensuráveis de $\R^n$ em subconjuntos
mensuráveis de $\R^n$.
\end{cor}
\begin{proof}
Segue da Observação~\ref{thm:linearLips} e da Proposição~\ref{thm:propmenslocLips}.
\end{proof}

\end{section}

\begin{section}[Aplicações Lineares]{O Efeito de Aplicações Lineares sobre a Medida de Lebesgue}

O objetivo desta seção é provar o seguinte:
\begin{teo}\label{thm:mudvarcasolin}
Seja $T:\R^n\to\R^n$ uma aplicação linear. Para todo subconjunto mensurável $A$ de $\R^n$
temos que $T(A)$ é mensurável e:
\begin{equation}\label{eq:formuladet}
\leb\big(T(A)\big)=\vert\det T\vert\,\leb(A).
\end{equation}
\end{teo}
Em \eqref{eq:formuladet} denotamos por $\det T$\index[simbolos]{$\det T$} o {\em determinante\/}\index[indice]{determinante} de $T$, ou seja,
o determinante da matriz que representa $T$ na base canônica de $\R^n$. No que segue, {\em sempre identificaremos
aplicações lineares de $\R^m$ em $\R^n$ com as respectivas matrizes $n\times m$ que as representam com respeito às bases
canônicas}.

O restante da seção é dedicado à demonstração do Teorema~\ref{thm:mudvarcasolin}.
Note que a mensurabilidade de $T(A)$ já é garantida pelo Corolário~\ref{thm:linearmenstomens}.
Note também que se $T$ não é inversível então o Teorema~\ref{thm:mudvarcasolin} segue do Corolário~\ref{thm:corsubespaconula},
já que a imagem de $T$ é um subespaço próprio de $\R^n$ e $\det T=0$.
Se $T$ é inversível, a estratégia da prova
do Teorema~\ref{thm:mudvarcasolin} é a seguinte. Inicialmente, observamos que se $T_1:\R^n\to\R^n$ e $T_2:\R^n\to\R^n$
são aplicações lineares tais que a igualdade \eqref{eq:formuladet} vale para $T=T_1$ e para $T=T_2$, para todo subconjunto
mensurável $A$ de $\R^n$, então a igualdade \eqref{eq:formuladet} também vale para $T=T_1T_2$; de fato, dado $A\subset\R^n$
mensurável, temos:
\begin{multline*}
\leb\big((T_1T_2)(A)\big)=\vert\det T_1\vert\,\leb\big(T_2(A)\big)=\vert\det T_1\vert\,\vert\det T_2\vert\,\leb(A)\\
=\vert\det(T_1T_2)\vert\,\leb(A).
\end{multline*}
A seguir, selecionamos alguns tipos de aplicações lineares que chamaremos de {\em elementares}; mostraremos então
que a igualdade \eqref{eq:formuladet} vale para aplicações lineares elementares e que toda aplicação linear inversível pode
ser escrita como um produto de aplicações lineares elementares.

\begin{defin}
Uma aplicação linear $E:\R^n\to\R^n$ é dita {\em elementar\/}\index[indice]{aplicacao linear@aplicação linear!elementar}\index[indice]{elementar!aplicacao linear@aplicação linear}
quando é de um dos seguintes tipos:
\begin{itemize}
\item[\textbf{tipo 1.}] $E=L_{i,j;c}$\index[simbolos]{$L_{i,j;c}$}, onde $i,j=1,\ldots,n$ são distintos, $c\in\R$ e:
\begin{equation}\label{eq:defEijc}
L_{i,j;c}(x_1,\ldots,x_i,\ldots,x_j,\ldots,x_n)=(x_1,\ldots,x_i+cx_j,\ldots,x_j,\ldots,x_n);
\end{equation}
\item[\textbf{tipo 2.}] $E=\widehat\sigma$\index[simbolos]{$\widehat\sigma$},
onde $\sigma:\{1,\ldots,n\}\to\{1,\ldots,n\}$ é uma bijeção e:
\begin{equation}\label{eq:defhatsigma}
\widehat\sigma(x_1,\ldots,x_n)=(x_{\sigma(1)},\ldots,x_{\sigma(n)});
\end{equation}
\item[\textbf{tipo 3.}] $E=D_\lambda$\index[simbolos]{$D_\lambda$},
onde $\lambda=(\lambda_1,\ldots,\lambda_n)\in\R^n$, $\lambda_i\ne0$ para $i=1,\ldots,n$ e:
\begin{equation}\label{eq:defDlambda}
D_\lambda(x_1,\ldots,x_n)=(\lambda_1x_1,\ldots,\lambda_nx_n).
\end{equation}
\end{itemize}
\end{defin}
Obviamente as expressões \eqref{eq:defEijc}, \eqref{eq:defhatsigma} e \eqref{eq:defDlambda} definem isomorfismos lineares
de $\R^n$; em \eqref{eq:defEijc} escrevemos a definição de $L_{i,j;c}$ assumindo que $i<j$, mas obviamente uma fórmula
análoga define $L_{i,j;c}$ se $i>j$. O efeito da multiplicação à esquerda de uma matriz $T$ por uma matriz que representa
uma aplicação linear elementar $E$ nos dá o que chamamos de uma
{\em transformação elementar\/}\index[indice]{elementar!transformacao@transformação}\index[indice]{transformacao@transformação!elementar}
de matrizes; mais explicitamente, se $T$ é uma matriz $n\times n$ cujas linhas são vetores $\ell_1,\ldots,\ell_n\in\R^n$
e se $E$ é uma aplicação linear elementar então $ET$ é a matriz cujas linhas são:
\begin{itemize}
\item $\ell_1,\ldots,\ell_i+c\ell_j,\ldots,\ell_j,\ldots,\ell_n$, se $E=L_{i,j;c}$;
\item $\ell_{\sigma(1)},\ldots,\ell_{\sigma(n)}$, se $E=\widehat\sigma$;
\item $\lambda_1\ell_1,\ldots,\lambda_n\ell_n$, se $E=D_\lambda$.
\end{itemize}
As transformações elementares de matrizes associadas à multicação à esquerda por uma aplicação elementar de tipos~1, 2 e 3
serão respectivamente chamadas de {\em transformações elementares de tipos~1, 2 e 3}.

O seguinte resultado é padrão em textos elementares de Álgebra Linear.
\begin{lem}\label{thm:lemaescalona}
Se $T:\R^n\to\R^n$ é uma aplicação linear inversível então existe uma seqüência finita de transformações
elementares de matrizes que leva $T$ até a matriz identidade.
\end{lem}
\begin{proof}
Fazemos uma descrição sucinta do algorítmo que é conhecido como {\em escalonamento\/}\index[indice]{escalonamento} de matrizes.
Em primeiro lugar, como $T$ é inversível então algum elemento da primeira coluna de $T$ é não nulo; realizando uma transformação
elementar de tipo~2, podemos assumir que o elemento $T_{11}$ é não nulo e depois realizando uma transformação
elementar de tipo~3 podemos assumir que $T_{11}=1$. Agora, uma seqüência de $n-1$ transformações
elementares de tipo~1 nos permite anular os elementos $T_{j1}$, com $j=2,\ldots,n$. Nesse ponto, a primeira coluna de $T$
coincide com o primeiro vetor da base canônica de $\R^n$; daí a submatriz de $T$ obtida removendo a primeira linha e a
primeira coluna é inversível e podemos portanto repetir o algorítmo recursivamente na mesma. Obteremos então uma matriz
$T$ triangular superior em que todos os elementos da diagonal são iguais a $1$. Podemos agora realizar uma seqüência
de $\frac{n(n-1)}2$ transformações elementares de tipo~1 para anular os elementos de $T$ que estão acima da diagonal,
obtendo assim a matriz identidade.
\end{proof}

\begin{cor}\label{thm:corescalona}
Toda aplicação linear inversível $T:\R^n\to\R^n$ é um produto de aplicações lineares elementares.
\end{cor}
\begin{proof}
Segue do Lema~\ref{thm:lemaescalona} que existem
aplicações lineares elementares $E_1$, \dots, $E_k$ de modo que $E_1\cdots E_kT$ é igual à matriz identidade.
Daí $T=E_k^{-1}\cdots E_1^{-1}$. A conclusão segue da observação simples de que a inversa de uma aplicação linear
elementar é novamente uma aplicação linear elementar (de mesmo tipo).
\end{proof}

Em vista do Corolário~\ref{thm:corescalona} e das observações feitas anteriormente nesta seção, temos que
a demonstração do Teorema~\ref{thm:mudvarcasolin} ficará concluída assim que demonstrarmos o seguinte:
\begin{lem}
Se $T:\R^n\to\R^n$ é uma aplicação linear elementar então a igualdade \eqref{eq:formuladet} vale para todo subconjunto
mensurável $A$ de $\R^n$.
\end{lem}
\begin{proof}
Se $T$ é de tipo~2 ou 3 então a tese do lema segue respectivamente dos resultados dos
Exercícios~\ref{exe:permutacao} e \ref{exe:diagonal} (note que as aplicações lineares elementares de tipo~2 tem determinante
igual a $\pm1$). Resta então considerar o caso em que $T$ é uma aplicação linear elementar de tipo~1.
É fácil verificar que se $\sigma:\{1,\ldots,n\}\to\{1,\ldots,n\}$ é uma bijeção então:
\[\widehat\sigma^{-1}L_{i,j;c}\,\widehat\sigma=L_{\sigma(i),\sigma(j);c},\]
para todos $i,j=1,\ldots,n$ distintos e todo $c\in\R$. Podemos então reduzir a demonstração do lema apenas ao caso
em que $T=L_{n,1;c}$, $c\in\R$. No que segue, identificamos $\R^n$ com o produto $\R^{n-1}\times\R$ e usamos a notação
da Seção~\ref{sec:Fubini}; a aplicação $T$ escreve-se na forma:
\[T(x,y)=(x,y+cx_1),\quad x\in\R^{n-1},\ y\in\R.\]
Dado $A\subset\R^n$ então para todo $x\in\R^{n-1}$, a fatia vertical $T(A)_x$ do conjunto $T(A)$
coincide com a translação $A_x+cx_1$ da fatia vertical $A_x$ de $A$. Se $A$ é mensurável, temos que $T(A)$ também é
mensurável (vide Corolário~\ref{thm:linearmenstomens}); segue então da Proposição~\ref{thm:quasetodafatia} que:
\begin{multline*}
\leb\big(T(A)\big)=\int_{\R^{n-1}}\leb\big(T(A)_x\big)\,\dd\leb(x)=\int_{\R^{n-1}}\leb(A_x+cx_1)\,\dd\leb(x)\\
=\int_{\R^{n-1}}\leb(A_x)\,\dd\leb(x)=\leb(A),
\end{multline*}
onde na terceira igualdade usamos o Lema~\ref{thm:extmeastransinv}. Como $T$ é uma matriz triangular com elementos
da diagonal iguais a $1$, temos que $\det T=1$ e portanto a igualdade \eqref{eq:formuladet} fica demonstrada.
\end{proof}

\end{section}

\begin{section}{O Teorema de Mudança de Variáveis}
\label{sec:teomudvar}

Nesta seção nós provaremos o Teorema de Mudança de Variáveis para integais de Lebesgue em $\R^n$. Para um entendimento
completo do conteúdo desta seção serão necessários alguns conhecimentos básicos de Cálculo no $\R^n$, sobre os quais fazemos
uma rápida revisão na Seção~\ref{sec:appCalculo}.

O enunciado do teorema é o seguinte:
\begin{teo}[mudança de variáveis]\index[indice]{teorema!de mudanca de variaveis@de mudança de variáveis}\index[indice]{mudanca de variaveis@mudança de variáveis!teorema de}
\label{thm:teomudvar}
Seja $\phi:U\to\R^n$ uma aplicação {\em injetora\/} de classe $C^1$ definida num subconjunto
aberto $U$ de $\R^n$; suponha que a diferencial $\dd\phi(x)$ é um isomorfismo de $\R^n$, para
todo $x\in U$.
Dados um conjunto mensurável $A\subset\R^n$ contido em $U$ e uma função mensurável
$f:\phi(A)\to\overline\R$ então:
\begin{itemize}
\item o conjunto $\phi(A)$ é mensurável;
\item a função:
\begin{equation}\label{eq:funcaomudada}
A\ni y\longmapsto f\big(\phi(y)\big)\,\big\vert\det\dd\phi(y)\big\vert\in\overline\R
\end{equation}
é mensurável;
\item a função $f$ é quase integrável se e somente se a função \eqref{eq:funcaomudada} é
quase integrável e, nesse caso, vale a igualdade:
\begin{equation}\label{eq:eqmudvar}
\int_{\phi(A)}f(x)\,\dd\leb(x)=\int_Af\big(\phi(y)\big)\,\big\vert\det\dd\phi(y)\big\vert\,\dd\leb(y).
\end{equation}
\end{itemize}
\end{teo}

Note que, pelo Teorema da Função Inversa (Teorema~\ref{thm:TFI}), as hipóteses sobre $\phi$
no enunciado do Teorema~\ref{thm:teomudvar} são equivalentes à condição de que $\phi(U)$
seja aberto em $\R^n$ e que $\phi:U\to\phi(U)$
seja um difeomorfismo $C^1$. Note também que a mensurabilidade
de $\phi(A)$ é garantida pela Proposição~\ref{thm:propmenslocLips}, já que $\phi:U\to\R^n$
é uma função localmente Lipschitziana (veja Corolário~\ref{thm:C1locLips}).

Para demonstrar o Teorema~\ref{thm:teomudvar}, precisamos de alguns lemas preparatórios.

\begin{lem}\label{thm:difeoC1mensmens}
Seja $\phi:U\to\R^n$ uma função de classe $C^1$ num aberto $U\subset\R^n$ e suponha que a diferencial
$\dd\phi(x)$ é um isomorfismo de $\R^n$, para todo $x\in U$. Então, para todo subconjunto mensurável $E$
de $\R^n$ temos que $\phi^{-1}(E)$ é mensurável; em outras palavras, a função:
\[\phi:\big(U,\Lebmens(\R^n)\vert_U\big)\longrightarrow\big(\R^n,\Lebmens(\R^n)\big)\]
é mensurável.
\end{lem}
\begin{proof}
Pelo Teorema da Função Inversa (Teorema~\ref{thm:TFI}), cada $x\in U$ possui uma vizinhança aberta
$U_x$ contida em $U$ tal que $\phi(U_x)$ é aberto em $\R^n$ e $\phi\vert_{U_x}:U_x\to\phi(U_x)$ é um difeomorfismo
$C^1$. Daí a função $\psi_x=(\phi\vert_{U_x})^{-1}:\phi(U_x)\to U_x$ é localmente Lipschitziana
(veja Corolário~\ref{thm:C1locLips}) e portanto, pela Proposição~\ref{thm:propmenslocLips}, o conjunto
\[\psi_x\big(E\cap\phi(U_x)\big)=\phi^{-1}\big(E\cap\phi(U_x)\big)\cap U_x=\phi^{-1}(E)\cap U_x\]
é mensurável, para todo $x\in U$. A cobertura aberta $U=\bigcup_{x\in U}U_x$ possui uma subcobertura enumerável
$U=\bigcup_{i=1}^\infty U_{x_i}$ e portanto:
\[\phi^{-1}(E)=\bigcup_{i=1}^\infty\big(\phi^{-1}(E)\cap U_{x_i}\big),\]
donde segue que $\phi^{-1}(E)$ é mensurável.
\end{proof}

\begin{cor}\label{thm:fphimens}
Seja $\phi:U\to\R^n$ uma função de classe $C^1$ num aberto $U\subset\R^n$ tal que a diferencial
$\dd\phi(x)$ é um isomorfismo de $\R^n$, para todo $x\in U$. Dados um subconjunto $A$ de $U$, um espaço mensurável
$(X,\mathcal A)$ e uma função mensurável $f:\phi(A)\to X$ então a função $f\circ\phi\vert_A:A\to X$ é mensurável.
\end{cor}
\begin{proof}
Basta observar que $f\circ\phi\vert_A$ é igual à composta das funções mensuráveis:
\begin{gather*}
\phi\vert_A:\big(A,\Lebmens(\R^n)\vert_A\big)\longrightarrow\big(\phi(A),\Lebmens(\R^n)\vert_{\phi(A)}\big),\\
f:\big(\phi(A),\Lebmens(\R^n)\vert_{\phi(A)}\big)\longrightarrow(X,\mathcal A).\qedhere
\end{gather*}
\end{proof}

\begin{lem}\label{thm:lemaphiquaselin}
Seja $\phi:U\to\R^n$ uma função de classe $C^1$ num aberto $U\subset\R^n$ e suponha que a diferencial
$\dd\phi(y_0)$ é um isomorfismo de $\R^n$, para um certo $y_0\in U$. Então, para todo $\varepsilon>0$,
existe uma vizinhança aberta $V$ de $y_0$ contida em $U$ tal que para todo conjunto mensurável $A\subset\R^n$
contido em $V$ temos que $\phi(A)$ é mensurável e vale a desigualdade:
\begin{equation}\label{eq:desigintdetprovisoria}
\leb\big(\phi(A)\big)\le(1+\varepsilon)\int_A\big\vert\det\dd\phi(y)\big\vert\,\dd\leb(y).
\end{equation}
\end{lem}
\begin{proof}
Em primeiro lugar, observe que a mensurabilidade de $\phi(A)$ segue da Proposição~\ref{thm:propmenslocLips},
já que $\phi$ é localmente Lipschitziana (veja Corolário~\ref{thm:C1locLips}).
Seja $\varepsilon'>0$ tal que:
\[(1+\varepsilon')^{n+1}\le1+\varepsilon.\]
Denote por $T$ a diferencial de $\phi$ no ponto $y_0$. Como $T^{-1}\circ\dd\phi(y_0)$ é igual à aplicação identidade
e como a função $y\mapsto\Vert T^{-1}\circ\dd\phi(y)\Vert$ é contínua, segue que:
\begin{equation}\label{eq:desigNormadphi}
\big\Vert T^{-1}\circ\dd\phi(y)\big\Vert<1+\varepsilon',
\end{equation}
para todo $y$ em uma vizinhança suficientemente pequena de $y_0$. Usando também a continuidade da função
$y\mapsto\big\vert\det\dd\phi(y)\big\vert$, vemos que:
\begin{equation}\label{eq:desigmoddetphi}
\big\vert\det\dd\phi(y_0)\big\vert<(1+\varepsilon')\;\big\vert\det\dd\phi(y)\big\vert,
\end{equation}
para todo $y$ em uma vizinhança suficientemente pequena de $y_0$. Seja $V$ uma bola aberta centrada em $y_0$ contida
em $U$ tal que \eqref{eq:desigNormadphi} e \eqref{eq:desigmoddetphi} valem para todo $y\in V$.
Seja $A$ um subconjunto mensurável de $V$ e provemos
\eqref{eq:desigintdetprovisoria}. Usando o Teorema~\ref{thm:mudvarcasolin}, obtemos:
\begin{multline}\label{eq:desigphiA1}
\leb\big(\phi(A)\big)=\leb\big(TT^{-1}\phi(A)\big)=\vert\det T\vert\,\leb\big(T^{-1}\phi(A)\big)\\
=\big\vert\det\dd\phi(y_0)\big\vert\,\leb\big(T^{-1}\phi(A)\big).
\end{multline}
Para todo $y\in V$, segue da regra da cadeia (veja Corolário~\ref{thm:corregracadeia}) que:
\[\big\Vert\dd(T^{-1}\circ\phi)(y)\big\Vert=\big\Vert T^{-1}\circ\dd\phi(y)\big\Vert<1+\varepsilon',\]
e portanto, pela desigualdade do valor médio (veja Corolário~\ref{thm:cordesigmedLips}), a função $T^{-1}\circ\phi\vert_V$
é Lipschitziana com constante de Lipschitz $1+\varepsilon'$. Usando a Proposição~\ref{thm:propefeitoLips}, obtemos:
\begin{equation}\label{eq:desigphiA2}
\leb\big(T^{-1}\phi(A)\big)\le(1+\varepsilon')^n\leb(A).
\end{equation}
De \eqref{eq:desigmoddetphi}, obtemos:
\begin{multline}\label{eq:desigphiA3}
\big\vert\det\dd\phi(y_0)\big\vert\,\leb(A)=\int_A\big\vert\det\dd\phi(y_0)\big\vert\chilow{A}(y)\,\dd\leb(y)\\
\le(1+\varepsilon')\int_A\big\vert\det\dd\phi(y)\big\vert\,\dd\leb(y).
\end{multline}
De \eqref{eq:desigphiA1}, \eqref{eq:desigphiA2} e \eqref{eq:desigphiA3}, vem:
\begin{multline*}
\leb\big(\phi(A)\big)\le(1+\varepsilon')^n\,\big\vert\det\dd\phi(y_0)\big\vert\,\leb(A)
\le(1+\varepsilon')^{n+1}\int_A\big\vert\det\dd\phi(y)\big\vert\,\dd\leb(y)\\
\le(1+\varepsilon)\int_A\big\vert\det\dd\phi(y)\big\vert\,\dd\leb(y).\qedhere
\end{multline*}
\end{proof}

\begin{lem}\label{thm:lemadesigphi}
Seja $\phi:U\to\R^n$ uma função de classe $C^1$ num aberto $U\subset\R^n$ e suponha que a diferencial
$\dd\phi(y)$ é um isomorfismo de $\R^n$, para todo $y\in U$. Então, dado um conjunto mensurável $A\subset\R^n$
contido em $U$, temos que $\phi(A)$ é mensurável e vale a desigualdade:
\[\leb\big(\phi(A)\big)\le\int_A\big\vert\det\dd\phi(y)\big\vert\,\dd\leb(y).\]
\end{lem}
\begin{proof}
Seja dado $\varepsilon>0$. Pelo Lema~\ref{thm:lemaphiquaselin}, todo ponto $y_0\in U$ possui uma vizinhança aberta
$V_{y_0}$ contida em $U$ com a seguinte propriedade: se $A\subset\R^n$ é um conjunto mensurável contido em $V_{y_0}$ então
$\phi(A)$ é mensurável e vale a desigualdade \eqref{eq:desigintdetprovisoria}. Da cobertura aberta
$U=\bigcup_{y\in U}V_y$, podemos extrair uma subcobertura enumerável $U=\bigcup_{i=1}^\infty V_{y_i}$.
Para cada $i\ge1$, definimos:
\[W_i=\begin{cases}
V_{y_i}\setminus\bigcup_{j=1}^{i-1}V_{y_j},&\text{se $i\ge2$},\\
\hfil V_{y_1},&\text{se $i=1$},
\end{cases}\]
de modo que $U=\bigcup_{i=1}^\infty W_i$, cada $W_i$ é mensurável (não necessariamente aberto), $W_i\subset V_{y_i}$
e os conjuntos $W_i$ são dois a dois disjuntos. Agora, dado um conjunto mensurável arbitrário $A\subset\R^n$ contido
em $U$, temos:
\[\phi(A)=\bigcup_{i=1}^\infty\phi(A\cap W_i).\]
Como $A\cap W_i$ é um subconjunto mensurável de $V_{y_i}$, segue que $\phi(A\cap W_i)$ é mensurável e vale a desigualdade:
\[\leb\big(\phi(A\cap W_i)\big)\le(1+\varepsilon)\int_{A\cap W_i}\big\vert\det\dd\phi(y)\big\vert\,\dd\leb(y).\]
Vemos então que $\phi(A)$ é mensurável e além disso:
\begin{multline*}
\leb\big(\phi(A)\big)\le\sum_{i=1}^\infty\leb\big(\phi(A\cap W_i)\big)\le
(1+\varepsilon)\sum_{i=1}^\infty\int_{A\cap W_i}\big\vert\det\dd\phi(y)\big\vert\,\dd\leb(y)\\
=(1+\varepsilon)\int_A\big\vert\det\dd\phi(y)\big\vert\,\dd\leb(y),
\end{multline*}
onde na última igualdade usamos o resultado do Exercício~\ref{exe:propintegralmedida}. A conclusão final é obtida agora fazendo
$\varepsilon\to0$.
\end{proof}

\begin{cor}\label{thm:desigpreparatoria}
Seja $\phi:U\to\R^n$ uma função de classe $C^1$ num aberto $U\subset\R^n$ e suponha que a diferencial
$\dd\phi(y)$ é um isomorfismo de $\R^n$, para todo $y\in U$. Então, dado um conjunto mensurável $A\subset\R^n$
contido em $U$ e uma função mensurável $f:\phi(A)\to[0,+\infty]$ temos que $\phi(A)$ é mensurável,
a função \eqref{eq:funcaomudada} é mensurável e vale a desigualdade:
\begin{equation}\label{eq:desigpreparatoria}
\int_{\phi(A)}f(x)\,\dd\leb(x)\le\int_Af\big(\phi(y)\big)\,\big\vert\det\dd\phi(y)\big\vert\,\dd\leb(y).
\end{equation}
\end{cor}
\begin{proof}
Note que a mensurabilidade da função \eqref{eq:funcaomudada} segue do Corolário~\ref{thm:fphimens}.
Para provar a desigualdade \eqref{eq:desigpreparatoria},
suponhamos inicialmente que $f:\phi(A)\to[0,+\infty]$ é simples e mensurável. Então podemos escrever:
\[f=\sum_{i=1}^kc_i\chilow{E_i},\]
onde $c_i\in[0,+\infty]$ e $E_i$ é um subconjunto mensurável de $\phi(A)$, para todo $i=1,\ldots,k$.
Seja $A_i=\phi^{-1}(E_i)\cap A$, de modo que $A_i$ é mensurável (veja Lema~\ref{thm:difeoC1mensmens}) e $\phi(A_i)=E_i$.
Segue do Lema~\ref{thm:lemadesigphi} que:
\[\leb(E_i)=\leb\big(\phi(A_i)\big)\le\int_{A_i}\big\vert\det\dd\phi(y)\big\vert\,\dd\leb(y),\]
para $i=1,\ldots,k$ e portanto:
\begin{multline*}
\int_{\phi(A)}f(x)\,\dd\leb(x)=\sum_{i=1}^kc_i\leb(E_i)\le
\sum_{i=1}^kc_i\int_{A_i}\big\vert\det\dd\phi(y)\big\vert\,\dd\leb(y)\\
\shoveright{=\sum_{i=1}^kc_i\int_A\chilow{E_i}\big(\phi(y)\big)\,\big\vert\det\dd\phi(y)\big\vert\,\dd\leb(y)}\\
=\int_Af\big(\phi(y)\big)\,\big\vert\det\dd\phi(y)\big\vert\,\dd\leb(y).
\end{multline*}
Demonstramos então a desigualdade \eqref{eq:desigpreparatoria} no caso em que $f$ é simples
e mensurável. Seja agora $f:\phi(A)\to[0,+\infty]$ uma função mensurável arbitrária.
Temos que existe uma seqüência $(f_k)_{k\ge1}$ de funções simples e mensuráveis
$f_k:\phi(A)\to[0,+\infty]$ tal que $f_k\nearrow f$; daí:
\[\int_{\phi(A)}f_k(x)\,\dd\leb(x)\le\int_Af_k\big(\phi(y)\big)\,\big\vert\det\dd\phi(y)\big\vert\,\dd\leb(y),\]
para todo $k\ge1$. A desigualdade \eqref{eq:desigpreparatoria} é obtida agora fazendo
$k\to\infty$ e usando o Teorema da Convergência Monotônica.
\end{proof}

\begin{proof}[Prova do Teorema~\ref{thm:teomudvar}]
Começamos supondo que $f$ é não negativa. A mensurabilidade de $\phi(A)$ e da função
\eqref{eq:funcaomudada} já foram estabelecidas no Corolário~\ref{thm:desigpreparatoria}.
Já temos também a desigualdade \eqref{eq:desigpreparatoria}. A desigualdade oposta segue da
aplicação do próprio Corolário~\ref{thm:desigpreparatoria} num contexto diferente.
Recorde que, pelo Teorema da Função Inversa (Teorema~\ref{thm:TFI}), $\phi(U)$ é um aberto
de $\R^n$ e $\phi:U\to\phi(U)$ é um difeomorfismo $C^1$; aplicamos então o Corolário~\ref{thm:desigpreparatoria}
ao difeomorfismo inverso $\psi=\phi^{-1}:\phi(U)\to\R^n$, à função $g:A\to[0,+\infty]$ definida por:
\[g(y)=f\big(\phi(y)\big)\,\big\vert\det\dd\phi(y)\big\vert,\quad y\in A,\]
e ao conjunto mensurável $B=\phi(A)\subset\phi(U)$. Obtemos a desigualdade:
\begin{equation}\label{eq:desigpsig}
\int_{\psi(B)}g(y)\,\dd\leb(y)\le\int_Bg\big(\psi(x)\big)\,\big\vert\det\dd\psi(x)\big\vert\,\dd\leb(x).
\end{equation}
Temos (veja \eqref{eq:difinverso}):
\[g\big(\psi(x)\big)\,\big\vert\det\dd\psi(x)\big\vert
=f(x)\,\big\vert\det\dd\phi(y)\big\vert\,
\big\vert\det\dd(\phi^{-1})\big(\phi(y)\big)\big\vert=f(x),\]
onde $y=\phi^{-1}(x)$. Daí \eqref{eq:desigpsig} nos dá:
\[\int_Af\big(\phi(y)\big)\,\big\vert\det\dd\phi(y)\big\vert\,\dd\leb(y)\le
\int_{\phi(A)}f(x)\,\dd\leb(x),\]
provando \eqref{eq:eqmudvar}. Finalmente, se $f:\phi(A)\to\overline\R$
é uma função mensurável arbitrária então:
\begin{gather}
\int_{\phi(A)}f^+(x)\,\dd\leb(x)=\int_Af^+\big(\phi(y)\big)\,\big\vert\det\dd\phi(y)\big\vert\,\dd\leb(y),\label{eq:eqmudvarf+}\\
\int_{\phi(A)}f^-(x)\,\dd\leb(x)=\int_Af^-\big(\phi(y)\big)\,\big\vert\det\dd\phi(y)\big\vert\,\dd\leb(y);\label{eq:eqmudvarf-}
\end{gather}
a conclusão segue subtraindo \eqref{eq:eqmudvarf-} de \eqref{eq:eqmudvarf+}, tendo em mente que
as funções:
\[A\ni y\longmapsto f^+\big(\phi(y)\big)\,\big\vert\det\dd\phi(y)\big\vert,\quad
A\ni y\longmapsto f^-\big(\phi(y)\big)\,\big\vert\det\dd\phi(y)\big\vert\]
são respectivamente a parte positiva e a parte negativa da função \eqref{eq:funcaomudada}.
\end{proof}

\end{section}

\begin{section}[recordação de Cálculo no $\R^n$]{Apêndice à Seção~\ref{sec:teomudvar}: recordação de Cálculo no ${\R^n}$}
\label{sec:appCalculo}

Seja $U\subset\R^m$ um aberto e $\phi:U\to\R^n$ uma função. Recorde que $\phi$ é dita
{\em diferenciável\/}\index[indice]{diferenciavel@diferenciável!funcao@função}\index[indice]{funcao@função!diferenciavel@diferenciável}
num ponto $x\in U$ se existe uma aplicação linear $T:\R^m\to\R^n$ tal que (recorde Notação~\ref{not:normainfinito}):
\begin{equation}\label{eq:defdiff}
\lim_{h\to0}\frac{\phi(x+h)-\phi(x)-T(h)}{\Vert h\Vert_\infty}=0;
\end{equation}
essa aplicação linear é única quando existe e é dada por:\index[simbolos]{$\frac{\partial\phi}{\partial v}(x)$}
\[T(v)=\lim_{t\to0}\frac{\phi(x+tv)-\phi(x)}t\stackrel{\text{def}}=\frac{\partial\phi}{\partial v}(x),\]
para todo $v\in\R^m$. A aplicação linear $T$ é chamada a
{\em diferencial\/}\index[indice]{diferencial!de uma funcao@de uma função}
de $\phi$ no ponto $x$ e é denotada por $\dd\phi(x)$\index[simbolos]{$\dd\phi(x)$}. A matriz que representa
a diferencial $\dd\phi(x)$ com respeito às bases canônicas é chamada a
{\em matriz Jacobiana\/}\index[indice]{matriz!Jacobiana}\index[indice]{Jacobiana!matriz}
de $\phi$ no ponto $x$. No que segue, {\em usaremos a mesma notação para a diferencial $\dd\phi(x)$ e para a
matriz Jacobiana de $\phi$ no ponto $x$}. Temos:
\[\dd\phi(x)=\begin{pmatrix}
\frac{\partial\phi_1}{\partial x_1}(x)&\cdots&\frac{\partial\phi_1}{\partial x_m}(x)\\
\vdots&\ddots&\vdots\\
\frac{\partial\phi_n}{\partial x_1}(x)&\cdots&\frac{\partial\phi_n}{\partial x_m}(x)
\end{pmatrix},\]
onde $\phi=(\phi_1,\ldots,\phi_n)$ e $\frac{\partial\phi_i}{\partial x_j}(x)$ denota a derivada parcial
no ponto $x$ da função coordenada $\phi_i$ com respeito à $j$-ésima variável.
Se uma aplicação $\phi$ é diferenciável num ponto $x$ então $\phi$ é contínua nesse ponto.

Intuitivamente, \eqref{eq:defdiff} diz que $T=\dd\phi(x)$ é uma ``boa aproximação linear'' para $\phi$ numa vizinhança
de $x$. Mais explicitamente, quando o ponto $x\in\R^m$ sofre um deslocamento (vetorial) $\Delta x$
então o ponto $y=\phi(x)\in\R^n$ sofre um deslocamento (vetorial) $\Delta y=\phi(x+\Delta x)-\phi(x)$ e a diferenciabilidade
de $\phi$ no ponto $x$ nos diz que $\Delta y$ é aproximadamente uma função linear de $\Delta x$; mais precisamente,
existe uma aplicação linear $\dd\phi(x)\stackrel{\text{def}}=T$, tal que $\Delta y$ difere de $T(\Delta x)$
por uma quantidade que vai a zero mais rápido que $\Vert\Delta x\Vert_\infty$, quando $\Delta x\to0$.

Quando uma aplicação $\phi:U\to\R^n$ definida num aberto $U$ de $\R^m$ é diferenciável em todos os pontos de $U$
dizemos simplesmente que ela é {\em diferenciável\/} em $U$; dizemos que $\phi$ é
{\em de classe $C^1$}\index[indice]{classe C1@classe $C^1$!funcao de@função de}\index[indice]{funcao@função!de classe C1@de classe $C^1$}
em $U$ se $\phi$ é diferenciável em $U$ e se a função $U\ni x\mapsto\dd\phi(x)$ é contínua. Sabe-se
que {\em uma função $\phi$ é de classe $C^1$ num aberto $U$ se e somente se as derivadas parciais $\frac{\partial \phi_i}{\partial x_j}(x)$,
$i=1,\ldots,n$, $j=1,\ldots,m$, existem e são contínuas em todos os pontos $x\in U$}.

Enunciamos agora alguns teoremas básicos de Cálculo no $\R^n$ que usamos na Seção~\ref{sec:teomudvar}.
\begin{teo}[regra da cadeia]\index[indice]{regra!da cadeia}\index[indice]{cadeia!regra da}
Sejam $\phi:U\to\R^n$, $\psi:V\to\R^p$ funções tais que $\phi(U)\subset V$, onde $U$ é um aberto de $\R^m$
e $V$ é um aberto de $\R^n$. Se $\phi$ é diferenciável num ponto $x\in U$ e $\psi$ é diferenciável no ponto
$\phi(x)$ então a função composta $\psi\circ\phi$ é diferenciável no ponto $x$ e sua diferencial é dada por:
\[\dd(\psi\circ\phi)(x)=\dd\psi\big(\phi(x)\big)\circ\dd\phi(x).\]
\end{teo}

Segue diretamente da definição de diferenciabilidade que toda aplicação linear $T:\R^m\to\R^n$
é diferenciável em $\R^m$ e $\dd T(x)=T$, para todo $x\in\R^m$. Dessa observação e da regra da cadeia obtemos:
\begin{cor}\label{thm:corregracadeia}
Seja $\phi:U\to\R^n$ uma função definida num aberto $U\subset\R^m$, diferenciável num ponto $x\in U$. Se
$T:\R^n\to\R^p$ é uma aplicação linear então $T\circ\phi$ é diferenciável no ponto $x$ e sua diferencial é
dada por:
\[\dd(T\circ\phi)(x)=T\circ\dd\phi(x).\eqno{\qed}\]
\end{cor}

Para o teorema a seguir,
o leitor deve recordar a Notação~\ref{not:normainfinito} e a Observação~\ref{thm:linearLips}, onde definimos a norma de uma
aplicação linear.
\begin{teo}[desigualdade do valor médio]\index[indice]{desigualdade!do valor medio@do valor médio}\index[indice]{valor medio@valor médio!desigualdade do}
Seja $\phi:U\to\R^n$ uma função definida num aberto $U\subset\R^m$ e sejam fixados dois pontos $x,y\in U$. Suponha
que a função $\phi$ é contínua em todos os pontos do {\em segmento de reta fechado}:\index[indice]{segmento de reta}\index[simbolos]{$[x,y]$}
\[[x,y]=\big\{x+\theta(y-x):0\le\theta\le1\big\}\]
e é diferenciável em todos os pontos do {\em segmento de reta aberto}:\index[simbolos]{$\left]x,y\right[$}
\[\left]x,y\right[=\big\{x+\theta(y-x):0<\theta<1\big\}.\]
Então existe $\theta\in\left]0,1\right[$ tal que vale a desigualdade:
\[\Vert\phi(y)-\phi(x)\Vert_\infty\le\big\Vert\dd\phi\big(x+\theta(y-x)\big)\big\Vert\Vert y-x\Vert_\infty.\]
\end{teo}

Recorde que um subconjunto $X$ de $\R^n$ é dito {\em convexo\/}\index[indice]{convexo!conjunto}\index[indice]{conjunto!convexo}
se para todos $x,y\in X$ o segmento de reta $[x,y]$ está contido em $X$.
\begin{cor}\label{thm:cordesigmedLips}
Sejam $\phi:U\to\R^n$ uma função definida num aberto $U\subset\R^m$ e suponha que $\phi$ é diferenciável em todos
os pontos de um subconjunto convexo $X$ de $U$. Se existe $k\ge0$ tal que $\Vert\dd\phi(x)\Vert\le k$, para todo $x\in X$
então a função $\phi\vert_X$ é Lipschitziana com constante de Lipschitz $k$.\qed
\end{cor}

\begin{cor}\label{thm:C1locLips}
Uma função $\phi:U\to\R^n$ de classe $C^1$ num aberto $U\subset\R^m$ é localmente Lipschitziana.
\end{cor}
\begin{proof}
Segue do Corolário~\ref{thm:cordesigmedLips}, observando que a função $x\mapsto\Vert\dd\phi(x)\Vert$ é contínua e portanto
limitada numa bola suficientemente pequena centrada num ponto dado $x\in U$.
\end{proof}

\begin{defin}
Se $U$, $V\subset\R^n$ são abertos então um {\em difeomorfismo\/}\index[indice]{difeomorfismo}
de $U$ para $V$ é uma bijeção diferenciável
$\phi:U\to V$ cuja inversa $\phi^{-1}:V\to U$ também é diferenciável. Dizemos que $\phi:U\to V$ é um {\em difeomorfismo $C^1$}
se $\phi$ é bijetora e se $\phi$ e $\phi^{-1}$ são ambas de classe $C^1$.
\end{defin}
Se $\phi:U\to V$ é um difeomorfismo então segue da regra da cadeia que para todo $x\in U$ a diferencial
$\dd\phi(x):\R^n\to\R^n$ é um isomorfismo de $\R^n$ cujo inverso é dado por:
\begin{equation}\label{eq:difinverso}
\big(\dd\phi(x)\big)^{-1}=\dd(\phi^{-1})\big(\phi(x)\big).
\end{equation}
Temos a seguinte recíproca para essa afirmação:
\begin{teo}[da função inversa]\index[indice]{teorema!da funcao inversa@da função inversa}\index[indice]{funcao inversa@função inversa!teorema da}
\label{thm:TFI}
Seja $\phi:U\to\R^n$ uma função de classe $C^1$ definida num aberto $U\subset\R^n$. Se $x\in U$ é tal que a diferencial
$\dd\phi(x)$ é um isomorfismo de $\R^n$ então existe uma vizinhança aberta $U_0$ de $x$ contida em $U$
tal que $\phi(U_0)$ é aberto em $\R^n$ e $\phi\vert_{U_0}:U_0\to\phi(U_0)$ é um difeomorfismo $C^1$. Além do mais,
se $\dd\phi(x)$ é um isomorfismo de $\R^n$ para todo $x\in U$ então:
\begin{itemize}
\item $\phi$ é uma {\em aplicação aberta}\index[indice]{aplicacao@aplicação!aberta}\index[indice]{aberta!aplicacao@aplicação},
i.e., $\phi$ leva subconjuntos abertos de $U$ em subconjuntos abertos de $\R^n$;
\item se $U_0$ é um aberto qualquer contido em $U$ tal que $\phi\vert_{U_0}$ é injetora então $\phi\vert_{U_0}:U_0\to\phi(U_0)$
é um difeomorfismo $C^1$.
\end{itemize}
\end{teo}

\end{section}

\section*{Exercícios para o Capítulo~\ref{CHP:MUDVAR}}

\subsection*{O Efeito de Aplicações Lineares sobre a Medida de Lebesgue}

\begin{exercise}
Dados pontos $p_1,\ldots,p_{n+1}\in\R^n$, então o {\em simplexo\/}\index[indice]{simplexo}
de vértices $p_1,\ldots,p_{n+1}$ é definido por:
\begin{equation}\label{eq:defsimplexogeral}
\Big\{\sum_{i=1}^{n+1}a_ip_i:a_i\ge0,\ i=1,\ldots,n+1,\ \sum_{i=1}^{n+1}a_i=1\Big\}.
\end{equation}
Mostre que o simplexo \eqref{eq:defsimplexogeral} é mensurável e determine
uma expressão para a sua medida de Lebesgue.
\end{exercise}

\subsection*{O Teorema de Mudança de Variáveis}

\begin{exercise}
Dados $(x_0,y_0)\in\R^2$ e $r>0$, mostre que o disco:
\[\big\{(x,y)\in\R^2:(x-x_0)^2+(y-y_0)^2\le r^2\big\}\]
é mensurável e determine sua medida de Lebesgue.
\end{exercise}

\begin{exercise}
Considere a aplicação $\phi:\left]0,+\infty\right[\times\R\to\R^2$ definida por:
\[\phi(\rho,\theta)=(\rho\cos\theta,\rho\,\sen\theta),\]
para todos $\rho\in\left]0,+\infty\right[$, $\theta\in\R$.
\begin{itemize}
\item Calcule $\det\dd\phi(\rho,\theta)$.
\item Se $A=\left]0,1\right]\times[0,4\pi]$ e $f:\R^2\to\R$ denota a função constante
e igual a $1$, calcule as integrais:
\[\int_{\phi(A)}f(x,y)\,\dd\leb(x,y),\quad
\int_A\big\vert\det\dd\phi(\rho,\theta)\big\vert\,\dd\leb(\rho,\theta).\]
\item Explique o que está acontecendo, em vista do Teorema~\ref{thm:teomudvar}.
\end{itemize}
\end{exercise}

\begin{exercise}
Seja $A$ um subconjunto de $\R^n$ e $p=(p_1,\ldots,p_{n+1})$ um ponto de $\R^{n+1}$
com $p_{n+1}\ne0$. Identifiquemos $\R^{n+1}$ com o produto $\R^n\times\R$.
O {\em cone de base $A$ e vértice $p$\/}\index[indice]{cone}
é definido por:\index[simbolos]{$C(A,p)$}
\[C(A,p)=\bigcup_{x\in A}[(x,0),p]=\big\{(x,0)+t\big(p-(x,0)\big):x\in A,\ t\in[0,1]\big\}.\]
Considere a função $\phi:\R^n\times\left]0,1\right[\to\R^{n+1}$
definida por:
\[\phi(x,t)=(x,0)+t\big(p-(x,0)\big),\]
para todos $x\in\R^n$, $t\in\left]0,1\right[$. Mostre que:
\begin{itemize}
\item $\phi$ é injetora, de classe $C^1$ e $\det\dd\phi(x,t)=(1-t)^np_{n+1}$,
para todos $x\in\R^n$, $t\in\left]0,1\right[$;
\item se $A$ é mensurável então o cone $C(A,p)$ é mensurável e sua medida de Lebesgue
é dada por:
\[\leb\big(C(A,p)\big)=\frac{\leb(A)\vert p_{n+1}\vert}{n+1}.\]
\end{itemize}
\end{exercise}

\begin{exercise}\label{exe:integralGauss}
Mostre que:
\[\Big(\int_0^{+\infty}e^{-x^2}\,\dd\leb(x)\Big)^2=\int_Qe^{-(x^2+y^2)}\,\dd\leb(x,y),\]
onde $Q=\left[0,+\infty\right[\times\left[0,+\infty\right[$; use essa identidade,
juntamente com uma mudança de variáveis apropriada, para calcular a integral
$\int_0^{+\infty}e^{-x^2}\,\dd\leb(x)$.
\end{exercise}

\end{chapter}

\begin{chapter}{Alguns Tópicos de Análise Funcional}
\label{CHP:TOPANFUNC}

Neste capítulo supõe-se que o leitor tenha familiaridade com conceitos básicos da teoria dos espaços métricos.

\begin{section}{Espaços Normados e com Produto Interno}

Seja $E$ um espaço vetorial sobre $\K$,\index[simbolos]{$\K$} onde $\K$ denota o corpo dos números reais ou o corpo dos números complexos.
\begin{defin}
Uma {\em semi-norma\/}\index[indice]{semi-norma} em $E$ é uma aplicação:\index[simbolos]{$\Vert\cdot\Vert$}
\[E\ni x\longmapsto\Vert x\Vert\in\R\]
satisfazendo as seguintes condições:
\begin{itemize}
\item[(a)] $\Vert x\Vert\ge0$, para todo $x\in E$;
\item[(b)] $\Vert\lambda x\Vert=\vert\lambda\vert\Vert x\Vert$, para todo $\lambda\in\K$ e todo $x\in E$;
\item[(c)] $\Vert x+y\Vert\le\Vert x\Vert+\Vert y\Vert$, para todos $x,y\in E$ ({\em desigualdade triangular}).
\end{itemize}
Uma {\em norma\/}\index[indice]{norma} em $E$ é uma semi-norma que satisfaz a condição adicional:
\begin{itemize}
\item[(d)] $\Vert x\Vert>0$, para todo $x\in E$ com $x\ne0$.
\end{itemize}
Um {\em espaço vetorial normado sobre $\K$\/}\index[indice]{espaco vetorial@espaço vetorial!normado}\index[indice]{normado!espaco vetorial@espaço vetorial}
(ou, mais abreviadamente, um {\em espaço normado})\index[indice]{espaco@espaço!normado} é um par
$(E,\Vert\cdot\Vert)$, onde $E$ é um espaço vetorial sobre $\K$ e $\Vert\cdot\Vert$ é uma norma em $E$.
\end{defin}
Note que fazendo $\lambda=0$ na condição~(b) obtemos:
\[\Vert0\Vert=0.\]

Dado um espaço vetorial normado $(E,\Vert\cdot\Vert)$, nós definimos:\index[simbolos]{$d(x,y)$}
\begin{equation}\label{eq:metricanorma}
d(x,y)=\Vert x-y\Vert,
\end{equation}
para todos $x,y\in E$. Temos que $d:E\times E\to\R$ é uma métrica em $E$; de fato, segue das condições~(a)
e (d) que, para todos $x,y\in E$, $d(x,y)\ge0$ e que $d(x,y)=0$ se e somente $x=y$. Usando a condição~(b), obtemos:
\[d(x,y)=\Vert x-y\Vert=\big\Vert(-1)\cdot(y-x)\big\Vert=\Vert y-x\Vert=d(y,x),\]
e usando a condição~(c) obtemos:
\[d(x,z)=\Vert x-z\Vert=\big\Vert(x-y)+(y-z)\big\Vert\le\Vert x-y\Vert+\Vert y-z\Vert=d(x,y)+d(y,z),\]
para todos $x,y,z\in E$. Dizemos que a métrica $d$ definida em \eqref{eq:metricanorma} é a métrica
{\em associada\/} à (ou {\em determinada} pela)\index[indice]{metrica@métrica!associada a uma norma} norma $\Vert\cdot\Vert$.
{\em Nós sempre assumiremos que um espaço normado $(E,\Vert\cdot\Vert)$ está munido da métrica $d$ associada a sua norma}.
Temos então que todo espaço normado é também um espaço métrico (para uma recíproca, veja
o Exercício~\ref{exe:metricanorma}).

\begin{defin}
Um {\em espaço de Banach sobre $\K$\/}\index[indice]{espaco@espaço!de Banach}\index[indice]{Banach!espaco de@espaço de} é um espaço normado
$(E,\Vert\cdot\Vert)$ sobre $\K$ tal que a métrica associada à norma $\Vert\cdot\Vert$ é completa.
\end{defin}

\begin{example}\label{exa:BoundedXK}
Dado um conjunto arbitrário $X$, então o conjunto $\K^X$\index[simbolos]{$\K^X$} de todas as funções $f:X\to\K$ possui uma estrutura
natural de espaço vetorial sobre $\K$ definida por:
\[(f+g)(x)=f(x)+g(x),\quad(\lambda f)(x)=\lambda f(x),\]
para todos $x\in X$, $\lambda\in\K$, $f,g\in\K^X$. O conjunto $\Bounded(X,\K)$\index[simbolos]{$\Bounded(X,\K)$} de todas as funções
limitadas $f:X\to\K$ é um subespaço de $\K^X$ e a aplicação $\Vert\cdot\Vert_\Sup:\Bounded(X,\K)\to\R$\index[simbolos]{$\Vert\cdot\Vert_\Sup$} definida
por:
\[\Vert f\Vert_\Sup=\sup_{x\in X}\big\vert f(x)\big\vert,\]
para toda $f\in\Bounded(X,\K)$ é uma norma em $\Bounded(X,\K)$. A norma $\Vert\cdot\Vert_\Sup$ é chamada a {\em norma do supremo\/}\index[indice]{norma!do supremo}
e a métrica $d_\Sup$ associada a $\Vert\cdot\Vert_\Sup$ é chamada a {\em métrica do supremo}.\index[indice]{metrica@métrica!do supremo}
Temos que uma seqüência $(f_n)_{n\ge1}$ em $\Bounded(X,\K)$ converge para uma função $f\in\Bounded(X,\K)$
(resp., é de Cauchy) com respeito à métrica $d_\Sup$ se e somente se $(f_n)_{n\ge1}$ converge uniformemente
para $f$ (resp., é uniformemente de Cauchy). Se $(f_n)_{n\ge1}$ é uma seqüência uniformemente de Cauchy
então $(f_n)_{n\ge1}$ também é pontualmente de Cauchy e portanto existe $f:X\to\K$ tal que
$(f_n)_{n\ge1}$ converge para $f$ pontualmente; segue do resultado do Exercício~\ref{exe:Cauchy3}
que $(f_n)_{n\ge1}$ converge para $f$ uniformemente. Se todas as funções $f_n$ são limitadas, é fácil ver que
$f$ também é limitada, e portanto a métrica $d_\Sup$ é completa e $\Bounded(X,\K)$ é um espaço de Banach.
Se $X$ é um espaço métrico (ou, mais geralmente, um espaço topológico) então o conjunto
$\Cb(X,\K)$\index[simbolos]{$\Cb(X,\K)$} de todas as funções contínuas e limitadas $f:X\to\K$ é um subespaço
de $\Bounded(X,\K)$; como o limite uniforme de funções contínuas é contínua, segue que $\Cb(X,\K)$ é fechado
em $\Bounded(X,\K)$ e portanto também completo com a métrica (induzida por) $d_\Sup$. Segue que $\Cb(X,\K)$
também é um espaço de Banach munido da norma (induzida por) $\Vert\cdot\Vert_\Sup$. Note que se $X$ é compacto
então toda função contínua $f:X\to\K$ é limitada, de modo que $\Cb(X,\K)$ coincide com o espaço
$C(X,\K)$\index[simbolos]{$C(X,\K)$} de todas as funções contínuas $f:X\to\K$.
\end{example}

\begin{defin}
Seja $E$ um espaço vetorial sobre $\K$. Um {\em produto interno\/}\index[indice]{produto interno} em $E$ é uma
aplicação:\index[simbolos]{$\langle\cdot,\cdot\rangle$}
\[E\times E\ni(x,y)\longmapsto\langle x,y\rangle\in\K\]
satisfazendo as seguintes condições:
\begin{itemize}
\item[(a)] $\langle\lambda x+x',y\rangle=\lambda\langle x,y\rangle+\langle x',y\rangle$
e $\langle x,\lambda y+y'\rangle=\bar\lambda\langle x,y\rangle+\langle x,y'\rangle$, para todos
$x,y,x',y'\in E$ e todo $\lambda\in\K$, onde $\bar\lambda$ denota o complexo conjugado de $\lambda$;
\item[(b)] $\langle x,y\rangle=\overline{\langle y,x\rangle}$, para todos $x,y\in E$;
\item[(c)] $\langle x,x\rangle>0$, para todo $x\in E$ com $x\ne0$.
\end{itemize}
\end{defin}
Note que a condição~(b) implica que $\langle x,x\rangle$ é real, para todo $x\in E$, de modo que faz sentido
falar em $\langle x,x\rangle>0$, na condição~(c). Quando $\K=\R$ então $\bar\lambda=\lambda$ para todo $\lambda\in\K$,
de modo que as condições~(a) e (b) podem ser substituídas respectivamente por:
\begin{itemize}
\item[(a')] $\langle\lambda x+x',y\rangle=\lambda\langle x,y\rangle+\langle x',y\rangle$
e $\langle x,\lambda y+y'\rangle=\lambda\langle x,y\rangle+\langle x,y'\rangle$, para todos
$x,y,x',y'\in E$ e todo $\lambda\in\K$;
\item[(b')] $\langle x,y\rangle=\langle y,x\rangle$, para todos $x,y\in E$.
\end{itemize}
Fazendo $\lambda=1$ na condição~(a) obtemos:
\[\langle x+x',y\rangle=\langle x,y\rangle+\langle x',y\rangle,\quad
\langle x,y+y'\rangle=\langle x,y\rangle+\langle x,y'\rangle,\]
para todos $x,y,x',y'\in E$; daí:
\[\langle0,y\rangle=\langle0+0,y\rangle=\langle0,y\rangle+\langle0,y\rangle,\quad
\langle x,0\rangle=\langle x,0+0\rangle=\langle x,0\rangle+\langle x,0\rangle\]
e portanto:
\[\langle x,0\rangle=0,\quad\langle 0,y\rangle=0,\]
para todos $x,y\in E$. Fazendo $x'=0$, $y'=0$ na condição~(a) obtemos então:
\[\langle\lambda x,y\rangle=\lambda\langle x,y\rangle,\quad\langle x,\lambda y\rangle=\bar\lambda\langle x,y\rangle,\]
para todos $x,y\in E$ e todo $\lambda\in\K$.

O primeiro resultado não trivial sobre produtos internos que provaremos é o seguinte:
\begin{lem}[desigualdade de Cauchy--Schwarz]\index[indice]{Cauchy--Schwarz!desigualdade de}\index[indice]{desigualdade de!Cauchy--Schwarz}
\label{thm:CauchySchwarz}
Seja $E$ um espaço vetorial sobre $\K$ e $\langle\cdot,\cdot\rangle$ um produto interno em $E$. Então:
\begin{equation}\label{eq:CauSch}
\big\vert\langle x,y\rangle\big\vert\le\langle x,x\rangle^{\frac12}\langle y,y\rangle^{\frac12},
\end{equation}
para todos $x,y\in E$; a igualdade em \eqref{eq:CauSch} vale se e somente se $x$ e $y$ são linearmente dependentes.
\end{lem}
\begin{proof}
Se $x$ e $y$ são linearmente dependentes então ou $y=\lambda x$ ou $x=\lambda y$, para algum $\lambda\in\K$;
daí é fácil ver que vale a igualdade em \eqref{eq:CauSch}. Suponhamos então que $x$ e $y$ são linearmente independentes
e provemos que vale a desigualdade estrita em \eqref{eq:CauSch}.
Provemos primeiramente que:
\begin{equation}\label{eq:CSreal}
\big\vert\Re\langle x,y\rangle\big\vert<\langle x,x\rangle^{\frac12}\langle y,y\rangle^{\frac12},
\end{equation}
onde $\Re\lambda$\index[simbolos]{$\Re\lambda$} denota a parte
real\index[indice]{parte real!de um numero complexo@de um número complexo}\index[indice]{numero complexo@número complexo!parte real de}
de um número complexo $\lambda$.
Considere a função $p:\R\to\R$ definida por:
\[p(t)=\langle x+ty,x+ty\rangle,\]
para todo $t\in\R$. Temos:
\[p(t)=\langle x,x\rangle+t\langle x,y\rangle+t\langle y,x\rangle+t^2\langle y,y\rangle=
\langle x,x\rangle+2t\Re\langle x,y\rangle+t^2\langle y,y\rangle,\]
para todo $t\in\R$. Como $x$ e $y$ são linearmente independentes, temos que $x+ty$ é não nulo, para todo $t\in\R$
e portanto $p(t)>0$, para todo $t\in\R$. Mas $p$ é uma função polinomial do segundo grau e portanto
seu discriminante:
\[\Delta=4\big(\Re\langle x,y\rangle\big)^2-4\langle x,x\rangle\langle y,y\rangle\]
deve ser negativo. Daí \eqref{eq:CSreal} segue diretamente. Seja agora $\lambda\in\K$ com
$\vert\lambda\vert=1$ tal que $\lambda\langle x,y\rangle$ é real\footnote{%
Dado um número complexo $z$, evidentemente existe um número complexo $\lambda$ de módulo $1$ tal que $\lambda z$
é real; basta tomar $\lambda=\frac{\bar z}{\vert z\vert}$, se $z\ne0$ e $\lambda=1$ se $z=0$.}; trocando $x$ por
$\lambda x$ em \eqref{eq:CSreal} obtemos:
\[\big\vert\Re\langle\lambda x,y\rangle\big\vert<\langle\lambda x,\lambda x\rangle^{\frac12}\langle y,y\rangle^{\frac12},\]
donde segue a desigualdade:
\[\big\vert\langle x,y\rangle\big\vert<\langle x,x\rangle^{\frac12}\langle y,y\rangle^{\frac12}.\qedhere\]
\end{proof}

\begin{cor}
Seja $E$ um espaço vetorial sobre $\K$ e $\langle\cdot,\cdot\rangle$ um produto interno em $E$. Então a aplicação
$\Vert\cdot\Vert:E\to\R$ definida por:\index[simbolos]{$\Vert x\Vert$}
\begin{equation}\label{eq:normaprodint}
\Vert x\Vert=\langle x,x\rangle^{\frac12},
\end{equation}
para todo $x\in E$, é uma norma em $E$.
\end{cor}
\begin{proof}
Note que, como $\langle x,x\rangle\ge0$ para todo $x\in E$, a aplicação $\Vert\cdot\Vert$ está bem definida
e $\Vert x\Vert\ge0$, para todo $x\in E$; além do mais, $\Vert x\Vert>0$, para $x\in E$ não nulo. Dados $\lambda\in\K$,
$x\in E$, temos:
\[\Vert\lambda x\Vert=\langle\lambda x,\lambda x\rangle^{\frac12}=\big(\lambda\bar\lambda\langle x,x\rangle\big)^{\frac12}
=\big(\vert\lambda\vert^2\langle x,x\rangle\big)^{\frac12}=\vert\lambda\vert\Vert x\Vert.\]
Finalmente, a desigualdade triangular é obtida da desigualdade de Cauchy--Schwarz, através dos cálculos abaixo:
\begin{multline*}
\Vert x+y\Vert^2=\langle x+y,x+y\rangle=\langle x,x\rangle+\langle x,y\rangle+\langle y,x\rangle+\langle y,y\rangle\\
=\Vert x\Vert^2+\Vert y\Vert^2+2\Re\langle x,y\rangle\le\Vert x\Vert^2+\Vert y\Vert^2+2\big\vert\langle x,y\rangle\big\vert\\
\le\Vert x\Vert^2+\Vert y\Vert^2+2\Vert x\Vert\Vert y\Vert=\big(\Vert x\Vert+\Vert y\Vert\big)^2,
\end{multline*}
onde $x,y\in E$.
\end{proof}
A norma definida em \eqref{eq:normaprodint} é chamada a norma {\em associada ao\/} (ou {\em determinada pelo})\index[indice]{norma!associada a um produto interno}
produto interno $\langle\cdot,\cdot\rangle$.

\begin{defin}
Seja $E$ um espaço vetorial sobre $\K$ e $\langle\cdot,\cdot\rangle$ um produto interno em $E$. O par
$(E,\langle\cdot,\cdot\rangle)$ é chamado um {\em espaço pré-Hilbertiano sobre $\K$}.\index[indice]{espaco@espaço!pre-Hilbertiano@pré-Hilbertiano}
Se a métrica associada à norma associada ao produto interno $\langle\cdot,\cdot\rangle$ for completa, dizemos
que $(E,\langle\cdot,\cdot\rangle)$ é um {\em espaço de Hilbert sobre $\K$}.\index[indice]{espaco@espaço!de Hilbert}
\end{defin}
{\em Nós sempre assumiremos que um espaço pré-Hilbertiano $(E,\langle\cdot,\cdot\rangle)$ está munido da norma
\eqref{eq:normaprodint} associada ao seu produto interno}. Vemos então que todo espaço pré-Hilbertiano é um espaço normado
e todo espaço de Hilbert é um espaço de Banach. Nem toda norma está associada a um produto interno, como segue facilmente
do seguinte:
\begin{lem}[identidade do paralelogramo]\index[indice]{paralelogramo!identidade do}\index[indice]{identidade!do paralelogramo}
Se $E$ é um espaço vetorial sobre $\K$, $\langle\cdot,\cdot\rangle$ é um produto interno em $E$ e $\Vert\cdot\Vert$
é a norma associada a $\langle\cdot,\cdot\rangle$ então:
\begin{equation}\label{eq:paralelogramo}
\Vert x+y\Vert^2+\Vert x-y\Vert^2=2\big(\Vert x\Vert^2+\Vert y\Vert^2\big),
\end{equation}
para todos $x,y\in E$.
\end{lem}
\begin{proof}
Temos:
\[\Vert x+y\Vert^2=\langle x+y,x+y\rangle=\langle x,x\rangle+\langle x,y\rangle+\langle y,x\rangle+\langle y,y\rangle,\]
donde:
\begin{equation}\label{eq:normaxplusy}
\Vert x+y\Vert^2=\Vert x\Vert^2+\Vert y\Vert^2+2\Re\langle x,y\rangle;
\end{equation}
similarmente:
\begin{equation}\label{eq:normaxminusy}
\Vert x-y\Vert^2=\Vert x\Vert^2+\Vert y\Vert^2-2\Re\langle x,y\rangle.
\end{equation}
A conclusão é obtida somando \eqref{eq:normaxplusy} e \eqref{eq:normaxminusy}.
\end{proof}
No Exercício~\ref{exe:supparalelogramo} pedimos ao leitor para mostrar que a norma do supremo não satisfaz a identidade do paralelogramo
(exceto pelo caso trivial, em que o domínio tem um único ponto). Vemos então que nem toda norma está associada a um produto
interno. No Exercício~\ref{exe:critparalelogramo} apresentamos um leitor um roteiro para demonstrar que toda norma
que satisfaz a identidade do paralelogramo está associada a um único produto interno.

Se $F$ é um subconjunto fechado não vazio de $\R^n$ e $K$ é um subconjunto compacto não vazio de $\R^n$ então
existem pontos $x\in F$, $y\in K$ tais que $d(x,y)=d(F,K)$, ou seja, a distância mínima entre $F$ e $K$ é explicitamente
realizada\footnote{%
Esse não é o caso se supusermos apenas que $F$ e $K$ são fechados. Por exemplo, se $F=\big\{\big(x,\frac1x\big):x>0\big\}$
e $K=\R\times\{0\}$ então $F$ e $K$ são fechados e disjuntos em $\R^2$, mas $d(K,F)=0$.}.
Como é ilustrado no exemplo a seguir, se $(M,d)$ é um espaço métrico arbitrário, $F\ne\emptyset$ é fechado em $M$
e $K\ne\emptyset$ é um subconjunto compacto de $M$,
não é verdade em geral que a distância mínima entre $F$ e $K$ é realizada, mesmo sob a hipótese que o espaço métrico
$M$ seja completo (é verdade, no entanto, que se $K$ e $F$ são disjuntos então $d(K,F)>0$).

\begin{example}
Seja $C\big([0,1],\R\big)$ o espaço de Banach das funções con\-tí\-nuas $f:[0,1]\to\R$ munido da norma do supremo
(veja Exemplo~\ref{exa:BoundedXK}) e seja $E$ o subespaço de $C\big([0,1],\R\big)$ constituído pelas funções
contínuas $f:[0,1]\to\R$ tais que $f(0)=f(1)=0$. Claramente $E$ é um subespaço fechado de $C\big([0,1],\R\big)$ e
portanto é também um espaço de Banach. Seja $H$ o subconjunto de $E$ definido por:
\[H=\big\{f\in E:{\textstyle\int_0^1f\,\dd\leb=1}\big\}.\]
Se uma seqüência $(f_n)_{n\ge1}$ em $C\big([0,1],\R\big)$ converge uniformemente para $f$ então (veja Exercício~\ref{exe:intconvunif}):
\[\lim_{n\to\infty}\int_0^1f_n\,\dd\leb=\int_0^1f\,\dd\leb,\]
donde segue que $H$ é fechado em $E$. Vamos mostrar que a distância
\[d_\Sup(0,H)=\inf_{f\in H}\Vert f\Vert_\Sup\]
da função nula até $H$ é igual a $1$,
mas que não existe nenhuma função $f\in H$ com $d_\Sup(0,f)=\Vert f\Vert_\Sup=1$. Em primeiro lugar, mostremos que não existe
$f\in H$ com $\Vert f\Vert_\Sup\le1$. De fato, suponha por absurdo que $f\in H$ e $\Vert f\Vert_\Sup\le1$. Daí
$1-f\ge0$ e $\int_0^1(1-f)\,\dd\leb=0$, donde $f=1$ quase sempre; mas $f(0)=f(1)=0$ e a continuidade de $f$ implicam
que $f$ é menor que $1$ numa vizinhança de $\{0,1\}$, o que nos dá uma contradição. Vamos agora mostrar que para todo
$\varepsilon>0$ existe $f\in H$ com $\Vert f\Vert_\Sup\le1+\varepsilon$. Obviamente podemos supor sem perda de generalidade
que $\varepsilon\le1$; seja $\eta=\frac\varepsilon{1+\varepsilon}\in\left]0,\frac12\right]$ e considere
a função $f:[0,1]\to\R$ definida por:
\[f(x)=\begin{cases}
\frac{1+\varepsilon}\eta\,x,&\text{se $0\le x\le\eta$},\\
1+\varepsilon,&\text{se $\eta\le x\le1-\eta$},\\
\frac{1+\varepsilon}\eta\,(1-x),&\text{se $1-\eta\le x\le1$}.
\end{cases}\]
É fácil ver que $f\in H$ e que $\Vert f\Vert_\Sup=1+\varepsilon$. Logo $d_\Sup(0,H)=1$,
mas não existe $f\in H$ com $\Vert f\Vert_\Sup=1$.
\end{example}

\begin{prop}\label{thm:distanciaminima}
Seja $(E,\langle\cdot,\cdot\rangle)$ um espaço pré-Hilbertiano sobre $\K$ e seja $C\subset E$ um subconjunto
completo, não vazio, tal que $\frac12(p+q)\in C$, para todos $p,q\in C$. Então para todo $x\in E$ existe um único
ponto $p\in C$ tal que $d(x,p)=d(x,C)$.
\end{prop}
\begin{proof}
Sejam $x\in E$, $p,q\in C$. Aplicando a identidade do paralelogramo \eqref{eq:paralelogramo} aos vetores
$x-p$, $x-q$, obtemos:
\[\Vert2x-p-q\Vert^2+\Vert p-q\Vert^2=2\big(\Vert x-p\Vert^2+\Vert x-q\Vert^2\big),\]
e portanto:
\begin{equation}\label{eq:magicadistpq}
\Vert p-q\Vert^2=2\big(\Vert x-p\Vert^2+\Vert x-q\Vert^2-2\big\Vert x-\tfrac12(p+q)\big\Vert^2\big).
\end{equation}
Seja $c=d(x,C)\ge0$. Provemos primeiramente a unicidade de $p$. Se $p,q\in C$ são tais que $d(x,p)=d(x,q)=c$
então $d\big(x,\frac12(p+q)\big)\ge c$, de modo que \eqref{eq:magicadistpq} nos dá:
\[\Vert p-q\Vert^2\le2(c^2+c^2-2c^2)=0,\]
e daí $p=q$. Para provar a existência de $p$, seja $(p_n)_{n\ge1}$ uma seqüência em $C$ tal que
$d(x,p_n)<c+\frac1n$, para todo $n\ge1$. Usando \eqref{eq:magicadistpq} com $p=p_n$, $q=p_m$
e observando que $d\big(x,\frac12(p_n+p_m)\big)\ge c$, obtemos:
\[\Vert p_n-p_m\Vert^2<2\big[\big(c+\tfrac1n\big)^2+\big(c+\tfrac1m\big)^2-2c^2\big]
=2\big(\tfrac{2c}n+\tfrac{2c}m+\tfrac1{n^2}+\tfrac1{m^2}\big),\]
donde segue que $(p_n)_{n\ge1}$ é uma seqüência de Cauchy. Já que $C$ é completo, existe $p\in C$
com $p_n\to p$ e daí fazendo $n\to\infty$ em $d(x,p_n)<c+\frac1n$ obtemos $d(x,p)\le c$. Como
obviamente $d(x,p)\ge c$, a conclusão segue.
\end{proof}

\begin{defin}
Seja $(E,\langle\cdot,\cdot\rangle)$ um espaço pré-Hilbertiano sobre $\K$. Dois vetores $x,y\in E$ são ditos
{\em ortogonais\/}\index[indice]{ortogonal!vetor}\index[indice]{vetores!ortogonais} se $\langle x,y\rangle=0$. Seja $S$ um subconjunto
de $E$. O {\em complemento ortogonal\/}\index[indice]{ortogonal!complemento}\index[indice]{complemento!ortogonal} de $S$,
denotado por $S^\perp$,\index[simbolos]{$S^\perp$} é conjunto dos vetores $x\in E$ que são
ortogonais a todos os vetores de $S$, isto é:
\[S^\perp=\big\{x\in E:\text{$\langle x,y\rangle=0$, para todo $y\in S$}\big\}.\]
\end{defin}
É fácil ver que o complemento ortogonal de um subconjunto $S$ de $E$ coincide com o complemento ortogonal
do subespaço vetorial de $E$ gerado por $S$, de modo que a noção de complemento ortogonal é particularmente interessante
apenas para subespaços vetoriais. É fácil ver também que o complemento ortogonal de um subconjunto arbitrário de $E$
é sempre um subespaço de $E$.

\begin{lem}\label{thm:maisproxperp}
Sejam $(E,\langle\cdot,\cdot\rangle)$ um espaço pré-Hilbertiano sobre $\K$ e $S$ um subespaço vetorial de $E$.
Dados $x\in E$, $p\in S$, então $d(x,p)=d(x,S)$ se e somente se $x-p\in S^\perp$.
\end{lem}
\begin{proof}
Sejam $x\in E$, $p\in S$ e suponha que $x-p\in S^\perp$. Dado $q\in S$ então $p-q\in S$ e portanto
os vetores $x-p$ e $p-q$ são ortogonais; pelo Teorema de Pitágoras (veja Exercício~\ref{exe:pitagoras}):
\[\Vert x-q\Vert^2=\big\Vert(x-p)+(p-q)\big\Vert^2=\Vert x-p\Vert^2+\Vert p-q\Vert^2\ge\Vert x-p\Vert^2,\]
donde $d(x,p)\le d(x,q)$, para todo $q\in S$. Isso mostra que $d(x,p)=d(x,S)$. Reciprocamente, suponha que
$d(x,p)=d(x,S)$. Dado $v\in S$, considere a função $\phi:\R\to\R$ definida por:
\[\phi(t)=d(x,p+tv)^2=\Vert(x-p)-tv\Vert^2,\]
para todo $t\in\R$. Temos:
\[\phi(t)=\Vert x-p\Vert^2-2t\Re\langle x-p,v\rangle+t^2\Vert v\Vert^2,\]
para todo $t\in\R$. Como $d(x,p)=d(x,S)$, a função $\phi$ possui um mínimo global em $t=0$ e portanto:
\[\phi'(0)=-2\Re\langle x-p,v\rangle=0.\]
Concluímos que:
\begin{equation}\label{eq:Realxminuspv}
\Re\langle x-p,v\rangle=0,
\end{equation}
para todo $v\in S$. Se $\K=\R$, a demonstração já está completa. Se $\K=\C$, podemos trocar $v$ por $iv$ em \eqref{eq:Realxminuspv},
o que nos dá:
\[\Re\langle x-p,iv\rangle=-\Re\big(i\langle x-p,v\rangle\big)=\Im\langle x-p,v\rangle=0,\]
onde $\Im\lambda$\index[simbolos]{$\Im\lambda$} denota a parte imaginária de um número complexo
$\lambda$.\index[indice]{parte imaginaria@parte imaginária!de um numero complexo@de um número complexo}%
\index[indice]{numero complexo@número complexo!parte imaginaria de@parte imaginária de}
Daí:
\[\langle x-p,v\rangle=0\]
e a demonstração está completa.
\end{proof}

\begin{defin}
Sejam $(E,\langle\cdot,\cdot\rangle)$ um espaço pré-Hilbertiano sobre $\K$ e $S$ um subespaço vetorial de $E$.
Dado $x\in E$ então um ponto $p\in S$ com $x-p\in S^\perp$ é dito uma {\em projeção ortogonal\/}\index[indice]{projecao@projeção!ortogonal}\index[indice]{ortogonal!projecao@projeção}
de $x$ em $S$.
\end{defin}
Temos que a projeção ortogonal de $x$ em $S$ é única quando existe; de fato, se $p,q\in S$ e
$x-p,x-q\in S^\perp$ então $p-q=(x-q)-(x-p)\in S^\perp$ e $p-q\in S$, de modo que $\langle p-q,p-q\rangle=0$
e $p-q=0$.

\begin{cor}\label{thm:existeprojort}
Sejam $(E,\langle\cdot,\cdot\rangle)$ um espaço pré-Hilbertiano sobre $\K$ e $S$ um subespaço vetorial de $E$.
Suponha que $S$ é completo (esse é o caso, por exemplo, se $(E,\langle\cdot,\cdot\rangle)$ é um espaço de Hilbert
e $S$ é fechado em $E$). Então todo $x\in E$ admite uma (única) projeção ortogonal $p\in S$.
\end{cor}
\begin{proof}
Segue da Proposição~\ref{thm:distanciaminima} que existe $p\in S$ com $d(x,p)=d(x,S)$.
O Lema~\ref{thm:maisproxperp} nos diz então que $x-p\in S^\perp$.
\end{proof}

\begin{cor}\label{thm:cosSoplusSperp}
Sejam $(E,\langle\cdot,\cdot\rangle)$ um espaço pré-Hilbertiano sobre $\K$ e $S$ um subespaço vetorial de $E$.
Suponha que $S$ é completo (esse é o caso, por exemplo, se $(E,\langle\cdot,\cdot\rangle)$ é um espaço de Hilbert
e $S$ é fechado em $E$). Então $E=S\oplus S^\perp$.
\end{cor}
\begin{proof}
Se $v\in S\cap S^\perp$ então $\langle v,v\rangle=0$, de modo que $v=0$. O Corolário~\ref{thm:existeprojort}
implica que todo elemento de $E$ é soma de um elemento de $S$ com um elemento de $S^\perp$, i.e., $E=S+S^\perp$.
A conclusão segue.
\end{proof}

\end{section}

\begin{section}{Aplicações Lineares Contínuas}

\begin{lem}\label{thm:equivlincont}
Sejam $(E,\norma\cdot E)$, $(F,\norma\cdot F)$ espaços normados sobre $\K$
e $T:E\to F$ uma aplicação linear. As seguintes afirmações são equivalentes:
\begin{itemize}
\item[(a)] $T$ é contínua;
\item[(b)] $T$ é contínua na origem;
\item[(c)] $T$ é limitada em alguma vizinhança da origem, i.e., existem $c\ge0$ e uma vizinhança $V$ da origem
em $E$ tal que $\norma{T(x)}F\le c$, para todo $x\in V$;
\item[(d)] $T$ é limitada na bola unitária de $E$, i.e., existe $c\ge0$ tal que $\norma{T(x)}F\le c$,
para todo $x\in E$ com $\norma xE$;
\item[(e)] $T$ é limitada na esfera unitária de $E$, i.e., existe $c\ge0$ tal que $\norma{T(x)}F\le c$,
para todo $x\in E$ com $\norma xE=1$;
\item[(f)] existe $c\ge0$ tal que $\norma{T(x)}F\le c\norma xE$, para todo $x\in E$;
\item[(g)] $T$ é Lipschitziana.
\end{itemize}
\end{lem}
\begin{proof}\
\begin{bulletindent}
\item (a)$\Rightarrow$(b).

Trivial.

\item (b)$\Rightarrow$(c).

Dado, por exemplo, $\varepsilon=1$, existe $\delta>0$ tal que:
\[\bignorma{T(x)-T(0)}F=\bignorma{T(x)}F<\varepsilon,\]
para todo $x\in E$ com $\norma xE<\delta$. Daí $T$ é limitada na bola aberta de centro na origem de $E$ e raio $\delta$.

\item (c)$\Rightarrow$(d).

Por hipótese, existem $r>0$ e $c\ge0$ tais que $\norma{T(x)}F\le c$, para todo $x\in E$ com $\norma xE\le r$.
Daí, se $x\in E$ é tal que $\norma xE\le1$ então $\norma{rx}E\le r$ e portanto $\norma{T(rx)}F\le c$;
logo $\norma{T(x)}F\le\frac cr$, para todo $x\in E$ com $\norma xE\le1$.

\item (d)$\Rightarrow$(e).

Trivial.

\item (e)$\Rightarrow$(f).

Seja $c\ge0$ tal que $\norma{T(x)}F\le c$, para todo $x\in E$ com $\norma xE=1$. Afirmamos que
$\norma{T(x)}F\le c\norma xE$, para todo $x\in E$. De fato, se $x=0$ essa desigualdade é trivial. Se $x\ne0$,
tomamos $y=\frac x{\subnorma xE}$, de modo que $\norma yE=1$ e:
\[\frac{\bignorma{T(x)}F}{\norma xE}=\bignorma{T(y)}F\le c.\]
A conclusão segue.

\item (f)$\Rightarrow$(g).

Seja $c\ge0$ tal que $\norma{T(x)}F\le c\norma xE$, para todo $x\in E$. Dados, $x,y\in E$, então:
\[\bignorma{T(x)-T(y)}F=\bignorma{T(x-y)}F\le c\norma{x-y}E,\]
de modo que $c$ é uma constante de Lipschitz para $T$.

\item (g)$\Rightarrow$(a).

Trivial.\qedhere
\end{bulletindent}
\end{proof}

\begin{defin}
Sejam $(E,\norma\cdot E)$, $(F,\norma\cdot F)$ espaços normados sobre $\K$. Uma aplicação
linear $T:E\to F$ é dita {\em limitada\/}\index[indice]{limitada!aplicacao linear@aplicação linear}%
\index[indice]{aplicacao linear@aplicação linear!limitada}\index[indice]{linear!aplicacao limitada@aplicação limitada}
se satisfaz uma das (e portanto todas as) condições equivalentes
que aparecem no enunciado do Lema~\ref{thm:equivlincont}.
\end{defin}
Normalmente, uma função $f$ cujo contra-domínio é um espaço métrico é dita {\em limitada\/} quando sua imagem
é um conjunto limitado. No caso de aplicações lineares, usamos a expressão ``limitada'' com um significado
um pouco diferente, i.e., dizemos que uma aplicação linear entre espaços normados é limitada quando sua restrição
à bola unitária do domínio é limitada no sentido mais usual da palavra. Esse uso ligeiramente ambíguo da palavra
``limitada'' é completamente usual e não causa confusão, já que uma aplicação linear não nula nunca pode ter imagem
limitada (veja Exercício~\ref{exe:limitadamesmo}).

Sejam $(E,\norma\cdot E)$, $(F,\norma\cdot F)$ espaços normados sobre $\K$. Denotamos
por $\Lin(E,F)$\index[simbolos]{$\Lin(E,F)$} o conjunto de todas as aplicações lineares limitadas $T:E\to F$.
Segue facilmente do resultado do Exercício~\ref{exe:operacoescontinuas} que $\Lin(E,F)$ é um subespaço
vetorial do espaço de {\em todas\/} as aplicações lineares $T:E\to F$.

\begin{defin}\label{thm:normaapllinear}
Sejam $(E,\norma\cdot E)$, $(F,\norma\cdot F)$ espaços normados sobre $\K$.
Se $T:E\to F$ é uma aplicação linear limitada então a
{\em norma\/}\index[indice]{norma!de uma aplicacao linear@de uma aplicação linear}%
\index[indice]{aplicacao linear@aplicação linear!norma de}\index[indice]{linear!aplicacao@aplicação!norma de}
de $T$ é definida por:
\begin{equation}\label{eq:normaopEF}
\Vert T\Vert=\sup\big\{\norma{T(x)}F:\text{$x\in E$ e $\norma xE\le1$}\big\}\in\left[0,+\infty\right[.
\end{equation}
\end{defin}
Quando o espaço $E$ é não nulo, então a norma de uma aplicação linear $T:E\to F$ coincide
com o supremo de $\norma{T(x)}F$ quando $x$ percorre a esfera unitária de $E$ (veja
Exercício~\ref{exe:supnaesfera}). Deixamos a cargo do leitor a verificação de que
\eqref{eq:normaopEF} define uma norma no espaço vetorial $\Lin(E,F)$ (veja Exercício~\ref{exe:normaop}).

\begin{lem}\label{thm:normacomposta}
Sejam $(E,\norma\cdot E)$, $(F,\norma\cdot F)$ espaços normados sobre $\K$.
Então:
\begin{equation}\label{eq:normaTx}
\norma{T(x)}F\le\Vert T\Vert\norma xE,
\end{equation}
para todos $T\in\Lin(E,F)$, $x\in E$. Se $(G,\norma\cdot G)$ é um espaço normado sobre $\K$
e $T\in\Lin(E,F)$, $S\in\Lin(F,G)$ então:
\[\Vert S\circ T\Vert\le\Vert S\Vert\Vert T\Vert.\]
\end{lem}
\begin{proof}
Seja $x\in E$. Se $x=0$, então a desigualdade \eqref{eq:normaTx} é trivial.
Senão, seja $y=\frac x{\subnorma xE}$; temos $\norma yE=1$ e portanto:
\[\frac{\bignorma{T(x)}F}{\norma xE}=\bignorma{T(y)}F\le\Vert T\Vert,\]
donde a desigualdade \eqref{eq:normaTx} segue.
Seja $S\in\Lin(F,G)$; para todo $x\in E$ com $\norma xE\le1$, temos:
\[\bignorma{(S\circ T)(x)}G=\bignorma{S\big(T(x)\big)}G\le
\Vert S\Vert\bignorma{T(x)}F\le\Vert S\Vert\Vert T\Vert,\]
donde $\Vert S\circ T\Vert\le\Vert S\Vert\Vert T\Vert$.
\end{proof}

\begin{prop}\label{thm:LinBanach}
Sejam $(E,\norma\cdot E)$, $(F,\norma\cdot F)$ espaços normados sobre $\K$. Se $(F,\norma\cdot F)$
é um espaço de Banach então também $\Lin(E,F)$ é um espaço de Banach.
\end{prop}
\begin{proof}
Seja $(T_n)_{n\ge1}$ uma seqüência de Cauchy em $\Lin(E,F)$. Para todo $x\in E$, temos:
\[\bignorma{T_n(x)-T_m(x)}F\le\Vert T_n-T_m\Vert\norma xE,\]
para todos $n,m\ge1$; segue que $\big(T_n(x)\big)_{n\ge1}$ é uma seqüência de Cauchy em $F$.
Como $F$ é completo, podemos definir uma aplicação $T:E\to F$ fazendo:
\[T(x)=\lim_{n\to\infty}T_n(x),\]
para todo $x\in E$. Se $x,y\in E$ e $\lambda\in\K$ então segue do resultado
do Exercício~\ref{exe:operacoescontinuas} que:
\begin{gather*}
T(x+y)=\lim_{n\to\infty}T_n(x+y)=\lim_{n\to\infty}\big(T_n(x)+T_n(y)\big)=T(x)+T(y),\\
T(\lambda x)=\lim_{n\to\infty}T_n(\lambda x)=\lim_{n\to\infty}\big(\lambda T_n(x)\big)
=\lambda T(x),
\end{gather*}
donde $T$ é linear. Seja dado $\varepsilon>0$. Vamos mostrar que existe $n_0\ge1$ tal que
$T_n-T$ é limitada e $\Vert T_n-T\Vert\le\varepsilon$, para todo $n\ge n_0$; seguirá
daí automaticamente que $T$ é limitada, já que $T=T_n-(T_n-T)$, com $T_n$ e $T_n-T$
ambas limitadas. Como $(T_n)_{n\ge1}$ é de Cauchy, existe $n_0\ge1$ tal que
$\Vert T_n-T_m\Vert<\varepsilon$, para todos $n,m\ge n_0$. Fixado $x\in E$
com $\norma xE\le1$ então:
\[\bignorma{T_n(x)-T_m(x)}F<\varepsilon;\]
fazendo $m\to\infty$ (com $n$ e $x$ fixados), obtemos:
\[\bignorma{T_n(x)-T(x)}F\le\varepsilon,\]
para todo $n\ge n_0$ e todo $x\in E$ com $\norma xE\le1$. Daí
$\Vert T_n-T\Vert\le\varepsilon$, para todo $n\ge n_0$. Isso mostra que
$\lim_{n\to\infty}T_n=T$ em $\Lin(E,F)$, o que completa a demonstração.
\end{proof}

\begin{defin}
Sejam $E$, $F$ espaços vetoriais complexos. Uma aplicação $T:E\to F$ é dita
{\em linear-conjugada\/}\index[indice]{aplicacao@aplicação!linear-conjugada}%
\index[indice]{linear-conjugada!aplicacao@aplicação}\index[indice]{conjugada!linear} se:
\[T(x+y)=T(x)+T(y)\quad\text{e}\quad T(\lambda x)=\bar\lambda T(x),\]
para todos $x,y\in E$ e todo $\lambda\in\C$.
\end{defin}
Se $E$, $F$ são espaços vetoriais complexos e $T:E\to F$ é uma aplicação linear-conjugada
então $T$ é $\R$-linear, i.e., a aplicação $T:E\vert_\R\to F\vert_\R$ é linear,
onde $E\vert_\R$, $F\vert_\R$ denotam as realificações dos espaços complexos $E$ e $F$
respectivamente (veja Definição~\ref{thm:realificacao}). Obtemos daí o seguinte:
\begin{lem}
O resultado do Lema~\ref{thm:equivlincont} também vale se supusermos que $\K=\C$ e
que $T:E\to F$ é uma aplicação linear-conjugada.
\end{lem}
\begin{proof}
Basta aplicar o Lema~\ref{thm:equivlincont} para a aplicação linear $T:E\vert_\R\to F\vert_\R$,
observando que $\norma\cdot E$ e $\norma\cdot F$ também são normas nos espaços vetorais
reais $E\vert_\R$ e $F\vert_\R$, respectivamente (veja Exercício~\ref{exe:mesmanormanareal}).
\end{proof}
Dizemos que uma aplicação linear-conjugada $T:E\to F$ é
{\em limitada\/}\index[indice]{limitada!aplicacao linear-conjugada@aplicação linear-conjugada}%
\index[indice]{aplicacao linear-conjugada@aplicação linear-conjugada!limitada}\index[indice]{linear-conjugada!aplicacao limitada@aplicação limitada}
quando satisfaz uma das (e portanto todas as) condições equivalentes que aparecem no enunciado do Lema~\ref{thm:equivlincont}.
Definimos então a {\em norma\/}\index[indice]{norma!de uma aplicacao linear-conjugada@de uma aplicação linear-\hfil\break conjugada}%
\index[indice]{aplicacao linear-conjugada@aplicação linear-conjugada!norma de}\index[indice]{linear-conjugada!aplicacao@aplicação!norma de}
de $T$ como em \eqref{eq:normaopEF}. O Lema~\ref{thm:normacomposta}, a Proposição~\ref{thm:LinBanach}
e o resultado dos Exercícios~\ref{exe:supnaesfera} e \ref{exe:normaop} todos possuem versões correspondentes para
aplicações lineares-conjugadas (e as correspondentes demonstrações são perfeitamente análogas às demonstrações
das versões originais).
Observamos também que uma aplicação linear-conjugada pode ser transformada em uma aplicação
linear se trocarmos o sinal da estrutura complexa de seu domínio ou de seu contra-domínio
(veja Exercícios~\ref{exe:espconjugado1}, \ref{exe:espconjugado2}, \ref{exe:espconjugado3}
e \ref{exe:espconjugado4}). Tal observação permite obter de forma imediata as versões para aplicações
lineares-conjugadas dos resultados que demonstramos sobre aplicações lineares.

\begin{defin}
Seja $(E,\norma\cdot E)$, $(F,\norma\cdot F)$ espaços vetoriais sobre $\K$ e seja $T:E\to F$ uma aplicação
linear (resp., linear conjugada). Dizemos que $T$ é uma
{\em imersão isométrica linear\/}\index[indice]{imersao isometrica@imersão isométrica!linear (conjugada)}
(resp., {\em imersão isométrica linear-conjugada}) se:
\begin{equation}\label{eq:condisometrica}
\bignorma{T(x)}F=\norma xE,
\end{equation}
para todo $x\in E$. Se, além do mais, $T$ é sobrejetora, dizemos que $T$ é uma
{\em isometria linear\/}\index[indice]{isometria!linear (conjugada)} (resp., {\em isometria linear-conjugada}).
\end{defin}
A condição \eqref{eq:condisometrica} implica diretamente que $\Ker(T)=\{0\}$, i.e., que $T$ é injetora.
Assim, toda isometria linear (resp., isometria linear conjugada) é bijetora e é fácil ver que sua inversa
$T^{-1}:F\to E$ também é uma isometria linear (resp., isometria linear conjugada). Além do mais,
é claro que toda imersão isométrica linear (e toda imersão isométrica linear-conjugada) $T$ é limitada
e que $\Vert T\Vert=1$, a menos que seu domínio seja o espaço nulo.

\end{section}

\begin{section}{Funcionais Lineares e o Espaço Dual}

Seja $(E,\Vert\cdot\Vert)$ um espaço vetorial normado sobre $\K$. Por um {\em funcional
linear\/}\index[indice]{funcional linear}\index[indice]{linear!funcional} em $E$ nós entendemos uma
aplicação linear $\alpha:E\to\K$ cujo contra-domínio
é o corpo de escalares $\K$. Um funcional linear $\alpha:E\to\K$ é dito
{\em limitado\/}\index[indice]{funcional linear!limitado}\index[indice]{limitado!funcional linear}
quando a aplicação linear $\alpha:E\to\K$ for limitada. O conjunto:\index[simbolos]{$E^*$}
\[E^*=\big\{\alpha:\text{$\alpha$ é um funcional linear limitado em $E$}\big\}=\Lin(E,\K)\]
é chamado o {\em espaço dual\/}\index[indice]{espaco@espaço!dual}\index[indice]{dual!espaco@espaço} de $E$.
Como caso particular da Definição~\ref{thm:normaapllinear}, nós
temos:\index[indice]{norma!de um funcional linear}\index[indice]{funcional linear!norma de}
\[\Vert\alpha\Vert=\sup\big\{\vert\alpha(x)\vert:\text{$x\in E$ e $\Vert x\Vert\le1$}\big\}.\]

\begin{prop}
Se $(E,\Vert\cdot\Vert)$ é um espaço vetorial normado sobre $\K$ então seu dual
$E^*$ é um espaço de Banach.
\end{prop}
\begin{proof}
Segue da Proposição~\ref{thm:LinBanach}, já que o corpo de escalares $\K$ é completo.
\end{proof}

Observamos que em álgebra linear normalmente define-se o dual (tam\-bém
chamado de {\em dual algébrico}) de um espaço vetorial
$E$ como sendo o espaço de {\em todos\/} os funcionais lineares em $E$. Por isso,
o espaço dual que nós definimos acima é muitas vezes chamado o {\em dual topológico\/}
de $E$. Nós não teremos nenhum uso para a noção de dual algébrico e portanto usamos
``dual'' como sinônimo de ``dual topológico''.

\begin{example}
Seja $(E,\langle\cdot,\cdot\rangle)$ um espaço pré-Hilbertiano sobre $\K$. Para todo
$v\in E$, a aplicação:
\[\alpha_v:E\ni x\longmapsto\langle x,v\rangle\in\K\]
é um funcional linear em $E$. Segue da desigualdade de Cauchy--Schwarz (Lema~\ref{thm:CauchySchwarz})
que:
\[\big\vert\alpha_v(x)\big\vert\le\Vert v\Vert\Vert x\Vert,\]
donde vemos que $\alpha_v$ é limitado e $\Vert\alpha_v\Vert\le\Vert v\Vert$.
Afirmamos que:
\begin{equation}\label{eq:normaalphav}
\Vert\alpha_v\Vert=\Vert v\Vert,
\end{equation}
para todo $v\in E$. De fato, se $v=0$, essa igualdade é trivial; senão, tomamos
$w=\frac v{\Vert v\Vert}$, de modo que $\Vert w\Vert=1$ e:
\[\Vert\alpha_v\Vert\ge\big\vert\alpha_v(w)\big\vert=\frac{\langle v,v\rangle}{\Vert v\Vert}
=\Vert v\Vert,\]
provando \eqref{eq:normaalphav}. É fácil ver que:
\[\alpha_{v+w}=\alpha_v+\alpha_w,\quad\alpha_{\lambda v}=\bar\lambda\alpha_v,\]
para todos $v,w\in E$, $\lambda\in\K$, donde segue que a aplicação:
\begin{equation}\label{eq:aplicacaoRiesz}
E\ni v\longmapsto\alpha_v\in E^*
\end{equation}
é linear para $\K=\R$ e linear-conjugada para $\K=\C$. A igualdade \eqref{eq:normaalphav} nos diz que
a aplicação \eqref{eq:aplicacaoRiesz} é uma imersão isométrica linear para $\K=\R$ e uma imersão isométrica
linear-conjugada para $\K=\C$. A aplicação \eqref{eq:aplicacaoRiesz} é chamada a
{\em aplicação de Riesz\/}\index[indice]{Riesz!aplicacao de@aplicação de}\index[indice]{aplicacao@aplicação!de Riesz}
do espaço pré-Hilbertiano $E$.
\end{example}

Temos o seguinte:
\begin{teo}[de representação de Riesz]\index[indice]{Riesz!teorema de representacao de@teorema de representação de}%
\index[indice]{teorema!de representacao de Riesz@de representação de Riesz}
\label{thm:RieszHilbert}
Seja $(E,\langle\cdot,\cdot\rangle)$ um espaço de Hilbert sobre $\K$. Então a aplicação de Riesz
\eqref{eq:aplicacaoRiesz} é uma isometria linear para $\K=\R$ e uma isometria linear-conjugada para $\K=\C$.
\end{teo}

A demonstração do Teorema~\ref{thm:RieszHilbert} usa o seguinte:
\begin{lem}\label{thm:lemaKeralphacodim1}
Seja $E$ um espaço vetorial sobre $\K$ e $\alpha:E\to\K$ um funcional linear não nulo. Dado $x\in E$
com $\alpha(x)\ne0$ então $\Ker(\alpha)\cup\{x\}$ é um conjunto de geradores para $E$.
\end{lem}
\begin{proof}
Seja $y\in E$; queremos encontrar $\lambda\in\K$ e $n\in\Ker(\alpha)$ com $y=n+\lambda x$. Basta então encontrar
$\lambda\in\K$ com $y-\lambda x\in\Ker(\alpha)$; mas:
\[\alpha(y-\lambda x)=\alpha(y)-\lambda\alpha(x),\]
donde basta tomar $\lambda=\frac{\alpha(y)}{\alpha(x)}$.
\end{proof}

\begin{cor}\label{thm:coralphaldbeta}
Seja $E$ um espaço vetorial sobre $\K$. Dados funcionais lineares $\alpha:E\to\K$, $\beta:E\to\K$, se
$\Ker(\alpha)\subset\Ker(\beta)$ então existe $\lambda\in\K$ com $\beta=\lambda\alpha$.
\end{cor}
\begin{proof}
Se $\alpha=0$ então $\Ker(\alpha)=E=\Ker(\beta)$, donde $\beta=0$ e basta tomar $\lambda=0$. Se $\alpha\ne0$,
seja $x\in E$ com $\alpha(x)\ne0$ e tome $\lambda=\frac{\beta(x)}{\alpha(x)}$. Temos que os funcionais lineares
$\beta$ e $\lambda\alpha$ coincidem em $\Ker(\alpha)\cup\{x\}$; segue então do Lema~\ref{thm:lemaKeralphacodim1} que
$\beta=\lambda\alpha$.
\end{proof}

\begin{proof}[Demonstração do Teorema~\ref{thm:RieszHilbert}]
Seja $(E,\langle\cdot,\cdot\rangle)$ um espaço de Hilbert sobre $\K$. Já sabemos que a aplicação de Riesz
\eqref{eq:aplicacaoRiesz} é uma imersão isométrica (linear para $\K=\R$ e linear-conjugada para $\K=\C$), de modo que é suficiente
demonstrar que ela é sobrejetora. Seja $\alpha\in E^*$ um funcional linear limitado em $E$. Devemos encontrar
$v\in E$ com $\alpha=\alpha_v$. Se $\alpha=0$, basta tomar $v=0$. Suponha então que $\alpha\ne0$. Como
$\alpha$ é contínuo, temos que $\Ker(\alpha)=\alpha^{-1}(0)$ é um subespaço fechado de $E$, de modo que
$E=\Ker(\alpha)\oplus\big(\Ker(\alpha)\big)^\perp$, pelo Corolário~\ref{thm:cosSoplusSperp}.
Como $\alpha\ne0$, temos $\Ker(\alpha)\ne E$ e portanto existe $x\in\big(\Ker(\alpha)\big)^\perp$
com $x\ne0$. Vemos então que o funcional linear $\alpha_x$ é nulo em $\Ker(\alpha)$ e portanto,
pelo Corolário~\ref{thm:coralphaldbeta}, existe $\lambda\in\K$ com $\alpha_x=\lambda\alpha$.
Como $x\ne0$, temos que $\alpha_x\ne0$ e portanto $\lambda\ne0$. Concluímos então que:
\[\alpha=\lambda^{-1}\alpha_x=\alpha_v,\]
onde $v=\bar\lambda^{-1}x\in E$. Isso completa a demonstração.
\end{proof}

\end{section}

\begin{section}[Espaços $L^p$]{Espaços ${L^p}$}
\label{sec:espacosLp}

Seja $(X,\mathcal A,\mu)$ um espaço de medida. Dados uma função mensurável $f:X\to\C$ e $p\in\left[1,+\infty\right[$,
nós definimos:
\begin{equation}\label{eq:defnormap}
\Vert f\Vert_p=\Big(\int_X\vert f\vert^p\,\dd\mu\Big)^{\frac1p}\in[0,+\infty],
\end{equation}
onde convencionamos que $(+\infty)^\alpha=+\infty$, para qualquer $\alpha>0$.
Note que $\vert f\vert^p:X\to\left[0,+\infty\right[$ é uma função mensurável, já que a função:
\[\R^2\cong\C\ni z\longmapsto\vert z\vert^p\in\left[0,+\infty\right[\]
é contínua e portanto Borel mensurável (veja Lema~\ref{thm:contmens} e Corolário~\ref{thm:corphidefs}).
Como a função $\vert f\vert^p$ é não negativa,
segue que a integral em \eqref{eq:defnormap} sempre existe (sendo possivelmente igual a $+\infty$).
Note que, pelo resultado do Exercício~\ref{exe:intzerofzeroqs}, temos:
\begin{equation}\label{eq:normapzero}
\Vert f\Vert_p=0\Longleftrightarrow f=0\ \qs.
\end{equation}

\begin{notation}
Denotamos por $\mathcal L^p(X,\mathcal A,\mu;\K)$\index[simbolos]{$\mathcal L^p(X,\mathcal A,\mu;\K)$} o
conjunto de todas as funções mensuráveis
$f:X\to\K$ tais que $\Vert f\Vert_p<+\infty$, onde $\K=\R$ ou $\K=\C$.
\end{notation}

Temos o seguinte:
\begin{lem}\label{thm:Lpsubespaco}
Dados um espaço de medida $(X,\mathcal A,\mu)$ e $p\in\left[1,+\infty\right[$ então
$\mathcal L^p(X,\mathcal A,\mu;\K)$ é um subespaço do espaço vetorial (sobre $\K$)
de todas as funções $f:X\to\K$.
\end{lem}
\begin{proof}
A função nula obviamente está em $\mathcal L^p(X,\mathcal A,\mu;\K)$.
Dados uma função mensurável $f:X\to\K$ e $\lambda\in\K$ então a função $\lambda f$
é mensurável e:
\[\Vert\lambda f\Vert_p=\Big(\int_X\vert\lambda f\vert^p\,\dd\mu\Big)^{\frac1p}
=\Big(\int_X\vert\lambda\vert^p\vert f\vert^p\,\dd\mu\Big)^{\frac1p}
=\Big(\vert\lambda\vert^p\int_X\vert f\vert^p\,\dd\mu\Big)^{\frac1p},\]
donde:
\begin{equation}\label{eq:normaplambdaf}
\Vert\lambda f\Vert_p=\vert\lambda\vert\,\Vert f\Vert_p.
\end{equation}
Daí $\lambda f\in\mathcal L^p(X,\mathcal A,\mu;\K)$ sempre que
$f\in\mathcal L^p(X,\mathcal A,\mu;\K)$. Para mostrar que $\mathcal L^p(X,\mathcal A,\mu;\K)$
é fechado por somas, usamos a desigualdade:
\begin{equation}\label{eq:aplusbpowerp}
(a+b)^p\le2^p(a^p+b^p),
\end{equation}
válida para quaisquer números reais não negativos $a$, $b$. Para provar \eqref{eq:aplusbpowerp},
seja $c=\max\{a,b\}$, de modo que $c^p=\max\{a^p,b^p\}$ e $a+b\le c+c=2c$; temos:
\[(a+b)^p\le(2c)^p=2^pc^p\le2^p(a^p+b^p).\]
Agora, dadas $f,g\in\mathcal L^p(X,\mathcal A,\mu;\K)$, então:
\[\vert f+g\vert^p\le2^p\big(\vert f\vert^p+\vert g\vert^p\big),\]
de modo que:
\[\int_X\vert f+g\vert^p\,\dd\mu\le2^p\Big(\int_X\vert f\vert^p\,\dd\mu
+\int_X\vert g\vert^p\,\dd\mu\Big)<+\infty,\]
e portanto $f+g\in\mathcal L^p(X,\mathcal A,\mu;\K)$.
\end{proof}

Nosso próximo objetivo é estabelecer que $\Vert\cdot\Vert_p$ é uma semi-norma
no espaço vetorial $\mathcal L^p(X,\mathcal A,\mu;\K)$.
\begin{lem}[desigualdade de Minkowski]\index[indice]{desigualdade!de Minkowski}\index[indice]{Minkowski!desigualdade de}
\label{thm:Minkowski}
Seja $(X,\mathcal A,\mu)$ um espaço de medida e sejam $f:X\to\C$, $g:X\to\C$ funções mensuráveis. Dado
$p\in\left[1,+\infty\right[$, então:
\begin{equation}\label{eq:Minkowski}
\Vert f+g\Vert_p\le\Vert f\Vert_p+\Vert g\Vert_p.
\end{equation}
\end{lem}

\begin{cor}
Se $(X,\mathcal A,\mu)$ é um espaço de medida e $p\in\left[1,+\infty\right[$ então $\Vert\cdot\Vert_p$
é uma semi-norma em $\mathcal L^p(X,\mathcal A,\mu;\K)$.
\end{cor}
\begin{proof}
Obviamente, $\Vert f\Vert_p$ é um número real não negativo, para toda $f\in\mathcal L^p(X,\mathcal A,\mu;\K)$.
Vimos em \eqref{eq:normaplambdaf} que $\Vert\lambda f\Vert_p=\vert\lambda\vert\,\Vert f\Vert_p$, para todo $\lambda\in\K$
e toda $f\in\mathcal L^p(X,\mathcal A,\mu;\K)$. Finalmente, a desigualdade triangular para $\Vert\cdot\Vert_p$ é
exatamente a desigualdade de Minkowski (Lema~\ref{thm:Minkowski}).
\end{proof}

A prova da desigualdade de Minkowski tem alguma similaridade com a prova da desigualdade triangular para
a norma \eqref{eq:normaprodint} associada a um produto interno. Na prova da desigualdade triangular para
a norma \eqref{eq:normaprodint}, utilizamos a desigualdade de Cauchy--Schwarz. Para provar a desigualdade de Minkowski,
vamos usar o seguinte:
\begin{lem}[desigualdade de Hölder]\index[indice]{desigualdade!de Holder@de Hölder}\index[indice]{Holder@Hölder!desigualdade de}
\label{thm:Holder}
Seja $(X,\mathcal A,\mu)$ um espaço de medida e sejam $f:X\to\C$, $g:X\to\C$ funções mensuráveis. Dados
$p,q\in\left]1,+\infty\right[$ tais que $\frac1p+\frac1q=1$ então:
\begin{equation}\label{eq:Holder}
\int_X\vert fg\vert\,\dd\mu\le\Vert f\Vert_p\,\Vert g\Vert_q.
\end{equation}
Supondo $\Vert f\Vert_p<+\infty$ e $\Vert g\Vert_q<+\infty$ então a igualdade ocorre em \eqref{eq:Holder} se e somente
se existe $\lambda\ge0$ tal que:
\begin{equation}\label{eq:condigHolder}
\vert g\vert^q=\lambda\vert f\vert^p\ \qs\quad\text{ou}\quad\vert f\vert^p=\lambda\vert g\vert^q\ \qs.
\end{equation}
\end{lem}

Vamos por um momento assumir a desigualdade de Hölder e demonstrar a desigualdade de Minkowski.
\begin{proof}[Demonstração do Lema~\ref{thm:Minkowski}]
Se $p=1$ então:
\begin{multline*}
\Vert f+g\Vert_p=\int_X\vert f+g\vert\,\dd\mu\le\int_X\big(\vert f\vert+\vert g\vert\big)\,\dd\mu
=\int_X\vert f\vert\,\dd\mu+\int_X\vert g\vert\,\dd\mu\\
=\Vert f\Vert_p+\Vert g\Vert_p.
\end{multline*}
Suponha agora que $p>1$ e seja $q=\frac p{p-1}\in\left]1,+\infty\right[$,
de modo que $\frac1p+\frac1q=1$. Temos:
\begin{multline}\label{eq:Minkowski1}
\big(\Vert f+g\Vert_p\big)^p=\int_X\vert f+g\vert^p\,\dd\mu=\int_X\vert f+g\vert^{p-1}\vert f+g\vert\,\dd\mu\\
\le\int_X\vert f+g\vert^{p-1}\big(\vert f\vert+\vert g\vert\big)\,\dd\mu\\
=\int_X\vert f\vert\vert f+g\vert^{p-1}\,\dd\mu+\int_X\vert g\vert\vert f+g\vert^{p-1}\,\dd\mu.
\end{multline}
Usando a desigualdade de Hölder (Lema~\ref{thm:Holder}), obtemos:
\begin{gather*}
\begin{aligned}
\int_X\vert f\vert\vert f+g\vert^{p-1}\,\dd\mu\le\Vert f\Vert_p\,\big\Vert\vert f+g\vert^{p-1}\big\Vert_q
&=\Vert f\Vert_p\Big(\int_X\vert f+g\vert^p\Big)^{\frac1q}\\
&=\Vert f\Vert_p\big(\Vert f+g\Vert_p\big)^{\frac pq},
\end{aligned}\\
\begin{aligned}
\int_X\vert g\vert\vert f+g\vert^{p-1}\,\dd\mu\le\Vert g\Vert_p\,\big\Vert\vert f+g\vert^{p-1}\big\Vert_q
&=\Vert g\Vert_p\Big(\int_X\vert f+g\vert^p\Big)^{\frac1q}\\
&=\Vert g\Vert_p\big(\Vert f+g\Vert_p\big)^{\frac pq};
\end{aligned}
\end{gather*}
daí:
\begin{equation}\label{eq:Minkowski2}
\int_X\vert f\vert\vert f+g\vert^{p-1}\,\dd\mu+\int_X\vert g\vert\vert f+g\vert^{p-1}\,\dd\mu
\le\big(\Vert f\Vert_p+\Vert g\Vert_p\big)\big(\Vert f+g\Vert_p\big)^{\frac pq}.
\end{equation}
De \eqref{eq:Minkowski1} e \eqref{eq:Minkowski2} vem:
\[\big(\Vert f+g\Vert_p\big)^p\le\big(\Vert f\Vert_p+\Vert g\Vert_p\big)\big(\Vert f+g\Vert_p\big)^{\frac pq},\]
e:
\begin{equation}\label{eq:Minkowski3}
\Vert f+g\Vert_p\big(\Vert f+g\Vert_p\big)^{\frac pq}\le\big(\Vert f\Vert_p+\Vert g\Vert_p\big)\big(\Vert f+g\Vert_p\big)^{\frac pq}.
\end{equation}
Se $0<\Vert f+g\Vert_p<+\infty$ então \eqref{eq:Minkowski} segue diretamente de \eqref{eq:Minkowski3}.
Se $\Vert f+g\Vert_p=0$, a desigualdade \eqref{eq:Minkowski} é trivial. Finalmente,
se $\Vert f+g\Vert_p$ é igual a $+\infty$ então o Lema~\ref{thm:Lpsubespaco} implica que $\Vert f\Vert_p=+\infty$
ou $\Vert g\Vert_p=+\infty$, donde a desigualdade \eqref{eq:Minkowski} segue.
\end{proof}

Passemos à prova da desigualdade de Hölder. Precisamos do seguinte:
\begin{lem}[desigualdade entre as médias]\index[indice]{desigualdade!entre as medias@entre as médias}%
\index[indice]{medias@méidas!desigualdade entre as}
\label{thm:desigmedias}
Dados $n\ge1$, números reais positivos $\alpha_1$, \dots, $\alpha_n$ com $\alpha_1+\cdots+\alpha_n=1$
e números reais não negativos $x_1$, \dots, $x_n$ então:
\begin{equation}\label{eq:desigmedias}
x_1^{\alpha_1}\cdots x_n^{\alpha_n}\le\alpha_1x_1+\cdots+\alpha_nx_n.
\end{equation}
A igualdade vale em \eqref{eq:desigmedias} se e somente se $x_1=\cdots=x_n$.
\end{lem}
\begin{proof}
A prova deste lema usa algus fatos simples da teoria de funções convexas, que serão demonstrados
na Seção~\ref{sec:convexas}. A função exponencial $\exp:\R\ni x\mapsto e^x\in\R$ possui derivada segunda positiva e portanto
é estritamente convexa (Corolário~\ref{thm:secderpos}). Dados então números reais positivos
$\alpha_1$, \dots, $\alpha_n$ com $\alpha_1+\cdots+\alpha_n=1$
e números reais $y_1$, \dots, $y_n$, temos (Proposição~\ref{thm:Jensenfinito}):
\[\exp(\alpha_1y_1+\cdots+\alpha_ny_n)\le\alpha_1\exp(y_1)+\cdots+\alpha_n\exp(y_n),\]
onde a igualdade vale se e somente se $y_1=\cdots=y_n$. Se todos os $x_i$ são positivos, a conclusão é obtida fazendo $y_i=\ln(x_i)$, $i=1,\ldots,n$.
O caso em que algum $x_i$ é igual a zero é trivial, já que o lado esquerdo de \eqref{eq:desigmedias} é zero
e o lado direito de \eqref{eq:desigmedias} é não negativo, sendo igual a zero se e somente se $x_1=\cdots=x_n=0$.
\end{proof}

\begin{cor}\label{thm:cormedias}
Sejam $p,q\in\left]1,+\infty\right[$ com $\frac1p+\frac1q=1$ e $a,b\ge0$. Então:
\begin{equation}\label{eq:cormedias}
ab\le\frac{a^p}p+\frac{b^q}q;
\end{equation}
a igualdade vale em \eqref{eq:cormedias} se e somente se $a^p=b^q$.
\end{cor}
\begin{proof}
Use o Lema~\ref{thm:desigmedias} com $n=2$, $x_1=a^p$, $x_2=b^q$, $\alpha_1=\frac1p$ e $\alpha_2=\frac1q$.
\end{proof}

\begin{proof}[Demonstração do Lema~\ref{thm:Holder}]
Se $\Vert f\Vert_p=0$ ou $\Vert g\Vert_q=0$ então $f=0$ \qs\ ou $g=0$ \qs\ (veja \eqref{eq:normapzero}) e
$\int_X\vert fg\vert\,\dd\mu=0$, donde o resultado segue trivialmente.
Suponhamos que $\Vert f\Vert_p>0$ e $\Vert g\Vert_q>0$. Se $\Vert f\Vert_p=+\infty$
ou $\Vert g\Vert_q=+\infty$ então $\Vert f\Vert_p\,\Vert g\Vert_q=+\infty$, donde também o resultado é trivial.
Podemos supor então que as normas $\Vert f\Vert_p$ e $\Vert g\Vert_q$ são positivas e finitas. Sendo $\Vert f\Vert_p$ e $\Vert g\Vert_q$
ambas positivas, vemos que a existência de $\lambda\ge0$ tal que \eqref{eq:condigHolder} vale é equivalente à existência de
$\lambda>0$ tal que $\vert g\vert^q=\lambda\vert f\vert^p$ \qs; integrando essa igualdade dos dois lados, vem:
\[\lambda=\frac{\big(\Vert g\Vert_q\big)^q}{\big(\Vert f\Vert_p\big)^p},\]
donde a condição \eqref{eq:condigHolder} é na verdade equivalente a:
\begin{equation}\label{eq:condigHolderequiv}
\Big(\frac{\vert f\vert}{\Vert f\Vert_p}\Big)^p=\Big(\frac{\vert g\vert}{\Vert g\Vert_q}\Big)^q\ \qs.
\end{equation}
Sejam:
\[\tilde f=\frac{\vert f\vert}{\Vert f\Vert_p},\quad\tilde g=\frac{\vert g\vert}{\Vert g\Vert_q}.\]
A desigualdade de Hölder \eqref{eq:Holder} é equivalente a:
\begin{equation}\label{eq:Holder1}
\int_X\tilde f\tilde g\,\dd\mu\le1.
\end{equation}
A igualdade \eqref{eq:normaplambdaf} nos dá:
\[\Vert\tilde f\Vert_p=1,\quad\Vert\tilde g\Vert_q=1.\]
Usando o Corolário~\ref{thm:cormedias}, obtemos:
\begin{equation}\label{eq:tildeftildegpq}
\tilde f\tilde g\le\frac{\tilde f^p}p+\frac{\tilde g^q}q;
\end{equation}
integrando, vem:
\begin{equation}\label{eq:passoHolder}
\int_X\tilde f\tilde g\,\dd\mu\le\int_X\Big(\frac{\tilde f^p}p+\frac{\tilde g^q}q\Big)\,\dd\mu=
\frac1p\big(\Vert\tilde f\Vert_p\big)^p+\frac1q\big(\Vert\tilde g\Vert_q\big)^q
=\frac1p+\frac1q=1,
\end{equation}
provando \eqref{eq:Holder1} (e também \eqref{eq:Holder}). Temos que a igualdade em \eqref{eq:Holder} é equivalente
à igualdade em \eqref{eq:Holder1} que, por sua vez, é equivalente à afirmação que a desigualdade que aparece em
\eqref{eq:passoHolder} é uma igualdade; usando \eqref{eq:tildeftildegpq} e o resultado do Exercício~\ref{exe:flegintigualqs},
vemos então que a igualdade em \eqref{eq:Holder} é equivalente a:
\begin{equation}\label{eq:tildeftildegpqig}
\tilde f\tilde g=\frac{\tilde f^p}p+\frac{\tilde g^q}q\ \qs.
\end{equation}
Pelo Corolário~\ref{thm:cormedias}, \eqref{eq:tildeftildegpqig} é equivalente a:
\[\tilde f^p=\tilde g^q\ \qs,\]
que, por sua vez, é equivalente a \eqref{eq:condigHolderequiv}. Isso completa a demonstração.
\end{proof}

Em vista de \eqref{eq:normapzero}, temos que $\Vert\cdot\Vert_p$ não é em geral uma norma em $\mathcal L^p(X,\mathcal A,\mu;\K)$.
Para obtermos um espaço normado, consideramos um quociente de $\mathcal L^p(X,\mathcal A,\mu;\K)$ (veja
Exercício~\ref{exe:seminormaviranorma}). Considere o seguinte subespaço de $\mathcal L^p(X,\mathcal A,\mu;\K)$:
\begin{equation}\label{eq:subespacoquasenulas}
\big\{f\in\mathcal L^p(X,\mathcal A,\mu;\K):\Vert f\Vert_p=0\big\}=\big\{f\in\mathcal L^p(X,\mathcal A,\mu;\K):f=0\ \qs\big\}.
\end{equation}
Denotamos por $L^p(X,\mathcal A,\mu;\K)$ o espaço quociente de $\mathcal L^p(X,\mathcal A,\mu;\K)$
por \eqref{eq:subespacoquasenulas}. Pelo resultado do Exercício~\ref{exe:seminormaviranorma}, temos que a semi-norma $\Vert\cdot\Vert_p$
em $\mathcal L^p(X,\mathcal A,\mu;\K)$ induz uma norma no espaço quociente $L^p(X,\mathcal A,\mu;\K)$, de modo que a norma
de uma classe de equivalência é igual à semi-norma de um representante qualquer da classe.
Denotaremos essa norma em $L^p(X,\mathcal A,\mu;\K)$ também por $\Vert\cdot\Vert_p$. Dada uma função
$f$ em $\mathcal L^p(X,\mathcal A,\mu;\K)$ então a classe de equivalência de $f$ em $L^p(X,\mathcal A,\mu;\K)$
é precisamente o conjunto de todas as funções mensuráveis $g:X\to\K$ que são iguais a $f$ quase sempre. Nós
em geral adotaremos o abuso de notação de denotar a classe de funções mensuráveis iguais a $f$ quase sempre também por $f$;
assim, nós diremos ``seja $f\in L^p(X,\mathcal A,\mu;\K)$'', onde $f$ denotará uma função $f\in\mathcal L^p(X,\mathcal A,\mu;\K)$
(que é identificada com o elemento de $L^p(X,\mathcal A,\mu;\K)$ que é a classe de funções mensuráveis iguais a $f$
quase sempre.

\begin{prop}\label{thm:Lpcompleto}
Se $(X,\mathcal A,\mu)$ é um espaço de medida e $p\in\left[1,+\infty\right[$ então o espaço normado $L^p(X,\mathcal A,\mu;\K)$ é completo
e é portanto um espaço de Banach sobre $\K$.
\end{prop}

A demonstração da Proposição~\ref{thm:Lpcompleto} usa os dois seguintes lemas.
\begin{lem}\label{thm:limiteqsemLp}
Sejam $(X,\mathcal A,\mu)$ um espaço de medida e $p\in\left[1,+\infty\right[$.
Seja $(f_n)_{n\ge1}$ uma seqüência de Cauchy em
$L^p(X,\mathcal A,\mu;\K)$ que converge pontualmente quase sempre para uma função mensurável $f:X\to\K$.
Então $f$ está em $L^p(X,\mathcal A,\mu;\K)$ e $(f_n)_{n\ge1}$ converge para $f$ em $L^p(X,\mathcal A,\mu;\K)$.
\end{lem}
\begin{proof}
É suficiente mostrar que $\Vert f_n-f\Vert_p\to0$; de fato, isso implica em particular que $\Vert f_n-f\Vert_p<+\infty$
para $n$ suficientemente grande, donde segue que $f_n-f\in L^p(X,\mathcal A,\mu;\K)$ e portanto também
$f$ está em $L^p(X,\mathcal A,\mu;\K)$. Seja dado $\varepsilon>0$. Como $(f_n)_{n\ge1}$ é uma seqüência de Cauchy em
$L^p(X,\mathcal A,\mu;\K)$, existe $n_0\ge1$ tal que $\Vert f_n-f_m\Vert_p<\varepsilon$, para todos $n,m\ge n_0$, isto é:
\[\int_X\vert f_n-f_m\vert^p\,\dd\mu<\varepsilon^p,\]
para todos $n,m\ge n_0$. Fixando $n\ge n_0$ e usando o Lema de Fatou (Proposição~\ref{thm:lemaFatou}) para a seqüência
$\big(\vert f_n-f_m\vert^p\big)_{m\ge1}$, obtemos:
\[\int_X\liminf_{m\to\infty}\vert f_n-f_m\vert^p\,\dd\mu\le\liminf_{m\to\infty}\int_X\vert f_n-f_m\vert^p\,\dd\mu\le
\varepsilon^p,\]
para todo $n\ge n_0$. Mas $f_m\to f$ \qs\ implica que:
\[\liminf_{m\to\infty}\vert f_n-f_m\vert^p=\vert f_n-f\vert^p\ \qs\]
e portanto:
\[\int_X\vert f_n-f\vert^p\,\dd\mu=\int_X\liminf_{m\to\infty}\vert f_n-f_m\vert^p\,\dd\mu\le\varepsilon^p,\]
para todo $n\ge n_0$. Isso prova que $\Vert f_n-f\Vert_p\le\varepsilon$, para todo $n\ge n_0$ e completa a demonstração.
\end{proof}

\begin{lem}\label{thm:LemaLpmedida}
Sejam $(X,\mathcal A,\mu)$ um espaço de medida, $p\in\left[1,+\infty\right[$ e
$(f_n)_{n\ge1}$ uma seqüência em $L^p(X,\mathcal A,\mu;\K)$.
Se $(f_n)_{n\ge1}$ é de Cauchy em
$L^p(X,\mathcal A,\mu;\K)$ (resp., converge em $L^p(X,\mathcal A,\mu;\K)$ para $f$) então $(f_n)_{n\ge1}$
é de Cauchy em medida (resp., converge para $f$ em medida).
\end{lem}
\begin{proof}
A tese segue facilmente da seguinte observação: dadas funções mensuráveis $f,g:X\to\K$ então para todo
$\varepsilon>0$ vale a desigualdade:
\[\mu\Big(\big\{x\in X:\big\vert f(x)-g(x)\big\vert\ge\varepsilon\big\}\Big)\le\frac{\big(\Vert f-g\Vert_p\big)^p}{\varepsilon^p}.\]
Para provar a observação, seja $A=\big\{x\in X:\big\vert f(x)-g(x)\big\vert\ge\varepsilon\big\}$; então:
\[\big(\Vert f-g\Vert_p\big)^p=\int_X\vert f-g\vert^p\,\dd\mu\ge\int_A\vert f-g\vert^p\,\dd\mu\ge\varepsilon^p\mu(A).\qedhere\]
\end{proof}

\begin{proof}[Demonstração da Proposição~\ref{thm:Lpcompleto}]
Seja $(f_n)_{n\ge1}$ uma seqüência de Cauchy em $L^p(X,\mathcal A,\mu;\K)$. Pelo Lema~\ref{thm:LemaLpmedida},
a seqüência $(f_n)_{n\ge1}$ tam\-bém é de Cauchy em medida e portanto, pelo Lema~\ref{thm:CauchyMedida},
existe uma subseqüência $(f_{n_k})_{k\ge1}$ de $(f_n)_{n\ge1}$ que converge pontualmente quase sempre
para uma função mensurável $f:X\to\K$. O Lema~\ref{thm:limiteqsemLp} nos diz então que
$f$ está em $L^p(X,\mathcal A,\mu;\K)$ e que $(f_{n_k})_{k\ge1}$ converge para $f$ em
$L^p(X,\mathcal A,\mu;\K)$. A conclusão segue da seguinte observação elementar: se uma seqüência de Cauchy
num espaço métrico possui uma subseqüência convergente então a própria seqüência de Cauchy também é convergente
(para o mesmo limite).
\end{proof}

\end{section}

\begin{section}[Funções Convexas]{Apêndice à Seção~\ref{sec:espacosLp}: funções convexas}
\label{sec:convexas}

\begin{defin}
Seja $I\subset\R$ um intervalo. Uma função $f:I\to\R$ é dita
{\em convexa\/}\index[indice]{convexa!funcao@função}\index[indice]{funcao@função!convexa} quando para todos $x,y\in I$
e todo $t\in[0,1]$ vale a desigualdade:
\begin{equation}\label{eq:defconvexa}
f\big((1-t)x+ty\big)\le(1-t)f(x)+tf(y).
\end{equation}
Dizemos que $f$ é
{\em estritamente convexa\/}\index[indice]{estritamente convexa!funcao@função}\index[indice]{funcao@função!convexa!estritamente}
quando para todos $x,y\in I$ com $x\ne y$ e todo $t\in\left]0,1\right[$ vale a desigualdade estrita:
\begin{equation}\label{eq:defestritconvexa}
f\big((1-t)x+ty\big)<(1-t)f(x)+tf(y).
\end{equation}
\end{defin}
Claramente a igualdade vale em \eqref{eq:defconvexa} quando $x=y$ ou $t\in\{0,1\}$;
além do mais, as desigualdades \eqref{eq:defconvexa} e \eqref{eq:defestritconvexa}
não se alteram quando trocamos $x$ por $y$ e $t$ por $1-t$. Vemos então que
$f$ é convexa (resp., estritamente convexa) se e somente se a desigualdade
\eqref{eq:defconvexa} (resp., a desigualdade estrita \eqref{eq:defestritconvexa})
vale para todos $x,y\in I$ com $x<y$ e para todo $t\in\left]0,1\right[$.
Evidentemente toda função estritamente convexa é convexa.

Geometricamente, a desigualdade \eqref{eq:defconvexa} diz que o trecho do gráfico de $f$ entre os pontos
$\big(x,f(x)\big)$ e $\big(y,f(y)\big)$ está abaixo da correspondente reta\index[indice]{secante!reta}\index[indice]{reta!secante} secante.
Vamos explorar algumas conseqüências
da definição de convexidade. Dados $x,y\in I$ com $x\ne y$, vamos denotar por $\ca(f;x,y)$ o
coeficiente angular\index[indice]{coeficiente!angular}\index[indice]{angular!coeficiente}
da reta que passa pelos pontos $\big(x,f(x)\big)$ e $\big(y,f(y)\big)$; mais explicitamente:\index[simbolos]{$\ca(f;x,y)$}
\[\ca(f;x,y)=\frac{f(y)-f(x)}{y-x}=\ca(f;y,x).\]
Se $x,y\in\R$ e $x<y$, temos uma bijeção estritamente crescente:
\begin{equation}\label{eq:bijtw}
[0,1]\ni t\longmapsto w=(1-t)x+ty=x+t(y-x)\in[x,y],
\end{equation}
cuja inversa é dada por:
\[[x,y]\ni w\longmapsto t=\frac{w-x}{y-x}\in[0,1].\]
Se $x,y\in I$, $x<y$ e $t\in[0,1]$, $w\in[x,y]$ estão relacionados por \eqref{eq:bijtw}
então:
\[f\big((1-t)x+ty\big)\le(1-t)f(x)+tf(y)\Longleftrightarrow
f(w)\le\frac{y-w}{y-x}f(x)+\frac{w-x}{y-x}f(y).\]
Note também que:
\begin{multline}\label{eq:igualdadesfaceis}
\frac{y-w}{y-x}f(x)+\frac{w-x}{y-x}f(y)=f(x)+\ca(f;x,y)(w-x)\\
=f(y)+\ca(f;x,y)(w-y),
\end{multline}
para todo $w\in\R$. Vemos então que:
\begin{equation}\label{eq:equivsca}
\begin{aligned}
f\big((1-t)x+ty\big)\le(1-t)f(x)+tf(y)&\Longleftrightarrow\ca(f;x,w)\le\ca(f;x,y)\\
&\Longleftrightarrow\ca(f;x,y)\le\ca(f;w,y),
\end{aligned}
\end{equation}
e:
\begin{equation}\label{eq:equivsca2}
\begin{aligned}
f\big((1-t)x+ty\big)=(1-t)f(x)+tf(y)&\Longleftrightarrow\ca(f;x,w)=\ca(f;x,y)\\
&\Longleftrightarrow\ca(f;x,y)=\ca(f;w,y),
\end{aligned}
\end{equation}
onde $x,y\in I$, $x<y$, e $t\in\left]0,1\right[$, $w\in\left]x,y\right[$ estão relacionados
por \eqref{eq:bijtw}. Observamos também que se $x,y,w\in I$, $x<w<y$ então existe
$\theta\in\left]0,1\right[$ tal que:
\begin{equation}\label{eq:cacombconvex}
\ca(f;x,y)=(1-\theta)\ca(f;x,w)+\theta\ca(f;w,y);
\end{equation}
de fato, basta tomar $\theta=\frac{y-w}{y-x}$. Segue de \eqref{eq:cacombconvex} que
$\ca(f;x,y)$ pertence ao intervalo fechado de extremidades $\ca(f;x,w)$ e $\ca(f;w,y)$;
levando em conta \eqref{eq:equivsca} e \eqref{eq:equivsca2} vemos então também que:
\begin{equation}\label{eq:equivsca3}
\begin{aligned}
\ca(f;x,w)\le\ca(f;x,y)&\Longleftrightarrow\ca(f;x,y)\le\ca(f;w,y)\\
&\Longleftrightarrow\ca(f;x,w)\le\ca(f;w,y),
\end{aligned}
\end{equation}
e:
\begin{equation}\label{eq:equivsca4}
\begin{aligned}
\ca(f;x,w)=\ca(f;x,y)&\Longleftrightarrow\ca(f;x,y)=\ca(f;w,y)\\
&\Longleftrightarrow\ca(f;x,w)=\ca(f;w,y).
\end{aligned}
\end{equation}
Temos então o seguinte:
\begin{lem}\label{thm:lemasecantes}
Seja $f:I\to\R$ uma função definida num intervalo $I\subset\R$. As seguintes condições
são equivalentes:
\begin{itemize}
\item $f$ é convexa;
\item $\ca(f;x,w)\le\ca(f;x,y)$, para todos $x,y,w\in I$ com $x<w<y$;
\item $\ca(f;x,y)\le\ca(f;w,y)$, para todos $x,y,w\in I$ com $x<w<y$;
\item $\ca(f;x,w)\le\ca(f;w,y)$, para todos $x,y,w\in I$ com $x<w<y$.
\end{itemize}
Similarmente, são equivalentes as condições:
\begin{itemize}
\item $f$ é estritamente convexa;
\item $\ca(f;x,w)<\ca(f;x,y)$, para todos $x,y,w\in I$ com $x<w<y$;
\item $\ca(f;x,y)<\ca(f;w,y)$, para todos $x,y,w\in I$ com $x<w<y$;
\item $\ca(f;x,w)<\ca(f;w,y)$, para todos $x,y,w\in I$ com $x<w<y$.
\end{itemize}
\end{lem}
\begin{proof}
Segue de \eqref{eq:equivsca}, \eqref{eq:equivsca2}, \eqref{eq:equivsca3}
e \eqref{eq:equivsca4}.
\end{proof}

Uma função convexa só deixa de ser estritamente convexa se for afim em algum trecho de seu domínio.
Esse é o conteúdo do seguinte:
\begin{cor}
Seja $f:I\to\R$ uma função convexa num intervalo $I\subset\R$. Dados $x,y\in I$ com $x<y$ então são equivalentes:
\item[(a)] existe $t\in\left]0,1\right[$ tal que $f\big((1-t)x+ty\big)=(1-t)f(x)+tf(y)$;
\item[(b)] $f$ é {\em afim}\index[indice]{funcao@função!afim}\index[indice]{afim!funcao@função} no intervalo $[x,y]$,
i.e., existem $a,b\in\R$ com $f(w)=aw+b$, para todo $w\in[x,y]$;
\item[(c)] $f\big((1-t)x+ty\big)=(1-t)f(x)+tf(y)$, para todo $t\in[0,1]$.
\end{cor}
\begin{proof}\
\begin{bulletindent}
\item (a)$\Rightarrow$(b).

Seja $t_0\in\left]0,1\right[$ com $f\big((1-t_0)x+t_0y\big)=(1-t_0)f(x)+t_0f(y)$
e seja $w_0\in\left]x,y\right[$ relacionado com $t_0$ pela bijeção \eqref{eq:bijtw}.
Por \eqref{eq:equivsca2}, temos:
\begin{equation}\label{eq:cafxyw0}
\ca(f;x,w_0)=\ca(f;x,y),\quad\ca(f;w_0,y)=\ca(f;x,y).
\end{equation}
O Lema~\ref{thm:lemasecantes} nos diz que as funções:
\[I\cap\left]x,+\infty\right[\ni w\longmapsto\ca(f;x,w),\quad
I\cap\left]-\infty,y\right[\ni w\longmapsto\ca(f;w,y),\]
são crescentes e portanto \eqref{eq:cafxyw0} implica que
$\ca(f;x,w)=\ca(f;x,y)$, para todo $w\in[w_0,y]$ e que
$\ca(f;w,y)=\ca(f;x,y)$, para todo $w\in[x,w_0]$.
Daí:
\[f(w)=\ca(f;x,y)(w-x)+f(x),\]
para todo $w\in[w_0,y]$ e:
\[f(w)=\ca(f;x,y)(w-y)+f(y)\stackrel{\eqref{eq:igualdadesfaceis}}=\ca(f;x,y)(w-x)+f(x),\]
para todo $w\in[x,w_0]$.

\item (b)$\Rightarrow$(c).

Temos:
\begin{multline*}
(1-t)f(x)+tf(y)=(1-t)(ax+b)+t(ay+b)=a\big((1-t)x+ty\big)+b\\
=f\big((1-t)x+ty\big).
\end{multline*}

\item (c)$\Rightarrow$(a).

Trivial.\qedhere
\end{bulletindent}
\end{proof}

\begin{cor}\label{thm:convnotstrict}
Seja $f:I\to\R$ uma função convexa num intervalo $I\subset\R$. Então $f$ não é estritamente convexa se e somente se
existem $x,y\in I$ e $a,b\in\R$ com $x<y$ e $f(w)=aw+b$, para todo $w\in[x,y]$.\qed
\end{cor}

Dada uma função $f:I\to\R$ definida num intervalo $I\subset\R$ e $x\in I$ um ponto que não é a extremidade direita de $I$,
denotamos por $\dd^+f(x)$ a {\em derivada à direita\/}\index[indice]{derivada!a direita@à direita} de $f$ no ponto $x$,
definida por:\index[simbolos]{$\dd^+f(x)$}
\[\dd^+f(x)=\lim_{h\to0^+}\frac{f(x+h)-f(x)}h=\lim_{y\to x^+}\ca(f;x,y)\in\overline\R,\]
desde que esse limite exista em $\overline\R$. Similarmente, se $x\in I$ não é a extremidade esquerda de $I$, denotamos por $\dd^-f(x)$
a {\em derivada à esquerda\/}\index[indice]{derivada!a esquerda@à esquerda} de $f$ no ponto $x$,
definida por:\index[simbolos]{$\dd^-f(x)$}
\[\dd^-f(x)=\lim_{h\to0^-}\frac{f(x+h)-f(x)}h=\lim_{y\to x^-}\ca(f;x,y)\in\overline\R,\]
desde que esse limite exista em $\overline\R$.

\begin{lem}\label{thm:existemderlat}
Seja $f:I\to\R$ uma função convexa num intervalo $I\subset\R$. Se $x\in I$
não é a extremidade direita então a derivada à direita $\dd^+f(x)$ existe em $\overline\R$
e vale a desigualdade:
\begin{equation}\label{eq:existedmaisf}
\dd^+f(x)\le\ca(f;x,v)<+\infty,
\end{equation}
para todo $v\in I$ com $v>x$. Similarmente, se $x\in I$ não é a extremidade esquerda
então a derivada à esquerda $\dd^-f(x)$ existe em $\overline\R$
e vale a desigualdade:
\begin{equation}\label{eq:existedmenosf}
-\infty<\ca(f;u,x)\le\dd^-f(x),
\end{equation}
para todo $u\in I$ com $u<x$. Quando $x\in I$ é um ponto interior então existem e são finitas
ambas as derivadas laterais $\dd^+f(x)$, $\dd^-f(x)$ e vale a desigualdade:
\begin{equation}\label{eq:dmaismenos}
\dd^-f(x)\le\dd^+f(x).
\end{equation}
Se $f$ é estritamente convexa então as desigualdades em \eqref{eq:existedmaisf}
e \eqref{eq:existedmenosf} são estritas.
\end{lem}
\begin{proof}
Suponha que $x\in I$ não é a extremidade direita.
O Lema~\ref{thm:lemasecantes} nos diz que a função:
\begin{equation}\label{eq:funccafxv}
I\cap\left]x,+\infty\right[\ni v\longmapsto\ca(f;x,v)
\end{equation}
é crescente; segue então que o limite $\lim_{v\to x^+}\ca(f;x,v)$ existe em $\overline\R$ e:
\[\dd^+f(x)=\lim_{v\to x^+}\ca(f;x,v)=\inf_{\substack{v>x\\v\in I}}\ca(f;x,v),\]
o que prova \eqref{eq:existedmaisf}. Similarmente, se $x\in I$ não é a extremidade
esquerda então o Lema~\ref{thm:lemasecantes} nos diz que a função:
\begin{equation}\label{eq:funccafux}
I\cap\left]-\infty,x\right[\ni u\longmapsto\ca(f;u,x)
\end{equation}
é crescente, donde o limite $\lim_{u\to x^-}\ca(f;u,x)$ existe em $\overline\R$ e:
\[\dd^-f(x)=\lim_{u\to x^-}\ca(f;u,x)=\sup_{\substack{u<x\\u\in I}}\ca(f;u,x),\]
provando \eqref{eq:existedmenosf}. Se $x\in I$ é um ponto interior então
o Lema~\ref{thm:lemasecantes} nos diz que:
\[\ca(f;u,x)\le\ca(f;x,v),\]
sempre que $u,v\in I$, $u<x<v$; daí:
\[-\infty<\dd^-f(x)=\sup_{\substack{u<x\\u\in I}}\ca(f;u,x)\le
\inf_{\substack{v>x\\v\in I}}\ca(f;x,v)=\dd^+f(x)<+\infty,\]
provando \eqref{eq:dmaismenos}.
Finalmente, se $f$ é estritamente convexa então o Lema~\ref{thm:lemasecantes} nos diz que
as funções \eqref{eq:funccafxv} e \eqref{eq:funccafux} são estritamente crescentes,
o que prova as desigualdades estritas em \eqref{eq:existedmaisf} e \eqref{eq:existedmenosf}.
\end{proof}

\begin{cor}\label{thm:cordercresc}
Seja $f:I\to\R$ uma função convexa num intervalo $I\subset\R$. Dados $x,y\in I$ com $x<y$ então:
\[\dd^+f(x)\le\dd^-f(y).\]
A desigualdade é estrita se $f$ é estritamente convexa.
\end{cor}
\begin{proof}
Os Lemas~\ref{thm:lemasecantes} e \ref{thm:existemderlat} implicam que para todo $w$ em
$\left]x,y\right[$, temos:
\[\dd^+f(x)\le\ca(f;x,w)\le\ca(f;w,y)\le\dd^-f(y).\]
Se $f$ é estritamente convexa então:
\[\dd^+f(x)<\ca(f;x,w)<\ca(f;w,y)<\dd^-f(y).\qedhere\]
\end{proof}

\begin{cor}
Se $f:I\to\R$ é uma função convexa num intervalo $I\subset\R$ então $f$ é contínua em todos os pontos interiores
de $I$.
\end{cor}
\begin{proof}
Seja $x$ um ponto no interior de $I$. A existência e a finitude
das derivadas laterais $\dd^+f(x)$ e $\dd^-f(x)$ implica que os limites laterais
$\lim_{y\to x^+}f(y)$ e $\lim_{y\to x^-}f(y)$ existem e são iguais a $f(x)$; de fato:
\begin{gather*}
\lim_{y\to x^+}\big(f(y)-f(x)\big)=\lim_{y\to x^+}\big(\ca(f;x,y)(y-x)\big)=\dd^+f(x)\cdot0=0,\\
\lim_{y\to x^-}\big(f(y)-f(x)\big)=\lim_{y\to x^-}\big(\ca(f;x,y)(y-x)\big)=\dd^-f(x)\cdot0=0.
\end{gather*}
Logo $f$ é contínua no ponto $x$.
\end{proof}

\begin{lem}\label{thm:retasuporte}
Seja $f:I\to\R$ uma função convexa num intervalo $I\subset\R$. Se $x\in I$ é um ponto interior e
$a\in\R$ é tal que $\dd^-f(x)\le a\le\dd^+f(x)$ então:
\begin{equation}\label{eq:retasuporte}
f(y)\ge f(x)+a(y-x),
\end{equation}
para todo $y\in I$. A desigualdade em \eqref{eq:retasuporte} é estrita se $f$ é estritamente convexa e $y\ne x$.
\end{lem}
\begin{proof}
Se $y=x$, vale a igualdade em \eqref{eq:retasuporte}. Se $y>x$, a desigualdade \eqref{eq:retasuporte}
é equivalente a $\ca(f;x,y)\ge a$; mas segue do Lema~\ref{thm:existemderlat} que:
\[\ca(f;x,y)\ge\dd^+f(x)\ge a.\]
Similarmente, se $y<x$, a desigualdade \eqref{eq:retasuporte}
é equivalente a $\ca(f;y,x)\le a$; mas segue do Lema~\ref{thm:existemderlat} que:
\[\ca(f;y,x)\le\dd^-f(x)\le a.\]
Se $f$ é estritamente convexa então $\ca(f;x,y)>\dd^+f(x)\ge a$ para $y>x$
e $\ca(f;y,x)<\dd^-f(x)\le a$ para $y<x$.
\end{proof}

Dada uma função $f:I\to\R$ e um ponto $x\in I$, então uma reta:
\[L:\R\ni y\longmapsto ay+b\in\R\]
tal que $f(x)=L(x)$ e $f(y)\ge L(y)$ para todo $y\in I$ é dita uma {\em reta suporte\/}\index[indice]{reta!suporte}\index[indice]{suporte!reta} para
$f$ no ponto $x$. O Lema~\ref{thm:retasuporte} nos
diz então que {\em uma função convexa $f:I\to\R$ num intervalo $I$ possui uma reta suporte em todo ponto $x$ do interior
do intervalo $I$}.

\begin{lem}\label{thm:supconvexas}
Seja $(f_\lambda)_{\lambda\in\Lambda}$ uma família não vazia de funções convexas $f_\lambda:I\to\R$ num intervalo $I$.
Se para todo $x\in I$ o supremo $\sup_{\lambda\in\Lambda}f_\lambda(x)$ é finito então a função
$f=\sup_{\lambda\in\Lambda}f_\lambda$ é convexa em $I$.
\end{lem}
\begin{proof}
Sejam $x,y\in I$ e $t\in[0,1]$. Para todo $\lambda\in\Lambda$, temos:
\[f_\lambda\big((1-t)x+ty\big)\le(1-t)f_\lambda(x)+tf_\lambda(y)\le(1-t)f(x)+tf(y);\]
tomando o supremo em $\lambda$, obtemos $f\big((1-t)x+ty\big)\le(1-t)f(x)+tf(y)$.
\end{proof}

\begin{cor}\label{thm:retasupconv}
Seja $f:I\to\R$ uma função definida num intervalo $I\subset\R$. Se $f$ admite uma
reta suporte em todo ponto de $I$ então $f$ é convexa.
\end{cor}
\begin{proof}
Para cada $x\in I$, seja $L_x:\R\to\R$ uma reta suporte para $f$. Obviamente:
\[\sup_{x\in I}L_x(y)\le f(y),\]
para todo $y\in I$. Além do mais:
\[\sup_{x\in I}L_x(y)\ge L_y(y)=f(y),\]
para todo $y\in I$, donde $\sup_{x\in I}L_x=f$. Como cada $L_x$ é convexa
(veja Exercício~\ref{exe:retaconvexa}), segue do Lema~\ref{thm:supconvexas} que $f$ é convexa.
\end{proof}

\begin{prop}\label{thm:convexderivavel}
Seja $f:I\to\R$ uma função derivável num intervalo $I\subset\R$. As seguintes condições
são equivalentes:
\begin{itemize}
\item[(a)] $f$ é convexa;
\item[(b)] a derivada $f'$ é crescente;
\item[(c)] $f(y)\ge f(x)+f'(x)(y-x)$, para todos $x,y\in I$ (o gráfico de $f$ fica acima
de suas retas tangentes).
\end{itemize}
\end{prop}
\begin{proof}\
\begin{bulletindent}
\item (a)$\Rightarrow$(b).

Segue do Corolário~\ref{thm:cordercresc}.

\item (b)$\Rightarrow$(c).

Seja $x\in I$ fixado e considere a função $g:I\to\R$ definida por:
\[g(y)=f(y)-\big(f(x)+f'(x)(y-x)\big),\]
para todo $y\in I$. Temos $g'(y)=f'(y)-f'(x)\le0$, se $y\in I\cap\left]-\infty,x\right]$ e $g'(y)\ge0$, se $y\in I\cap\left[x,+\infty\right[$;
segue que $g$ é decrescente em $I\cap\left]-\infty,x\right]$ e crescente em
$I\cap\left[x,+\infty\right[$. Como $g(x)=0$, concluímos que $g(y)\ge0$, para todo $y\in I$.

\item (c)$\Rightarrow$(a).

Segue do Corolário~\ref{thm:retasupconv}.\qedhere
\end{bulletindent}
\end{proof}

\begin{cor}
Seja $f:I\to\R$ uma função duas vezes derivável num intervalo $I\subset\R$. Então
$f$ é convexa se e somente se a sua derivada segunda $f''$ é não negativa.
\end{cor}
\begin{proof}
Evidentemente $f'$ é crescente se e somente se $f''$ é não negativa.
\end{proof}

Temos a seguinte versão da Proposição~\ref{thm:convexderivavel} para convexidade estrita.
\begin{prop}
Seja $f:I\to\R$ uma função derivável num intervalo $I\subset\R$. As seguintes condições
são equivalentes:
\begin{itemize}
\item[(a)] $f$ é estritamente convexa;
\item[(b)] a derivada $f'$ é estritamente crescente;
\item[(c)] $f(y)>f(x)+f'(x)(y-x)$, para todos $x,y\in I$ com $x\ne y$.
\end{itemize}
\end{prop}
\begin{proof}\
\begin{bulletindent}
\item (a)$\Rightarrow$(b).

Segue do Corolário~\ref{thm:cordercresc}.

\item (b)$\Rightarrow$(c).

Argumentamos como na demonstração da implicação (b)$\Rightarrow$(c) na Proposição~\ref{thm:convexderivavel},
mas agora temos que $g'(y)<0$ para $y\in I$ com $y<x$ e $g'(y)>0$ para $y\in I$ com $y>x$;
daí $g$ é estritamente decrescente em $I\cap\left]-\infty,x\right]$ e estritamente crescente em
$I\cap\left[x,+\infty\right[$. Concluímos então que $g(y)>0$, para $y\in I$ com $y\ne x$.

\item (c)$\Rightarrow$(a).

O Corolário~\ref{thm:retasupconv} nos garante que $f$ é convexa. Se $f$ não fosse
estritamente convexa, existiriam $u,v\in I$, $a,b\in\R$ com $u<v$ e:
\[f(w)=aw+b,\]
para todo $w\in[u,v]$ (Corolário~\ref{thm:convnotstrict}). Daí, para todos $x,y\in[u,v]$,
teríamos:
\[f(x)+f'(x)(y-x)=ax+b+a(y-x)=f(y),\]
contradizendo~(c).\qedhere
\end{bulletindent}
\end{proof}

\begin{cor}\label{thm:secderpos}
Seja $f:I\to\R$ uma função duas vezes derivável num intervalo $I\subset\R$. Se sua
derivada segunda $f''$ é positiva então $f$ é estritamente convexa.
\end{cor}
\begin{proof}
Evidentemente $f'$ é estritamente crescente se $f''$ é positiva.
\end{proof}

\begin{example}
A recíproca do Corolário~\ref{thm:secderpos} não é verdadeira.
De fato, a função $f:\R\ni x\mapsto x^4\in\R$ é estritamente convexa, pois sua derivada
$f'(x)=4x^3$ é estritamente crescente. No entanto, temos $f''(0)=0$.
\end{example}

\begin{prop}\label{thm:Jensenfinito}
Seja $f:I\to\R$ uma função convexa num intervalo $I\subset\R$. Dados $x_1,\ldots,x_n\in I$
e números reais não negativos $\alpha_1$, \dots, $\alpha_n$ com $\alpha_1+\cdots+\alpha_n=1$ então
$\alpha_1x_1+\cdots+\alpha_nx_n\in I$ e:
\begin{equation}\label{eq:convexan}
f(\alpha_1x_1+\cdots+\alpha_nx_n)\le\alpha_1f(x_1)+\cdots+\alpha_nf(x_n);
\end{equation}
se $f$ é estritamente convexa e $\alpha_1,\ldots,\alpha_n>0$ então a igualdade vale em \eqref{eq:convexan}
se e somente se $x_1=\cdots=x_n$.
\end{prop}
\begin{proof}
Se $a=\min\{x_1,\ldots,x_n\}$ e $b=\max\{x_1,\ldots,x_n\}$ então $a,b\in I$ e:
\[a=(\alpha_1+\cdots+\alpha_n)a\le\alpha_1x_1+\cdots+\alpha_nx_n\le
(\alpha_1+\cdots+\alpha_n)b=b,\]
donde $\alpha_1x_1+\cdots+\alpha_nx_n\in I$. Para provar o restante da tese, usamos indução em $n$.
O resultado é óbvio no caso que $n=1$. Assumindo o resultado válido para
um certo $n\ge1$, sejam $x_1,\ldots,x_{n+1}\in I$ e $\alpha_1,\ldots,\alpha_{n+1}\ge0$
com $\alpha_1+\cdots+\alpha_{n+1}=1$. Vamos mostrar que:
\begin{multline}\label{eq:convexnplus1}
f(\alpha_1x_1+\cdots+\alpha_nx_n+\alpha_{n+1}x_{n+1})\\
\le\alpha_1f(x_1)+\cdots+\alpha_nf(x_n)+\alpha_{n+1}f(x_{n+1}).
\end{multline}
Se $\alpha_{n+1}=1$, vale a igualdade em \eqref{eq:convexnplus1}. Suponha que $\alpha_{n+1}<1$.
Seja $\alpha'_i=\frac{\alpha_i}{1-\alpha_{n+1}}$, $i=1,\ldots,n$. Claramente $\alpha'_1,\ldots,\alpha'_n\ge0$
e $\alpha'_1+\cdots+\alpha'_n=1$, donde a hipótese de indução nos dá:
\[f(\alpha'_1x_1+\cdots+\alpha'_nx_n)\le\alpha'_1f(x_1)+\cdots+\alpha'_nf(x_n).\]
Temos:
\begin{multline}\label{eq:convexinducao}
f(\alpha_1x_1+\cdots+\alpha_nx_n+\alpha_{n+1}x_{n+1})\\
=f\big((1-\alpha_{n+1})(\alpha'_1x_1+\cdots+\alpha'_nx_n)+\alpha_{n+1}x_{n+1}\big)\\
\le(1-\alpha_{n+1})f(\alpha'_1x_1+\cdots+\alpha'_nx_n)+\alpha_{n+1}f(x_{n+1})\\
\le(1-\alpha_{n+1})\big(\alpha'_1f(x_1)+\cdots+\alpha'_nf(x_n)\big)+\alpha_{n+1}f(x_{n+1})\\
=\alpha_1f(x_1)+\cdots+\alpha_nf(x_n)+\alpha_{n+1}f(x_{n+1}),
\end{multline}
onde na primeira desigualdade usamos $\alpha'_1x_1+\cdots+\alpha'_nx_n\in I$
e a convexidade de $f$. Isso prova \eqref{eq:convexnplus1}.
Suponha agora que $\alpha_1,\ldots,\alpha_{n+1}>0$,
$f$ é estritamente convexa e vale a igualdade em \eqref{eq:convexnplus1}. Daí
todas as desigualdades em \eqref{eq:convexinducao} são igualdades e portanto:
\begin{equation}\label{eq:convigualalphalinha}
\alpha'_1x_1+\cdots+\alpha'_nx_n=x_{n+1},
\end{equation}
e:
\[f(\alpha'_1x_1+\cdots+\alpha'_nx_n)=\alpha'_1f(x_1)+\cdots+\alpha'_nf(x_n).\]
Como $\alpha'_1,\ldots,\alpha'_n>0$, a hipótese de indução nos dá $x_1=\cdots=x_n$.
Finalmente, \eqref{eq:convigualalphalinha} implica que $x_1=\cdots=x_n=x_{n+1}$.
\end{proof}

\end{section}

\section*{Exercícios para o Capítulo~\ref{CHP:TOPANFUNC}}

\subsection*{Espaços Normados e com Produto Interno}

\begin{exercise}\label{exe:outratriangular}
Sejam $E$ um espaço vetorial sobre $\K$ e $\Vert\cdot\Vert$ uma semi-norma em $E$. Mostre que:
\[\big\vert\Vert x\Vert-\Vert y\Vert\big\vert\le\Vert x-y\Vert,\]
para todos $x,y\in E$. Conclua que se $(E,\Vert\cdot\Vert)$ é um espaço normado
então a norma $\Vert\cdot\Vert:E\to\R$ é uma aplicação Lipschitziana
(e portanto uniformemente contínua), se $E$ é munido da métrica associada a $\Vert\cdot\Vert$.
\end{exercise}

\begin{exercise}\label{exe:operacoescontinuas}
Seja $(E,\Vert\cdot\Vert)$ um espaço normado sobre $\K$. Se $E$ é munido da métrica associada a $\Vert\cdot\Vert$,
mostre que as aplicações:
\[E\times E\ni(x,y)\longmapsto x+y\in E,\quad\K\times E\ni(\lambda,x)\longmapsto\lambda x\in E,\]
são contínuas.
\end{exercise}

\begin{exercise}\label{exe:metricanorma}
Seja $(E,\Vert\cdot\Vert)$ um espaço normado sobre $\K$. Mostre que a métrica $d$ associada à norma $\Vert\cdot\Vert$ satisfaz
as seguintes condições:
\begin{itemize}
\item[(a)] $d(x+z,y+z)=d(x,y)$, para todos $x,y,z\in E$ ({\em invariância por translações});%
\index[indice]{invariancia por translacoes@invariância por translações}\index[indice]{metrica@métrica!invariante por translacoes@invariante por translações}
\item[(b)] $d(\lambda x,\lambda y)=\vert\lambda\vert\,d(x,y)$, para todo $\lambda\in\K$ e todos $x,y\in E$.
\end{itemize}
Reciprocamente, se $d$ é uma métrica num espaço vetorial $E$ sobre $\K$ que satisfaz as condições~(a) e (b) acima,
mostre que existe uma única norma $\Vert\cdot\Vert$ em $E$ tal que $d$ é a métrica associada a $\Vert\cdot\Vert$.
\end{exercise}

\begin{exercise}
Seja $E$ um espaço vetorial {\em real}. Para que uma métrica $d$ em $E$ seja a métrica associada
a uma norma em $E$, mostre que é suficiente que $d$ satisfaça a condição~(a) que está no enunciado do Exercício~\ref{exe:metricanorma}
e a condição:
\begin{itemize}
\item[(b')] $d(\lambda x,\lambda y)=\lambda\,d(x,y)$, para todo $\lambda>0$ e todos $x,y\in E$ ({\em homogeneidade positiva}).%
\index[indice]{homogeneidade positiva}\index[indice]{metrica@métrica!positivamente homogenea@positivamente homogênea}
\end{itemize}
\end{exercise}

\begin{exercise}
Sejam $E$ um espaço vetorial e $S\subset E$ um subespaço vetorial de $E$. Mostre que:
\begin{itemize}
\item a relação binária $\sim$ em $E$ definida por:
\[x\sim y\Longleftrightarrow x-y\in S,\quad x,y\in E,\]
é uma relação de equivalência em $E$;
\item para todo $x\in E$, a classe de equivalência de $x$ correspondente a $\sim$ é dada por:
\[x+S=\big\{x+v:v\in S\big\};\]
\item existe uma única estrutura de espaço vetorial no conjunto quociente $E/S=\big\{x+S:x\in E\big\}$
tal que a aplicação quociente:
\[q:E\ni x\longmapsto x+S\in E/S\]
é linear.
\end{itemize}
O espaço vetorial $E/S$\index[simbolos]{$E/S$} é chamado o
{\em espaço vetorial quociente\/}\index[indice]{espaco vetorial@espaço vetorial!quociente}\index[indice]{quociente!de um espaco vetorial@de um espaço vetorial}
de $E$ pelo subespaço $S$.
\end{exercise}

\begin{exercise}\label{exe:seminormaviranorma}
Seja $E$ um espaço vetorial sobre $\K$ e $\Vert\cdot\Vert$ uma semi-norma em $E$. Mostre que:
\begin{itemize}
\item o conjunto $N=\big\{x\in E:\Vert x\Vert=0\big\}$ é um subespaço de $E$;
\item existe uma única norma $\Vert\cdot\Vert'$ no espaço quociente $E/N$ tal que
$\Vert x+N\Vert'=\Vert x\Vert$, para todo $x\in E$.
\end{itemize}
\end{exercise}

\begin{exdefin}\label{thm:realificacao}
Se $E$ é um espaço vetorial complexo então o espaço vetorial real $E\vert_\R$\index[simbolos]{$E\vert_\R$} obtido de $E$ pela restrição
da multiplicação por escalares $\C\times E\ni(\lambda,x)\mapsto\lambda x\in E$ a $\R\times E$ é chamado a
{\em realificação\/}\index[indice]{realificacao@realificação!de um espaco vetorial@de um espaço vetorial} de $E$.
\end{exdefin}

\begin{exercise}\label{exe:mesmanormanareal}
Se $E$ é um espaço vetorial complexo e $\Vert\cdot\Vert$ é uma norma em $E$, mostre que $\Vert\cdot\Vert$ também
é uma norma na realificação $E\vert_\R$ de $E$. Dê exemplo de uma norma em $E\vert_\R$ que não é uma norma em $E$.
\end{exercise}

\begin{exercise}\label{exe:prodintrealifica}
Se $E$ é um espaço vetorial complexo e $\langle\cdot,\cdot\rangle$ é um produto interno em $E$, mostre que:
\[\langle x,y\rangle_\R=\Re\langle x,y\rangle,\quad x,y\in E,\]
define um produto interno $\langle\cdot,\cdot\rangle_\R$ em $E\vert_\R$. Mostre também que:
\begin{itemize}
\item[(a)] $\langle\cdot,\cdot\rangle$ e $\langle\cdot,\cdot\rangle_\R$ determinam a mesma norma
(conclua que $(E,\langle\cdot,\cdot\rangle)$ é um espaço de Hilbert complexo se e somente se
$(E\vert_\R,\langle\cdot,\cdot\rangle_\R)$ é um espaço de Hilbert real);
\item[(b)] $\langle ix,iy\rangle_\R=\langle x,y\rangle_\R$ e $\langle ix,y\rangle_\R=-\langle x,iy\rangle_\R$, para todos $x,y\in E$;
\item[(c)] $\langle x,y\rangle=\langle x,y\rangle_\R-i\langle ix,y\rangle_\R$, para todos $x,y\in E$.
\end{itemize}
\end{exercise}

\begin{exercise}\label{exe:langleranglezero}
Seja $E$ um espaço vetorial complexo e $\langle\cdot,\cdot\rangle_0$ um produto interno em $E\vert_\R$.
Mostre que as duas seguintes condições são equivalentes:
\begin{itemize}
\item[(i)] $\langle ix,iy\rangle_0=\langle x,y\rangle_0$, para todos $x,y\in E$;
\item[(ii)] $\langle ix,y\rangle_0=-\langle x,iy\rangle_0$, para todos $x,y\in E$.
\end{itemize}
Assumindo que uma das (e portanto ambas as) condições acima são satisfeitas, mostre que:
\[\langle x,y\rangle=\langle x,y\rangle_0-i\langle ix,y\rangle_0,\quad x,y\in E,\]
define um produto interno $\langle\cdot,\cdot\rangle$ em $E$ tal que
$\langle x,y\rangle_0=\Re\langle x,y\rangle$, para todos $x,y\in E$.
\end{exercise}

\begin{exercise}\label{exe:supparalelogramo}
Seja $X$ um conjunto com mais de um ponto. Mostre que a norma do supremo (Exemplo~\ref{exa:BoundedXK})
definida no espaço vetorial $\Bounded(X,\K)$
das funções limitadas $f:X\to\K$ não satisfaz a identidade do paralelogramo (veja \eqref{eq:paralelogramo}).
Conclua que essa norma não está associada a nenhum produto interno.
\end{exercise}

\begin{exercise}[fórmula de polarização]\index[indice]{formula@fórmula!de polarizacao@de polarização}\index[indice]{polarizacao@polarização!formula de@fórmula de}
Seja $E$ um espaço vetorial sobre $\K$ e $\langle\cdot,\cdot\rangle$ um produto interno em $E$. Se $\K=\R$,
mostre que:
\begin{gather}
\label{eq:polarizacao}\langle x,y\rangle=\frac12\big(\Vert x+y\Vert^2-\Vert x\Vert^2-\Vert y\Vert^2\big),\\
\langle x,y\rangle=\frac14\big(\Vert x+y\Vert^2-\Vert x-y\Vert^2\big),
\end{gather}
para todos $x,y\in E$. Para $\K=\C$, mostre que:
\begin{gather*}
\Re\langle x,y\rangle=\frac12\big(\Vert x+y\Vert^2-\Vert x\Vert^2-\Vert y\Vert^2\big),\\
\Re\langle x,y\rangle=\frac14\big(\Vert x+y\Vert^2-\Vert x-y\Vert^2\big),
\end{gather*}
para todos $x,y\in E$. Use a fórmula que está no item~(c) do Exercício~\ref{exe:prodintrealifica} para
concluir que, também no caso $\K=\C$, podemos escrever uma fórmula para $\langle x,y\rangle$ usando
apenas a norma $\Vert\cdot\Vert$.
\end{exercise}

\begin{exercise}
Seja $(E,\langle\cdot,\cdot\rangle)$ um espaço pré-Hilbertiano sobre $\K$. Se $E$ é munido da métrica associada
à norma associada a $\langle\cdot,\cdot\rangle$, mostre que o produto interno $\langle\cdot,\cdot\rangle:E\times E\to\K$
é uma aplicação contínua.
\end{exercise}

\begin{exercise}\label{exe:Qlinear}
Este exercício contém um resultado preparatório que será usado na resolução do Exercício~\ref{exe:critparalelogramo}.
Sejam $E$, $E'$ espaços vetoriais sobre $\Q$ e seja $T:E\to E'$ um homomorfismo de grupos aditivos, i.e.,
$T(x+y)=T(x)+T(y)$, para todos $x,y\in E$. Mostre que $T$ é linear, i.e., mostre que $T(\lambda x)=\lambda T(x)$,
para todos $x\in E$, $\lambda\in\Q$.
\end{exercise}

\begin{exercise}\label{exe:critparalelogramo}
Seja $E$ um espaço vetorial sobre $\K$ e $\Vert\cdot\Vert$ uma norma em $E$ que satisfaz a identidade do paralelogramo
\eqref{eq:paralelogramo}. O objetivo deste exercício é mostrar que $\Vert\cdot\Vert$ está associada a um único
produto interno no espaço vetorial $E$.
\begin{itemize}
\item[(a)] Mostre que:
\[\Vert x+y\Vert^2+\Vert x+z\Vert^2+\Vert y+z\Vert^2=\Vert x+y+z\Vert^2+\Vert x\Vert^2+\Vert y\Vert^2+\Vert z\Vert^2,\]
para todos $x,y,z\in E$.
\item[(b)] Defina $\langle\cdot,\cdot\rangle:E\times E\to\R$ através da fórmula \eqref{eq:polarizacao}
e use o resultado do item~(a) para concluir que $\langle x+y,z\rangle=\langle x,z\rangle+\langle y,z\rangle$,
para todos $x,y,z\in E$.
\item[(c)] Use o resultado do item~(b) e o resultado do Exercício~\ref{exe:Qlinear} para concluir que
$\langle\lambda x,y\rangle=\lambda\langle x,y\rangle$, para todos $x,y\in E$ e todo $\lambda\in\Q$.
\item[(d)] Use o resultado do item~(d) e o resultado dos Exercícios~\ref{exe:outratriangular} e
\ref{exe:operacoescontinuas} para concluir que
$\langle\lambda x,y\rangle=\lambda\langle x,y\rangle$, para todos $x,y\in E$ e todo $\lambda\in\R$.
\item[(e)] Se $\K=\R$, mostre que \eqref{eq:polarizacao} define um produto interno $\langle\cdot,\cdot\rangle$
em $E$ e que $\Vert\cdot\Vert$ é a norma associada a $\langle\cdot,\cdot\rangle$.
\item[(f)] Se $\K=\C$, use o resultado do Exercício~\ref{exe:langleranglezero} para concluir que existe
um único produto interno em $E$ ao qual a norma $\Vert\cdot\Vert$ está associada.
\end{itemize}
\end{exercise}

\begin{exercise}[teorema de Pitágoras]\index[indice]{Pitagoras@Pitágoras!teorema de}\index[indice]{teorema!de Pitagoras@de Pitágoras}
\label{exe:pitagoras}
Seja $E$ um espaço vetorial sobre $\K$ e $\langle\cdot,\cdot\rangle$ um produto interno em $E$.
Se $x,y\in E$ são ortogonais, mostre que:
\[\Vert x+y\Vert^2=\Vert x\Vert^2+\Vert y\Vert^2.\]
\end{exercise}

\subsection*{Aplicações Lineares Contínuas}

\begin{exercise}\label{exe:limitadamesmo}
Mostre que:
\begin{itemize}
\item um espaço vetorial normado não nulo nunca é um espaço métrico limitado;
\item uma aplicação linear não nula entre espaços normados nunca possui imagem limitada.
\end{itemize}
\end{exercise}

\begin{exercise}\label{exe:supnaesfera}
Sejam $(E,\norma\cdot E)$, $(F,\norma\cdot F)$ espaços normados sobre $\K$,
com $E$ não nulo. Se $T:E\to F$ é uma aplicação linear limitada, mostre que:
\[\Vert T\Vert=\sup\big\{\norma{T(x)}F:\text{$x\in E$ e $\norma xE=1$}\big\}.\]
\end{exercise}

\begin{exercise}\label{exe:normaop}
Mostre que \eqref{eq:normaopEF} define uma norma no espaço vetorial $\Lin(E,F)$.
\end{exercise}

\begin{exercise}\label{exe:espconjugado1}
Seja $E$ um espaço vetorial complexo. Denotemos por $\overline E$ o conjunto $E$
munido da mesma soma de $E$ e da operação de multiplicação por escalares complexos
definida por:
\[\C\times E\ni(\lambda,x)\longmapsto\bar\lambda x\in E.\]
Mostre que:
\begin{itemize}
\item $\overline E$ também é um espaço vetorial complexo;
\item uma aplicação $\Vert\cdot\Vert:E\to\R$ é uma norma em $E$ se e somente se ela é uma norma
em $\overline E$.
\end{itemize}
Dizemos que $\overline E$ é o espaço vetorial
{\em conjugado\/}\index[indice]{espaco vetorial@espaço vetorial!conjugado}%
\index[indice]{conjugado!de um espaco vetorial@de um espaço vetorial} a $E$.
\end{exercise}

\begin{exercise}\label{exe:espconjugado2}
Sejam $E$ um espaço vetorial complexo e $\overline E$ seu espaço vetorial conjugado. Mostre que:
\begin{itemize}
\item o espaço vetorial conjugado de $\overline E$ é $E$;
\item se $S$ é um subespaço vetorial de $E$ então o espaço vetorial conjugado a $S$
é um subespaço vetorial de $\overline E$.
\end{itemize}
\end{exercise}

\begin{exercise}\label{exe:espconjugado3}
Sejam $E$ um espaço vetorial complexo e $\overline E$ seu espaço vetorial conjugado.
Se $\langle\cdot,\cdot\rangle$ é um produto interno em $E$, mostre que:
\begin{equation}\label{eq:prodintconjugado}
E\times E\ni(x,y)\longmapsto\overline{\langle x,y\rangle}\in\C
\end{equation}
é um produto interno em $\overline E$. Mostre que $\langle\cdot,\cdot\rangle$
e \eqref{eq:prodintconjugado} definem a mesma norma em $E$.
\end{exercise}

\begin{exercise}\label{exe:espconjugado4}
Sejam $E$, $F$ espaços vetoriais complexos e $\overline E$, $\overline F$ seus espaços
vetoriais conjugados. Mostre que as seguintes afirmações são equivalentes sobre
uma aplicação $T:E\to F$:
\begin{itemize}
\item $T:E\to F$ é linear;
\item $T:\overline E\to\overline F$ é linear.
\end{itemize}
Mostre também que são equivalentes as seguintes afirmações:
\begin{itemize}
\item $T:E\to F$ é linear-conjugada;
\item $T:\overline E\to F$ é linear;
\item $T:E\to\overline F$ é linear.
\end{itemize}
\end{exercise}

\subsection*{Funções Convexas}

\begin{exercise}\label{exe:retaconvexa}
Se $f:\R\to\R$ é uma função afim, i.e., existem $a,b\in\R$ com $f(x)=ax+b$ para todo
$x\in\R$, mostre que $f$ é convexa.
\end{exercise}

\end{chapter}

\begin{chapter}{Construção de Medidas}
\label{CHP:CONSTRUCAO}

\begin{section}{Medidas em Classes de Conjuntos}

Recorde da Definição~\ref{thm:defespacomedida} que um espaço de medida consiste de um conjunto $X$, de
uma $\sigma$-álgebra $\mathcal A$ de partes de $X$ e de uma medida $\mu$ definida nessa $\sigma$-álgebra.
Uma $\sigma$-álgebra de partes de $X$ é uma coleção de subconjuntos de $X$ que inclui o próprio conjunto $X$
e que é fechada por todas as operações conjuntistas, desde que realizadas apenas uma quantidade enumerável de
vezes (veja Definição~\ref{thm:defalgsigmaalg} e Observação~\ref{thm:obsXnaalg}). Espaços de medida são ambientes
confortáveis para o desenvolvimento de uma teoria de integração (veja Capítulo~\ref{CHP:INTEGRAL}) justamente
porque a classe dos conjuntos mensuráveis (i.e., a $\sigma$-álgebra) é fechada pelas várias operações conjuntistas
que precisamos fazer durante o desenvolvimento da teoria. Em contra-partida, $\sigma$-álgebras são muitas vezes
classes de conjuntos um tanto complexas e não é sempre fácil construir exemplos não triviais de medidas definidas
em $\sigma$-álgebras (considere, por exemplo, o trabalho que tivemos na Seção~\ref{sec:MedLebRn} para construir a medida
de Lebesgue). Nosso objetivo agora é o de mostrar como construir uma medida numa $\sigma$-álgebra a partir
de uma medida definida {\it a priori\/} apenas em uma classe de conjuntos mais simples. Começamos então
definindo a noção de medida em uma classe de conjuntos arbitrária.

\begin{defin}\label{thm:defmedidageral}
Seja $\mathcal C$ uma classe\footnote{%
Neste texto, as palavras ``classe'' e ``conjunto'' têm exatamente o mesmo significado. Observamos que em textos de teoria
dos conjuntos e lógica, quando teorias axiomáticas como NBG e KM (vide \cite{Mendelson})
são expostas, as palavras ``classe'' e ``conjunto'' têm significados diferentes (a saber: uma classe $X$ é um {\em conjunto\/}
quando existe uma classe $Y$ tal que $X\in Y$).}\index[indice]{classe} de conjuntos tal que o conjunto vazio $\emptyset$ pertence a $\mathcal C$.
Uma {\em medida finitamente aditiva\/}\index[indice]{medida!finitamente aditiva}\index[indice]{finitamente aditiva!medida}
em $\mathcal C$ é uma função $\mu:\mathcal C\to[0,+\infty]$ tal que $\mu(\emptyset)=0$ e tal que,
se $(A_k)_{k=1}^t$ é uma seqüência
finita de elementos dois a dois disjuntos de $\mathcal C$ {\em tal que $\bigcup_{k=1}^t A_k$ também está
em $\mathcal C$}, então:
\begin{equation}\label{eq:finitaddmathcalC}
\mu\Big(\bigcup_{k=1}^t A_k\Big)=\sum_{k=1}^t\mu(A_k).
\end{equation}
Uma {\em medida\/}\index[indice]{medida!numa classe de conjuntos} em $\mathcal C$
é uma função $\mu:\mathcal C\to[0,+\infty]$ tal que $\mu(\emptyset)=0$ e tal que, se $(A_k)_{k\ge1}$ é
uma seqüência de elementos dois a dois disjuntos de $\mathcal C$ {\em tal que $\bigcup_{k=1}^\infty A_k$ também está
em $\mathcal C$}, então:
\begin{equation}\label{eq:sigmaddmathcalC}
\mu\Big(\bigcup_{k=1}^\infty A_k\Big)=\sum_{k=1}^\infty\mu(A_k).
\end{equation}
\end{defin}
Claramente toda medida é também uma medida finitamente aditiva; basta tomar $A_k=\emptyset$ para todo $k>t$
em \eqref{eq:sigmaddmathcalC}.

Observe que uma medida finitamente aditiva pode em geral não ser uma medida. A expressão ``finitamente aditiva''
não deve ser encarada como um adjetivo que está sendo acrescido à palavra ``medida''; deve-se pensar
na expressão ``medida finitamente aditiva'' como sendo um substantivo. Para evitar mal-entendidos, usaremos muitas vezes
a expressão {\em medida $\sigma$-aditiva\/}\index[indice]{medida!sigma aditiva@$\sigma$-aditiva}\index[indice]{sigma aditiva@$\sigma$-aditiva!medida}
como sendo um sinônimo de ``medida'', quando queremos enfatizar que não estamos falando apenas de uma medida
finitamente aditiva.

\begin{rem}
Alguns comentários de natureza conjuntista: para $t=0$, a igualdade \eqref{eq:finitaddmathcalC} é equivalente a
$\mu(\emptyset)=0$, de modo que a condição $\mu(\emptyset)=0$ na definição de medida
finitamente aditiva é redundante se admitirmos $t=0$ em \eqref{eq:finitaddmathcalC}. A condição \eqref{eq:sigmaddmathcalC},
no entanto, não implica $\mu(\emptyset)=0$ pois essa condição é consistente com $\mu(A)=+\infty$, para todo $A\in\mathcal C$
(veja Exercício~\ref{exe:tudoinfinito}).

Note que se $\mathcal C$ é uma classe de conjuntos arbitrária então sempre existe um conjunto $X$ tal que
$\mathcal C\subset\wp(X)$, isto é, tal que todo elemento de $\mathcal C$ é um subconjunto de $X$. De fato, basta
tomar $X=\bigcup_{A\in\mathcal C}A$.
\end{rem}

\begin{rem}
A expressão ``classe de conjuntos'' usada na Definição~\ref{thm:defmedidageral} é redundante
se entendemos que a teoria dos conjuntos usada no texto está fundamentada por uma teoria axiomática como ZFC, já que em ZFC {\em todo
objeto é um conjunto\/} e portanto todo conjunto $\mathcal C$ é também um conjunto de conjuntos. Na prática, no entanto,
alguns objetos (como números naturais ou números reais) não são costumeiramente pensados como conjuntos e por questões
didáticas consideramos que seja mais claro na Definição~\ref{thm:defmedidageral} (e em outras
situações similares) enfatizar que $\mathcal C$ é uma classe de conjuntos.
\end{rem}

Muito pouco pode-se provar sobre medidas em classes de conjuntos arbitrárias $\mathcal C$ (com $\emptyset\in\mathcal C$),
pois é bem possível que, a menos de situações triviais, não existam seqüências de elementos dois a dois disjuntos
de $\mathcal C$ cuja união está em $\mathcal C$. Em particular, a tese dos Lemas~\ref{thm:muAminusB}
e \ref{thm:setlimits} não são em geral satisfeitas para medidas $\mu$ em classes de conjuntos arbitrárias, como ilustra o seguinte
exemplo.
\begin{example}\label{exa:patolC}
Seja $\N=\{0,1,2,\ldots\}$\index[simbolos]{$\N$} o conjunto dos números naturais\index[indice]{numeros naturais@números naturais}\index[indice]{naturais!numeros@números}
e considere a classe de conjuntos $\mathcal C$ definida por:
\[\mathcal C=\{\emptyset,\N\}\cup\big\{\{0,1,\ldots,n\}:n\in\N\big\}.\]
Dados $A,B\in\mathcal C$, se $A\cap B=\emptyset$ então necessariamente $A=\emptyset$ ou $B=\emptyset$.
Segue que {\em qualquer\/} função $\mu:\mathcal C\to[0,+\infty]$ com $\mu(\emptyset)=0$ é uma medida
em $\mathcal C$. É fácil exibir então medidas em $\mathcal C$ para as quais as teses dos Lemas~\ref{thm:muAminusB}
e \ref{thm:setlimits} não são satisfeitas.
\end{example}

\begin{rem}\label{thm:rembastadois}
Seja $\mathcal C$ uma classe de conjuntos com $\emptyset\in\mathcal C$ e seja $\mu:\mathcal C\to[0,+\infty]$
uma função tal que $\mu(\emptyset)=0$. Para verificar que $\mu$ é uma medida finitamente aditiva em $\mathcal C$, {\em não
é\/} suficiente verificar que:
\begin{equation}\label{eq:addbastadois}
\mu(A\cup B)=\mu(A)+\mu(B),
\end{equation}
para todos $A,B\in\mathcal C$ com $A\cap B=\emptyset$ e $A\cup B\in\mathcal C$ (veja Exercício~\ref{exe:naobastadois}).
No entanto, se a classe $\mathcal C$ é
{\em fechada por uniões finitas\/}\index[indice]{classe!fechada!por unioes finitas@por uniões finitas}%
\index[indice]{fechada!por unioes finitas@por uniões finitas}
(i.e., se $A\cup B\in\mathcal C$, para todos $A,B\in\mathcal C$) então evidentemente $\mu:\mathcal C\to[0,+\infty]$ é uma medida finitamente aditiva se e somente
se $\mu(\emptyset)=0$ e \eqref{eq:addbastadois} é satisfeita, para todos $A,B\in\mathcal C$ com $A\cap B=\emptyset$ e $A\cup B\in\mathcal C$.
\end{rem}

Se a classe de conjuntos $\mathcal C$ onde a medida $\mu$ está definida é fechada por diferenças então as patologias
observadas no Exemplo~\ref{exa:patolC} não ocorrem. Esse é o conteúdo do seguinte:
\begin{lem}\label{thm:classerazoavel}
Seja $\mathcal C$ uma classe de conjuntos tal que $\emptyset\in\mathcal C$ e tal que $A_2\setminus A_1\in\mathcal C$,
para todos $A_1,A_2\in\mathcal C$ tais que $A_1\subset A_2$ (diz-se nesse caso que a classe de conjuntos $\mathcal C$ é
{\em fechada por diferença própria}\index[indice]{diferenca propria@diferença própria}). Temos que:
\begin{itemize}
\item[(a)] se $\mu:\mathcal C\to[0,+\infty]$ é uma medida finitamente aditiva então dados $A_1,A_2\in\mathcal C$ com
$A_1\subset A_2$, temos $\mu(A_1)\le\mu(A_2)$;
\item[(b)] se $\mu:\mathcal C\to[0,+\infty]$ é uma medida finitamente aditiva então dados $A_1,A_2\in\mathcal C$ com
$A_1\subset A_2$ e $\mu(A_1)<+\infty$, temos:
\[\mu(A_2\setminus A_1)=\mu(A_2)-\mu(A_1);\]
\item[(c)] se $\mu:\mathcal C\to[0,+\infty]$ é uma medida $\sigma$-aditiva e se $(A_k)_{k\ge1}$ é uma seqüência
de elementos de $\mathcal C$ tal que $A_k\nearrow A$ e $A\in\mathcal C$ então:
\begin{equation}\label{eq:mueolim2}
\mu(A)=\lim_{k\to\infty}\mu(A_k);
\end{equation}
\item[(d)] se $\mu:\mathcal C\to[0,+\infty]$ é uma medida $\sigma$-aditiva e se $(A_k)_{k\ge1}$ é uma seqüência
de elementos de $\mathcal C$ tal que $A_k\searrow A$, $A\in\mathcal C$ e $\mu(A_1)<+\infty$ então a igualdade \eqref{eq:mueolim2}
vale.
\end{itemize}
Suponha adicionalmente que $A_2\setminus A_1\in\mathcal C$, para todos $A_1,A_2\in\mathcal C$. Então valem também:
\begin{itemize}
\item[(e)] se $\mu:\mathcal C\to[0,+\infty]$ é uma medida finitamente aditiva então dados $A,A_1,\ldots,A_t\in\mathcal C$
com $A\subset\bigcup_{k=1}^tA_k$, temos:
\[\mu(A)\le\sum_{k=1}^t\mu(A_k);\]
\item[(f)] se $\mu:\mathcal C\to[0,+\infty]$ é uma medida $\sigma$-aditiva então dados $A\in\mathcal C$ e uma seqüência
$(A_k)_{k\ge1}$ em $\mathcal C$ com $A\subset\bigcup_{k=1}^\infty A_k$, temos:
\[\mu(A)\le\sum_{k=1}^\infty\mu(A_k).\]
\end{itemize}
\end{lem}
\begin{proof}
A demonstração dos itens (a), (b), (c) e (d) é idêntica à demonstração dos Lemas~\ref{thm:muAminusB} e \ref{thm:setlimits}.
Passemos à demonstração do item~(e). Para cada $k=1,\ldots,t$, seja $A'_k=A_k\cap A$, de modo que:
\[A=\bigcup_{k=1}^tA'_k.\]
Observamos que $A'_k\in\mathcal C$, para todo $k$; de fato:
\[A'_k=A_k\setminus(A_k\setminus A).\]
Agora sejam $B_1=A'_1$ e $B_k=A'_k\setminus(A'_1\cup A'_2\cup\ldots\cup A'_{k-1})$, para $k=2,\ldots,t$; note que:
\[B_k=\big(\cdots\big((A'_k\setminus A'_1)\setminus A'_2\big)\cdots\big)\setminus A'_{k-1},\quad k=2,\ldots,t,\]
de modo que $B_k\in\mathcal C$, para todo $k$. Pelo resultado do Exercício~\ref{exe:disjuntar}, os conjuntos $B_k$ são
dois a dois disjuntos\footnote{%
Na verdade, o Exercício~\ref{exe:disjuntar} considera uma seqüência infinita de conjuntos, mas basta fazer
$A'_k=\emptyset$, para todo $k>t$.} e:
\[A=\bigcup_{k=1}^tA'_k=\bigcup_{k=1}^tB_k.\]
Daí, usando o fato que $\mu$ é finitamente aditiva e o resultado do item~(a), obtemos:
\[\mu(A)=\sum_{k=1}^t\mu(B_k)\le\sum_{k=1}^t\mu(A_k),\]
já que $B_k\subset A'_k\subset A_k$, para todo $k=1,\ldots,t$. Isso completa a demonstração do item~(e). A demonstração
do item~(f) é totalmente análoga, basta trocar $t$ por $\infty$ no argumento acima.
\end{proof}

Temos pouco interesse em estudar medidas em classes de conjuntos totalmente arbitrárias. Vamos então introduzir
algumas classes de conjuntos sobre as quais será interessante definir medidas. Recorde da Definição~\ref{thm:defalgsigmaalg}
(veja também Observação~\ref{thm:obsXnaalg}) que uma álgebra de partes de um conjunto $X$ é uma coleção de partes de $X$ que
inclui o próprio $X$ e que é fechada por união finita e complementação. Embora durante o estudo da teoria de integração
seja interessante assumir que o espaço $X$ subjacente a um espaço de medida $(X,\mathcal A,\mu)$ seja um conjunto
mensurável, quando desenvolvemos a teoria de construção de medidas é prático trabalhar também com medidas definidas
em classes de conjuntos $\mathcal C\subset\wp(X)$ que não incluem o espaço $X$ entre seus elementos.
Temos então a seguinte:
\begin{defin}
Seja $\mathcal R$ uma classe de conjuntos. Dizemos que $\mathcal R$ é um {\em anel\/}\index[indice]{anel} se $\mathcal R$ é não vazio
e se as seguintes condições são satisfeitas:
\begin{itemize}
\item[(a)] $A\setminus B\in\mathcal R$, para todos $A,B\in\mathcal R$;
\item[(b)] $A\cup B\in\mathcal R$, para todos $A,B\in\mathcal R$.
\end{itemize}
Dizemos que $\mathcal R$ é um {\em $\sigma$-anel\/}\index[indice]{sigma anel@$\sigma$-anel} se $\mathcal R$ é não vazio,
satisfaz a condição (a) acima e também a condição:
\begin{itemize}
\item[(b')] $\bigcup_{k=1}^\infty A_k\in\mathcal R$, para toda seqüência $(A_k)_{k\ge1}$ de elementos
de $\mathcal R$.
\end{itemize}
\end{defin}
Note que todo $\sigma$-anel é também um anel. De fato, se $\mathcal R$ é um $\sigma$-anel
e se $A,B\in\mathcal R$, podemos tomar $A_1=A$ e $A_k=B$ para todo $k\ge2$ na condição (b');
daí $A\cup B=\bigcup_{k=1}^\infty A_k\in\mathcal R$.

\begin{rem}
Se $\mathcal R$ é um anel (em particular, se $\mathcal R$ é um $\sigma$-anel) então o conjunto vazio $\emptyset$ é
um elemento de $\mathcal R$. De fato, como $\mathcal R$ é não vazio, existe um elemento $A\in\mathcal R$;
daí $\emptyset=A\setminus A\in\mathcal R$.
\end{rem}

Se $X$ é um conjunto e $\mathcal R\subset\wp(X)$ é uma coleção de partes de $X$ então é fácil ver que
$\mathcal R$ é uma álgebra (resp., uma $\sigma$-álgebra) de partes de $X$ se e somente se $\mathcal R$ é um anel
(resp., um $\sigma$-anel) tal que $X\in\mathcal R$ (veja Exercício~\ref{exe:anelalgebra}).

Temos o seguinte análogo do Lema~\ref{thm:propalgebras} para anéis e $\sigma$-anéis.
\begin{lem}\label{thm:propaneis}
Se $\mathcal R$ é um anel e se $A,B\in\mathcal R$ então $A\cap B\in\mathcal R$. Além do mais, se $\mathcal R$ é um
$\sigma$-anel e se $(A_k)_{k\ge1}$ é uma seqüência de elementos de $\mathcal R$
então $\bigcap_{k=1}^\infty A_k\in\mathcal R$.
\end{lem}
\begin{proof}
Seja $\mathcal R$ um anel e sejam $A,B\in\mathcal R$. Então os conjuntos $A\cup B$, $A\setminus B$, $B\setminus A$
estão todos em $\mathcal R$ e portanto:
\[A\cap B=(A\cup B)\setminus[(A\setminus B)\cup(B\setminus A)]\in\mathcal R.\]
Suponha agora que $\mathcal R$ é um $\sigma$-anel e seja $(A_k)_{k\ge1}$ uma seqüência de elementos de $\mathcal R$.
Então:
\[\bigcap_{k=1}^\infty A_k=A_1\setminus\Big(\bigcup_{k=1}^\infty(A_1\setminus A_k)\Big),\]
e portanto $\bigcap_{k=1}^\infty A_k\in\mathcal R$.
\end{proof}

Infelizmente, a classe dos intervalos da reta real não é um anel, pois não é fechada por uniões finitas. Para
incluir essa importante classe de conjuntos na nossa teoria, precisamos da seguinte:
\begin{defin}
Seja $\mathcal S$ uma classe de conjuntos. Dizemos que $\mathcal S$ é um {\em semi-anel\/}\index[indice]{semi anel@semi-anel}
se $\mathcal S$ é não vazio e se as seguintes condições são satisfeitas:
\begin{itemize}
\item[(a)] $A\cap B\in\mathcal S$, para todos $A,B\in\mathcal S$;
\item[(b)] se $A,B\in\mathcal S$ então existem $k\ge1$ e conjuntos $C_1,\ldots,C_k\in\mathcal S$, dois a dois disjuntos,
de modo que $A\setminus B=\bigcup_{i=1}^kC_i$.
\end{itemize}
\end{defin}
Segue diretamente do Lema~\ref{thm:propaneis} que todo anel é um semi-anel (note que se $A\setminus B\in\mathcal S$ então
podemos tomar $k=1$ e $C_1=A\setminus B$ na condição (b)).

\begin{rem}
Se $\mathcal S$ é um semi-anel então $\emptyset\in\mathcal S$. De fato, como $\mathcal S$ é não vazio, existe um elemento
$A\in\mathcal S$; daí existem $k\ge1$ e conjuntos $C_1,\ldots,C_k\in\mathcal S$ dois a dois disjuntos
de modo que $\emptyset=A\setminus A=\bigcup_{i=1}^kC_i$. Portanto $C_i=\emptyset$, para todo $i=1,\ldots,k$.
\end{rem}

Um semi-anel de subconjuntos de um conjunto $X$ que possui o próprio $X$ como elemento é às vezes chamado uma
{\em semi-álgebra\/}\index[indice]{semi algebra@semi-álgebra} de partes de $X$ (veja Exercício~\ref{exe:semialgebra}). Nós não teremos nenhum uso para
essa terminologia.

\begin{example}
O conjuntos de todos os intervalo da reta real (incluindo aí o vazio e os conjuntos unitários) é um semi-anel.
De fato, a interseção de dois intervalos é sempre um intervalo e a diferença de dois intervalos é ou um intervalo
ou uma união de dois intervalos disjuntos. Verifica-se facilmente também que a coleção:
\begin{equation}\label{eq:semiintervalos}
\mathcal S=\big\{\left]a,b\right]:a,b\in\R,\ a\le b\big\}\subset\wp(\R)
\end{equation}
é um semi-anel (note que $\left]a,b\right]=\emptyset$ para $a=b$).
\end{example}

\begin{notation}
Dadas classes de conjuntos $\mathcal C_1$ e $\mathcal C_2$ nós escrevemos:\index[simbolos]{$\mathcal C_1\Times\mathcal C_2$}
\[\mathcal C_1\Times\mathcal C_2=\big\{A_1\times A_2:A_1\in\mathcal C_1,\ A_2\in\mathcal C_2\big\}.\]
\end{notation}

\begin{lem}\label{thm:prodsemianel}
Se $\mathcal S_1$, $\mathcal S_2$ são semi-anéis então $\mathcal S_1\Times\mathcal S_2$
também é um semi-anel.
\end{lem}
\begin{proof}
Obviamente $\mathcal S_1\Times\mathcal S_2$ é não vazio, já que $\mathcal S_1$ e
$\mathcal S_2$ são não vazios. Dados
$A_1,B_1\in\mathcal S_1$ e $A_2,B_2\in\mathcal S_2$ então:
\[(A_1\times A_2)\cap(B_1\times B_2)=(A_1\cap B_1)\times(A_2\cap B_2);\]
como $A_1\cap B_1\in\mathcal S_1$, $A_2\cap B_2\in\mathcal S_2$, segue que
$(A_1\times A_2)\cap(B_1\times B_2)\in\mathcal S_1\Times\mathcal S_2$. Temos também:
\[(A_1\times A_2)\setminus(B_1\times B_2)=U_1\cup U_2\cup U_3,\]
onde:
\begin{gather*}
U_1=(A_1\setminus B_1)\times(A_2\cap B_2),\quad U_2=(A_1\cap B_1)\times(A_2\setminus B_2),\\
U_3=(A_1\setminus B_1)\times(A_2\setminus B_2).
\end{gather*}
Os conjuntos $U_i$, $i=1,2,3$ são dois a dois disjuntos; para completar a demonstração, basta ver que cada
$U_i$ é uma união finita disjunta de elementos de $\mathcal S_1\Times\mathcal S_2$. Como $\mathcal S_1$, $\mathcal S_2$ são semi-anéis, podemos
escrever:
\[A_1\setminus B_1=\bigcup_{i=1}^kC_i,\quad A_2\setminus B_2=\bigcup_{j=1}^lD_j,\]
com $C_1,\ldots,C_k\in\mathcal S_1$ dois a dois disjuntos e $D_1,\ldots,D_l\in\mathcal S_2$ dois a dois disjuntos.
Daí:
\begin{equation}\label{eq:Uis}
\begin{gathered}
U_1=\bigcup_{i=1}^k\big(C_i\times(A_2\cap B_2)\big),\quad
U_2=\bigcup_{j=1}^l\big((A_1\cap B_1)\times D_j\big),\\
U_3=\bigcup_{i=1}^k\bigcup_{j=1}^l(C_i\times D_j),
\end{gathered}
\end{equation}
onde $C_i\times(A_2\cap B_2)\in\mathcal S_1\Times\mathcal S_2$, $(A_1\cap B_1)\times D_j\in\mathcal S_1\Times\mathcal S_2$
e $C_i\times D_j\in\mathcal S_1\Times\mathcal S_2$, para todos $i=1,\ldots,k$, $j=1,\ldots,l$. Claramente as uniões em \eqref{eq:Uis}
são disjuntas e a demonstração está completa.
\end{proof}

\begin{cor}
Se $\mathcal S_1$, \dots, $\mathcal S_n$ são semi-anéis então a classe de conjuntos:
\[\mathcal S_1\Times\cdots\Times\mathcal S_n=\big\{A_1\times A_2\times\cdots\times A_n:A_1\in\mathcal S_1,\ldots,A_n\in\mathcal S_n\big\}\]
é um semi-anel.
\end{cor}
\begin{proof}
Segue diretamente do Lema~\ref{thm:prodsemianel} usando indução.
\end{proof}

Uma medida (ou uma medida finitamente aditiva) $\mu:\mathcal C\to[0,+\infty]$ numa classe de conjuntos $\mathcal C$
é dita {\em finita\/}\index[indice]{medida!finita} quando $\mu(A)<+\infty$, para todo $A\in\mathcal C$. Vamos agora determinar
as medidas finitamente aditivas finitas no semi-anel \eqref{eq:semiintervalos}.

Dizemos que uma função $F:I\to\overline\R$ definida num subconjunto $I$ de $\overline\R$ é
{\em crescente\/}\index[indice]{crescente!funcao@função}\index[indice]{funcao@função!crescente} (resp.,
{\em decrescente}\index[indice]{decrescente!funcao@função}\index[indice]{funcao@função!decrescente}) quando $F(x)\le F(y)$ (resp., $F(x)\ge F(y)$)
para todos $x,y\in I$ com $x\le y$. Dizemos que $F:I\to\overline\R$ é
{\em estritamente crescente\/}\index[indice]{estritamente crescente!funcao@função}\index[indice]{funcao@função!crescente!estritamente} (resp.,
{\em estritamente decrescente}\index[indice]{estritamente decrescente!funcao@função}\index[indice]{funcao@função!decrescente!estritamente})
quando $F(x)<F(y)$ (resp., $F(x)>F(y)$), para todos $x,y\in I$ com $x<y$.
\begin{prop}\label{thm:propmuF}
Seja $\mathcal S\subset\wp(\R)$ o semi-anel definido em \eqref{eq:semiintervalos}.
Se $F:\R\to\R$ é uma função crescente então a função $\mu_F:\mathcal S\to\left[0,+\infty\right[$ definida por:
\begin{equation}\label{eq:defmuF}
\mu_F\big(\left]a,b\right]\big)=F(b)-F(a),\index[simbolos]{$\mu_F$}
\end{equation}
para todos $a,b\in\R$ com $a\le b$ é uma medida finitamente aditiva finita em $\mathcal S$. Além do mais,
toda medida finitamente aditiva finita $\mu:\mathcal S\to\left[0,+\infty\right[$ em $\mathcal S$ é igual a $\mu_F$,
para alguma função crescente $F:\R\to\R$; se $F:\R\to\R$, $G:\R\to\R$ são funções crescentes então
$\mu_F=\mu_G$ se e somente se a função $F-G$ é constante.
\end{prop}
\begin{proof}
Todo elemento não vazio de $\mathcal S$ se escreve de modo único na forma $\left]a,b\right]$,
com $a,b\in\R$. O conjunto vazio é igual a $\left]a,a\right]$ para todo $a\in\R$; como $F(a)-F(a)=0$,
para todo $a\in\R$, segue que a função $\mu_F$ está de fato bem definida pela igualdade \eqref{eq:defmuF}.
Além do mais, o fato de $F$ ser crescente implica que $\mu_F$ toma valores em $\left[0,+\infty\right[$
e evidentemente $\mu_F(\emptyset)=0$. Sejam $a,b\in\R$, $a_i,b_i\in\R$, $i=1,\ldots,k$ com
$a\le b$, $a_i\le b_i$, $i=1,\ldots,k$,
\[\left]a,b\right]=\bigcup_{i=1}^k\left]a_i,b_i\right]\]
e suponha que os intervalos $\left]a_i,b_i\right]$, $i=1,\ldots,k$, sejam dois a dois disjuntos. Vamos mostrar que:
\begin{equation}\label{eq:tesepropmuF}
F(b)-F(a)=\sum_{i=1}^k\big(F(b_i)-F(a_i)\big).
\end{equation}
Se $\left]a,b\right]=\emptyset$ então $\left]a_i,b_i\right]=\emptyset$ para todo $i=1,\ldots,k$ e os dois lados
de \eqref{eq:tesepropmuF} são nulos. Suponha então que $a<b$. Podemos desconsiderar os índices $i$ tais que
$\left]a_i,b_i\right]=\emptyset$, pois isso não altera o lado direito de \eqref{eq:tesepropmuF}; suponha então
que $a_i<b_i$, para todo $i=1,\ldots,k$. Fazendo, se necessário, uma permutação nos índices $i$, podemos supor
que $a_1\le a_2\le\cdots\le a_k$. O Lema~\ref{thm:quaseobvio} que provaremos logo a seguir nos diz então que:
\[a=a_1<b_1=a_2<b_2=\cdots=a_i<b_i=\cdots=a_k<b_k=b,\]
donde a igualdade \eqref{eq:tesepropmuF} segue. Isso completa a demonstração do fato que $\mu_F$ é uma medida
finitamente aditiva finita em $\mathcal S$. Note que se $F:\R\to\R$, $G:\R\to\R$ são funções crescentes então
$\mu_F=\mu_G$ se e somente se:
\[F(b)-F(a)=\mu_F\big(\left]a,b\right]\big)=\mu_G\big(\left]a,b\right]\big)=G(b)-G(a),\]
para todos $a,b\in\R$ com $a\le b$; logo $\mu_F=\mu_G$ se e somente se:
\[F(b)-G(b)=F(a)-G(a),\]
para todos $a,b\in\R$ com $a\le b$. Isso prova que $\mu_F=\mu_G$ se e somente se $F-G$ é uma função constante.
Finalmente, seja $\mu:\mathcal S\to\left[0,+\infty\right[$ uma medida finitamente aditiva finita em $\mathcal S$
e vamos mostrar que existe uma função crescente $F:\R\to\R$ tal que $\mu=\mu_F$. Defina $F:\R\to\R$ fazendo:
\[F(x)=\begin{cases}
\hfil\mu\big(\left]0,x\right]\big),&\text{se $x\ge0$},\\
-\mu\big(\left]x,0\right]\big),&\text{se $x<0$}.
\end{cases}\]
Vamos mostrar que:
\begin{equation}\label{eq:muehmuF}
\mu\big(\left]a,b\right]\big)=F(b)-F(a),
\end{equation}
para todos $a,b\in\R$ com $a\le b$; seguirá então automaticamente que $F$ é crescente, já que $\mu\big(\left]a,b\right]\big)\ge0$,
para todos $a,b\in\R$. Em primeiro lugar, se $0\le a\le b$ então $\left]0,b\right]$ é igual à união disjunta
de $\left]0,a\right]$ com $\left]a,b\right]$; logo:
\[F(b)=\mu\big(\left]0,b\right]\big)=\mu\big(\left]0,a\right]\big)+\mu\big(\left]a,b\right]\big)=
F(a)+\mu\big(\left]a,b\right]\big),\]
donde \eqref{eq:muehmuF} é satisfeita. Similarmente, se $a\le b<0$, mostra-se \eqref{eq:muehmuF}
observando que $\left]a,0\right]$ é igual à união disjunta de $\left]a,b\right]$ com $\left]b,0\right]$.
Para completar a demonstração de \eqref{eq:muehmuF}, consideramos o caso em que $a<0\le b$; daí
$\left]a,b\right]$ é igual à união disjunta de $\left]a,0\right]$ com $\left]0,b\right]$, donde:
\[\mu\big(\left]a,b\right]\big)=\mu\big(\left]a,0\right]\big)+\mu\big(\left]0,b\right]\big)
=F(b)-F(a).\]
Isso completa a demonstração de \eqref{eq:muehmuF}. Concluímos então que $F$ é uma função crescente e que $\mu=\mu_F$.
\end{proof}

\begin{lem}\label{thm:quaseobvio}
Sejam $a_i,b_i\in\R$, $i=1,\ldots,k$, $a,b\in\R$ tais que $a<b$, $a_i<b_i$, $a_i\le a_{i+1}$, $i=1,\ldots,k$,
\[\left]a,b\right]=\bigcup_{i=1}^k\left]a_i,b_i\right]\]
e tais que os intervalos $\left]a_i,b_i\right]$, $i=1,\ldots,k$, sejam dois a dois disjuntos. Então
$a=a_1$, $b=b_k$ e $b_i=a_{i+1}$ para $i=1,\ldots,k-1$.
\end{lem}
\begin{proof}
Dividimos a demonstração em passos.

\begin{stepindent}
\item\label{itm:intquebrado1}
{\em $b_i\le a_{i+1}$, para $i=1,\ldots,k-1$}.

Seja $i=1,\ldots,k-1$ fixado e suponha por absurdo que $a_{i+1}<b_i$. Seja $c$ o mínimo
entre $b_i$ e $b_{i+1}$. Daí:
\[a_i\le a_{i+1}<c\le b_{i+1},\quad a_i<c\le b_i,\]
donde $c\in\left]a_i,b_i\right]\cap\left]a_{i+1},b_{i+1}\right]\ne\emptyset$, contradizendo nossas hipóteses.

\item {\em $b_i=a_{i+1}$, para $i=1,\ldots,k-1$}.

Seja $i=1,\ldots,k-1$ fixado. Temos $b_i\in\left]a_i,b_i\right]$, $b_{i+1}\in\left]a_{i+1},b_{i+1}\right]$,
donde $b_i,b_{i+1}\in\left]a,b\right]$; também:
\[a<b_i\le a_{i+1}<b_{i+1}\le b,\]
donde $a_{i+1}\in\left]a,b\right]$. Sabemos então que existe $j=1,\ldots,k$ tal que
$a_{i+1}\in\left]a_j,b_j\right]$. Se fosse $1\le j\le i-1$, teríamos:
\[b_j\le a_{j+1}\le a_i<b_i\le a_{i+1},\]
donde $a_{i+1}\not\in\left]a_j,b_j\right]$; por outro lado, se fosse $i+1\le j\le k$, teríamos
$a_{i+1}\le a_j$, donde novamente $a_{i+1}\not\in\left]a_j,b_j\right]$. Vemos então que a única
possibilidade é $j=i$, isto é, $a_{i+1}\in\left]a_i,b_i\right]$. Logo $a_{i+1}\le b_i$
e portanto, pelo passo~\ref{itm:intquebrado1}, $a_{i+1}=b_i$.

\item {\em $b_k=b$}.

Temos $b_k\in\left]a_k,b_k\right]$, donde $b_k\in\left]a,b\right]$ e $b_k\le b$. Por outro lado,
$b\in\left]a,b\right]$ implica $b\in\left]a_i,b_i\right]$, para algum $i=1,\ldots,k$. Se $i=k$ então
$b\le b_k$ e portanto $b_k=b$. Senão, $b\le b_i=a_{i+1}\le a_k<b_k$, contradizendo $b_k\le b$.

\item {\em $a_1=a$}.

Para todo $i=1,\ldots,k$, temos $a_1\le a_i$, donde $a_1\not\in\left]a_i,b_i\right]$ e portanto
$a_1\not\in\left]a,b\right]$; logo $a_1\le a$ ou $a_1>b$.
Como $a_1\le a_k<b_k=b$, vemos que $a_1\le a$. Suponha por absurdo que
$a_1<a$. Seja $c$ o mínimo entre $b_1$ e $a$; temos $a_1<c\le b_1$, donde $c\in\left]a_1,b_1\right]\subset\left]a,b\right]$
e $c>a$, o que nos dá uma contradição.\qedhere
\end{stepindent}
\end{proof}

Recorde da Definição~\ref{thm:defsigmagerada} que se $\mathcal C$ é uma coleção arbitrária de partes de um conjunto
$X$ então a $\sigma$-álgebra de partes de $X$ gerada por $\mathcal C$ é a menor $\sigma$-álgebra de partes de $X$
que contém $\mathcal C$. De forma totalmente análoga, podemos definir as noções de álgebra, anel e $\sigma$-anel
gerados por uma dada classe de conjuntos.
\begin{defin}\label{thm:defalgerada}
Se $X$ é um conjunto arbitrário e se $\mathcal C\subset\wp(X)$ é uma coleção arbitrária de partes de $X$ então
a {\em álgebra de partes de $X$ gerada por $\mathcal C$\/}\index[indice]{algebra@álgebra!gerada por uma colecao de conjuntos@gerada por uma coleção\hfil\break
de conjuntos} é a menor álgebra $\mathcal A$ de partes de $X$ que contém $\mathcal C$, i.e., $\mathcal A$ é uma
álgebra de partes de $X$ tal que:
\begin{enumerate}
\item\label{itm:algera1} $\mathcal C\subset\mathcal A$;
\item\label{itm:algera2} se $\mathcal A'$ é uma álgebra de partes de $X$ tal que $\mathcal C\subset\mathcal A'$ então
$\mathcal A\subset\mathcal A'$.
\end{enumerate}
Dizemos também que $\mathcal C$ é um
{\em conjunto de geradores\/}\index[indice]{conjunto!de geradores!para uma algebra@para uma álgebra}\index[indice]{geradores!para uma algebra@para uma álgebra}
para a álgebra $\mathcal A$.
\end{defin}
No Exercício~\ref{exe:algerada} pedimos ao leitor para justificar o fato de que a álgebra
gerada por uma coleção $\mathcal C\subset\wp(X)$ está de fato bem definida,
ou seja, existe uma única álgebra $\mathcal A$ satisfazendo as propriedades \eqref{itm:algera1} e \eqref{itm:algera2}
acima.

\begin{defin}\label{thm:anelgerado}
Se $\mathcal C$ é uma classe de conjuntos arbitrária então o {\em anel gerado por $\mathcal C$\/}\index[indice]{anel!gerado por uma colecao de conjuntos@gerado por uma coleção\hfil\break
de conjuntos} (resp., o {\em $\sigma$-anel gerado por $\mathcal C$}\index[indice]{sigma anel@$\sigma$-anel!gerado por uma colecao de conjuntos@gerado por uma coleção\hfil\break
de conjuntos}) é o menor anel (resp., $\sigma$-anel) $\mathcal R$ que contém $\mathcal C$, i.e., $\mathcal R$ é um anel
(resp., $\sigma$-anel) tal que:
\begin{enumerate}
\item\label{itm:anelger1} $\mathcal C\subset\mathcal R$;
\item\label{itm:anelger2} se $\mathcal R'$ é um anel (resp., $\sigma$-anel) tal que $\mathcal C\subset\mathcal R'$ então
$\mathcal R\subset\mathcal R'$.
\end{enumerate}
Dizemos também que $\mathcal C$ é um {\em conjunto de geradores\/}\index[indice]{conjunto!de geradores!para um anel}\index[indice]{geradores!para um anel}%
\index[indice]{conjunto!de geradores!para um sigma anel@para um $\sigma$-anel}\index[indice]{geradores!para um sigma anel@para um $\sigma$-anel}
para o anel (resp., $\sigma$-anel) $\mathcal R$.
\end{defin}
No Exercício~\ref{exe:anelgerado} pedimos ao leitor para justificar o fato de que o anel (resp., $\sigma$-anel)
gerado por uma classe de conjuntos $\mathcal C$ está de fato bem definido,
ou seja, existe um único anel (resp., $\sigma$-anel) $\mathcal R$ satisfazendo as propriedades \eqref{itm:anelger1}
e \eqref{itm:anelger2} acima.

É interessante observar que não é possível definir uma noção de semi-anel gerado por uma classe de conjuntos
(veja Exercício~\ref{exe:naosemianelgera}).

Dada uma medida $\mu$ num semi-anel $\mathcal S$, nós gostaríamos de estendê-la para o anel (e até para o $\sigma$-anel)
gerado por $\mathcal S$. Extensões de medidas para $\sigma$-anéis serão estudadas na Seção~\ref{sec:TeoExtend}.
No momento, nós mostraremos apenas como estender uma medida de um semi-anel para o anel gerado pelo mesmo.
Para isso, precisaremos entender melhor a estrutura do anel gerado por um dado semi-anel.

O próximo lema nos dá uma caracterização diferente para o conceito de anel.
\begin{lem}\label{thm:outradefanel}
Seja $\mathcal R$ uma classe de conjuntos não vazia. Então $\mathcal R$ é um anel se e somente se as seguintes condições
são satisfeitas:
\begin{itemize}
\item[(a)] $A\setminus B\in\mathcal R$, para todos $A,B\in\mathcal R$ tais que $B\subset A$;
\item[(b)] $A\cup B\in\mathcal R$, para todos $A,B\in\mathcal R$ com $A\cap B=\emptyset$;
\item[(c)] $A\cap B\in\mathcal R$, para todos $A,B\in\mathcal R$.
\end{itemize}
\end{lem}
\begin{proof}
Se $\mathcal R$ é um anel então as condições (a) e (b) são satisfeitas por definição e a condição (c) é satisfeita
pelo Lema~\ref{thm:propaneis}. Reciprocamente, suponha que $\mathcal R$ é uma classe de conjuntos não vazia
satisfazendo as condições (a), (b) e (c) acima. Dados $A,B\in\mathcal R$, devemos mostrar que $A\setminus B$
e $A\cup B$ estão em $\mathcal R$. Pela condição (c), temos que $A\cap B\in\mathcal R$; como:
\[A\setminus B=A\setminus(A\cap B),\]
e $A\cap B\subset A$, segue da condição (a) que $A\setminus B$ está em $\mathcal R$. Também, como:
\[A\cup B=(A\setminus B)\cup B,\]
e os conjuntos $A\setminus B\in\mathcal R$ e $B\in\mathcal R$ são disjuntos, segue da condição (b) que
$A\cup B\in\mathcal R$.
\end{proof}

\begin{lem}\label{thm:semigeraanel}
Seja $\mathcal S$ um semi-anel. O anel $\mathcal R$ gerado por $\mathcal S$ é igual ao conjunto das uniões
finitas disjuntas de elementos de $\mathcal S$, ou seja:
\[\mathcal R=\Big\{\bigcup_{k=1}^tA_k:\text{$A_1,\ldots,A_t\in\mathcal S$ dois a dois disjuntos},\ t\ge1\big\}.\]
\end{lem}
\begin{proof}
Sabemos que o anel gerado por $\mathcal S$ contém $\mathcal R$, já que o anel gerado por $\mathcal S$
é uma classe de conjuntos fechada por uniões finitas que contém $\mathcal S$. Para mostrar que $\mathcal R$ contém o anel
gerado por $\mathcal S$, é suficiente mostrar que $\mathcal R$ é um anel, já que obviamente $\mathcal R$ contém $\mathcal S$.
Para mostrar que $\mathcal R$ é um anel, usamos o Lema~\ref{thm:outradefanel}. É evidente que $\mathcal R$ satisfaz
a condição (b) do enunciado do Lema~\ref{thm:outradefanel}. Para ver que $\mathcal S$ é fechado por interseções finitas
(i.e., satisfaz a condição (c) do enunciado do Lema~\ref{thm:outradefanel}), sejam $A,B\in\mathcal R$ e escreva:
\[A=\bigcup_{k=1}^tA_k,\quad B=\bigcup_{l=1}^rB_l,\]
com $A_1,\ldots,A_t\in\mathcal S$ dois a dois disjuntos e $B_1,\ldots,B_r\in\mathcal S$ dois a dois disjuntos. Temos:
\[A\cap B=\bigcup_{k=1}^t\bigcup_{l=1}^r(A_k\cap B_l),\]
onde $A_k\cap B_l\in\mathcal S$ para todos $k=1,\ldots,t$, $l=1,\ldots,r$ e os conjuntos $A_k\cap B_l$ são dois a dois
disjuntos. Finalmente, mostraremos que $A\setminus B\in\mathcal R$, para todos $A,B\in\mathcal R$. Suponha primeiramente
que $B\in\mathcal S$; daí:
\begin{equation}\label{eq:SgeraR1}
A\setminus B=\bigcup_{k=1}^t(A_k\setminus B),
\end{equation}
onde $A=\bigcup_{k=1}^tA_k$ e $A_1,\ldots,A_t\in\mathcal S$ são dois a dois disjuntos. Como $A_k$ e $B$ estão em
$\mathcal S$, temos que $A_k\setminus B$ é uma união finita disjunta de elementos de $\mathcal S$, isto é,
$A_k\setminus B\in\mathcal R$, para todo $k=1,\ldots,t$. Como a união em \eqref{eq:SgeraR1} é disjunta, segue
que $A\setminus B\in\mathcal R$, já que $\mathcal R$ satisfaz a condição (b) do enunciado do Lema~\ref{thm:outradefanel}.
Finalmente, dados $A,B\in\mathcal R$ arbitrários, temos:
\[A\setminus B=\bigcap_{l=1}^r(A\setminus B_l),\]
onde $B=\bigcup_{l=1}^rB_l$ e $B_1,\ldots,B_l\in\mathcal S$ são dois a dois disjuntos. Como $A\in\mathcal R$ e
$B_l\in\mathcal S$, temos que $A\setminus B_l\in\mathcal R$, para todo $l=1,\ldots,r$, pelo que
acabamos de demonstrar; concluímos então que $A\setminus B\in\mathcal R$, já que $\mathcal R$ é fechado por
interseções finitas.
\end{proof}

\begin{cor}
Se $\mathcal S$ é um semi-anel então toda união finita de elementos de $\mathcal S$ é também igual a uma união
finita disjunta de (possivelmente outros) elementos de $\mathcal S$; em particular, o anel gerado por $\mathcal S$
coincide também com o conjunto das uniões finitas (não necessariamente disjuntas) de elementos de $\mathcal S$.
\end{cor}
\begin{proof}
Toda união finita de elementos de $\mathcal S$ pertence ao anel $\mathcal R$ gerado por $\mathcal S$; mas,
pelo Lema~\ref{thm:semigeraanel}, todo elemento de $\mathcal R$ é igual a uma união finita disjunta de elementos
de $\mathcal S$.
\end{proof}

Estamos agora em condições de prova o seguinte:
\begin{teo}[pequeno teorema da extensão]
\label{thm:peqteoext}\index[indice]{teorema!da extensao@da extensão!pequeno}\index[indice]{pequeno!teorema da extensao@teorema da extensão}
Seja $\mu:\mathcal S\to[0,+\infty]$ uma medida finitamente aditiva num semi-anel $\mathcal S$ e seja
$\mathcal R$ o anel gerado por $\mathcal S$. Então:
\begin{itemize}
\item[(a)] existe uma única medida finitamente aditiva $\tilde\mu:\mathcal R\to[0,+\infty]$ em $\mathcal R$
tal que $\tilde\mu\vert_{\mathcal S}=\mu$;
\item[(b)] $\mu$ é uma medida $\sigma$-aditiva em $\mathcal S$ se e somente se $\tilde\mu$ é uma medida $\sigma$-aditiva
em $\mathcal R$.
\end{itemize}
\end{teo}
\begin{proof}
Pelo Lema~\ref{thm:semigeraanel}, todo $A\in\mathcal R$ se escreve na forma $A=\bigcup_{k=1}^tA_k$,
com $A_1,\ldots,A_t\in\mathcal S$ dois a dois disjuntos. Se $\tilde\mu$ é uma medida finitamente aditiva
em $\mathcal R$ que estende $\mu$ então obrigatoriamente:
\begin{equation}\label{eq:deftildemu}
\tilde\mu(A)=\sum_{k=1}^t\mu(A_k),
\end{equation}
o que prova a unicidade de $\tilde\mu$. Para provar a existência de $\tilde\mu$, usamos a igualdade \eqref{eq:deftildemu}
para definir $\tilde\mu$, onde $A=\bigcup_{k=1}^tA_k$ e $A_1,\ldots,A_t\in\mathcal S$ são dois a dois disjuntos.
Precisamos, no entanto, mostrar primeiramente que $\tilde\mu$ está bem definida, já que é possível que:
\[A=\bigcup_{k=1}^tA_k=\bigcup_{l=1}^rA'_l,\]
com $A_1,\ldots,A_t\in\mathcal S$ dois a dois disjuntos e $A'_1,\ldots,A'_r\in\mathcal S$ dois a dois
disjuntos. Nesse caso, devemos verificar que:
\begin{equation}\label{eq:peqteoext3}
\sum_{k=1}^t\mu(A_k)=\sum_{l=1}^r\mu(A'_l).
\end{equation}
Note que, para todo $k=1,\ldots,t$, temos:
\[A_k=A_k\cap A=\bigcup_{l=1}^r(A_k\cap A'_l),\]
onde $A_k\cap A'_l\in\mathcal S$, para $l=1,\ldots,r$ e os conjuntos $A_k\cap A'_l$ são dois a dois disjuntos.
Como também $A_k\in\mathcal S$, o fato de $\mu$ ser uma medida finitamente aditiva em $\mathcal S$ implica que:
\begin{equation}\label{eq:peqteoext1}
\mu(A_k)=\sum_{l=1}^r\mu(A_k\cap A'_l).
\end{equation}
De maneira análoga, vemos que:
\begin{equation}\label{eq:peqteoext2}
\mu(A'_l)=\sum_{k=1}^t\mu(A'_l\cap A_k),
\end{equation}
para todo $l=1,\ldots,r$. De \eqref{eq:peqteoext1} e \eqref{eq:peqteoext2} vem:
\[\sum_{k=1}^t\mu(A_k)=\sum_{k=1}^t\sum_{l=1}^r\mu(A_k\cap A'_l)=\sum_{l=1}^r\sum_{k=1}^t\mu(A'_l\cap A_k)
=\sum_{l=1}^r\mu(A'_l),\]
o que prova \eqref{eq:peqteoext3}. Logo $\tilde\mu$ está bem definida. Devemos verificar agora que $\tilde\mu$
é uma medida finitamente aditiva em $\mathcal R$. É óbvio que $\tilde\mu\vert_{\mathcal S}=\mu$ e em particular
$\tilde\mu(\emptyset)=0$. Como $\mathcal R$ é fechado por uniões finitas, é suficiente demonstrar que:
\[\tilde\mu(A\cup B)=\tilde\mu(A)+\tilde\mu(B),\]
para todos $A,B\in\mathcal R$ com $A\cap B=\emptyset$ (veja Observação~\ref{thm:rembastadois}).
Dados $A,B\in\mathcal R$ com $A\cap B=\emptyset$, escrevemos:
\[A=\bigcup_{k=1}^tA_k,\quad B=\bigcup_{l=1}^rB_l,\]
com $A_1,\ldots,A_t\in\mathcal S$ dois a dois disjuntos e $B_1,\ldots,B_r\in\mathcal S$ dois a dois disjuntos.
Daí $A\cup B$ é união disjunta de $A_1,\ldots,A_t,B_1,\ldots,B_r\in\mathcal S$ e portanto:
\[\tilde\mu(A\cup B)=\sum_{k=1}^t\mu(A_k)+\sum_{l=1}^r\mu(B_l)=\tilde\mu(A)+\tilde\mu(B).\]
Isso completa a demonstração de que $\tilde\mu$ é uma medida finitamente aditiva. Note que se $\tilde\mu$
é uma medida $\sigma$-aditiva então obviamente $\mu$ também é uma medida $\sigma$-aditiva, já que $\mu$ é
apenas uma restrição de $\tilde\mu$. Suponha então que $\mu$ é uma medida $\sigma$-aditiva e vamos mostrar
que $\tilde\mu$ também é. Seja $(A_k)_{k\ge1}$ uma seqüência de elementos dois a dois disjuntos de $\mathcal R$
e suponha que $A=\bigcup_{k=1}^\infty A_k\in\mathcal R$; devemos mostrar que:
\begin{equation}\label{eq:tildemusigmadd}
\tilde\mu(A)=\sum_{k=1}^\infty\tilde\mu(A_k).
\end{equation}
Suponha inicialmente que $A\in\mathcal S$. Cada $A_k\in\mathcal R$ pode ser escrito na forma:
\[A_k=\bigcup_{u=1}^{r_k}A_{ku},\]
com $A_{k1},\ldots,A_{kr_k}\in\mathcal S$ dois a dois disjuntos. Daí $A\in\mathcal S$ é igual à união
disjunta dos conjuntos $A_{ku}$, $k\ge1$, $u=1,\ldots,r_k$, sendo que todos os $A_{ku}$ estão em $\mathcal S$;
como $\mu$ é uma medida $\sigma$-aditiva em $\mathcal S$, segue que:
\[\tilde\mu(A)=\mu(A)=\sum_{k=1}^\infty\sum_{u=1}^{r_k}\mu(A_{ku}).\]
Mas, pela definição de $\tilde\mu$, temos $\tilde\mu(A_k)=\sum_{u=1}^{r_k}\mu(A_{ku})$, e portanto a igualdade
\eqref{eq:tildemusigmadd} fica demonstrada no caso em que $A\in\mathcal S$. Vamos agora ao caso geral;
como $A\in\mathcal R$, é possível escrever $A$ na forma:
\[A=\bigcup_{l=1}^rB_l,\]
com $B_1,\ldots,B_r\in\mathcal S$ dois a dois disjuntos. Temos então:
\[A_k=A_k\cap A=\bigcup_{l=1}^r(A_k\cap B_l),\quad B_l=B_l\cap A=\bigcup_{k=1}^\infty(B_l\cap A_k),\]
para todo $k\ge1$ e todo $l=1,\ldots,r$. Usando respectivemente o fato que $\tilde\mu$ é uma medida
finitamente aditiva e a parte da $\sigma$-aditividade de $\tilde\mu$ que já foi demonstrada, obtemos:
\[\tilde\mu(A_k)=\sum_{l=1}^r\tilde\mu(A_k\cap B_l),\quad\tilde\mu(B_l)=\sum_{k=1}^\infty\tilde\mu(B_l\cap A_k).\]
Então:
\[\tilde\mu(A)=\sum_{l=1}^r\tilde\mu(B_l)=\sum_{l=1}^r\sum_{k=1}^\infty\tilde\mu(B_l\cap A_k)=
\sum_{k=1}^\infty\sum_{l=1}^r\tilde\mu(A_k\cap B_l)=\sum_{k=1}^\infty\tilde\mu(A_k),\]
o que prova \eqref{eq:tildemusigmadd} e completa a demonstração.
\end{proof}

\begin{cor}\label{thm:cortambemnosemianel}
As afirmações que aparecem nos itens (a), (c), (d), (e) e (f) do enunciado do Lema~\ref{thm:classerazoavel}
são verdadeiras sob a hipótese de que a classe de conjuntos $\mathcal C$ seja um semi-anel;
a afirmação que aparece no item (b) também é verdadeira, sob a hipótese de que $A_2\setminus A_1$
esteja em $\mathcal C$.
\end{cor}
\begin{proof}
Seja $\mathcal R$ o anel gerado pelo semi-anel $\mathcal C$ e seja $\tilde\mu$
a medida finitamente aditiva em $\mathcal R$ que estende $\mu:\mathcal C\to[0,+\infty]$. Como o anel $\mathcal R$
é fechado por diferenças, o Lema~\ref{thm:classerazoavel} pode ser aplicado
à medida finitamente aditiva $\tilde\mu$. A conclusão segue.
\end{proof}

\begin{example}\label{exa:muFsigmaimplFcontdir}
Seja $F:\R\to\R$ uma função crescente e considere a medida finitamente aditiva
finita $\mu_F:\mathcal S\to\left[0,+\infty\right[$ correspondente a $F$ definida
no enunciado da Proposição~\ref{thm:propmuF}. Vamos determinar uma condição
necessária sobre $F$ para que $\mu_F$ seja uma medida $\sigma$-aditiva.
Note em primeiro lugar que, como a função $F$ é crescente, então para todo $a\in\R$
o limite à direita:\index[indice]{funcao@função!continua@contínua!a direita@à direita}%
\index[indice]{continuidade!a direita@à direita}\index[indice]{limite!a direita@à direita}\index[simbolos]{$F(a^+)$}
\[F(a^+)\stackrel{\text{def}}=\lim_{x\to a^+}F(x)\in\R\]
existe e é maior ou igual a $F(a)$.
Suponha que $\mu_F$ seja uma medida $\sigma$-aditiva e seja $a\in\R$ fixado.
Nós temos:
\[\left]a,a+\tfrac1n\right]\searrow\emptyset;\]
pelo Corolário~\ref{thm:cortambemnosemianel}, isso nos dá:
\[0=\mu_F(\emptyset)=\lim_{n\to\infty}\mu_F\big(\left]a,a+\tfrac1n\right]\!\big)
=\lim_{n\to\infty}\big[F\big(a+\tfrac1n\big)-F(a)\big]=F(a^+)-F(a).\]
Logo $F(a)=F(a^+)$ e portanto $F$ é contínua à direita. Na verdade, nós veremos
adiante na Proposição~\ref{thm:carmedLebSti} que $\mu_F$ é uma medida $\sigma$-aditiva se e somente
se a função crescente $F$ é contínua à direita.
\end{example}

\begin{subsection}{O critério da classe compacta}

Nós vimos no Teorema~\ref{thm:peqteoext} que toda medida finitamente aditiva $\mu$
num semi-anel $\mathcal S$ estende-se de modo único a uma medida finitamente aditiva $\tilde\mu$
no anel gerado por $\mathcal S$; a extensão $\tilde\mu$ é uma medida $\sigma$-aditiva se e
somente se $\mu$ o for. Enquanto é muitas vezes possível mostrar por técnicas
elementares que uma função $\mu:\mathcal S\to[0,+\infty]$ é uma medida finitamente
aditiva (veja, por exemplo, a demonstração da Proposição~\ref{thm:propmuF}), a situação
não é tão simples quando se quer provar a $\sigma$-aditividade\footnote{%
Uma análise ingênua da situação poderia levar a crer que, sob hipóteses adequadas para a função $F$,
poder-se-ia provar a $\sigma$-aditividade de $\mu_F$ na Proposição~\ref{thm:propmuF}
utilizando alguma versão do Lema~\ref{thm:quaseobvio} para seqüências infinitas de intervalos
$\left]a_i,b_i\right]$. A situação não é tão simples, como mostra o Exercício~\ref{exe:poiseh}.}
de $\mu$. Nesta subseção nós provaremos um critério prático para verificação da $\sigma$-aditividade
de uma medida finitamente aditiva num semi-anel. Como corolário, nós determinaremos exatamente
quais são as funções crescentes $F:\R\to\R$ para as quais a medida finitamente aditiva $\mu_F$
é uma medida $\sigma$-aditiva.

Precisamos da seguinte:
\begin{defin}
Uma classe de conjuntos $\mathcal C$ é dita
{\em compacta\/}\index[indice]{compacta!classe de conjuntos}\index[indice]{classe!compacta} quando para toda seqüência
$(C_k)_{k\ge1}$ em $\mathcal C$ com $\bigcap_{k=1}^\infty C_k=\emptyset$ existe $t\ge1$
tal que $\bigcap_{k=1}^tC_k=\emptyset$.
\end{defin}

\begin{example}\label{exa:classedecompactos}
Se $\mathcal C$ é uma classe arbitrária de subconjuntos compactos de $\R^n$ então $\mathcal C$ é uma
classe compacta. De fato, seja $(C_k)_{k\ge1}$ uma seqüência em $\mathcal C$ com
$\bigcap_{k=1}^\infty C_k=\emptyset$. Temos:
\[\R^n=\bigcup_{k=1}^\infty C_k^\compl,\]
e em particular os conjuntos $(C_k^\compl)_{k\ge1}$ constituem uma cobertura aberta
do compacto $C_1$. Logo existem $t_1,\ldots,t_r\ge1$ tais que:
\[C_1\subset C_{t_1}^\compl\cup\ldots\cup C_{t_r}^\compl.\]
Tomando $t=\max\{t_1,\ldots,t_r\}$ então:
\[C_1\cap\ldots\cap C_t\subset C_1\cap C_{t_1}\cap\ldots\cap C_{t_r}=\emptyset.\]
\end{example}

\begin{prop}[critério da classe compacta]\index[indice]{criterio@critério!da classe compacta}%
\index[indice]{compacta!criterio da classe@critério da classe}\index[indice]{classe!compacta!criterio da@critério da}%
\label{thm:critcompclass}
Seja $\mu:\mathcal S\to[0,+\infty]$ uma medida finitamente aditiva num semi-anel $\mathcal S$.
Suponha que existe uma classe compacta $\mathcal C$ tal que para todo $A\in\mathcal S$
e para todo $\varepsilon>0$ existem $B\in\mathcal S$, $C\in\mathcal C$ tais que
$B\subset C\subset A$ e:
\begin{equation}\label{eq:critcompclass}
\mu(A)<\mu(B)+\varepsilon.
\end{equation}
Então $\mu$ é uma medida $\sigma$-aditiva.
\end{prop}
Note que as hipóteses da Proposição~\ref{thm:critcompclass} implicam em particular
que a medida finitamente aditiva $\mu$ é finita; de fato, a desigualdade \eqref{eq:critcompclass}
implica que $\mu(A)<+\infty$.

Antes de demonstrar a Proposição~\ref{thm:critcompclass}, precisamos de alguns resultados
preparatórios. O próximo lema nos dá um critério para a $\sigma$-aditividade de medidas
finitamente aditivas em anéis.
\begin{lem}\label{thm:critcontsupvazio}
Seja $\mu:\mathcal R\to[0,+\infty]$ uma medida finitamente aditiva num anel $\mathcal R$.
Suponha que para toda seqüência $(A_k)_{k\ge1}$ em $\mathcal R$ tal que $A_k\searrow\emptyset$
temos $\lim_{k\to\infty}\mu(A_k)=0$. Então $\mu$ é uma medida $\sigma$-aditiva em $\mathcal R$.
\end{lem}
\begin{proof}
Seja $(B_k)_{k\ge1}$ uma seqüência de elementos dois a dois disjuntos de $\mathcal R$
tal que a união $B=\bigcup_{k=1}^\infty B_k$ está em $\mathcal R$; vamos mostrar
que $\mu(B)=\sum_{k=1}^\infty\mu(B_k)$. Para cada $k\ge1$, seja
$A_k=\bigcup_{i=k+1}^\infty B_i$; temos:
\begin{equation}\label{eq:BseAs}
B=B_1\cup\ldots\cup B_k\cup A_k,
\end{equation}
onde a união em \eqref{eq:BseAs} é disjunta. Note que cada $A_k$ está em $\mathcal R$, já que
$A_k=B\setminus(B_1\cup\ldots\cup B_k)$. Como $\mu$ é uma medida finitamente aditiva
em $\mathcal R$, obtemos:
\begin{equation}\label{eq:BseAs2}
\mu(B)=\mu(A_k)+\sum_{i=1}^k\mu(B_i).
\end{equation}
Claramente $A_k\searrow\emptyset$ e portanto $\lim_{k\to\infty}\mu(A_k)=0$; a conclusão
é obtida fazendo $k\to\infty$ em \eqref{eq:BseAs2}.
\end{proof}

\begin{cor}\label{thm:critcompclassanel}
A Proposição~\ref{thm:critcompclass} é verdadeira sob a hipótese adicional de que
$\mathcal S$ seja um anel.
\end{cor}
\begin{proof}
Já que $\mathcal S$ é um anel, podemos usar o Lema~\ref{thm:critcontsupvazio} para
estabelecer o fato de que $\mu$ é uma medida $\sigma$-aditiva. Seja $(A_k)_{k\ge1}$
uma seqüência em $\mathcal S$ tal que $A_k\searrow\emptyset$. Vamos mostrar que:
\begin{equation}\label{eq:limAkzero}
\lim_{k\to\infty}\mu(A_k)=0.
\end{equation}
Seja dado $\varepsilon>0$ e para cada $k\ge1$ sejam
$B_k\in\mathcal S$, $C_k\in\mathcal C$ tais que:
\[B_k\subset C_k\subset A_k,\quad\mu(A_k)<\mu(B_k)+\frac\varepsilon{2^k}.\]
Temos:
\[\bigcap_{k=1}^\infty C_k\subset\bigcap_{k=1}^\infty A_k=\emptyset,\]
e portanto existe $t\ge1$ tal que $\bigcap_{k=1}^tC_k=\emptyset$; daí $\bigcap_{k=1}^tB_k=\emptyset$
e portanto:
\[A_t=\bigcup_{k=1}^t(A_t\setminus B_k)\subset\bigcup_{k=1}^t(A_k\setminus B_k).\]
Usando o item~(e) do Lema~\ref{thm:classerazoavel}, obtemos:
\[\mu(A_t)\le\sum_{k=1}^t\mu(A_k\setminus B_k)=\sum_{k=1}^t\big(\mu(A_k)-\mu(B_k)\big)
<\sum_{k=1}^t\frac\varepsilon{2^k}<\sum_{k=1}^\infty\frac\varepsilon{2^k}=\varepsilon.\]
Logo $\mu(A_k)\le\mu(A_t)<\varepsilon$, para todo $k\ge t$, o que prova \eqref{eq:limAkzero}
e completa a demonstração.
\end{proof}

Para demonstrar a Proposição~\ref{thm:critcompclass} nós consideraremos a medida finitamente
aditiva $\tilde\mu$ que estende $\mu$ para o anel $\mathcal R$ gerado por $\mathcal S$
e nós usaremos uma classe compacta $\widetilde{\mathcal C}$ de modo que $\tilde\mu$
e $\widetilde{\mathcal C}$ satisfaçam as hipóteses da Proposição~\ref{thm:critcompclass}.
\begin{lem}\label{thm:Cftambemcompacta}
Seja $\mathcal C$ uma classe compacta e seja $\widetilde{\mathcal C}$ a classe formada por
todas as uniões finitas de elementos de $\mathcal C$, isto é:
\[\widetilde{\mathcal C}=\Big\{\bigcup_{k=1}^tC_k:C_1,\ldots,C_t\in\mathcal C,\ t\ge1\Big\}.\]
Então $\widetilde{\mathcal C}$ também é uma classe compacta.
\end{lem}
\begin{proof}
Seja $(\widetilde C_k)_{k\ge1}$ uma seqüência de elementos de $\widetilde{\mathcal C}$
tal que $\bigcap_{k=1}^\infty\widetilde C_k=\emptyset$; devemos mostrar que existe $t\ge1$ tal que
$\bigcap_{k=1}^t\widetilde C_k=\emptyset$. Suponha por absurdo que $\bigcap_{k=1}^t\widetilde C_k\ne\emptyset$,
para todo $t\ge1$. Para cada $k\ge1$ escrevemos:
\[\widetilde C_k=\bigcup_{i\in I_k}C_{ki},\]
onde $I_k$ é um conjunto finito não vazio e $C_{ki}\in\mathcal C$, para todo $i\in I_k$.
Para cada $t\ge1$, existe $x_t\in\bigcap_{k=1}^t\widetilde C_k$ e portanto
para cada $k=1,\ldots,t$, temos $x_t\in\widetilde C_k$; daí existe um índice
$i^t_k\in I_k$ tal que $x_t\in C_{ki^t_k}$. Em particular:
\begin{equation}\label{eq:xtestala}
x_t\in\bigcap_{k=1}^tC_{ki^t_k}\ne\emptyset.
\end{equation}
Nós vamos construir uma seqüência $(j_k)_{k\ge1}\in\prod_{k=1}^\infty I_k$ tal que:
\[\bigcap_{k=1}^\infty C_{kj_k}\ne\emptyset;\]
uma vez que essa seqüência esteja construída, teremos:
\begin{equation}\label{eq:classecomcontr}
\bigcap_{k=1}^\infty C_{kj_k}\subset\bigcap_{k=1}^\infty\widetilde C_k\ne\emptyset,
\end{equation}
o que nos dará uma contradição e completará a demonstração. Nosso plano é construir
indutivamente uma seqüência $(j_k)_{k\ge1}\in\prod_{k=1}^\infty I_k$ tal que para todo
$k\ge1$, existam uma infinidade de índices $t\ge k$ tais que:
\begin{equation}\label{eq:forinfmanyt}
(j_1,\ldots,j_k)=(i^t_1,\ldots,i^t_k).
\end{equation}
Em primeiro lugar, como $(i^t_1)_{t\ge1}$ é uma seqüência de elementos do conjunto
finito $I_1$, deve existir $j_1\in I_1$ tal que $j_1=i^t_1$ para uma infinidade de índices
$t\ge1$. Suponha que tenhamos construído $(j_1,\ldots,j_k)\in I_1\times\cdots\times I_k$
de modo que a igualdade \eqref{eq:forinfmanyt} é válida para uma infinidade de índices $t\ge k$.
Se $T$ é o conjunto infinito constituído pelos índices $t\ge k+1$ tais que a igualdade \eqref{eq:forinfmanyt}
é satisfeita então $(i^t_{k+1})_{t\in T}$ é uma família infinita de elementos do conjunto
finito $I_{k+1}$ e portanto existe $j_{k+1}\in I_{k+1}$ tal que $j_{k+1}=i^t_{k+1}$,
para uma infinidade de índices $t\in T$; daí:
\[(j_1,\ldots,j_k,j_{k+1})=(i^t_1,\ldots,i^t_k,i^t_{k+1}),\]
para uma infinidade de índices $t\ge k+1$. Nós obtivemos então uma seqüência
$(j_k)_{k\ge1}\in\prod_{k=1}^\infty I_k$ tal que para todo $k\ge1$ a igualdade
\eqref{eq:forinfmanyt} é satisfeita para uma infinidade de índices $t\ge k$;
em particular, para todo $k\ge1$ existe $t\ge k$ tal que a igualdade \eqref{eq:forinfmanyt}
é satisfeita e daí, usando \eqref{eq:xtestala}, obtemos:
\[\bigcap_{r=1}^kC_{rj_r}=\bigcap_{r=1}^kC_{ri^t_r}\supset\bigcap_{r=1}^tC_{ri^t_r}\ne\emptyset.\]
Como $\mathcal C$ é uma classe compacta e $\bigcap_{r=1}^kC_{rj_r}\ne\emptyset$ para todo
$k\ge1$, segue que $\bigcap_{r=1}^\infty C_{rj_r}\ne\emptyset$. Obtivemos então \eqref{eq:classecomcontr},
o que nos dá uma contradição e completa a demonstração.
\end{proof}

\begin{proof}[Demonstração da Proposição~\ref{thm:critcompclass}]
Seja $\mathcal R$ o anel gerado por $\mathcal S$ e $\tilde\mu:\mathcal R\to[0,+\infty]$
a única medida finitamente aditiva em $\mathcal R$ que estende $\mu$ (veja Teorema~\ref{thm:peqteoext});
nós vamos mostrar que $\tilde\mu$ é uma medida $\sigma$-aditiva em $\mathcal R$ e isso completará
a demonstração. Seja $\widetilde{\mathcal C}$ a classe compacta definida no enunciado
do Lema~\ref{thm:Cftambemcompacta}; vamos mostrar que para todo $A\in\mathcal R$ e para
todo $\varepsilon>0$ existem $B\in\mathcal R$, $C\in\widetilde{\mathcal C}$ tais que
$B\subset C\subset A$ e $\tilde\mu(A)<\tilde\mu(B)+\varepsilon$. Seguirá então do
Corolário~\ref{thm:critcompclassanel} que $\tilde\mu$ é uma medida $\sigma$-aditiva.
Pelo Lema~\ref{thm:semigeraanel}, podemos escrever $A=\bigcup_{k=1}^tA_k$, com
$A_1,\ldots,A_t\in\mathcal S$ dois a dois disjuntos e $t\ge1$. Para cada $k=1,\ldots,t$,
existem $B_k\in\mathcal S$ e $C_k\in\mathcal C$ com $B_k\subset C_k\subset A_k$
e $\mu(A_k)<\mu(B_k)+\frac\varepsilon t$. Tomando $B=\bigcup_{k=1}^tB_k\in\mathcal R$
e $C=\bigcup_{k=1}^tC_k\in\widetilde{\mathcal C}$ então $B\subset C\subset A$ e:
\[\tilde\mu(A)=\sum_{k=1}^t\mu(A_k)<\sum_{k=1}^t\big(\mu(B_k)+\tfrac\varepsilon t\big)
=\tilde\mu(B)+\varepsilon,\]
já que os conjuntos $B_1,\ldots,B_t\in\mathcal S$ são dois a dois disjuntos. Isso completa
a demonstração.
\end{proof}

Como aplicação da Proposição~\ref{thm:critcompclass}, nós vamos determinar
para quais funções crescentes $F:\R\to\R$ a medida finitamente aditiva correspondente
$\mu_F$ é $\sigma$-aditiva.
\begin{prop}\label{thm:carmedLebSti}
Seja $F:\R\to\R$ uma função crescente e seja $\mu_F:\mathcal S\to\left[0,+\infty\right[$
a medida finitamente aditiva finita correspondente a $F$ definida no enunciado da Proposição~\ref{thm:propmuF}.
Então $\mu_F$ é uma medida $\sigma$-aditiva se e somente se a função $F$ é contínua à direita.
\end{prop}
\begin{proof}
Nós já vimos no Exemplo~\ref{exa:muFsigmaimplFcontdir} que se $\mu_F$ é uma medida
$\sigma$-aditiva então a função $F$ é contínua à direita.
Reciprocamente, suponha que $F$ é contínua à direita e vamos demonstrar que a medida
finitamente aditiva $\mu_F$ é uma medida $\sigma$-aditiva. Seja:
\[\mathcal C=\{\emptyset\}\cup\big\{[a,b]:a,b\in\R,\ a\le b\big\};\]
pelo Exemplo~\ref{exa:classedecompactos}, $\mathcal C$ é uma classe compacta.
Vamos verificar as hipóteses da Proposição~\ref{thm:critcompclass}. Sejam
dados $A\in\mathcal S$ e $\varepsilon>0$. Se $A=\emptyset$, tomamos
$B=C=\emptyset$. Se $A\ne\emptyset$ então $A=\left]a,b\right]$ com $a,b\in\R$, $a<b$
e, como $F$ é contínua à direita no ponto $a$, existe $\delta>0$ tal que
$a<a+\delta<b$ e $F(a+\delta)<F(a)+\varepsilon$. Tomamos então
$B=\left]a+\delta,b\right]\in\mathcal S$, $C=[a+\delta,b]\in\mathcal C$, de modo
que $B\subset C\subset A$ e:
\[\mu_F(A)=F(b)-F(a)<F(b)-F(a+\delta)+\varepsilon=\mu_F(B)+\varepsilon.\qedhere\]
\end{proof}

\end{subsection}

\end{section}

\begin{section}{Classes Monotônicas e Classes ${\sigma}$-aditivas}

Seja $\mathcal C$ uma classe de conjuntos e $\mathcal A$ o $\sigma$-anel gerado por $\mathcal C$.
Digamos que nós saibamos que todo elemento de $\mathcal C$ satisfaz uma certa propriedade $\mathbb P$
e que nós gostaríamos de verificar que todo elemento de $\mathcal A$ também satisfaz $\mathbb P$.
Em geral pode ser inviável apresentar uma descrição concreta dos elementos de $\mathcal A$ a partir
dos elementos de $\mathcal C$ (compare a complexidade do $\sigma$-anel dos Boreleanos da reta com
a simplicidade da classe dos intervalos da reta, que é um conjunto de geradores para os Boreleanos).
Uma possibilidade seria mostrar que a classe de todos os conjuntos que satisfazem a propriedade $\mathbb P$ é um
$\sigma$-anel; isso implicaria imediatamente que tal classe deve conter $\mathcal A$. No entanto,
existem situações concretas em que é difícil verificar que a classe dos conjuntos que satisfazem a propriedade
$\mathbb P$ é fechada por uniões enumeráveis e diferenças, mas é simples verificar que tal classe é fechada por
operações como uniões (finitas ou enumeráveis) disjuntas ou diferenças próprias. O objetivo desta seção
é o de provar resultados do seguinte tipo: se a classe de conjuntos $\mathcal C$ satisfaz certas hipóteses
e se a classe de conjuntos que satisfaz a propriedade $\mathbb P$ é fechada por certas operações então
todo elemento do $\sigma$-anel $\mathcal A$ gerado por $\mathcal C$ satisfaz a propriedade $\mathbb P$.

Começamos com a seguinte:
\begin{defin}
Seja $\mathcal E$ uma classe de conjuntos. Dizemos que $\mathcal E$ é uma
{\em classe monotônica\/}\index[indice]{classe!monotonica@monotônica}\index[indice]{monotonica@monotônica!classe}
se $\emptyset\in\mathcal E$ e se $\mathcal E$ satisfaz as seguintes condições:
\begin{itemize}
\item se $(A_k)_{k\ge1}$ é uma seqüência em $\mathcal E$ e $A_k\nearrow A$ então $A\in\mathcal E$;
\item se $(A_k)_{k\ge1}$ é uma seqüência em $\mathcal E$ e $A_k\searrow A$ então $A\in\mathcal E$.
\end{itemize}
Dizemos que $\mathcal E$ é uma
{\em classe $\sigma$-aditiva\/}\index[indice]{classe!sigma aditiva@$\sigma$-aditiva}\index[indice]{sigma aditiva@$\sigma$-aditiva!classe}
se $\mathcal E$ é não vazia e satisfaz as seguintes condições:
\begin{itemize}
\item se $A,B\in\mathcal E$ e $A\cap B=\emptyset$ então $A\cup B\in\mathcal E$;
\item se $A,B\in\mathcal E$ e $B\subset A$ então $A\setminus B\in\mathcal E$;
\item se $(A_k)_{k\ge1}$ é uma seqüência em $\mathcal E$ e $A_k\nearrow A$ então $A\in\mathcal E$.
\end{itemize}
\end{defin}
Claramente, se $\mathcal E$ é uma classe $\sigma$-aditiva então $\emptyset\in\mathcal E$; de fato,
tome qualquer $A\in\mathcal E$ e observe que $\emptyset=A\setminus A\in\mathcal E$.

\begin{example}\label{exa:muigualnu}
Sejam $\mu:\mathcal A\to\left[0,+\infty\right[$, $\nu:\mathcal A\to\left[0,+\infty\right[$ medidas finitas num
$\sigma$-anel $\mathcal A$. Usando o Lema~\ref{thm:classerazoavel}, vê-se facilmente que a classe de conjuntos:
\begin{equation}\label{eq:muigualnu}
\big\{A\in\mathcal A:\mu(A)=\nu(A)\big\}
\end{equation}
é ao mesmo tempo uma classe $\sigma$-aditiva e uma classe monotônica.
\end{example}

Em analogia à Definição~\ref{thm:anelgerado}, enunciamos a seguinte:
\begin{defin}\label{thm:sigmonotgerado}
Se $\mathcal C$ é uma classe de conjuntos arbitrária então a
{\em classe monotônica gerada por $\mathcal C$\/}\index[indice]{classe!monotonica@monotônica!gerada por uma colecao de conjuntos@gerada por uma coleção\hfil\break
de conjuntos} (resp., a
{\em classe $\sigma$-aditiva gerada por $\mathcal C$}\index[indice]{classe!sigma aditiva@$\sigma$-aditiva!gerada por uma colecao de conjuntos@gerada por uma coleção\hfil\break
de conjuntos}) é a menor classe monotônica (resp., classe $\sigma$-aditiva) $\mathcal E$ que contém $\mathcal C$, i.e.,
$\mathcal E$ é uma classe monotônica (resp., classe $\sigma$-aditiva) tal que:
\begin{enumerate}
\item\label{itm:sigmonotger1} $\mathcal C\subset\mathcal E$;
\item\label{itm:sigmonotger2} se $\mathcal E'$ é uma classe monotônica (resp., classe $\sigma$-aditiva) tal que
$\mathcal C\subset\mathcal E'$ então $\mathcal E\subset\mathcal E'$.
\end{enumerate}
Dizemos também que $\mathcal C$ é um {\em conjunto de geradores\/}\index[indice]{conjunto!de geradores!para uma classe monotonica@para uma classe\hfil\break monotônica}%
\index[indice]{geradores!para uma classe monotonica@para uma classe\hfil\break monotônica}\index[indice]{conjunto!de geradores!para uma classe sigma aditiva@para uma classe\hfil\break$\sigma$-aditiva}%
\index[indice]{geradores!para uma classe sigma aditiva@para uma classe $\sigma$-aditiva}
para a classe monotônica (resp., classe $\sigma$-aditiva) $\mathcal E$.
\end{defin}
No Exercício~\ref{exe:sigmonotgerado} pedimos ao leitor para justificar o fato de que a classe monotônica (resp.,
a classe $\sigma$-aditiva) gerada por uma classe de conjuntos $\mathcal C$ está de fato bem definida,
ou seja, existe uma única classe monotônica (resp., classe $\sigma$-aditiva) $\mathcal E$ satisfazendo as propriedades
\eqref{itm:sigmonotger1} e \eqref{itm:sigmonotger2} acima.

Temos o seguinte:
\begin{lem}[lema da classe monotônica]\index[indice]{lema!da classe monotonica@da classe monotônica}\index[indice]{classe!monotonica@monotônica!lema da}
\label{thm:lemmonclass}
Se $\mathcal R$ é um anel então a classe monotônica gerada por $\mathcal R$ coincide com o $\sigma$-anel
gerado por $\mathcal R$. Em particular, toda classe monotônica que contém um anel $\mathcal R$ contém
também o $\sigma$-anel gerado por $\mathcal R$.
\end{lem}
\begin{proof}
Seja $\mathcal A$ o $\sigma$-anel gerado por $\mathcal R$ e seja $\mathcal E$ a classe monotônica gerada
por $\mathcal R$. Segue do Lema~\ref{thm:propaneis} que todo $\sigma$-anel é uma classe monotônica e portanto
$\mathcal A$ contém $\mathcal E$. Para mostrar que $\mathcal E$ contém $\mathcal A$, basta verificar que
$\mathcal E$ é um $\sigma$-anel. Dividimos o restante da demonstração em vários passos.

\begin{stepindent}
\item\label{itm:monotclass1}
{\em Se $A,B\in\mathcal E$ então $A\cup B\in\mathcal E$}.

Seja $B\in\mathcal E$ fixado e considere a classe de conjuntos:
\begin{equation}\label{eq:unionBnocalE}
\big\{A\in\mathcal A:A\cup B\in\mathcal E\big\}.
\end{equation}
Afirmamos que \eqref{eq:unionBnocalE} é uma classe monotônica. De fato, o conjunto vazio está em \eqref{eq:unionBnocalE}
pois $B\in\mathcal E$. Se $(A_k)_{k\ge1}$ é uma seqüência em \eqref{eq:unionBnocalE} e $A_k\nearrow A$ então
$A\in\mathcal A$ e $A_k\cup B\in\mathcal E$, para todo $k\ge1$. Como $(A_k\cup B)\nearrow(A\cup B)$ e
$\mathcal E$ é uma classe monotônica, segue que $A\cup B\in\mathcal E$; portanto $A$ está em \eqref{eq:unionBnocalE}.
Se $(A_k)_{k\ge1}$ é uma seqüência em \eqref{eq:unionBnocalE} com $A_k\searrow A$ então verifica-se de modo análogo
que $A$ está em \eqref{eq:unionBnocalE}, usando o fato que $(A_k\cup B)\searrow(A\cup B)$. Logo \eqref{eq:unionBnocalE}
é uma classe monotônica. Se o conjunto $B$ estiver no anel $\mathcal R$ então $A\cup B\in\mathcal R\subset\mathcal E$
para todo $A\in\mathcal R$ e portanto \eqref{eq:unionBnocalE} contém $\mathcal R$; concluímos então que \eqref{eq:unionBnocalE}
contém $\mathcal E$, o que prova que $A\cup B\in\mathcal E$, para todo $A\in\mathcal E$ e todo $B\in\mathcal R$.
Seja agora $B\in\mathcal E$ arbitrário. Pelo que acabamos de mostrar, \eqref{eq:unionBnocalE} contém $\mathcal R$; logo
\eqref{eq:unionBnocalE} também contém $\mathcal E$. Isso prova que $A\cup B\in\mathcal E$, para todos $A,B\in\mathcal E$.

\item {\em Se $(A_k)_{k\ge1}$ é uma seqüência em $\mathcal E$ então $\bigcup_{k=1}^\infty A_k\in\mathcal E$}.

Para cada $t\ge1$, seja $B_t=\bigcup_{k=1}^tA_k$; temos $B_t\nearrow\bigcup_{k=1}^\infty A_k$ e $B_t\in\mathcal E$
para todo $t\ge1$, pelo passo~\ref{itm:monotclass1}. Como $\mathcal E$ é uma classe monotônica, segue que
$\bigcup_{k=1}^\infty A_k\in\mathcal E$.

\item\label{itm:monotclass3}
{\em Se $A\in\mathcal E$ e $B\in\mathcal R$ então $A\setminus B\in\mathcal E$}.

Seja $B\in\mathcal R$ fixado e considere a classe de conjuntos:
\begin{equation}\label{eq:menosBnocalE}
\big\{A\in\mathcal A:A\setminus B\in\mathcal E\big\}.
\end{equation}
Afirmamos que \eqref{eq:menosBnocalE} é uma classe monotônica. De fato, o conjunto vazio está em \eqref{eq:menosBnocalE},
pois $\emptyset\in\mathcal E$. Se $(A_k)_{k\ge1}$ é uma seqüência em \eqref{eq:menosBnocalE} tal que
$A_k\nearrow A$ (resp., tal que $A_k\searrow A$) então $A\in\mathcal A$ e $A_k\setminus B\in\mathcal E$, para todo $k\ge1$;
conclui-se então que $A$ está em \eqref{eq:menosBnocalE} observando-se que $(A_k\setminus B)\nearrow(A\setminus B)$
(resp., que $(A_k\setminus B)\searrow(A\setminus B)$). Isso prova que \eqref{eq:menosBnocalE} é uma
classe monotônica. Claramente, se $A\in\mathcal R$ então
$A\setminus B\in\mathcal R\subset\mathcal E$ e portanto \eqref{eq:menosBnocalE} contém $\mathcal R$;
concluímos então que \eqref{eq:menosBnocalE} contém $\mathcal E$, o que prova que $A\setminus B\in\mathcal E$,
para todo $A\in\mathcal E$ e todo $B\in\mathcal R$.

\item {\em Se $A,B\in\mathcal E$ então $A\setminus B\in\mathcal E$}.

Seja $A\in\mathcal E$ fixado e considere a classe de conjuntos:
\begin{equation}\label{eq:AmenosnocalE}
\big\{B\in\mathcal A:A\setminus B\in\mathcal E\big\}.
\end{equation}
De forma análoga ao que foi feito no passo~\ref{itm:monotclass3}, prova-se que \eqref{eq:AmenosnocalE}
é uma classe monotônica usando o fato que $B_k\nearrow B$ (resp., $B_k\searrow B$) implica
em $(A\setminus B_k)\searrow(A\setminus B)$ (resp., $(A\setminus B_k)\nearrow(A\setminus B)$).
Finalmente, o que provamos no passo~\ref{itm:monotclass3} implica que \eqref{eq:AmenosnocalE} contém
$\mathcal R$ e daí segue que \eqref{eq:AmenosnocalE} contém $\mathcal E$. Logo $A\setminus B\in\mathcal E$,
para todos $A,B\in\mathcal E$.\qedhere
\end{stepindent}
\end{proof}

\begin{lem}[lema da classe $\sigma$-aditiva]\index[indice]{lema!da classe sigma aditiva@da classe $\sigma$-aditiva}\index[indice]{classe!sigma aditiva@$\sigma$-aditiva!lema da}
\label{thm:lemsigmaclass}
Seja $\mathcal C$ uma classe de conjuntos {\em fechada por interseções finitas},\index[indice]{classe!fechada!por intersecoes finitas@por interseções finitas}%
\index[indice]{fechada!por intersecoes finitas@por interseções finitas}
i.e., $A\cap B\in\mathcal C$, para
todos $A,B\in\mathcal C$. Então a classe $\sigma$-aditiva gerada por $\mathcal C$ coincide com o $\sigma$-anel
gerado por $\mathcal C$. Em particular, toda classe $\sigma$-aditiva que contém uma classe fechada por interseções
finitas $\mathcal C$ contém também o $\sigma$-anel gerado por $\mathcal C$.
\end{lem}
\begin{proof}
Seja $\mathcal A$ o $\sigma$-anel gerado por $\mathcal C$ e seja $\mathcal E$ a classe $\sigma$-aditiva gerada
por $\mathcal C$. Evidentemente todo $\sigma$-anel é uma classe $\sigma$-aditiva e portanto
$\mathcal A$ contém $\mathcal E$. Para mostrar que $\mathcal E$ contém $\mathcal A$, basta verificar que
$\mathcal E$ é um $\sigma$-anel. Assuma por um momento que já tenhamos mostrado que $\mathcal E$ é fechado
por interseções finitas. Segue então do Lema~\ref{thm:outradefanel} que $\mathcal E$ é um anel.
Além do mais, se $(A_k)_{k\ge1}$ é uma seqüência em $\mathcal E$ então $B_t=\bigcup_{k=1}^tA_k\in\mathcal E$ para
todo $t\ge1$ e $B_t\nearrow\bigcup_{k=1}^\infty A_k$, donde $\bigcup_{k=1}^\infty A_k\in\mathcal E$. Concluímos então
que $\mathcal E$ é um $\sigma$-anel. Para completar a demonstração, verificaremos que $A\cap B\in\mathcal E$,
para todos $A,B\in\mathcal E$. Seja $B\in\mathcal A$ fixado e considere a classe de conjuntos:
\begin{equation}\label{eq:interBnocalE}
\big\{A\in\mathcal A:A\cap B\in\mathcal E\big\}.
\end{equation}
Afirmamos que \eqref{eq:interBnocalE} é uma classe $\sigma$-aditiva. Em primeiro lugar, o conjunto vazio está em
\eqref{eq:interBnocalE}, pois $\emptyset\cap B=\emptyset\in\mathcal E$. Dados $A_1$ e $A_2$ em \eqref{eq:interBnocalE}
com $A_1\cap A_2=\emptyset$ então $A_1\cap B,A_2\cap B\in\mathcal E$ e $(A_1\cap B)\cap(A_2\cap B)=\emptyset$;
como $\mathcal E$ é uma classe $\sigma$-aditiva, segue que $(A_1\cap B)\cup(A_2\cap B)=(A_1\cup A_2)\cap B\in\mathcal E$
e portanto $A_1\cup A_2$ está em \eqref{eq:interBnocalE}. Suponha agora que $A_1$ e $A_2$ são elementos de \eqref{eq:interBnocalE}
com $A_1\subset A_2$. Temos $A_1\cap B,A_2\cap B\in\mathcal E$ e $A_1\cap B\subset A_2\cap B$; segue então que
$(A_2\cap B)\setminus(A_1\cap B)\in\mathcal E$. Como $(A_2\cap B)\setminus(A_1\cap B)=(A_2\setminus A_1)\cap B$,
concluímos que $A_2\setminus A_1$ está em \eqref{eq:interBnocalE}. Para concluir a demonstração de que \eqref{eq:interBnocalE}
é uma classe $\sigma$-aditiva, seja $(A_k)_{k\ge1}$ uma seqüência em \eqref{eq:interBnocalE} com $A_k\nearrow A$.
Daí $A_k\cap B\in\mathcal E$, para todo $k\ge1$ e $(A_k\cap B)\nearrow(A\cap B)$; segue então que $A\cap B\in\mathcal E$
e portanto $A$ está em \eqref{eq:interBnocalE}. Demonstramos então que \eqref{eq:interBnocalE} é uma classe
$\sigma$-aditiva. Se o conjunto $B$ pertence a $\mathcal C$ então $A\cap B\in\mathcal C\subset\mathcal E$,
para todo $A\in\mathcal C$ e portanto \eqref{eq:interBnocalE} contém $\mathcal C$; concluímos então que
\eqref{eq:interBnocalE} contém $\mathcal E$, isto é, $A\cap B\in\mathcal E$, para todo $A\in\mathcal E$ e todo
$B\in\mathcal C$. Seja agora $B\in\mathcal E$ arbitrário. Pelo que acabamos de mostrar, \eqref{eq:interBnocalE}
contém $\mathcal C$ e portanto contém $\mathcal E$; concluímos então que $A\cap B\in\mathcal E$, para todos
$A,B\in\mathcal E$.
\end{proof}

Vejamos agora algumas aplicações interessantes dos Lemas~\ref{thm:lemmonclass}
e \ref{thm:lemsigmaclass}.
\begin{lem}\label{thm:uniqueext}
Sejam $\mu:\mathcal A\to\left[0,+\infty\right[$, $\nu:\mathcal A\to\left[0,+\infty\right[$
medidas finitas num $\sigma$-anel $\mathcal A$. Seja $\mathcal C\subset\mathcal A$
uma classe fechada por interseções finitas tal que $\mathcal A$ é o $\sigma$-anel
gerado por $\mathcal C$. Se $\mu(A)=\nu(A)$ para todo $A\in\mathcal C$ então $\mu=\nu$.
\end{lem}
\begin{proof}
Como vimos no Exemplo~\ref{exa:muigualnu}, a classe \eqref{eq:muigualnu} formada pelos conjuntos
onde $\mu$ e $\nu$ coincidem é uma classe $\sigma$-aditiva. Como \eqref{eq:muigualnu} contém
$\mathcal C$ e $\mathcal C$ é fechada por interseções finitas, segue do Lema~\ref{thm:lemsigmaclass}
que \eqref{eq:muigualnu} contém $\mathcal A$. Logo $\mu=\nu$.
\end{proof}
O Lema~\ref{thm:uniqueext} pode ser pensado como um lema de unicidade de extensão de medidas;
de fato, um enunciado alternativo para o Lema~\ref{thm:uniqueext} é o seguinte:
uma medida finita $\mu$ numa classe de conjuntos $\mathcal C$ fechada por interseções
finitas extende-se no máximo de uma maneira a uma medida finita no $\sigma$-anel gerado
por $\mathcal C$. Logo adiante apresentaremos uma generalização desse resultado.

Outro resultado interessante é o seguinte lema de aproximação.
\begin{lem}\label{thm:aproxlema}
Seja $\mu:\mathcal A\to\left[0,+\infty\right[$ uma medida finita num $\sigma$-anel $\mathcal A$
e seja $\mathcal R\subset\mathcal A$ um anel tal que $\mathcal A$ é o $\sigma$-anel
gerado por $\mathcal R$. Então para todo $A\in\mathcal A$ e todo $\varepsilon>0$ existe
$B\in\mathcal R$ tal que $\mu(A\bigtriangleup B)<\varepsilon$. Em particular, pelo
resultado do Exercício~\ref{exe:mudifsym}, temos $\big\vert\mu(A)-\mu(B)\big\vert<\varepsilon$.
\end{lem}
\begin{proof}
Considere a classe de conjuntos:
\begin{equation}\label{eq:dapraaprox}
\big\{A\in\mathcal A:\text{para todo $\varepsilon>0$, existe $B\in\mathcal R$ tal que
$\mu(A\bigtriangleup B)<\varepsilon$}\big\}.
\end{equation}
Evidentemente \eqref{eq:dapraaprox} contém o anel $\mathcal R$. Se mostrarmos que
\eqref{eq:dapraaprox} é uma classe monotônica, a tese seguirá do Lema~\ref{thm:lemmonclass}.
Seja $(A_k)_{k\ge1}$ uma seqüência em \eqref{eq:dapraaprox} tal que $A_k\nearrow A$
(resp., tal que $A_k\searrow A$) e seja dado $\varepsilon>0$.
Temos que $A_k\bigtriangleup A=A\setminus A_k$ (resp.,
que $A_k\bigtriangleup A=A_k\setminus A$) e portanto $(A_k\bigtriangleup A)\searrow\emptyset$.
Como a medida $\mu$ é finita, obtemos $\lim_{k\to\infty}\mu(A_k\bigtriangleup A)=0$
e portanto existe $k\ge1$ tal que $\mu(A_k\bigtriangleup A)<\frac\varepsilon2$.
Como $A_k$ está em \eqref{eq:dapraaprox}, existe $B\in\mathcal R$ tal que
$\mu(A_k\bigtriangleup B)<\frac\varepsilon2$. Mas (veja Exercício~\ref{exe:destriangle}):
\[A\bigtriangleup B\subset(A\bigtriangleup A_k)\cup(A_k\bigtriangleup B)\]
e portanto, pelo item~(e) do Lema~\ref{thm:classerazoavel}:
\[\mu(A\bigtriangleup B)\le\mu(A\bigtriangleup A_k)+\mu(A_k\bigtriangleup B)<
\frac\varepsilon2+\frac\varepsilon2=\varepsilon.\]
Isso prova que $A$ está em \eqref{eq:dapraaprox} e completa a demonstração.
\end{proof}

A hipótese de finitude das medidas nos Lemas~\ref{thm:uniqueext} e \ref{thm:aproxlema}
é muito restritiva. Vamos agora relaxar essa hipótese.
\begin{defin}
Seja $\mathcal C$ uma classe de conjuntos com $\emptyset\in\mathcal C$
e seja $\mu:\mathcal C\to[0,+\infty]$ uma medida em $\mathcal C$. Dizemos que
um conjunto $A$ (não necessariamente em $\mathcal C$) é
{\em $\sigma$-finito com respeito a $\mu$\/}\index[indice]{conjunto!sigma finito@$\sigma$-finito}%
\index[indice]{sigma finito@$\sigma$-finito!conjunto} se existe uma seqüência $(A_k)_{k\ge1}$
em $\mathcal C$ tal que $A\subset\bigcup_{k=1}^\infty A_k$ e $\mu(A_k)<+\infty$, para todo
$k\ge1$. Dizemos que a medida $\mu$ é {\em $\sigma$-finita\/}\index[indice]{medida!sigma finita@$\sigma$-finita}%
\index[indice]{sigma finita@$\sigma$-finita!medida} se todo conjunto $A$ {\em em $\mathcal C$\/}
é $\sigma$-finito com respeito a $\mu$.
\end{defin}
Evidentemente, se $A\in\mathcal C$ e $\mu(A)<+\infty$ então $A$ é $\sigma$-finito
com respeito a $\mu$; em particular, toda medida finita é $\sigma$-finita.
Note que se $A$ é $\sigma$-finito com respeito a $\mu$ e $B\subset A$
então $B$ também é $\sigma$-finito com respeito a $\mu$. Em particular,
se $X$ é um conjunto tal que $\mathcal C\subset\wp(X)$ e $X\in\mathcal C$ então
$\mu$ é $\sigma$-finita se e somente se $X$ é $\sigma$-finito com respeito a $\mu$, isto é,
se somente se existe uma seqüência $(A_k)_{k\ge1}$ em $\mathcal C$ tal que $X=\bigcup_{k=1}^\infty A_k$
e $\mu(A_k)<+\infty$, para todo $k\ge1$.

\begin{rem}\label{thm:tambemehsigmafin}
Seja $\mathcal C$ uma classe de conjuntos com $\emptyset\in\mathcal C$
e seja $\mu:\mathcal C\to[0,+\infty]$ uma medida em $\mathcal C$; denote
por $\mathcal A$ o $\sigma$-anel gerado por $\mathcal C$. Se a medida $\mu$ é
$\sigma$-finita então todo elemento $A$ de $\mathcal A$ é
$\sigma$-finito com respeito a $\mu$. De fato, pelo resultado do Exercício~\ref{exe:sigmanoheredit},
todo $A\in\mathcal A$ pode ser coberto por uma união
enumerável de elementos de $\mathcal C$; mas cada elemento de $\mathcal C$ pode ser coberto
por uma união enumerável de elementos de $\mathcal C$ de medida finita. Logo $A$ pode
ser coberto por uma união enumerável de elementos de $\mathcal C$ de medida finita.
Vemos então que se $\bar\mu:\mathcal A\to[0,+\infty]$ é uma medida em $\mathcal A$ que estende $\mu$
então a $\sigma$-finitude de $\mu$ implica na $\sigma$-finitude de $\bar\mu$ (a recíproca
não é verdadeira, como veremos no Exemplo~\ref{exa:naobastanocalA}).
\end{rem}

\begin{notation}
Seja $\mathcal C$ uma classe de conjuntos e $X$ um conjunto arbitrário. Denotamos
por $\mathcal C\vert_X$ a classe de conjuntos:\index[simbolos]{$\mathcal C\vert_X$}
\[\mathcal C\vert_X=\big\{A\cap X:A\in\mathcal C\big\}.\]
\end{notation}
Note que se a classe de conjuntos $\mathcal C$ é fechada por interseções finitas
e se $X\in\mathcal C$ então:
\[\mathcal C\vert_X=\big\{A\in\mathcal C:A\subset X\big\}=\mathcal C\cap\wp(X).\]
Em particular, nesse caso, se $\emptyset\in\mathcal C$ e $\mu:\mathcal C\to[0,+\infty]$
é uma medida em $\mathcal C$ então a restrição de $\mu$ a $\mathcal C\vert_X$ também
é uma medida.

\begin{lem}\label{thm:gerarestr}
Se $\mathcal A$ é um $\sigma$-anel e $X$ é um conjunto arbitrário então $\mathcal A\vert_X$
é um $\sigma$-anel. Além do mais, se $\mathcal A$ é o $\sigma$-anel gerado por uma classe
de conjuntos $\mathcal C$ então $\mathcal A\vert_X$ é o $\sigma$-anel gerado por
$\mathcal C\vert_X$.
\end{lem}
\begin{proof}
A prova de que $\mathcal A\vert_X$ é um $\sigma$-anel é deixada a cargo do leitor
(veja Exercício~\ref{exe:AvertYanel}). Seja $\mathcal B$ o $\sigma$-anel gerado
por $\mathcal C\vert_X$. Como $\mathcal A\vert_X$ é um $\sigma$-anel que contém $\mathcal C\vert_X$,
temos que $\mathcal B\subset\mathcal A\vert_X$. Para mostrar que $\mathcal A\vert_X$
está contido em $\mathcal B$, considere a coleção de conjuntos:
\begin{equation}\label{eq:provaCXgeraAX}
\big\{A\in\mathcal A:A\cap X\in\mathcal B\big\}.
\end{equation}
Verifica-se diretamente que \eqref{eq:provaCXgeraAX} é um $\sigma$-anel. Como
\eqref{eq:provaCXgeraAX} contém $\mathcal C$ e $\mathcal A$ é o $\sigma$-anel
gerado por $\mathcal C$, concluímos que \eqref{eq:provaCXgeraAX} contém $\mathcal A$,
ou seja, $A\cap X\in\mathcal B$, para todo $A\in\mathcal A$. Isso mostra que
$\mathcal A\vert_X\subset\mathcal B$ e completa a demonstração.
\end{proof}

Estamos agora em condições de generalizar os Lemas~\ref{thm:uniqueext} e \ref{thm:aproxlema}.
\begin{lem}\label{thm:sigmauniqueext}
Sejam $\mu:\mathcal A\to[0,+\infty]$, $\nu:\mathcal A\to[0,+\infty]$
medidas num $\sigma$-anel $\mathcal A$. Seja $\mathcal C\subset\mathcal A$
uma classe fechada por interseções finitas tal que\footnote{%
A\label{foot:sigmauniqueext} hipótese $\emptyset\in\mathcal C$ só foi colocada para que possamos dizer que
$\mu\vert_{\mathcal C}$ é uma medida (recorde Definição~\ref{thm:defmedidageral}).
Note, no entanto, que essa hipótese não é nada restritiva
já que $\mu(\emptyset)=\nu(\emptyset)=0$ e que a classe $\mathcal C$ é fechada por interseções finitas
se e somente se $\mathcal C\cup\{\emptyset\}$ o é; podemos então sempre
substituir $\mathcal C$ por $\mathcal C\cup\{\emptyset\}$ se necessário.}
$\emptyset\in\mathcal C$;
suponha que $\mathcal A$ é o $\sigma$-anel
gerado por $\mathcal C$. Se $\mu\vert_{\mathcal C}=\nu\vert_{\mathcal C}$
e $A\in\mathcal A$ é $\sigma$-finito com respeito a $\mu\vert_{\mathcal C}$
então $\mu(A)=\nu(A)$. Em particular, pela Observação~\ref{thm:tambemehsigmafin},
se a medida $\mu\vert_{\mathcal C}$ é $\sigma$-finita
e se $\mu\vert_{\mathcal C}=\nu\vert_{\mathcal C}$ então $\mu=\nu$.
\end{lem}
\begin{proof}
Seja $X\in\mathcal C$ com $\mu(X)<+\infty$. Afirmamos que $\mu$ e $\nu$ coincidem
em $\mathcal A\vert_X$. De fato, $\mathcal C\vert_X$ é uma classe fechada por interseções
finitas que gera o $\sigma$-anel $\mathcal A\vert_X$ (veja Lema~\ref{thm:gerarestr});
como $\mu$ e $\nu$ coincidem em $\mathcal C\vert_X\subset\mathcal C$,
segue do Lema~\ref{thm:uniqueext} aplicado às medida finitas $\mu\vert_{\mathcal A\vert_X}$
e $\nu\vert_{\mathcal A\vert_X}$ que $\mu$ e $\nu$ coincidem em $\mathcal A\vert_X$.
Seja agora $A\in\mathcal A$ e suponha que $A$ é $\sigma$-finito com respeito a $\mu\vert_{\mathcal C}$;
mostremos que $\mu(A)=\nu(A)$. Seja $(X_k)_{k\ge1}$ uma seqüência em $\mathcal C$ com $A\subset\bigcup_{k=1}^\infty X_k$
e $\mu(X_k)<+\infty$, para todo $k\ge1$. Para cada $k\ge1$ seja
$Y_k=X_k\setminus\bigcup_{i=0}^{k-1}X_i$, onde $X_0=\emptyset$. Pelo resultado
do Exercício~\ref{exe:disjuntar}, os conjuntos $(Y_k)_{k\ge1}$ são dois a dois disjuntos
e $\bigcup_{k=1}^\infty X_k=\bigcup_{k=1}^\infty Y_k$. Temos
$A=\bigcup_{k=1}^\infty(Y_k\cap A)$ e portanto:
\[\mu(A)=\sum_{k=1}^\infty\mu(Y_k\cap A),\quad\nu(A)=\sum_{k=1}^\infty\nu(Y_k\cap A).\]
Mas para todo $k\ge1$ temos $Y_k\cap A\in\mathcal A\cap\wp(X_k)=\mathcal A\vert_{X_k}$
e segue da primeira parte da demonstração que $\mu(Y_k\cap A)=\nu(Y_k\cap A)$, para todo
$k\ge1$. Logo $\mu(A)=\nu(A)$.
\end{proof}

\begin{cor}\label{thm:coruniquext}
Seja $\mathcal C$ uma classe de conjuntos fechada por interseções finitas tal que $\emptyset\in\mathcal C$.
Se $\mu:\mathcal C\to[0,+\infty]$ é uma medida $\sigma$-finita então $\mu$ possui no máximo
uma extensão a uma medida no $\sigma$-anel gerado por $\mathcal C$; uma tal extensão (caso
exista) é automaticamente $\sigma$-finita.
\end{cor}
\begin{proof}
Segue do Lema~\ref{thm:sigmauniqueext} e da Observação~\ref{thm:tambemehsigmafin}.
\end{proof}
Resultados sobre existência de extensões de medidas para $\sigma$-anéis serão estudados
mais adiante na Seção~\ref{sec:TeoExtend}.

\begin{example}\label{exa:naobastanocalA}
Nas hipóteses do Lema~\ref{thm:sigmauniqueext} é realmente necessário supor que
$A$ seja $\sigma$-finito com respeito a $\mu\vert_{\mathcal C}$; assumir apenas
a $\sigma$-finitude com respeito a $\mu$ e a $\nu$ não é suficiente.
De fato, considere as medidas $\mu:\wp(\R)\to[0,+\infty]$ e
$\nu:\wp(\R)\to[0,+\infty]$ definidas por:
\[\mu(A)=\vert A\cap\Q\vert,\quad\nu(A)=\big\vert A\cap\big(\Q\setminus\{0\}\big)\big\vert,\]
para todo $A\subset\R$, onde $\vert E\vert\in\N\cup\{+\infty\}$\index[simbolos]{$\vert E\vert$} denota o número de elementos
de um conjunto $E$. Seja $\mathcal S$ o semi-anel constituído pelos intervalos da forma
$\left]a,b\right]$, $a,b\in\R$ (veja \eqref{eq:semiintervalos}). Temos $\mu(A)=\nu(A)=+\infty$,
para todo $A\in\mathcal S$, $A\ne\emptyset$. Se $\mathcal A$ é o $\sigma$-anel
gerado por $\mathcal S$ (pelo resultado do Exercício~\ref{exe:anelBorelint},
$\mathcal A$ coincide com a $\sigma$-álgebra de Borel de $\R$) então as medidas $\mu\vert_{\mathcal A}$
e $\nu\vert_{\mathcal A}$ são ambas $\sigma$-finitas. De fato, para todo $x\in\R$,
o conjunto unitário $\{x\}$ está em $\mathcal A$ e:
\[\R=(\R\setminus\Q)\cup\bigcup_{x\in\Q}\{x\};\]
além do mais, $\mu(\R\setminus\Q)=\nu(\R\setminus\Q)=0$, $\mu\big(\{x\}\big)=1$
e $\nu\big(\{x\}\big)\le1$, para todo $x\in\Q$. No entanto, temos
$\mu\vert_{\mathcal A}\ne\nu\vert_{\mathcal A}$, já que $\mu\big(\{0\}\big)=1$
e $\nu\big(\{0\}\big)=0$.
\end{example}

Generalizamos agora o Lema~\ref{thm:aproxlema}.
\begin{lem}\label{thm:aproxsigmafin}
Seja $\mu:\mathcal A\to[0,+\infty]$ uma medida num $\sigma$-anel $\mathcal A$
e seja $\mathcal R\subset\mathcal A$ um anel tal que $\mathcal A$ é o $\sigma$-anel
gerado por $\mathcal R$. Suponha que $A\in\mathcal A$ é $\sigma$-finito com respeito
a $\mu\vert_{\mathcal R}$ (pela Observação~\ref{thm:tambemehsigmafin}, esse é o caso,
por exemplo, se a medida $\mu\vert_{\mathcal R}$
é $\sigma$-finita). Se $\mu(A)<+\infty$ então para todo $\varepsilon>0$ existe
$B\in\mathcal R$ tal que $\mu(A\bigtriangleup B)<\varepsilon$. Em particular, pelo
resultado do Exercício~\ref{exe:mudifsym}, temos $\big\vert\mu(A)-\mu(B)\big\vert<\varepsilon$.
\end{lem}
\begin{proof}
Seja $A\in\mathcal A$ um conjunto $\sigma$-finito com respeito a $\mu\vert_{\mathcal R}$
tal que $\mu(A)<+\infty$; seja dado $\varepsilon>0$.
Seja $(X_k)_{k\ge1}$ uma seqüência em $\mathcal R$ com $A\subset\bigcup_{k=1}^\infty X_k$
e $\mu(X_k)<+\infty$, para todo $k\ge1$. Para cada $k\ge1$ seja $Y_k=\bigcup_{i=1}^kX_i$,
de modo que $Y_k\in\mathcal R$, $\mu(Y_k)<+\infty$ e $(A\setminus Y_k)\searrow\emptyset$.
Como $\mu(A)<+\infty$, temos $\lim_{k\to\infty}\mu(A\setminus Y_k)=0$ e portanto existe $k\ge1$ tal que
$\mu(A\setminus Y_k)<\frac\varepsilon2$. Seja $A'=A\cap Y_k$; daí
$A\bigtriangleup A'=A\setminus Y_k$ e portanto:
\[\mu(A\bigtriangleup A')<\frac\varepsilon2,\]
e $A'\in\mathcal A\cap\wp(Y_k)=\mathcal A\vert_{Y_k}$. Vamos agora aplicar o
Lema~\ref{thm:aproxlema} à medida finita $\mu\vert_{\mathcal A\vert_{Y_k}}$; para
isso, note que, pelo Exercício~\ref{exe:AvertYanel}, $\mathcal R\vert_{Y_k}$ é um anel e,
pelo Lema~\ref{thm:gerarestr}, $\mathcal A\vert_{Y_k}$ é o $\sigma$-anel gerado por $\mathcal R\vert_{Y_k}$.
Vemos então que existe $B\in\mathcal R\vert_{Y_k}\subset\mathcal R$
tal que $\mu(A'\bigtriangleup B)<\frac\varepsilon2$. Daí (veja Exercício~\ref{exe:destriangle}):
\[A\bigtriangleup B\subset(A\bigtriangleup A')\cup(A'\bigtriangleup B)\]
e portanto:
\[\mu(A\bigtriangleup B)\le\mu(A\bigtriangleup A')+\mu(A'\bigtriangleup B)
<\frac\varepsilon2+\frac\varepsilon2=\varepsilon,\]
o que completa a demonstração.
\end{proof}

\begin{example}
Sejam $\mu$, $\mathcal S$ e $\mathcal A$ definidos como no Exemplo~\ref{exa:naobastanocalA};
seja $\mathcal R$ o anel gerado por $\mathcal S$, de modo que $\mathcal A$ é o $\sigma$-anel
gerado por $\mathcal R$. Pelo Lema~\ref{thm:semigeraanel}, todo elemento de
$\mathcal R$ é uma união finita disjunta de elementos de $\mathcal S$ e portanto
$\mu(A)=+\infty$, para todo $A\in\mathcal R$ com $A\ne\emptyset$. Temos que a medida
$\mu\vert_{\mathcal A}$ é $\sigma$-finita (mas $\mu\vert_{\mathcal R}$ não é). Dado
$A\in\mathcal A$ com $\mu(A)<+\infty$ (por exemplo, se $A$ é um subconjunto finito de $\Q$)
então $\mu(A\bigtriangleup B)=+\infty$, para todo $B\in\mathcal R$ com $B\ne\emptyset$
(veja Exercício~\ref{exe:mudifsym}). Logo a tese do Lema~\ref{thm:aproxsigmafin} não é verdadeira
para a medida $\mu\vert_{\mathcal A}$ e para o anel $\mathcal R$, embora $\mu\vert_{\mathcal A}$
seja $\sigma$-finita.
\end{example}

\begin{example}
Seja $\mathcal A=\wp(\N)$ e $\mathcal R$ o conjunto das partes finitas de $\N$; daí $\mathcal R$
é um anel e $\mathcal A$ é o $\sigma$-anel gerado por $\mathcal R$. Seja $\mu:\mathcal A\to[0,+\infty]$
a medida de contagem (veja a Definição~\ref{thm:defmedcont}). Note que $\mu\vert_{\mathcal R}$
é uma medida finita (e portanto $\sigma$-finita). Dado $A\in\mathcal A$ com $\mu(A)=+\infty$
então para todo $B\in\mathcal R$ temos $\mu(A\bigtriangleup B)=+\infty$.
Vemos que a hipótese $\mu(A)<+\infty$ é essencial no Lema~\ref{thm:aproxsigmafin}.
\end{example}

\end{section}

\begin{section}{Medidas Exteriores e o Teorema da Extensão}
\label{sec:TeoExtend}

A estratégia que nós usamos na Seção~\ref{sec:MedLebRn} para construir a medida
de Lebesgue em $\R^n$ foi a seguinte: nós definimos inicialmente a medida exterior
de Lebesgue $\leb^*:\wp(\R^n)\to[0,+\infty]$ (que não é sequer uma medida finitamente aditiva)
e depois uma $\sigma$-álgebra $\Lebmens(\R^n)$ de partes de $\R^n$ restrita a qual a
medida exterior $\leb^*$ é uma medida. Essa estratégia trata-se na verdade de um caso
particular de um procedimento geral. Nesta seção nós definiremos o conceito
geral de medida exterior e mostraremos que a toda medida exterior pode-se associar
naturalmente um $\sigma$-anel restrita ao qual a medida exterior é uma medida.

Começamos definindo as classes de conjuntos que servirão como domínio para as medidas exteriores.
\begin{defin}
Uma classe de conjuntos $\mathcal H$ é dita um {\em $\sigma$-anel
hereditário\/}\index[indice]{sigma anel@$\sigma$-anel!hereditario@hereditário}\index[indice]{hereditario@hereditário!sigma anel@$\sigma$-anel}
se $\mathcal H$ é não vazia e satisfaz as seguintes condições:
\begin{itemize}
\item[(a)] se $A\in\mathcal H$ e $B\subset A$ então $B\in\mathcal H$;
\item[(b)] se $(A_k)_{k\ge1}$ é uma seqüência de elementos de $\mathcal H$ então
$\bigcup_{k=1}^\infty A_k\in\mathcal H$.
\end{itemize}
\end{defin}
Evidentemente todo $\sigma$-anel hereditário é um $\sigma$-anel.

\begin{example}
Dado um conjunto arbitrário $X$ então a coleção $\wp(X)$ de todas as partes de $X$
é um $\sigma$-anel hereditário. A coleção de todos os subconjuntos enumeráveis de $X$
também é um $\sigma$-anel hereditário. Note que se $\mathcal H$ é um $\sigma$-anel hereditário
contido em $\wp(X)$ então $\mathcal H=\wp(X)$ se e somente se $X\in\mathcal H$.
\end{example}

\begin{defin}
Seja $\mathcal H$ um $\sigma$-anel hereditário. Uma {\em medida exterior\/}\index[indice]{medida!exterior} em $\mathcal H$
é uma função $\mu^*:\mathcal H\to[0,+\infty]$ satisfazendo as seguintes condições:
\begin{itemize}
\item[(a)] $\mu^*(\emptyset)=0$;
\item[(b)] ({\em monotonicidade})\index[indice]{monotonicidade} se $A,B\in\mathcal H$ e $A\subset B$ então
$\mu^*(A)\le\mu^*(B)$;
\item[(c)] ({\em $\sigma$-subaditividade})\index[indice]{sigma subaditividade@$\sigma$-subaditividade}\index[indice]{subaditividade!sigma@$\sigma$-}
se $(A_k)_{k\ge1}$ é uma seqüência de elementos de $\mathcal H$ então:
\begin{equation}\label{eq:mustarsubadd}
\mu^*\Big(\bigcup_{k=1}^\infty A_k\Big)\le\sum_{k=1}^\infty\mu^*(A_k).
\end{equation}
\end{itemize}
\end{defin}
Claramente, se $\mu^*:\mathcal H\to[0,+\infty]$ é uma medida exterior, então dados $A_1,\ldots,A_t\in\mathcal H$
temos:
\[\mu^*\Big(\bigcup_{k=1}^tA_k\Big)\le\sum_{k=1}^t\mu^*(A_k);\]
de fato, basta tomar $A_k=\emptyset$ para todo $k>t$ em \eqref{eq:mustarsubadd}.

\begin{example}
Segue dos Lemas~\ref{thm:lebmonotonica} e \ref{thm:sigmasubad} que a medida exterior
de Lebesgue $\leb^*$ é uma medida exterior no $\sigma$-anel hereditário $\wp(\R^n)$.
\end{example}

Usando a Proposição~\ref{thm:Caratheodory} como inspiração nós damos a seguinte:
\begin{defin}
Sejam $\mathcal H$ um $\sigma$-anel hereditário e $\mu^*:\mathcal H\to[0,+\infty]$
uma medida exterior. Um conjunto $E\in\mathcal H$ é dito
{\em $\mu^*$-mensurável\/}\index[indice]{mu mensuravel@$\mu^*$-mensurável}%
\index[indice]{mensuravel@mensurável!com respeito a uma medida exterior@com respeito à uma medida\hfil\break exterior}%
\index[indice]{conjunto!mu mensuravel@$\mu^*$-mensurável}%
\index[indice]{conjunto!mensuravel@mensurável!com respeito a uma medida exterior@com respeito à uma medida exterior}
se para todo $A\in\mathcal H$ vale a igualdade:
\begin{equation}\label{eq:Caratheodorygeral}
\mu^*(A)=\mu^*(A\cap E)+\mu^*(A\setminus E).
\end{equation}
\end{defin}
Como $A=(A\cap E)\cup(A\setminus E)$, temos $\mu^*(A)\le\mu^*(A\cap E)+\mu^*(A\setminus E)$,
para todos $A,E\in\mathcal H$ e portanto \eqref{eq:Caratheodorygeral} é na verdade equivalente a:
\[\mu^*(A)\ge\mu^*(A\cap E)+\mu^*(A\setminus E).\]

\begin{rem}\label{thm:lebstarmens}
Se $\leb^*:\wp(\R^n)\to[0,+\infty]$ denota a medida exterior de Lebesgue então segue
da Proposição~\ref{thm:Caratheodory} que os conjuntos $\leb^*$-mensuráveis são precisamente
os subconjuntos Lebesgue mensuráveis de $\R^n$.
\end{rem}

\begin{teo}\label{thm:teomustarmens}
Sejam $\mathcal H$ um $\sigma$-anel hereditário e $\mu^*:\mathcal H\to[0,+\infty]$
uma medida exterior. Então:
\begin{itemize}
\item[(a)] a coleção $\mathfrak M\subset\mathcal H$ de todos os conjuntos $\mu^*$-mensuráveis
é um $\sigma$-anel;
\item[(b)] dados $A\in\mathcal H$ e uma seqüência $(E_k)_{k\ge1}$ de elementos dois a dois
disjuntos de $\mathfrak M$ então:
\begin{equation}\label{eq:supersigmaadd}
\mu^*\Big(A\cap\bigcup_{k=1}^\infty E_k\Big)=\sum_{k=1}^\infty\mu^*(A\cap E_k);
\end{equation}
\item[(c)] a restrição de $\mu^*$ a $\mathfrak M$ é uma medida $\sigma$-aditiva;
\item[(d)] se $E\in\mathcal H$ é tal que $\mu^*(E)=0$ então $E\in\mathfrak M$.
\end{itemize}
\end{teo}
\begin{proof}
Seja $X$ um conjunto arbitrário tal que $A\subset X$, para todo $A\in\mathcal H$
(tome, por exemplo, $X=\bigcup_{A\in\mathcal H}A$). Convencionando que complementares
são sempre tomados com respeito a $X$, podemos reescrever a condição \eqref{eq:Caratheodorygeral}
na forma mais conveniente:
\[\mu^*(A)=\mu^*(A\cap E)+\mu^*(A\cap E^\compl).\]

A demonstração do teorema será dividida em vários passos.

\begin{stepindent}
\item\label{itm:mustaruniao1}
{\em Se $E_1,E_2\in\mathfrak M$ então $E_1\cup E_2\in\mathfrak M$}.

Seja dado $A\in\mathcal H$. Usando o fato que $E_1$ e $E_2$ são $\mu^*$-mensuráveis, obtemos:
\begin{multline}\label{eq:mustar1}
\mu^*(A)=\mu^*(A\cap E_1)+\mu^*(A\cap E_1^\compl)=\mu^*(A\cap E_1)\\
+\mu^*(A\cap E_1^\compl\cap E_2)+\mu^*(A\cap E_1^\compl\cap E_2^\compl)\\
=\mu^*(A\cap E_1)+\mu^*(A\cap E_1^\compl\cap E_2)+\mu^*\big(A\cap(E_1\cup E_2)^\compl\big);
\end{multline}
mas $A\cap(E_1\cup E_2)=(A\cap E_1)\cup(A\cap E_1^\compl\cap E_2)$ e portanto:
\begin{equation}\label{eq:mustar2}
\mu^*\big(A\cap(E_1\cup E_2)\big)\le\mu^*(A\cap E_1)+\mu^*(A\cap E_1^\compl\cap E_2).
\end{equation}
De \eqref{eq:mustar1} e \eqref{eq:mustar2} vem:
\begin{multline*}
\mu^*(A)=\mu^*(A\cap E_1)+\mu^*(A\cap E_1^\compl\cap E_2)+\mu^*\big(A\cap(E_1\cup E_2)^\compl\big)\\
\ge\mu^*\big(A\cap(E_1\cup E_2)\big)+\mu^*\big(A\cap(E_1\cup E_2)^\compl\big),
\end{multline*}
o que prova que $E_1\cup E_2\in\mathfrak M$.

\item\label{itm:mustaruniao2}
{\em Se $E_1,E_2\in\mathfrak M$, $A\in\mathcal H$ e $E_1\cap E_2=\emptyset$ então:}
\begin{equation}\label{eq:mustarpasso2}
\mu^*\big(A\cap(E_1\cup E_2)\big)=\mu^*(A\cap E_1)+\mu^*(A\cap E_2).
\end{equation}

Como $A\cap(E_1\cup E_2)\in\mathcal H$ e $E_1\in\mathfrak M$, temos:
\begin{multline*}
\mu^*\big(A\cap(E_1\cup E_2)\big)=\mu^*\big(A\cap(E_1\cup E_2)\cap E_1\big)+
\mu^*\big(A\cap(E_1\cup E_2)\cap E_1^\compl\big)\\
=\mu^*(A\cap E_1)+\mu^*(A\cap E_2),
\end{multline*}
onde na última igualdade usamos que $E_1\cap E_2=\emptyset$.

\item {\em Se $E_1,E_2\in\mathfrak M$ e $E_1\cap E_2=\emptyset$ então:}
\[\mu^*(E_1\cup E_2)=\mu^*(E_1)+\mu^*(E_2).\]

Basta tomar $A=E_1\cup E_2$ em \eqref{eq:mustarpasso2}.

\item\label{itm:mustardiffprop}
{\em Se $E_1,E_2\in\mathfrak M$ e $E_1\subset E_2$ então $E_2\setminus E_1\in\mathfrak M$}.

Seja dado $A\in\mathcal H$. Evidentemente:
\begin{equation}\label{eq:mustar3}
\mu^*\big(A\cap(E_2\setminus E_1)\big)=\mu^*(A\cap E_2\cap E_1^\compl);
\end{equation}
como $E_1\subset E_2$, temos $(E_2\setminus E_1)^\compl=E_1\cup E_2^\compl$ e portanto:
\begin{multline}\label{eq:mustar4}
\mu^*\big(A\cap(E_2\setminus E_1)^\compl\big)=\mu^*\big((A\cap E_1)\cup(A\cap E_2^\compl)\big)
\\\le\mu^*(A\cap E_1)+\mu^*(A\cap E_2^\compl)
=\mu^*(A\cap E_2\cap E_1)+\mu^*(A\cap E_2^\compl).
\end{multline}
Somando \eqref{eq:mustar3} e \eqref{eq:mustar4} obtemos:
\begin{multline*}
\mu^*\big(A\cap(E_2\setminus E_1)\big)+\mu^*\big(A\cap(E_2\setminus E_1)^\compl\big)
\le\mu^*(A\cap E_2\cap E_1^\compl)+\mu^*(A\cap E_2\cap E_1)\\
+\mu^*(A\cap E_2^\compl)=\mu^*(A\cap E_2)+\mu^*(A\cap E_2^\compl)=\mu^*(A),
\end{multline*}
o que prova que $E_2\setminus E_1\in\mathfrak M$.

\item\label{itm:mustardiff}
{\em Se $E_1,E_2\in\mathfrak M$ então $E_2\setminus E_1\in\mathfrak M$}.

Pelo passo~\ref{itm:mustaruniao1}, temos $E_1\cup E_2\in\mathfrak M$;
como $E_1\subset E_1\cup E_2$, segue do passo~\ref{itm:mustardiffprop} que
$(E_1\cup E_2)\setminus E_1\in\mathfrak M$. Mas $(E_1\cup E_2)\setminus E_1=E_2\setminus E_1$.

\item\label{itm:mustarcountun} {\em Se $(E_k)_{k\ge1}$ é uma seqüência de elementos dois a dois disjuntos de
$\mathfrak M$ então $\bigcup_{k=1}^\infty E_k\in\mathfrak M$}.

Usando indução e os passos~\ref{itm:mustaruniao1} e \ref{itm:mustaruniao2} obtemos
que $\bigcup_{k=1}^tE_k\in\mathfrak M$ e:
\[\mu^*\Big(A\cap\bigcup_{k=1}^tE_k\Big)=\sum_{k=1}^t\mu^*(A\cap E_k),\]
para todo $A\in\mathcal H$ e todo $t\ge1$; daí:
\begin{multline*}
\mu^*(A)=\mu^*\Big(A\cap\bigcup_{k=1}^tE_k\Big)+\mu^*\Big(A\cap\Big(\bigcup_{k=1}^tE_k\Big)^{\!\compl\,}\Big)\\
=\Big(\sum_{k=1}^t\mu^*(A\cap E_k)\Big)+\mu^*\Big(A\cap\Big(\bigcup_{k=1}^tE_k\Big)^{\!\compl\,}\Big).
\end{multline*}
Como $A\cap\big(\bigcup_{k=1}^tE_k\big)^\compl\supset A\cap\big(\bigcup_{k=1}^\infty E_k\big)^\compl$, obtemos:
\begin{equation}\label{eq:mustar5}
\mu^*(A)\ge\Big(\sum_{k=1}^t\mu^*(A\cap E_k)\Big)+\mu^*\Big(A\cap\Big(\bigcup_{k=1}^\infty E_k\Big)^{\!\compl\,}\Big);
\end{equation}
fazendo $t\to\infty$ em \eqref{eq:mustar5} vem:
\begin{multline}\label{eq:mustar6}
\mu^*(A)\ge\Big(\sum_{k=1}^\infty\mu^*(A\cap E_k)\Big)+\mu^*\Big(A\cap\Big(\bigcup_{k=1}^\infty E_k\Big)^{\!\compl\,}\Big)\\
\ge\mu^*\Big(A\cap\bigcup_{k=1}^\infty E_k\Big)+\mu^*\Big(A\cap\Big(\bigcup_{k=1}^\infty E_k\Big)^{\!\compl\,}\Big)\ge\mu^*(A),
\end{multline}
onde nas duas últimas desigualdades usamos a $\sigma$-subaditividade de $\mu^*$.
Isso prova que $\bigcup_{k=1}^\infty E_k\in\mathfrak M$.

\item\label{itm:mustarcountunn}
{\em Se $(E_k)_{k\ge1}$ é uma seqüência em $\mathfrak M$ então $\bigcup_{k=1}^\infty E_k\in\mathfrak M$}.

Para cada $k\ge1$, seja $F_k=E_k\setminus\bigcup_{i=0}^{k-1}E_i$, onde $E_0=\emptyset$.
Segue dos passos~\ref{itm:mustaruniao1} e \ref{itm:mustardiff} que $F_k\in\mathfrak M$,
para todo $k\ge1$. Além do mais, pelo resultado do Exercício~\ref{exe:disjuntar},
os conjuntos $(F_k)_{k\ge1}$ são dois a dois disjuntos e $\bigcup_{k=1}^\infty E_k=
\bigcup_{k=1}^\infty F_k$. Segue então do passo~\ref{itm:mustarcountun} que
$\bigcup_{k=1}^\infty E_k\in\mathfrak M$.

\item {\em O item~(a) da tese do teorema vale}.

Segue dos passos~\ref{itm:mustardiff} e \ref{itm:mustarcountunn}, já que obviamente
$\emptyset\in\mathfrak M$.

\item {\em O item~(b) da tese do teorema vale}.

Segue de \eqref{eq:mustar6} que:
\begin{equation}\label{eq:mustar7}
\mu^*(A)=\Big(\sum_{k=1}^\infty\mu^*(A\cap E_k)\Big)+\mu^*\Big(A\cap\Big(\bigcup_{k=1}^\infty E_k\Big)^{\!\compl\,}\Big),
\end{equation}
para qualquer $A\in\mathcal H$ e para qualquer seqüência $(E_k)_{k\ge1}$ de elementos dois a dois
disjuntos de $\mathfrak M$. A conclusão é obtida substituindo $A$ por
$A\cap\big(\bigcup_{k=1}^\infty E_k\big)$ em \eqref{eq:mustar7}.

\item {\em O item~(c) da tese do teorema vale}.

Basta fazer $A=\bigcup_{k=1}^\infty E_k$ em \eqref{eq:supersigmaadd}.

\item {\em O item~(d) da tese do teorema vale}.

Sejam $E\in\mathcal H$ com $\mu^*(E)=0$ e $A\in\mathcal H$. Segue da monotonicidade
de $\mu^*$ que:
\[\mu^*(A\cap E)+\mu^*(A\cap E^\compl)\le\mu^*(E)+\mu^*(A)=\mu^*(A).\]
Logo $E$ é $\mu^*$-mensurável.\qedhere
\end{stepindent}
\end{proof}

\begin{rem}
Seja $\mu^*:\wp(X)\to[0,+\infty]$ uma medida exterior, onde $X$ é um conjunto
arbitrário. É imediato que o próprio conjunto $X$ é $\mu^*$-mensurável.
Segue então do Teorema~\ref{thm:teomustarmens} que a coleção de conjuntos
$\mu^*$-mensuráveis é uma $\sigma$-álgebra de partes de $X$ (veja o Exercício~\ref{exe:anelalgebra}).
\end{rem}

\begin{example}
É bem possível que o $\sigma$-anel de conjuntos $\mu^*$-men\-su\-rá\-veis associado a uma medida
exterior $\mu^*$ seja completamente trivial. De fato, seja $X$ um conjunto arbitrário
e seja $\mathcal H\subset\wp(X)$ o $\sigma$-anel hereditário constituído pelos subconjuntos
enumeráveis de $X$. Defina $\mu^*:\mathcal H\to[0,+\infty]$ fazendo:
\[\mu^*(A)=\begin{cases}
\vert A\vert+1,&\text{se $A\ne\emptyset$},\\
\hfil0,&\text{se $A=\emptyset$},
\end{cases}\]
onde $\vert A\vert\in\N\cup\{+\infty\}$ denota o número de elementos de $A$. É fácil ver
que $\mu^*$ é uma medida exterior em $\mathcal H$. Seja $E\in\mathcal H$ com $E\ne\emptyset$
e $E\ne X$; podemos então escolher um ponto $x\in E$ e um ponto $y\in X\setminus E$.
Tomando $A=\{x,y\}\in\mathcal H$ então:
\[\mu^*(A\cap E)+\mu^*(A\setminus E)=2+2\ne3=\mu^*(A),\]
donde vemos que $E$ não é $\mu^*$-mensurável. Concluímos então que, se $X$ é enumerável
(de modo que $\mathcal H=\wp(X)$) então a $\sigma$-álgebra de conjuntos $\mu^*$-mensuráveis
é igual a $\{\emptyset,X\}$; se $X$ é não enumerável então o $\sigma$-anel de conjuntos
$\mu^*$-mensuráveis é $\{\emptyset\}$.
\end{example}

\begin{defin}\label{thm:sigmaherger}
Se $\mathcal C$ é uma classe de conjuntos arbitrária então o {\em $\sigma$-anel hereditário
gerado por $\mathcal C$\/}\index[indice]{sigma anel@$\sigma$-anel!hereditario@hereditário!gerado por uma colecao de conjuntos@gerado por uma coleção\hfil\break
de conjuntos} é o menor $\sigma$-anel hereditário $\mathcal H$ que contém $\mathcal C$, i.e., $\mathcal H$ é um $\sigma$-anel
hereditário tal que:
\begin{enumerate}
\item $\mathcal C\subset\mathcal H$;
\item\label{itm:anelherger2} se $\mathcal H'$ é um $\sigma$-anel hereditário tal que
$\mathcal C\subset\mathcal H'$ então $\mathcal H\subset\mathcal H'$.
\end{enumerate}
Dizemos também que $\mathcal C$ é um {\em conjunto de geradores\/}\index[indice]{conjunto!de geradores!para um sigma anel hereditario@para um $\sigma$-anel hereditário}%
\index[indice]{geradores!para um sigma anel hereditario@para um $\sigma$-anel hereditário}
para o $\sigma$-anel hereditário $\mathcal H$.
\end{defin}
A existência e unicidade do $\sigma$-anel hereditário gerado por uma classe de conjuntos
$\mathcal C$ pode ser demonstrada usando exatamente o mesmo roteiro que foi descrito
nos Exercícios~\ref{exe:sigmagerada}, \ref{exe:algerada}, \ref{exe:anelgerado} e \ref{exe:sigmonotgerado}.
No entanto, nós mostraremos a existência do $\sigma$-anel hereditário gerado por $\mathcal C$
exibindo explicitamente esse $\sigma$-anel hereditário (a unicidade do $\sigma$-anel hereditário
gerado por $\mathcal C$ é evidente). Se $\mathcal C=\emptyset$ então o $\sigma$-anel
hereditário gerado por $\mathcal C$ é $\mathcal H=\{\emptyset\}$; senão, temos o seguinte:
\begin{lem}
Seja $\mathcal C$ uma classe de conjuntos não vazia. O $\sigma$-anel hereditário gerado
por $\mathcal C$ é igual a:
\begin{equation}\label{eq:anelherger}
\mathcal H=\Big\{A:\text{existe uma seqüência $(A_k)_{k\ge1}$ em $\mathcal C$ com $A\subset\bigcup_{k=1}^\infty A_k$}\Big\}.
\end{equation}
\end{lem}
\begin{proof}
Pelo resultado do Exercício~\ref{exe:sigmanoheredit}, $\mathcal H$ é um $\sigma$-anel que
contém $\mathcal C$. Obviamente, se $A\in\mathcal H$ e $B\subset A$ então $B\in\mathcal H$,
de modo que $\mathcal H$ é um $\sigma$-anel hereditário que contém $\mathcal C$.
Para verificar a condição~\eqref{itm:anelherger2} que aparece na Definição~\ref{thm:sigmaherger},
observe que se $\mathcal H'$ é um $\sigma$-anel hereditário que contém $\mathcal C$ então
$\bigcup_{k=1}^\infty A_k\in\mathcal H'$, para toda seqüência $(A_k)_{k\ge1}$ de elementos
de $\mathcal C$; logo todo subconjunto de $\bigcup_{k=1}^\infty A_k$ está
em $\mathcal H'$, donde $\mathcal H\subset\mathcal H'$.
\end{proof}

\begin{example}\label{exa:mustarassocmu}
Seja $\mathcal C$ uma classe de conjuntos tal que $\emptyset\in\mathcal C$ e seja
dada uma aplicação $\mu:\mathcal C\to[0,+\infty]$ (não necessariamente uma medida)
tal que $\mu(\emptyset)=0$. Seja $\mathcal H$ (veja \eqref{eq:anelherger}) o
$\sigma$-anel hereditário gerado por $\mathcal C$. Vamos
definir uma medida exterior $\mu^*$ em $\mathcal H$ associada a $\mu$. Para
cada $A\in\mathcal H$, seja:
\[\mathcal C_\mu(A)=\Big\{\sum_{k=1}^\infty\mu(A_k):A\subset\bigcup_{k=1}^\infty A_k,\ A_k\in\mathcal C,\
\text{para todo $k\ge1$}\Big\},\]\index[simbolos]{$\mathcal C_\mu(A)$}
e defina:
\[\mu^*(A)=\inf\,\mathcal C_\mu(A).\]
Evidentemente, $\mu^*(A)\in[0,+\infty]$, para todo $A\in\mathcal H$ (note que
$\mathcal C_\mu(A)\ne\emptyset$, já que $A\in\mathcal H$). Observe também que para todo $A\in\mathcal C$, temos:
\begin{equation}\label{eq:mustarleqmu}
\mu^*(A)\le\mu(A);
\end{equation}
de fato, basta tomar $A_1=A$ e $A_k=\emptyset$ para todo $k\ge2$ para ver que $\mu(A)$ está em $\mathcal C_\mu(A)$.
Vamos mostrar que $\mu^*$ é uma medida
exterior em $\mathcal H$. De \eqref{eq:mustarleqmu} segue que $\mu^*(\emptyset)=0$.
Se $A,B\in\mathcal H$ e $A\subset B$ então $\mathcal C_\mu(B)\subset\mathcal C_\mu(A)$, donde $\mu^*(A)\le\mu^*(B)$,
provando a monotonicidade de $\mu^*$. Finalmente, provemos a $\sigma$-subaditividade
de $\mu^*$. Seja $(A_k)_{k\ge1}$ uma seqüência em $\mathcal H$. Dado $\varepsilon>0$
então para todo $k\ge1$ existe uma seqüência $(A_{ki})_{i\ge1}$ em $\mathcal C$
tal que:
\[A_k\subset\bigcup_{i=1}^\infty A_{ki}\quad\text{e}\quad
\sum_{i=1}^\infty\mu(A_{ki})\le\mu^*(A_k)+\frac\varepsilon{2^k}.\]
Daí $\bigcup_{k=1}^\infty A_k\subset\bigcup_{k=1}^\infty\bigcup_{i=1}^\infty A_{ki}$
e $\sum_{k=1}^\infty\sum_{i=1}^\infty\mu(A_{ki})\in\mathcal C_\mu\big(\bigcup_{k=1}^\infty A_k\big)$; portanto:
\[\mu^*\Big(\bigcup_{k=1}^\infty A_k\Big)\le\sum_{k=1}^\infty\sum_{i=1}^\infty\mu(A_{ki})
\le\sum_{k=1}^\infty\Big(\mu^*(A_k)+\frac\varepsilon{2^k}\Big)=\Big(\sum_{k=1}^\infty\mu^*(A_k)\Big)
+\varepsilon.\]
Como $\varepsilon>0$ é arbitrário, a $\sigma$-subaditividade de $\mu^*$ segue.
\end{example}

\begin{defin}
Seja $\mathcal C$ uma classe de conjuntos tal que $\emptyset\in\mathcal C$ e seja
dada uma aplicação $\mu:\mathcal C\to[0,+\infty]$ (não necessariamente uma medida)
tal que $\mu(\emptyset)=0$. Se $\mathcal H$ denota o $\sigma$-anel hereditário gerado por
$\mathcal C$ então a medida exterior $\mu^*:\mathcal H\to[0,+\infty]$ definida como
no Exemplo~\ref{exa:mustarassocmu} é chamada a
{\em medida exterior determinada por $\mu$}.\index[indice]{medida!exterior!determinada por mu@determinada por $\mu$}
\end{defin}

\begin{example}
Se $\mathcal C\subset\wp(\R^n)$ é a classe dos blocos retangulares $n$-dimensionais e
se $\mu:\mathcal C\to[0,+\infty]$ é a aplicação que associa a cada bloco retangular $n$-dimensional seu volume
então o $\sigma$-anel hereditário gerado por $\mathcal C$ é $\wp(\R^n)$ e a medida exterior $\mu^*$
determinada por $\mu$ coincide com a medida exterior de Lebesgue em $\R^n$.
\end{example}

Observamos que, mesmo se $\mu:\mathcal C\to[0,+\infty]$ é uma medida, é possível que tenhamos uma desigualdade
estrita em \eqref{eq:mustarleqmu} (veja Exercício~\ref{exe:mustarlessmu}). No entanto, temos o seguinte:
\begin{lem}\label{thm:mustarehmunoS}
Seja $\mathcal S$ um semi-anel e $\mu:\mathcal S\to[0,+\infty]$ uma medida em $\mathcal S$.
Denote por $\mathcal H$ o $\sigma$-anel hereditário gerado por $\mathcal S$ e por
$\mu^*:\mathcal H\to[0,+\infty]$ a medida exterior determinada por $\mu$. Então:
\begin{itemize}
\item[(a)] $\mu^*(A)=\mu(A)$, para todo $A\in\mathcal S$;
\item[(b)] todo elemento de $\mathcal S$ é $\mu^*$-mensurável.
\end{itemize}
\end{lem}
\begin{proof}
Seja $\mathcal R$ o anel gerado por $\mathcal S$ e seja $\tilde\mu:\mathcal R\to[0,+\infty]$
a única medida em $\mathcal R$ que estende $\mu$ (veja Teorema~\ref{thm:peqteoext}).
Nós vamos mostrar que $\mu^*(A)=\tilde\mu(A)$, para todo $A\in\mathcal R$; isso implicará
em particular que o item~(a) vale. Seja $A\in\mathcal R$ e sejam $A_1,\ldots,A_t\in\mathcal S$
dois a dois disjuntos de modo que $A=\bigcup_{k=1}^tA_k$ (veja Lema~\ref{thm:semigeraanel}).
Tomando $A_k=\emptyset$ para $k>t$ concluímos que:
\[\mu^*(A)\le\sum_{k=1}^\infty\mu(A_k)=\sum_{k=1}^t\mu(A_k)=\tilde\mu(A).\]
Por outro lado, se $(B_k)_{k\ge1}$ é uma seqüência em $\mathcal S$ tal que
$A\subset\bigcup_{k=1}^\infty B_k$ então segue do item~(f) do Lema~\ref{thm:classerazoavel} aplicado
à medida $\tilde\mu$ que:
\[\tilde\mu(A)\le\sum_{k=1}^\infty\tilde\mu(B_k)=\sum_{k=1}^\infty\mu(B_k),\]
donde $\tilde\mu(A)\le\mu^*(A)$. Provamos então que $\mu^*(A)=\tilde\mu(A)$.
Seja agora $E\in\mathcal S$ e provemos que $E$ é $\mu^*$-mensurável. Dado $A\in\mathcal H$
arbitrariamente, devemos verificar que $\mu^*(A)\ge\mu^*(A\cap E)+\mu^*(A\setminus E)$.
Para isso, é suficiente mostrar que para toda seqüência $(A_k)_{k\ge1}$ em $\mathcal S$
com $A\subset\bigcup_{k=1}^\infty A_k$ vale a desigualdade:
\begin{equation}\label{eq:emSehmens1}
\sum_{k=1}^\infty\mu(A_k)\ge\mu^*(A\cap E)+\mu^*(A\setminus E).
\end{equation}
Temos $A_k\cap E,A_k\setminus E\in\mathcal R$ e $A_k=(A_k\cap E)\cup(A_k\setminus E)$,
para todo $k\ge1$ e portanto:
\[\mu(A_k)=\tilde\mu(A_k)=\tilde\mu(A_k\cap E)+\tilde\mu(A_k\setminus E)
=\mu^*(A_k\cap E)+\mu^*(A_k\setminus E);\]
daí:
\begin{equation}\label{eq:emSehmens2}
\sum_{k=1}^\infty\mu(A_k)=\sum_{k=1}^\infty\mu^*(A_k\cap E)+\sum_{k=1}^\infty
\mu^*(A_k\setminus E).
\end{equation}
Como $A\cap E\subset\bigcup_{k=1}^\infty(A_k\cap E)$ e
$A\setminus E\subset\bigcup_{k=1}^\infty(A_k\setminus E)$, segue da monotonicidade
e da $\sigma$-subaditividade de $\mu^*$ que:
\begin{equation}\label{eq:emSehmens3}
\mu^*(A\cap E)\le\sum_{k=1}^\infty\mu^*(A_k\cap E),\quad
\mu^*(A\setminus E)\le\sum_{k=1}^\infty\mu^*(A_k\setminus E).
\end{equation}
De \eqref{eq:emSehmens2} e \eqref{eq:emSehmens3} segue \eqref{eq:emSehmens1}, o que prova
que $E$ é $\mu^*$-mensurável.
\end{proof}

Como conseqüência direta do Lema~\ref{thm:mustarehmunoS} e do Teorema~\ref{thm:teomustarmens}
obtemos o seguinte:
\begin{teo}[teorema da extensão]\index[indice]{teorema!da extensao@da extensão}%
\index[indice]{extensao@extensão!teorema da}\label{thm:teoextensao}
Seja $\mu:\mathcal S\to[0,+\infty]$ uma medida num semi-anel $\mathcal S$. Então
$\mu$ estende-se a uma medida no $\sigma$-anel gerado por $\mathcal S$; se $\mu$ é
$\sigma$-finita então essa extensão é única e $\sigma$-finita.
\end{teo}
\begin{proof}
Seja $\mu^*$ a medida exterior determinada por $\mu$;
pelo Teorema~\ref{thm:teomustarmens}, a coleção $\mathfrak M$ dos conjuntos $\mu^*$-mensuráveis
é um $\sigma$-anel e a restrição de $\mu^*$ a $\mathfrak M$ é uma medida. Mas, pelo Lema~\ref{thm:mustarehmunoS},
$\mathfrak M$ contém $\mathcal S$ e $\mu^*$ é uma extensão de $\mu$; logo $\mathfrak M$
contém o $\sigma$-anel gerado por $\mathcal S$ e a restrição de $\mu^*$ a esse $\sigma$-anel
é uma medida que estende $\mu$. Se $\mu$ é $\sigma$-finita então essa extensão é única e $\sigma$-finita,
pelo Corolário~\ref{thm:coruniquext}.
\end{proof}

\begin{example}\label{exa:Lebesgueagain}
Seja $\mathcal S$ o semi-anel constituído pelos intervalos da forma $\left]a,b\right]$, $a,b\in\R$
(veja \eqref{eq:semiintervalos}) e seja $\mu:\mathcal S\to[0,+\infty]$ definida por
$\mu\big(\left]a,b\right]\big)=b-a$, para todos $a,b\in\R$ com $a\le b$. Segue da Proposição~\ref{thm:carmedLebSti}
que $\mu$ é uma medida em $\mathcal S$. O $\sigma$-anel $\mathcal A=\Borel(\R)$ gerado por $\mathcal S$ é precisamente a $\sigma$-álgebra
de Borel de $\R$ (veja Exercício~\ref{exe:anelBorelint}). Como $\mu$ é finita (e portanto $\sigma$-finita),
o Teorema~\ref{thm:teoextensao} nos diz que $\mu$ estende-se de modo único a uma medida em $\Borel(\R)$.
Note que a medida de Lebesgue $\leb:\Lebmens(\R)\to[0,+\infty]$ construída na Seção~\ref{sec:MedLebRn}
restrita ao $\sigma$-anel $\Borel(\R)$ é uma medida em $\Borel(\R)$ que estende $\mu$. Concluímos então
que a restrição da medida de Lebesgue $\leb$ a $\Borel(\R)$ é precisamente a única extensão da medida
$\mu$ ao $\sigma$-anel $\Borel(\R)$.
\end{example}

\begin{example}
Sejam $\mathcal S$ o semi-anel constituído pelos intervalos da forma $\left]a,b\right]$, $a,b\in\R$
(veja \eqref{eq:semiintervalos}), $F:\R\to\R$ uma função crescente e contínua à direita e
$\mu_F:\mathcal S\to[0,+\infty]$ definida por $\mu_F\big(\left]a,b\right]\big)=F(b)-F(a)$, para todos
$a,b\in\R$ com $a\le b$. Segue da Proposição~\ref{thm:carmedLebSti} que $\mu_F$ é uma medida em $\mathcal S$.
Como $\mu_F$ é finita (e portanto $\sigma$-finita), o Teorema~\ref{thm:teoextensao} nos diz que $\mu_F$ estende-se
de modo único a uma medida (também $\sigma$-finita) no $\sigma$-anel $\mathcal A=\Borel(\R)$ gerado por $\mathcal S$.
Vamos denotar essa extensão de $\mu_F$ também por $\mu_F$. A medida $\mu_F:\Borel(\R)\to[0,+\infty]$\index[simbolos]{$\mu_F$} é chamada
a {\em medida de Lebesgue--Stieltjes\/}\index[indice]{medida!de Lebesgue--Stieltjes}%
\index[indice]{Lebesgue--Stieltjes!medida de} associada à função crescente e contínua à direita $F:\R\to\R$.
Note que se $\mu:\Borel(\R)\to[0,+\infty]$ é uma medida arbitrária que seja finita sobre intervalos limitados
então a Proposição~\ref{thm:propmuF} nos diz que existe uma função crescente $F:\R\to\R$ tal que
$\mu_F=\mu\vert_{\mathcal S}$; a função $F$ é única, a menos da possível adição de constantes.
Como $\mu\vert_{\mathcal S}$ é uma medida $\sigma$-aditiva, a Proposição~\ref{thm:carmedLebSti} nos diz que
$F$ é contínua à direita. Temos portanto que $\mu$ é a única extensão de $\mu_F:\mathcal S\to[0,\infty]$ a
$\Borel(\R)$. Concluímos então que {\em toda medida em $\Borel(\R)$ que seja finita sobre intervalos limitados
é a medida de Lebesgue--Stieltjes associada a alguma função crescente e contínua à direita $F:\R\to\R$; a função
$F$ é única, a menos da possível adição de constantes}.
\end{example}

Note que se $\mu:\mathcal S\to[0,+\infty]$ é uma medida num semi-anel $\mathcal S$ então
a extensão de $\mu$ que construímos está definida num $\sigma$-anel $\mathfrak M$ que
pode ser maior do que o $\sigma$-anel gerado por $\mathcal S$. Por exemplo,
se $\mathcal S$ e $\mu$ são definidos como no Exemplo~\ref{exa:Lebesgueagain} então
o $\sigma$-anel gerado por $\mathcal S$ é a $\sigma$-álgebra
de Borel da reta, mas o $\sigma$-anel $\mathfrak M$ de conjuntos
$\mu^*$-mensuráveis coincide com a $\sigma$-álgebra de conjuntos Lebesgue mensuráveis
(veja Exercício~\ref{exe:medextehLeb} e Observação~\ref{thm:lebstarmens}).

Vamos agora investigar um pouco mais a fundo o $\sigma$-anel $\mathfrak M$ e a extensão de
$\mu$ definida em $\mathfrak M$.

\begin{lem}\label{thm:geralenvelopes}
Sejam $\mathcal S$ um semi-anel, $\mu:\mathcal S\to[0,+\infty]$ uma medida em $\mathcal S$,
$\mathcal H$ o $\sigma$-anel hereditário gerado por $\mathcal S$, $\mu^*:\mathcal H\to[0,+\infty]$
a medida exterior determinada por $\mu$, $\mathfrak M$ o $\sigma$-anel
de conjuntos $\mu^*$-mensuráveis e $\mathcal A$ o $\sigma$-anel gerado por $\mathcal S$.
Então:
\begin{itemize}
\item[(a)] para todo $A\in\mathcal H$ existe $E\in\mathcal A$ tal que
$A\subset E$ e $\mu^*(A)=\mu^*(E)$;
\item[(b)] se $A\in\mathfrak M$ é $\sigma$-finito com respeito à medida $\mu^*\vert_{\mathfrak M}$
então existem $E,W\in\mathcal A$ tais que $W\subset A\subset E$,
e $\mu^*(A\setminus W)=\mu^*(E\setminus A)=0$.
\end{itemize}
\end{lem}
\begin{proof}
Começamos provando o item~(a). Seja $A\in\mathcal H$. Para todo $n\ge1$ existe
uma seqüência $(A_{nk})_{k\ge1}$ em $\mathcal S$ tal que $A\subset\bigcup_{k=1}^\infty A_{nk}$
e $\sum_{k=1}^\infty\mu(A_{nk})\le\mu^*(A)+\frac1n$; se $A_n=\bigcup_{k=1}^\infty A_{nk}$
então $A_n\in\mathcal A$, $A\subset A_n$ e da $\sigma$-subaditividade de $\mu^*$ vem:
\[\mu^*(A_n)\le\sum_{k=1}^\infty\mu^*(A_{nk})=\sum_{k=1}^\infty\mu(A_{nk})\le\mu^*(A)+\frac1n.\]
Tomando $E=\bigcap_{n=1}^\infty A_n$ então $E\in\mathcal A$, $A\subset E$ e:
\[\mu^*(A)\le\mu^*(E)\le\mu^*(A_n)\le\mu^*(A)+\frac1n,\]
para todo $n\ge1$; daí $\mu^*(E)=\mu^*(A)$, o que prova o item~(a). Passemos à prova
do item~(b). Seja $A\in\mathfrak M$ um conjunto $\sigma$-finito com respeito à medida
$\mu^*\vert_{\mathfrak M}$. Existe então uma seqüência $(X_k)_{k\ge1}$ em $\mathfrak M$
com $A\subset\bigcup_{k=1}^\infty X_k$ e $\mu^*(X_k)<+\infty$, para todo $k\ge1$.
Para cada $k\ge1$, como $A\cap X_k\in\mathfrak M\subset\mathcal H$, o item~(a) nos dá
$E_k\in\mathcal A$ com $A\cap X_k\subset E_k$ e $\mu^*(A\cap X_k)=\mu^*(E_k)$.
Como $A\cap X_k,E_k\in\mathfrak M$, $\mu^*\vert_{\mathfrak M}$ é uma medida
e $\mu^*(A\cap X_k)<+\infty$, obtemos:
\[\mu^*\big(E_k\setminus(A\cap X_k)\big)=\mu^*(E_k)-\mu^*(A\cap X_k)=0,\]
para todo $k\ge1$. Seja $E=\bigcup_{k=1}^\infty E_k\in\mathcal A$. Evidentemente:
\[A=\bigcup_{k=1}^\infty(A\cap X_k)\subset\bigcup_{k=1}^\infty E_k=E;\]
além do mais:
\[E\setminus A\subset\bigcup_{k=1}^\infty\big(E_k\setminus(A\cap X_k)\big),\]
e portanto:
\[\mu^*(E\setminus A)\le\sum_{k=1}^\infty\mu^*\big(E_k\setminus(A\cap X_k)\big)=0.\]
Vamos agora mostrar a existência de $W\in\mathcal A$ com $W\subset A$ e $\mu^*(A\setminus W)=0$.
Como $E\setminus A\in\mathcal H$, o item~(a) nos dá $N\in\mathcal A$ com
$E\setminus A\subset N$ e:
\[\mu^*(N)=\mu^*(E\setminus A)=0.\]
Tome $W=E\setminus N$; temos que $W\in\mathcal A$ e $W\subset A$. Além do mais, $A\setminus W\subset N$
e portanto $\mu^*(A\setminus W)\le\mu^*(N)=0$. Isso completa a demonstração.
\end{proof}

\begin{rem}
Pode ser interessante para o leitor comparar o enunciado do Lema~\ref{thm:geralenvelopes} aos enunciados
dos Lemas~\ref{thm:aproxGdelta}, \ref{thm:Gdeltazero} e do Corolário~\ref{thm:innerFsigma}.
\end{rem}

Vimos no Exemplo~\ref{exa:naobastanocalA} que uma medida não $\sigma$-finita
$\mu$ num semi-anel $\mathcal S$ pode admitir extensões $\sigma$-finitas para o $\sigma$-anel
gerado por $\mathcal S$. No entanto, se consideramos apenas a forma específica de construir extensões
que foi desenvolvida nesta seção então temos o seguinte:
\begin{lem}\label{thm:sigmatudoigual}
Sob as condições do Lema~\ref{thm:geralenvelopes}, as seguintes afirmações são equivalentes:
\begin{itemize}
\item[(a)] a medida $\mu:\mathcal S\to[0,+\infty]$ é $\sigma$-finita;
\item[(b)] a medida $\mu^*\vert_{\mathcal A}$ é $\sigma$-finita;
\item[(c)] a medida $\mu^*\vert_{\mathfrak M}$ é $\sigma$-finita.
\end{itemize}
\end{lem}
\begin{proof}\
\begin{bulletindent}
\item (a)$\Rightarrow$(b).

Segue da Observação~\ref{thm:tambemehsigmafin}.

\item (b)$\Rightarrow$(c).

Dado $A\in\mathfrak M$ então $A\in\mathcal H$ e portanto $A$ está contido numa união enumerável
de elementos de $\mathcal S$; segue então que $A$ está contido num elemento de $\mathcal A$.
A conclusão é obtida observando que todo elemento de $\mathcal A$ está contido numa união
enumerável de elementos de $\mathcal A$ de medida finita.

\item (c)$\Rightarrow$(a).

Dado $A\in\mathcal S$ então $A\in\mathfrak M$ e portanto existe uma seqüência $(A_k)_{k\ge1}$ em $\mathfrak M$
tal que $A\subset\bigcup_{k=1}^\infty A_k$ e $\mu^*(A_k)<+\infty$, para todo $k\ge1$.
Pela definição de $\mu^*$, se $\mu^*(A_k)<+\infty$ então $A_k$ está contido numa união
enumerável de elementos de $\mathcal S$ de medida finita; logo $A$ está contido numa união
enumerável de elementos de $\mathcal S$ de medida finita.\qedhere
\end{bulletindent}
\end{proof}

\begin{example}
Seja $X$ um conjunto não vazio. Considere o semi-anel $\mathcal S=\{\emptyset,X\}$ e
a medida $\mu:\mathcal S\to[0,+\infty]$ definida por $\mu(\emptyset)=0$ e $\mu(X)=+\infty$.
Temos que o $\sigma$-anel $\mathcal A$ gerado por $\mathcal S$ é igual a $\mathcal S$ e o
$\sigma$-anel hereditário gerado por $\mathcal S$ é igual a $\mathcal H=\wp(X)$.
É fácil ver que a medida exterior
$\mu^*:\wp(X)\to[0,+\infty]$ determinada por $\mu$
é dada por $\mu^*(\emptyset)=0$ e $\mu^*(A)=+\infty$, para todo $A\subset X$ não vazio.
Temos então que o $\sigma$-anel $\mathfrak M$ de conjuntos $\mu^*$-mensuráveis
é igual a $\wp(X)$. Esse exemplo ilustra a necessidade da hipótese de $\sigma$-finitude
no item~(b) do Lema~\ref{thm:geralenvelopes}.
\end{example}

\begin{subsection}{Completamento de medidas}

Seja $\mu:\mathcal A\to[0,+\infty]$ uma medida num $\sigma$-anel $\mathcal A$. É perfeitamente
possível que exista um conjunto $A\in\mathcal A$ com $\mu(A)=0$ tal que nem todo subconjunto
de $A$ está em $\mathcal A$.

\begin{defin}
Uma medida $\mu:\mathcal A\to[0,+\infty]$ num $\sigma$-anel $\mathcal A$ é dita
{\em completa\/}\index[indice]{medida!completa}\index[indice]{completa!medida} se para todo
$A\in\mathcal A$ com $\mu(A)=0$ e para todo $B\subset A$ temos $B\in\mathcal A$.
\end{defin}

\begin{prop}\label{thm:vemdemustarecompl}
Seja $\mu^*:\mathcal H\to[0,+\infty]$ uma medida exterior num $\sigma$-anel hereditário
$\mathcal H$. Se $\mathfrak M$ denota a coleção dos conjuntos $\mu^*$-mensuráveis então
a medida $\mu^*\vert_{\mathfrak M}$ é completa.
\end{prop}
\begin{proof}
Segue diretamente do item~(d) do Teorema~\ref{thm:teomustarmens}.
\end{proof}

\begin{lem}\label{thm:lemacompletar}
Seja $\mu:\mathcal A\to[0,+\infty]$ uma medida num $\sigma$-anel $\mathcal A$.
A classe de conjuntos:
\[\overline{\mathcal A}=\big\{A\cup N:\text{$A\in\mathcal A$ e existe $M\in\mathcal A$
com $N\subset M$ e $\mu(M)=0$}\big\}\]
é um $\sigma$-anel que contém $\mathcal A$ e existe uma única medida $\bar\mu:\overline{\mathcal A}\to[0,+\infty]$
em $\overline{\mathcal A}$ que estende $\mu$. A medida $\bar\mu$ é a {\em menor extensão
completa\/} de $\mu$, no sentido que:
\begin{itemize}
\item $\bar\mu$ é uma medida completa;
\item se $\mu':\mathcal A'\to[0,+\infty]$ é uma medida completa num $\sigma$-anel $\mathcal A'$
contendo $\mathcal A$ e se $\mu'$ estende $\mu$ então $\mathcal A'$ contém $\overline{\mathcal A}$
e $\mu'$ estende $\bar\mu$.
\end{itemize}
\end{lem}
\begin{proof}
Evidentemente $\mathcal A\subset\overline{\mathcal A}$. Seja $(A_k\cup N_k)_{k\ge1}$
uma se\-qüên\-cia em $\overline{\mathcal A}$, onde para cada $k\ge1$,
$A_k\in\mathcal A$ e existe $M_k\in\mathcal A$ com $N_k\subset M_k$ e $\mu(M_k)=0$.
Temos:
\begin{equation}\label{eq:countunbarcalA}
\bigcup_{k=1}^\infty(A_k\cup N_k)=\Big(\bigcup_{k=1}^\infty A_k\Big)\cup
\Big(\bigcup_{k=1}^\infty N_k\Big)\in\overline{\mathcal A},
\end{equation}
já que $\bigcup_{k=1}^\infty A_k\in\mathcal A$, $\bigcup_{k=1}^\infty N_k\subset\bigcup_{k=1}^\infty M_k\in\mathcal A$
e:
\[\mu\Big(\bigcup_{k=1}^\infty M_k\Big)\le\sum_{k=1}^\infty\mu(M_k)=0.\]
Sejam $A_1\cup N_1,A_2\cup N_2\in\overline{\mathcal A}$, com $A_1,A_2\in\mathcal A$,
$N_1\subset M_1$, $N_2\subset M_2$, $M_1,M_2\in\mathcal A$ e $\mu(M_1)=\mu(M_2)=0$.
Se $A=A_1\setminus(A_2\cup M_2)$ então $A\in\mathcal A$ e:
\[A\subset(A_1\cup N_1)\setminus(A_2\cup N_2);\]
podemos então escrever:
\[(A_1\cup N_1)\setminus(A_2\cup N_2)=A\cup N,\]
onde $N=[(A_1\cup N_1)\setminus(A_2\cup N_2)]\setminus A$. É fácil ver que
$N\subset N_1\cup M_2\subset M_1\cup M_2$. Como $M_1\cup M_2\in\mathcal A$
e $\mu(M_1\cup M_2)=0$, segue que $(A_1\cup N_1)\setminus(A_2\cup N_2)\in\overline{\mathcal A}$.
Isso prova que $\overline{\mathcal A}$ é um $\sigma$-anel.
Se $\bar\mu$ é uma medida em $\overline{\mathcal A}$ que estende $\mu$ então:
\[\mu(A)=\bar\mu(A)\le\bar\mu(A\cup N)\le\bar\mu(A)+\bar\mu(N)\le
\bar\mu(A)+\bar\mu(M)=\mu(A),\]
para todos $A,M\in\mathcal A$, $N\subset M$, com $\mu(M)=0$; daí:
\begin{equation}\label{eq:defbarmu}
\bar\mu(A\cup N)=\mu(A),
\end{equation}
para todos $A,M\in\mathcal A$, $N\subset M$, com $\mu(M)=0$. Isso prova a unicidade
de $\bar\mu$; para provar a existência, nós usaremos a igualdade \eqref{eq:defbarmu}
para definir $\bar\mu$ em $\overline{\mathcal A}$. Para verificar que $\bar\mu$ está
de fato bem definida, devemos mostrar que $\mu(A_1)=\mu(A_2)$, sempre que
$A_1\cup N_1=A_2\cup N_2$, $N_1\subset M_1$, $N_2\subset M_2$, $A_1,A_2,M_1,M_2\in\mathcal A$
e $\mu(M_1)=\mu(M_2)=0$. Temos:
\[A_1\bigtriangleup A_2\subset N_1\cup N_2\subset M_1\cup M_2,\]
donde $\mu(A_1\bigtriangleup A_2)=0$; segue então do resultado do Exercício~\ref{exe:mudifsym}
que $\mu(A_1)=\mu(A_2)$. Concluímos que $\bar\mu$ está bem definida e é claro que $\bar\mu$
estende $\mu$. Para verificar que $\bar\mu$ é uma medida em $\overline{\mathcal A}$, seja
$(A_k\cup N_k)_{k\ge1}$ uma seqüência de elementos dois a dois disjuntos
de $\overline{\mathcal A}$, onde para cada $k\ge1$,
$A_k\in\mathcal A$ e existe $M_k\in\mathcal A$ com $N_k\subset M_k$ e $\mu(M_k)=0$. Temos:
\[\bar\mu\Big(\bigcup_{k=1}^\infty(A_k\cup N_k)\Big)\stackrel{\eqref{eq:countunbarcalA}}=
\mu\Big(\bigcup_{k=1}^\infty A_k\Big)=\sum_{k=1}^\infty\mu(A_k)=\sum_{k=1}^\infty\bar\mu(A_k\cup N_k),\]
o que prova que $\bar\mu$ é uma medida. Vejamos que $\bar\mu$ é completa. Sejam dados
$A,M\in\mathcal A$, $N\subset M$ com $\mu(M)=0$ e $\bar\mu(A\cup N)=0$; daí
$\mu(A)=0$. Se $B$ é um subconjunto de $A\cup N$ então $B=\emptyset\cup B$, onde $\emptyset\in\mathcal A$,
$B\subset A\cup M\in\mathcal A$ e $\mu(A\cup M)=0$. Logo $B\in\overline{\mathcal A}$
e a medida $\bar\mu$ é completa.
Finalmente, seja $\mu':\mathcal A'\to[0,+\infty]$ uma medida completa que estende $\mu$,
definida num $\sigma$-anel $\mathcal A'$. Dados $A,M\in\mathcal A$, $N\subset M$ com $\mu(M)=0$
então $A,M\in\mathcal A'$ e $\mu'(M)=0$; como $\mu'$ é completa e $N\subset M$, temos
$N\in\mathcal A'$ e portanto $A\cup N\in\mathcal A'$. Isso prova que $\mathcal A'$ contém
$\overline{\mathcal A}$. Como a restrição de $\mu'$ a $\overline{\mathcal A}$ é uma medida
em $\overline{\mathcal A}$ que estende $\mu$, vemos que essa restrição deve coincidir
com $\bar\mu$; logo $\mu'$ é uma extensão de $\bar\mu$. Isso completa a demonstração.
\end{proof}

\begin{defin}
A medida $\bar\mu:\overline{\mathcal A}\to[0,+\infty]$ cuja existência e unicidade
é garantida pelo Lema~\ref{thm:lemacompletar} é chamada o
{\em completamento\/}\index[indice]{completamento!de uma medida}\index[indice]{medida!completamento de}
da medida $\mu$.
\end{defin}

\begin{rem}
Se $(X,\mathcal A,\mu)$ é um espaço de medida (i.e., $X$ é um conjunto,
$\mathcal A$ é uma $\sigma$-álgebra de partes de $X$ e $\mu$ é uma medida em $\mathcal A$)
e se $\bar\mu:\overline{\mathcal A}\to[0,+\infty]$
é o completamento de $\mu$ então $\overline{\mathcal A}$ é uma $\sigma$-álgebra de partes
de $X$, já que $\overline{\mathcal A}$ é um $\sigma$-anel e $X\in\mathcal A\subset\overline{\mathcal A}$
(veja Exercício~\ref{exe:anelalgebra}). Logo $(X,\overline{\mathcal A},\bar\mu)$ também é um espaço de medida;
nós dizemos então que $(X,\overline{\mathcal A},\bar\mu)$ é o
{\em completamento\/}\index[indice]{completamento!de um espaco de medida@de um espaço de medida}%
\index[indice]{espaco@espaço!de medida!completamento de} de $(X,\mathcal A,\mu)$.
\end{rem}

\begin{prop}\label{thm:nofrakMcompletamento}
Sob as condições do Lema~\ref{thm:geralenvelopes}, se a medida $\mu:\mathcal S\to[0,+\infty]$ é
$\sigma$-finita então a medida $\mu^*\vert_{\mathfrak M}$ é o completamento
da medida $\mu^*\vert_{\mathcal A}$.
\end{prop}
\begin{proof}
Pela Proposição~\ref{thm:vemdemustarecompl}, a medida $\mu^*\vert_{\mathfrak M}$ é
uma extensão completa de $\mu^*\vert_{\mathcal A}$ e portanto é uma extensão
do completamento de $\mu^*\vert_{\mathcal A}$. Para mostrar que $\mu^*\vert_{\mathfrak M}$
é o completamento de $\mu^*\vert_{\mathcal A}$, devemos verificar que para todo
$A\in\mathfrak M$ existem $W,M\in\mathcal A$ e $N\subset M$ com $A=W\cup N$
e $\mu^*(M)=0$. Como $\mu$ é $\sigma$-finita, temos que $A$ é $\sigma$-finito
com respeito a $\mu^*\vert_{\mathfrak M}$ (veja Lema~\ref{thm:sigmatudoigual}) e portanto,
pelo Lema~\ref{thm:geralenvelopes}, existe $W\in\mathcal A$ com $W\subset A$
e $\mu^*(A\setminus W)=0$. Tome $N=A\setminus W$, de modo que $A=W\cup N$.
Aplicando novamente o Lema~\ref{thm:geralenvelopes} obtemos $M\in\mathcal A$ com
$N\subset M$ e $\mu^*(M)=\mu^*(N)=0$. Isso completa a demonstração.
\end{proof}

Se $\mu$ e $\mathcal S$ são definidos como no Exemplo~\ref{exa:Lebesgueagain} e
se $\mu^*$ é a medida exterior determinada por $\mu$
então a $\sigma$-álgebra $\mathfrak M$ de conjuntos $\mu^*$-mensuráveis coincide
com a $\sigma$-álgebra $\Lebmens(\R)$ de subconjuntos Lebesgue mensuráveis da reta e a restrição
de $\mu^*$ a $\mathfrak M$ coincide exatamente com a medida de Lebesgue $\leb$. Temos duas
maneiras de verificar a validade dessa afirmação. Uma delas segue da Observação~\ref{thm:lebstarmens}
usando o resultado do Exercício~\ref{exe:medextehLeb}. A outra é a seguinte; vimos
no Exemplo~\ref{exa:Lebesgueagain} que se $\mathcal A$ é o $\sigma$-anel gerado por
$\mathcal S$ então $\mathcal A=\Borel(\R)$ coincide com a $\sigma$-álgebra
de Borel de $\R$ e a restrição de $\mu^*$ a $\mathcal A$ coincide com a restrição da medida
de Lebesgue a $\mathcal A$. A Proposição~\ref{thm:nofrakMcompletamento} nos diz
que $\mu^*\vert_{\mathfrak M}$ é o completamento de $\mu^*\vert_{\mathcal A}$
e o resultado do Exercício~\ref{exe:LebcomplBorel} nos diz que a medida de Lebesgue é o completamento
da restrição da medida de Lebesgue à $\sigma$-álgebra de Borel. Logo $\mu^*\vert_{\mathfrak M}$
é precisamente a medida de Lebesgue.

Vemos então que a medida de Lebesgue na reta poderia ser introduzida usando apenas a teoria desenvolvida
neste capítulo, sem que nenhuma menção fosse feita a resultados do Capítulo~\ref{CHP:LEBESGUE}.
De fato, podemos definir $\mu$ e $\mathcal S$ como no Exemplo~\ref{exa:Lebesgueagain},
tomar a única extensão de $\mu$ a uma medida no $\sigma$-anel $\mathcal A$
gerado por $\mathcal S$ (Teorema~\ref{thm:teoextensao}) e depois tomar o completamento dessa extensão; esse completamento
é exatamente a medida de Lebesgue em $\R$. Alternativamente, consideramos a medida exterior
$\mu^*$ determinada por $\mu$ e tomamos a restrição
de $\mu^*$ ao $\sigma$-anel de conjuntos $\mu^*$-mensuráveis; o resultado também é a
medida de Lebesgue.

Usando a teoria que desenvolveremos no Capítulo~\ref{CHP:PRODUTOS} nós veremos que também
a medida de Lebesgue em $\R^n$ pode ser construída sem a utilização da teoria
desenvolvida no Capítulo~\ref{CHP:LEBESGUE}.

\end{subsection}

\end{section}

\section*{Exercícios para o Capítulo~\ref{CHP:CONSTRUCAO}}

\subsection*{Medidas em Classes de Conjuntos}

\begin{exercise}\label{exe:tudoinfinito}
Seja $\mathcal C$ uma classe de conjuntos tal que $\emptyset\in\mathcal C$ e seja dada uma função $\mu:\mathcal C\to[0,+\infty]$
tal que, se $(A_k)_{k\ge1}$ é uma seqüência de elementos dois a dois disjuntos de $\mathcal C$
{\em tal que $\bigcup_{k=1}^\infty A_k$ também está em $\mathcal C$}, então a igualdade \eqref{eq:sigmaddmathcalC}
é satisfeita. Mostre que se $\mu(\emptyset)\ne0$ então $\mu(A)=+\infty$, para todo $A\in\mathcal C$.
\end{exercise}

\begin{exercise}\label{exe:naobastadois}
Considere a classe de conjuntos:
\[\mathcal C=\big\{\emptyset,\{0\},\{1\},\{2\},\{0,1,2\}\big\}\]
e defina $\mu:\mathcal C\to[0,+\infty]$ fazendo:
\[\mu(\emptyset)=0,\quad\mu\big(\{0\}\big)=\mu\big(\{1\}\big)=\mu\big(\{2\}\big)=1,\quad\mu\big(\{0,1,2\}\big)=2.\]
Mostre que $\mu(A\cup B)=\mu(A)+\mu(B)$, para todos $A,B\in\mathcal C$ tais que $A\cap B=\emptyset$ e $A\cup B\in\mathcal C$.
No entanto, observe que $\mu$ não é uma medida finitamente aditiva em $\mathcal C$.
\end{exercise}

\begin{exercise}\label{exe:anelalgebra}
Seja $X$ um conjunto e $\mathcal R\subset\wp(X)$ uma coleção de partes de $X$. Mostre que
$\mathcal R$ é uma álgebra (resp., uma $\sigma$-álgebra) de partes de $X$ se e somente se $\mathcal R$ é um anel
(resp., um $\sigma$-anel) tal que $X\in\mathcal R$.
\end{exercise}

\begin{exercise}\label{exe:semialgebra}
Seja $X$ um conjunto e $\mathcal S\subset\wp(X)$ uma coleção de partes de $X$. Dizemos que $\mathcal S$ é
uma {\em semi-álgebra\/}\index[indice]{semi algebra@semi-álgebra} de partes de $X$ se $\mathcal S$ é um semi-anel
e se $X\in\mathcal S$. Mostre que $\mathcal S\subset\wp(X)$ é uma semi-álgebra de partes de $X$ se e somente se
as seguintes condições são satisfeitas:
\begin{itemize}
\item[(a)] $A\cap B\in\mathcal S$, para todos $A,B\in\mathcal S$;
\item[(b)] se $A\in\mathcal S$ então existem $k\ge1$ e conjuntos $C_1,\ldots,C_k\in\mathcal S$, dois a dois disjuntos,
de modo que $A^\compl=X\setminus A=\bigcup_{i=1}^kC_i$;
\item[(c)] $X\in\mathcal S$.
\end{itemize}
Se $X$ é um conjunto finito com mais de um elemento e se $\mathcal S\subset\wp(X)$ é definido por:
\[\mathcal S=\{\emptyset\}\cup\big\{\{x\}:x\in X\big\},\]
mostre que $\mathcal S$ é uma classe não vazia de subconjuntos de $X$ satisfazendo as condições (a) e (b),
mas que $\mathcal S$ não é uma semi-álgebra de partes de $X$.
\end{exercise}

\begin{exercise}
Sejam $\mathcal A$ um anel e $A$, $B$ conjuntos. Mostre que se $A\in\mathcal A$ e $A\bigtriangleup B\in\mathcal A$
então também $B\in\mathcal A$.
\end{exercise}

\begin{exercise}\label{exe:algerada}
Seja $X$ um conjunto arbitrário.
\begin{itemize}
\item[(a)] Se $(\mathcal A_i)_{i\in I}$ é uma família não vazia de álgebras de partes de $X$, mostre que
$\mathcal A=\bigcap_{i\in I}\mathcal A_i$ também é uma álgebra de partes de $X$.
\item[(b)] Mostre que, fixada uma coleção $\mathcal C\subset\wp(X)$ de partes de $X$, existe {\em no máximo uma\/}
álgebra $\mathcal A$ de partes de $X$ satisfazendo as propriedades \eqref{itm:algera1} e
\eqref{itm:algera2} que aparecem na Definição~\ref{thm:defalgerada}.
\item[(c)] Dada uma coleção arbitrária $\mathcal C\subset\wp(X)$ de partes de $X$, mostre que a interseção de todas
as álgebras de partes de $X$ que contém $\mathcal C$ é uma álgebra de partes de $X$ que satisfaz as propriedades
\eqref{itm:algera1} e \eqref{itm:algera2} que aparecem na Definição~\ref{thm:defalgerada} (note que sempre existe
ao menos uma álgebra de partes de $X$ contendo $\mathcal C$, a saber, $\wp(X)$).
\end{itemize}
\end{exercise}

\begin{exercise}\label{exe:anelgerado}
\
\begin{itemize}
\item[(a)] Se $(\mathcal R_i)_{i\in I}$ é uma família não vazia de anéis (resp., de $\sigma$-anéis), mostre que
$\mathcal R=\bigcap_{i\in I}\mathcal R_i$ também é um anel (resp., $\sigma$-anel).
\item[(b)] Mostre que, fixada uma classe de conjuntos $\mathcal C$, existe {\em no máximo um\/}
anel (resp., $\sigma$-anel) $\mathcal R$ satisfazendo as propriedades \eqref{itm:anelger1} e
\eqref{itm:anelger2} que aparecem na Definição~\ref{thm:anelgerado}.
\item[(c)] Seja $\mathcal C$ uma classe de conjuntos arbitrária e seja $X$ um conjunto tal que
$\mathcal C\subset\wp(X)$ (por exemplo, tome $X=\bigcup_{A\in\mathcal C}A$). Mostre que a interseção de todos
os anéis (resp., $\sigma$-anéis) $\mathcal R\subset\wp(X)$ que contém $\mathcal C$ é um anel (resp., $\sigma$-anel)
que satisfaz as propriedades \eqref{itm:anelger1} e \eqref{itm:anelger2} que aparecem na Definição~\ref{thm:anelgerado}
(note que sempre existe ao menos um anel (resp., $\sigma$-anel) $\mathcal R\subset\wp(X)$
contendo $\mathcal C$, a saber, $\wp(X)$).
\end{itemize}
\end{exercise}

\begin{exercise}\label{exe:anelalgerada}
Sejam $X$ um conjunto e $\mathcal C$ uma coleção de subconjuntos de $X$. Mostre
que o anel (resp., o $\sigma$-anel) gerado por $\mathcal C$ coincide com a álgebra
(resp., a $\sigma$-álgebra) de partes de $X$ gerada por $\mathcal C$ se e somente se
$X$ pertence ao anel (resp., ao $\sigma$-anel) gerado por $\mathcal C$ (esse é o caso,
por exemplo, se $X\in\mathcal C$).
\end{exercise}

\begin{exercise}\label{exe:anelBorelint}
Mostre que o $\sigma$-anel gerado pelo semi-anel $\mathcal S$ constituído pelos intervalos da forma
$\left]a,b\right]$, $a,b\in\R$ (veja \eqref{eq:semiintervalos}) coincide com a $\sigma$-álgebra de Borel de $\R$.
\end{exercise}

\begin{exercise}\label{exe:naosemianelgera}
Sejam:
\[\mathcal S_1=\big\{\emptyset,\{1\},\{2,3\},\{1,2,3\}\big\},\quad
\mathcal S_2=\big\{\emptyset,\{1\},\{2\},\{3\},\{1,2,3\}\big\}.\]
\begin{itemize}
\item[(a)] Mostre que $\mathcal S_1$ e $\mathcal S_2$ são semi-anéis, mas $\mathcal S_1\cap\mathcal S_2$ não é
um semi-anel.
\item[(b)] Seja $\mathcal C=\mathcal S_1\cap\mathcal S_2$. Mostre que não existe um semi-anel $\mathcal S$
contendo $\mathcal C$ tal que $\mathcal S\subset\mathcal S'$ para todo semi-anel $\mathcal S'$ contendo
$\mathcal C$.
\end{itemize}
\end{exercise}

\begin{exercise}\label{exe:destriangle}
Dados conjuntos $A$, $B$ e $C$, mostre que:
\[A\bigtriangleup C\subset(A\bigtriangleup B)\cup(B\bigtriangleup C).\]
\end{exercise}

\begin{exercise}\label{exe:mudifsym}
Seja $\mu:\mathcal S\to[0,+\infty]$ uma medida finitamente aditiva num semi-anel $\mathcal S$
e sejam $A,B\in\mathcal S$ com $A\bigtriangleup B\in\mathcal S$. Se $\mu(A)<+\infty$ ou $\mu(B)<+\infty$,
mostre que:
\[\big\vert\mu(A)-\mu(B)\big\vert\le\mu(A\bigtriangleup B).\]
Conclua que se $\mu(A\bigtriangleup B)<+\infty$ então $\mu(A)$ é finito se e somente se
$\mu(B)$ é finito.
\end{exercise}

\begin{exdefin}
Sejam $I$ um conjunto e $<$ uma relação binária em $I$. Dizemos que $<$ é uma
{\em relação de ordem total\/}\index[indice]{relacao de ordem@relação de ordem!total}
no conjunto $I$ se as seguintes condições são satisfeitas:
\begin{itemize}
\item ({\em anti-reflexividade})\index[indice]{anti-reflexividade}\index[indice]{relacao@relação!anti-reflexiva}
para todo $a\in I$, não é o caso que $a<a$;
\item ({\em transitividade})\index[indice]{transitividade}\index[indice]{relacao@relação!transitiva}
para todos $a,b,c\in I$, se $a<b$ e $b<c$ então $a<c$;
\item ({\em tricotomia})\index[indice]{tricotomia} dados $a,b\in I$ então $a<b$, $b<a$ ou $a=b$.
\end{itemize}
Diz-se então que o par $(I,{<})$
é um {\em conjunto totalmente ordenado}.\index[indice]{conjunto!totalmente ordenado}\index[indice]{totalmente ordenado!conjunto}
Para $a,b\in I$, nós escrevemos $a>b$ quando $b<a$, $a\le b$ quando $a<b$ ou $a=b$ e
escrevemos $a\ge b$ quando $b\le a$. Definimos também:
\begin{gather*}
[a,b]=\big\{x\in I:\text{$a\le x$ e $x\le b$}\big\},\\
\left]a,b\right]=\big\{x\in I:\text{$a<x$ e $x\le b$}\big\},\\
\left[a,b\right[=\big\{x\in I:\text{$a\le x$ e $x<b$}\big\},\\
\left]a,b\right[=\big\{x\in I:\text{$a<x$ e $x<b$}\big\},\\
\end{gather*}
para todos $a,b\in I$.
\end{exdefin}

\begin{exercise}\label{exe:interstotord}
Seja $(I,<)$ um conjunto totalmente ordenado não vazio.
\begin{itemize}
\item[(a)] Mostre que a classe de conjuntos:
\[\mathcal S=\big\{\left]a,b\right]:a,b\in I,\ a\le b\big\}\]
é um semi-anel.
\item[(b)] Dados $a,b\in I$ com $a\le b$, mostre que $\left]a,b\right]=\emptyset$
se e somente se $a=b$.
\item[(c)] Dados $a,b,a',b'\in I$ com $a<b$, $a'<b'$, mostre que
$\left]a,b\right]=\left]a',b'\right]$ se e somente se $a=a'$ e $b=b'$.
\end{itemize}
\end{exercise}

\begin{exdefin}
Sejam $(I,<)$, $(I',<)$ conjuntos totalmente ordenados. Uma função $F:I\to I'$ é dita
{\em crescente\/}\index[indice]{crescente!funcao@função}\index[indice]{funcao@função!crescente} (resp.,
{\em decrescente}\index[indice]{decrescente!funcao@função}\index[indice]{funcao@função!decrescente}) se
$F(a)\le F(b)$ (resp., $F(a)\ge F(b)$) para todos $a,b\in I$ com $a\le b$.
\end{exdefin}

\begin{exercise}\label{exe:muFtotord}
Seja $(I,<)$ um conjunto totalmente ordenado não vazio e seja $\mathcal S$ o semi-anel definido
no enunciado do Exercício~\ref{exe:interstotord}.
\begin{itemize}
\item[(a)] Seja $F:I\to\R$ uma função crescente e defina $\mu_F:\mathcal S\to\left[0,+\infty\right[$
fazendo:
\[\mu_F\big(\left]a,b\right]\big)=F(b)-F(a),\]
para todos $a,b\in I$ com $a\le b$. Mostre que $\mu_F$ é uma medida finitamente aditiva finita em $\mathcal S$.
\item[(b)] Se $\mu:\mathcal S\to\left[0,+\infty\right[$ é uma medida finitamente aditiva
finita em $\mathcal S$, mostre que existe uma função crescente $F:I\to\R$ tal que
$\mu=\mu_F$.
\item[(c)] Dadas funções crescentes $F:I\to\R$, $G:I\to\R$, mostre que $\mu_F=\mu_G$
se e somente se a função $F-G$ é constante.
\end{itemize}
\end{exercise}

\begin{exercise}\label{exe:poiseh}
Seja:
\[\mathcal S=\big\{\left]a,b\right]\cap\Q:a,b\in\Q,\ a\le b\big\}\]
e defina $\mu:\mathcal S\to\left[0,+\infty\right[$ fazendo:
\[\mu\big(\left]a,b\right]\cap\Q\big)=b-a,\]
para todos $a,b\in\Q$ com $a\le b$. Pelo resultado do Exercício~\ref{exe:interstotord},
$\mathcal S$ é um semi-anel e pelo resultado do Exercício~\ref{exe:muFtotord},
$\mu$ é uma medida finitamente aditiva finita em $\mathcal S$ (note que
$\mu=\mu_F$, onde $F:\Q\to\R$ é a aplicação inclusão).
\begin{itemize}
\item[(a)] Dados $A\in\mathcal S$ e $\varepsilon>0$, mostre que existe uma seqüência $(A_n)_{n\ge1}$
em $\mathcal S$ tal que $A\subset\bigcup_{n=1}^\infty A_n$ e $\sum_{n=1}^\infty\mu(A_n)\le\varepsilon$.
\item[(b)] Conclua que $\mu$ não é uma medida $\sigma$-aditiva.
\end{itemize}
\end{exercise}

\begin{exercise}
Seja $X$ um espaço topológico Hausdorff e seja $\mathcal C$ uma classe arbitrária
de subconjuntos compactos de $X$. Mostre que $\mathcal C$ é uma classe compacta.
\end{exercise}

\subsection*{Classes Monotônicas e Classes ${\sigma}$-aditivas}

\begin{exercise}\label{exe:sigmonotgerado}
\
\begin{itemize}
\item[(a)] Se $(\mathcal E_i)_{i\in I}$ é uma família não vazia de classes monotônicas (resp., de classes
$\sigma$-aditivas), mostre que $\mathcal E=\bigcap_{i\in I}\mathcal E_i$ também é uma classe monotônica
(resp., uma classe $\sigma$-aditiva).
\item[(b)] Mostre que, fixada uma classe de conjuntos $\mathcal C$, existe {\em no máximo uma\/}
classe monotônica (resp., classe $\sigma$-aditiva) $\mathcal E$ satisfazendo as propriedades \eqref{itm:sigmonotger1} e
\eqref{itm:sigmonotger2} que aparecem na Definição~\ref{thm:sigmonotgerado}.
\item[(c)] Seja $\mathcal C$ uma classe de conjuntos arbitrária e seja $X$ um conjunto tal que
$\mathcal C\subset\wp(X)$ (por exemplo, tome $X=\bigcup_{A\in\mathcal C}A$). Mostre que a interseção de todas
as classes monotônicas (resp., classes $\sigma$-aditivas) $\mathcal E\subset\wp(X)$ que contém $\mathcal C$ é uma
classe monotônica (resp., classe $\sigma$-aditiva) que satisfaz as propriedades \eqref{itm:sigmonotger1} e \eqref{itm:sigmonotger2}
que aparecem na Definição~\ref{thm:sigmonotgerado} (note que sempre existe ao menos uma classe monotônica
(resp., classe $\sigma$-aditiva) $\mathcal E\subset\wp(X)$ contendo $\mathcal C$, a saber, $\wp(X)$).
\end{itemize}
\end{exercise}

\begin{exercise}\label{exe:AvertYanel}
Sejam $X$ um conjunto e $\mathcal A$ um anel (resp., um $\sigma$-anel).
Mostre que $\mathcal A\vert_X$ é também um anel (resp., um $\sigma$-anel).
\end{exercise}

\begin{exercise}\label{exe:restrsigmafin}
Seja $\mu:\mathcal S\to[0,+\infty]$ uma medida $\sigma$-finita num semi-anel $\mathcal S$.
Mostre que para todo $X\in\mathcal S$, a medida $\mu\vert_{\mathcal S\vert_X}$
também é $\sigma$-finita.
\end{exercise}

\begin{exercise}\label{exe:sigmanoheredit}
Seja $\mathcal C$ uma classe de conjuntos não vazia. Mostre que a coleção de conjuntos:
\[\Big\{A:\text{existe uma seqüência $(A_k)_{k\ge1}$ em $\mathcal C$ com $A\subset\bigcup_{k=1}^\infty A_k$}\Big\}\]
é um $\sigma$-anel que contém $\mathcal C$. Conclua que todo elemento do $\sigma$-anel
gerado por $\mathcal C$ está contido numa união enumerável de elementos de $\mathcal C$.
\end{exercise}

\begin{exercise}\label{exe:Xnosigmaneluncount}
Sejam $X$ um conjunto e $\mathcal C$ uma coleção de subconjuntos de $X$.
Mostre que $X$ pertence ao $\sigma$-anel gerado por $\mathcal C$ se e somente se
$X$ é igual a uma união enumerável de elementos de $\mathcal C$.
\end{exercise}

\subsection*{Medidas Exteriores e o Teorema da Extensão}

\begin{exercise}\label{exe:mustarlessmu}
Seja:
\[\mathcal C=\{\emptyset\}\cup\big\{\{0,1,\ldots,n\}:n\in\N\big\}\]
e considere a medida $\mu:\mathcal C\to[0,+\infty]$ definida por:
\[\mu(\emptyset)=0,\quad\mu\big(\{0,1,\ldots,n\}\big)=\frac1{2^n},\]
para todo $n\in\N$ (veja Exemplo~\ref{exa:patolC}).
Se $\mu^*:\wp(\N)\to[0,+\infty]$ é a medida exterior determinada por $\mu$,
mostre que $\mu^*(A)=0$, para todo $A\subset\N$. Conclua que a desigualdade estrita ocorre em \eqref{eq:mustarleqmu},
para todo $A\in\mathcal C$ não vazio.
\end{exercise}

\begin{exercise}
Sejam $\mathcal C$, $\mathcal D$ classes de conjuntos com $\emptyset\in\mathcal C$, $\emptyset\in\mathcal D$
e sejam $\mu:\mathcal C\to[0,+\infty]$, $\nu:\mathcal D\to[0,+\infty]$ funções tais que $\mu(\emptyset)=0$
e $\nu(\emptyset)=0$. Suponha que $\mathcal C$ e $\mathcal D$ geram o mesmo $\sigma$-anel hereditário
$\mathcal H$. Sejam $\mu^*:\mathcal H\to[0,+\infty]$ e $\nu^*:\mathcal H\to[0,+\infty]$
as medidas exteriores determinadas respectivamente pelas funções $\mu$ e $\nu$.
Mostre que as seguintes condições são equivalentes:
\begin{itemize}
\item[(a)] $\mu^*=\nu^*$;
\item[(b)] para todo $A\in\mathcal D$, temos $\mu^*(A)\le\nu(A)$ e para todo $A\in\mathcal C$, temos $\nu^*(A)\le\mu(A)$.
\end{itemize}
\end{exercise}

\begin{exercise}\label{exe:medextehLeb}
Sejam $\mathcal S$ e $\mu$ definidos como no Exemplo~\ref{exa:Lebesgueagain}. Mostre que
a medida exterior $\mu^*:\wp(\R)\to[0,+\infty]$ determinada por $\mu$
coincide com a medida exterior de Lebesgue.
\end{exercise}

\begin{exdefin}
Seja $\mu^*:\mathcal H\to[0,+\infty]$ uma medida exterior num $\sigma$-anel hereditário
$\mathcal H$. Dado $A\in\mathcal H$ então um {\em envelope mensurável\/}\index[indice]{envelope mensuravel@envelope mensurável}%
\index[indice]{mensuravel@mensurável!envelope} para $A$ é um conjunto $\mu^*$-mensurável $E$
tal que $A\subset E$ e tal que $\mu^*(E)=\mu^*(A)$. Dizemos que a medida exterior
$\mu^*$ possui a {\em propriedade do envelope mensurável\/}\index[indice]{propriedade!do envelope mensuravel@do envelope mensurável}%
\index[indice]{envelope mensuravel@envelope mensurável!propriedade do} se todo $A\in\mathcal H$ admite um
envelope mensurável.
\end{exdefin}

\begin{exercise}
Seja $\mathcal C$ a classe de conjuntos definida no Exemplo~\ref{exa:patolC} e seja
$\mu:\mathcal C\to[0,+\infty]$ a medida definida por:
\[\mu(\emptyset)=0,\quad\mu(\N)=2,\quad\mu\big(\{0,1,\ldots,n\}\big)=1,\]
para todo $n\in\N$. Seja $\mu^*:\wp(\N)\to[0,+\infty]$ a medida exterior determinada
por $\mu$. Mostre que:
\begin{itemize}
\item[(a)] $\mu^*(A)=1$, se $A\subset\N$ é finito não vazio e $\mu^*(A)=2$,
se $A\subset\N$ é infinito;
\item[(b)] os únicos conjuntos $\mu^*$-mensuráveis são o vazio e $\N$;
\item[(c)] $\mu^*$ não possui a propriedade do envelope mensurável.
\end{itemize}
\end{exercise}

\begin{exercise}
Seja $\mu^*:\mathcal H\to[0,+\infty]$ uma medida exterior num $\sigma$-anel hereditário
$\mathcal H$. Mostre que as seguintes condições são equivalentes:
\begin{itemize}
\item[(a)] existe um semi-anel $\mathcal S$ e uma medida $\mu:\mathcal S\to[0,+\infty]$
tal que $\mathcal H$ é o $\sigma$-anel hereditário gerado por $\mathcal S$ e $\mu^*$
é a medida exterior determinada por $\mu$;
\item[(b)] $\mu^*$ possui a propriedade do envelope mensurável;
\item[(c)] $\mathcal H$ é o $\sigma$-anel hereditário gerado por $\mathfrak M$
e $\mu^*$ é a medida exterior determinada pela medida $\mu^*\vert_{\mathfrak M}$,
onde $\mathfrak M$ denota o $\sigma$-anel de conjuntos $\mu^*$-mensuráveis.
\end{itemize}
\end{exercise}

\begin{exercise}
Seja $\mu^*:\mathcal H\to[0,+\infty]$ uma medida exterior num $\sigma$-anel hereditário
$\mathcal H$; suponha que $\mu^*$ possui a propriedade do envelope mensurável. Se
$(A_k)_{k\ge1}$ é uma seqüência em $\mathcal H$ com $A_k\nearrow A$, mostre que
$\mu^*(A)=\lim_{k\to\infty}\mu^*(A_k)$.
\end{exercise}

\begin{exdefin}
Seja $\mu^*:\mathcal H\to[0,+\infty]$ uma medida exterior num $\sigma$-anel hereditário
$\mathcal H$. A {\em medida interior determinada por $\mu^*$\/}\index[indice]{medida!interior!determinada por uma medida exterior}
é a aplicação $\mu_*:\mathcal H\to[0,+\infty]$ definida por:
\[\mu_*(A)=\sup\big\{\mu^*(E):\text{$E\subset A$ e $E$ é $\mu^*$-mensurável}\big\}\in[0,+\infty],\]
para todo $A\in\mathcal H$.
\end{exdefin}

\begin{exercise}
Seja $\mu^*:\mathcal H\to[0,+\infty]$ uma medida exterior num $\sigma$-anel hereditário
$\mathcal H$ e seja $\mu_*:\mathcal H\to[0,+\infty]$ a medida interior determinada
por $\mu_*$. Mostre que:
\begin{itemize}
\item[(a)] para todo $A\in\mathcal H$ temos $\mu_*(A)\le\mu^*(A)$;
\item[(b)] para todo $A\in\mathcal H$ existe um conjunto $\mu^*$-mensurável $E$ contido
em $A$ tal que $\mu^*(E)=\mu_*(A)$;
\item[(c)] $\mu_*(A)=\mu^*(A)$, para todo conjunto $\mu_*$-mensurável $A$;
\item[(d)] dados $A,B\in\mathcal H$ com $A\subset B$ então $\mu_*(A)\le\mu_*(B)$;
\item[(e)] dada uma seqüência $(A_k)_{k\ge1}$ de elementos dois a dois disjuntos de
$\mathcal H$ então:
\[\mu_*\Big(\bigcup_{k=1}^\infty A_k\Big)\ge\sum_{k=1}^\infty\mu_*(A_k);\]
\item[(f)] se $\mu^*$ possui a propriedade do envelope mensurável e se $A\in\mathcal H$
é tal que $\mu_*(A)=\mu^*(A)<+\infty$ então $A$ é $\mu^*$-mensurável.
\end{itemize}
\end{exercise}

\subsection*{Completamento de Medidas}

\begin{exercise}\label{exe:complintermed}
Sejam $\mu:\mathcal A\to[0,+\infty]$ uma medida num $\sigma$-anel $\mathcal A$,
$\bar\mu:\overline{\mathcal A}\to[0,+\infty]$ o completamento de $\mu$ e $\mu':\mathcal A'\to[0,+\infty]$
uma medida tal que $\mathcal A\subset\mathcal A'\subset\overline{\mathcal A}$ e $\mu'\vert_{\mathcal A}=\mu$.
Mostre que:
\begin{itemize}
\item $\bar\mu$ é uma extensão de $\mu'$;
\item $\bar\mu$ é o completamento de $\mu'$.
\end{itemize}
\end{exercise}

\begin{exercise}\label{exe:complsigmafin}
Seja $\mu:\mathcal A\to[0,+\infty]$ uma medida num $\sigma$-anel $\mathcal A$ e seja
$\bar\mu:\overline{\mathcal A}\to[0,+\infty]$ o completamento de $\mu$. Mostre que
$\mu$ é $\sigma$-finita se e somente se $\bar\mu$ é $\sigma$-finita.
\end{exercise}

\begin{exercise}
Sejam $(X,\mathcal A,\mu)$ um espaço de medida e $(X',\mathcal A')$ um espaço mensurável.
Suponha que $\mu$ é completa. Seja $Y$ um subconjunto mensurável de $X$ com $\mu(X\setminus Y)=0$
e seja $f:X\to X'$ uma função. Mostre que $f$ é mensurável se e somente se $f\vert_Y$
é mensurável.
\end{exercise}

\begin{exercise}\label{exe:fquaseiggmens}
Sejam $(X,\mathcal A,\mu)$ um espaço de medida e $(X',\mathcal A')$ um espaço mensurável.
Suponha que $\mu$ é completa. Dadas funções $f:X\to X'$, $g:X\to X'$ tais que
$f(x)=g(x)$ para quase todo $x\in X$, mostre que $f$ é mensurável se e somente se
$g$ é mensurável.
\end{exercise}

\begin{exercise}
Seja $(X,\mathcal A,\mu)$ um espaço de medida, com $\mu$ completa
e seja $f:X\to\overline\R$ uma função. Se $(f_k)_{k\ge1}$
é uma seqüência de funções mensuráveis $f_k:X\to\overline\R$ e se $f_k\to f$ \qs,
mostre que $f$ também é mensurável.
\end{exercise}

\begin{exdefin}
Uma $\sigma$-álgebra $\mathcal A$ de partes de um conjunto $X$ é dita
{\em separável\/}\index[indice]{separavel@separável!sigma algebra@$\sigma$-álgebra}%
\index[indice]{sigma algebra@$\sigma$-álgebra!separavel@separável} se existe um subconjunto
enumerável $\mathcal C$ de $\mathcal A$ tal que $\mathcal A$ é a $\sigma$-álgebra de
partes de $X$ gerada por $\mathcal C$.
\end{exdefin}

\begin{exercise}
Seja $\mathcal A$ uma $\sigma$-álgebra separável de partes de um conjunto $X$
e seja $Y$ um subconjunto de $X$.
Mostre que a $\sigma$-álgebra $\mathcal A\vert_Y$ também é separável.
\end{exercise}

\begin{exercise}
Seja $X$ um conjunto. Um subconjunto $A$ de $X$ é dito {\em coenumerável\/}\index[indice]{coenumeravel@coenumerável!subconjunto}%
\index[indice]{conjunto!coenumeravel@coenumerável}\index[indice]{subconjunto!coenumeravel@coenumerável} se o complementar
de $A$ em $X$ é enumerável. Seja $\mathcal A\subset\wp(X)$ a coleção constituída pelos subconjuntos enumeráveis de $X$
e pelos subconjuntos coenumeráveis de $X$.
\begin{itemize}
\item[(a)] Mostre que $\mathcal A$ é uma $\sigma$-álgebra de partes de $X$.
\item[(b)] Se $X$ é não enumerável, mostre que $\mathcal A$ não é separável.
\item[(c)] Dê exemplo de um conjunto $X$ e de $\sigma$-álgebras $\mathcal A$, $\mathcal B$ de partes de $X$
com $\mathcal A\subset\mathcal B$, de modo que $\mathcal B$ seja separável mas $\mathcal A$ não seja.
\end{itemize}
\end{exercise}

\begin{exercise}\label{exe:Borelsepar}
Mostre que a $\sigma$-álgebra de Borel de $\R^n$ e a $\sigma$-álgebra de Borel
da reta estendida são ambas separáveis.
\end{exercise}

\begin{exercise}\label{exe:quaseigualmens}
Seja $(X,\mathcal A,\mu)$ um espaço de medida e seja $(X',\mathcal A')$ um espaço mensurável;
denote por $\bar\mu:\overline{\mathcal A}\to[0,+\infty]$ o completamento da medida $\mu$.
Suponha que a $\sigma$-álgebra $\mathcal A'$ é separável. Dada uma função
mensurável $f:(X,\overline{\mathcal A})\to(X',\mathcal A')$,
mostre que:
\begin{itemize}
\item[(a)] existe um conjunto mensurável $Y\in\mathcal A$ tal que $\mu(X\setminus Y)=0$ e tal que a função
$f\vert_Y:(Y,\mathcal A\vert_Y)\to(X',\mathcal A')$ é mensurável;
\item[(b)] existe uma função mensurável $g:(X,\mathcal A)\to(X',\mathcal A')$ que é igual
a $f$ quase sempre, i.e., existe $Y\in\mathcal A$ tal que $f\vert_Y=g\vert_Y$
e tal que $\mu(X\setminus Y)=0$.
\end{itemize}
\end{exercise}

\end{chapter}

\begin{chapter}{Medidas Produto e o Teorema de Fubini}
\label{CHP:PRODUTOS}

\begin{section}[Produto de $\sigma$-Álgebras]{Produto de ${\sigma}$-Álgebras}

Sejam $(X,\mathcal A)$, $(Y,\mathcal B)$ espaços mensuráveis, i.e., $X$ e $Y$ são conjuntos,
$\mathcal A$ é uma $\sigma$-álgebra de partes de $X$ e $\mathcal B$ é uma $\sigma$-álgebra
de partes de $Y$. Segue do Lema~\ref{thm:prodsemianel} que a classe de conjuntos:
\[\mathcal A\Times\mathcal B=\big\{A\times B:A\in\mathcal A,\ B\in\mathcal B\big\}\subset\wp(X\times Y)\]
é um semi-anel; evidentemente, não é de se esperar que $\mathcal A\Times\mathcal B$
seja uma $\sigma$-álgebra de partes de $X\times Y$.

\begin{defin}
Sejam $(X,\mathcal A)$, $(Y,\mathcal B)$ espaços mensuráveis. A $\sigma$-álgebra
de partes de $X\times Y$ gerada por $\mathcal A\Times\mathcal B$, denotada por
$\mathcal A\otimes\mathcal B$,\index[simbolos]{$\mathcal A\otimes\mathcal B$} é chamada
a {\em $\sigma$-álgebra produto\/}\index[indice]{sigma algebra@$\sigma$-álgebra!produto}%
\index[indice]{produto!de sigma algebras@de $\sigma$-álgebras} de $\mathcal A$ por $\mathcal B$. O espaço mensurável
$(X\times Y,\mathcal A\otimes\mathcal B)$ é chamado o
{\em produto\/}\index[indice]{produto!de espacos mensuraveis@de espaços mensuráveis}%
\index[indice]{espaco@espaço!mensuravel@mensurável!produto} de $(X,\mathcal A)$ por $(Y,\mathcal B)$.
\end{defin}

O seguinte lema dá uma caracterização interessante para a $\sigma$-álgebra produto.
\begin{lem}\label{thm:sigmaprodmin}
Sejam $(X,\mathcal A)$, $(Y,\mathcal B)$ espaços mensuráveis e denote por
$\pi_1:X\times Y\to X$, $\pi_2:X\times Y\to Y$ as projeções. Então a $\sigma$-álgebra
produto $\mathcal A\otimes\mathcal B$ é a menor $\sigma$-álgebra de partes de
$X\times Y$ que torna as aplicações $\pi_1$ e $\pi_2$ ambas mensuráveis; mais explicitamente:
\begin{itemize}
\item as projeções:
\[\pi_1:(X\times Y,\mathcal A\otimes\mathcal B)\to(X,\mathcal A),\quad
\pi_2:(X\times Y,\mathcal A\otimes\mathcal B)\to(Y,\mathcal B)\]
são mensuráveis;
\item se $\mathcal P$ é uma $\sigma$-álgebra de partes de $X\times Y$ e se as projeções:
\[\pi_1:(X\times Y,\mathcal P)\to(X,\mathcal A),\quad
\pi_2:(X\times Y,\mathcal P)\to(Y,\mathcal B)\]
são mensuráveis então $\mathcal A\otimes\mathcal B\subset\mathcal P$.
\end{itemize}
\end{lem}
\begin{proof}
Para todo $A\in\mathcal A$, temos:
\[\pi_1^{-1}(A)=A\times Y\in\mathcal A\Times\mathcal B\subset\mathcal A\otimes\mathcal B,\]
donde $\pi_1$ é mensurável se $X\times Y$ é munido da $\sigma$-álgebra
produto. Similarmente, $\pi_2$ é mensurável se $X\times Y$ é munido da $\sigma$-álgebra
produto. Seja agora $\mathcal P$ uma $\sigma$-álgebra de partes de $X\times Y$ que torna
as projeções $\pi_1$ e $\pi_2$ ambas mensuráveis. Daí:
\[\pi_1^{-1}(A)\cap\pi_2^{-1}(B)=A\times B\in\mathcal P,\]
para todos $A\in\mathcal A$, $B\in\mathcal B$. Logo $\mathcal A\Times\mathcal B\subset\mathcal P$
e portanto, como $\mathcal P$ é uma $\sigma$-álgebra, temos $\mathcal A\otimes\mathcal B\subset\mathcal P$.
\end{proof}

A principal propriedade da $\sigma$-álgebra produto é expressa pelo seguinte:
\begin{lem}\label{thm:menscoordmensgeral}
Sejam $(X,\mathcal A)$, $(Y,\mathcal B)$, $(Z,\mathfrak C)$ espaços mensuráveis e
seja $f:Z\to X\times Y$ uma função com funções coordenadas $f_1:Z\to X$ e $f_2:Z\to Y$.
Se $X\times Y$ é munido da $\sigma$-álgebra produto $\mathcal A\otimes\mathcal B$ então
$f$ é mensurável se e somente se $f_1$ e $f_2$ são ambas mensuráveis.
\end{lem}
\begin{proof}
Sejam $\pi_1:X\times Y\to X$, $\pi_2:X\times Y\to Y$ as projeções; temos $f_1=\pi_1\circ f$
e $f_2=\pi_2\circ f$. Se $f$ é mensurável, então $f_1$ e $f_2$ também são mensuráveis,
sendo composições de funções mensuráveis. Suponha agora que $f_1$ e $f_2$ são mensuráveis
e provemos que $f$ é mensurável. Pelo Lema~\ref{thm:funcmensgeradores}, para estabelecer
a mensurabilidade de $f$ é suficiente verificar que:
\[f^{-1}(A\times B)\in\mathfrak C,\]
para todos $A\in\mathcal A$, $B\in\mathcal B$; a conclusão segue então da igualdade:
\[f^{-1}(A\times B)=f_1^{-1}(A)\cap f_2^{-1}(B).\qedhere\]
\end{proof}
No Exercício~\ref{exe:universalprod} pedimos ao leitor para demonstrar que a propriedade
constante do enunciado do Lema~\ref{thm:menscoordmensgeral} caracteriza completamente a $\sigma$-álgebra produto.

\begin{example}\label{thm:funcaotroca}
Sejam $(X,\mathcal A)$, $(Y,\mathcal B)$ espaços mensuráveis. Se os produtos $X\times Y$ e $Y\times X$ são
munidos respectivamente das $\sigma$-álgebras $\mathcal A\otimes\mathcal B$ e $\mathcal B\otimes\mathcal A$ então
segue do Lema~\ref{thm:menscoordmensgeral} que a função:
\[\sigma:X\times Y\ni(x,y)\longmapsto(y,x)\in Y\times X\]
é uma bijeção mensurável cuja aplicação inversa $\sigma^{-1}$ também é mensurável. De fato, as funções coordenadas
de $\sigma$ são as projeções do produto cartesiano $X\times Y$ e as funções coordenadas de $\sigma^{-1}$ são
as projeções do produto cartesiano $Y\times X$. Temos em particular que a bijeção $\sigma$ induz uma bijeção:
\[\mathcal A\otimes\mathcal B\ni U\longmapsto\sigma(U)\in\mathcal B\otimes\mathcal A\]
da $\sigma$-álgebra $\mathcal A\otimes\mathcal B$ sobre a $\sigma$-álgebra $\mathcal B\otimes\mathcal A$.
\end{example}

\begin{example}\label{exa:ixiy}
Sejam $(X,\mathcal A)$, $(Y,\mathcal B)$ espaços mensuráveis. Fixado $x\in X$ então segue do Lema~\ref{thm:menscoordmensgeral}
que a função:\index[simbolos]{$i_x$}
\[i_x:(Y,\mathcal B)\ni y\longmapsto(x,y)\in(X\times Y,\mathcal A\otimes\mathcal B)\]
é mensurável; de fato, a primeira coordenada de $i_x$ é uma função constante (veja Exercício~\ref{exe:constmens})
e a segunda coordenada de $i_x$ é a aplicação identidade. Similarmente, fixado $y\in Y$, vê-se que a aplicação:\index[simbolos]{$i^y$}
\[i^y:(X,\mathcal A)\ni x\longmapsto(x,y)\in(X\times Y,\mathcal A\otimes\mathcal B)\]
é mensurável.
\end{example}

\begin{example}\label{exa:fvarmens}
Sejam $(X,\mathcal A)$, $(Y,\mathcal B)$, $(Z,\mathfrak C)$ espaços mensuráveis. Seja
$f:X\times Y\to Z$ uma função mensurável, onde $X\times Y$ é munido da $\sigma$-álgebra produto $\mathcal A\otimes\mathcal B$.
Temos que para todo $x\in X$ a função:
\begin{equation}\label{eq:ytofxy}
Y\ni y\longmapsto f(x,y)\in Z
\end{equation}
é mensurável e para todo $y\in Y$ a função:
\begin{equation}\label{eq:xtofxy}
X\ni x\longmapsto f(x,y)\in Z
\end{equation}
é mensurável. De fato, a função \eqref{eq:ytofxy} é igual a $f\circ i_x$ e a função \eqref{eq:xtofxy} é igual
a $f\circ i^y$ (veja Exemplo~\ref{exa:ixiy}).
\end{example}

\begin{example}\label{exa:prodBorelRn}
Identificando $\R^m\times\R^n$ com $\R^{m+n}$ (veja \eqref{eq:identRmRnRmn}) então o produto da $\sigma$-álgebra de Borel de
$\R^m$ pela $\sigma$-álgebra de Borel de $\R^n$ coincide com a $\sigma$-álgebra de Borel
de $\R^{m+n}$, ou seja:
\begin{equation}\label{eq:BorelRmn}
\Borel(\R^{m+n})=\Borel(\R^m)\otimes\Borel(\R^n).
\end{equation}
De fato, as projeções $\pi_1:\R^{m+n}\to\R^m$, $\pi_2:\R^{m+n}\to\R^n$ são
contínuas e portanto são Borel mensuráveis, pelo Lema~\ref{thm:contmens}. Mais
explicitamente, temos que se $\R^{m+n}$ é munido da $\sigma$-álgebra de Borel
$\Borel(\R^{m+n})$ então $\pi_1$ e $\pi_2$ são ambas mensuráveis;
segue então do Lema~\ref{thm:sigmaprodmin} que:
\[\Borel(\R^m)\otimes\Borel(\R^n)\subset\Borel(\R^{m+n}).\]
Para mostrar a inclusão oposta, é suficiente mostrar que todo aberto $U$ de $\R^{m+n}$
pertence à $\sigma$-álgebra produto $\Borel(\R^m)\otimes\Borel(\R^n)$. Temos que para todo $z\in U$
existem abertos $V_z\subset\R^m$, $W_z\subset\R^n$ tais que $z\in V_z\times W_z\subset U$.
Além do mais, a cobertura aberta $U=\bigcup_{z\in U}(V_z\times W_z)$ possui uma subcobertura
enumerável, i.e., existe um subconjunto enumerável $E$ de $U$ tal que:
\[U=\bigcup_{z\in E}(V_z\times W_z).\]
Mas $V_z\in\Borel(\R^m)$, $W_z\in\Borel(\R^n)$
e $V_z\times W_z\in\Borel(\R^m)\Times\Borel(\R^n)$, para todo $z\in E$; segue então
que $U\in\Borel(\R^m)\otimes\Borel(\R^n)$, o que completa a demonstração de \eqref{eq:BorelRmn}.
\end{example}

Vejamos como produtos de $\sigma$-álgebras relacionam-se com restrições de $\sigma$-álgebras.
\begin{lem}\label{thm:lemarestrprod}
Sejam $(X,\mathcal A)$, $(Y,\mathcal B)$ espaços mensuráveis e $X_0\subset X$,
$Y_0\subset Y$ subconjuntos (não necessariamente mensuráveis). Então:
\[(\mathcal A\vert_{X_0})\otimes(\mathcal B\vert_{Y_0})=(\mathcal A\otimes\mathcal B)\vert_{X_0\times Y_0}.\]
\end{lem}
\begin{proof}
Temos que $(\mathcal A\vert_{X_0})\otimes(\mathcal B\vert_{Y_0})$ é a $\sigma$-álgebra
de partes de $X_0\times Y_0$ gerada por $(\mathcal A\vert_{X_0})\Times(\mathcal B\vert_{Y_0})$;
evidentemente:
\begin{align*}
(\mathcal A\vert_{X_0})\Times(\mathcal B\vert_{Y_0})&=
\big\{(A\cap X_0)\times(B\cap Y_0):A\in\mathcal A,\ B\in\mathcal B\big\}\\
&=\big\{(A\times B)\cap(X_0\times Y_0):A\in\mathcal A,\ B\in\mathcal B\big\}\\
&=(\mathcal A\Times\mathcal B)\vert_{X_0\times Y_0}.
\end{align*}
A conclusão segue do resultado do Exercício~\ref{exe:gerarestrita}.
\end{proof}

\begin{example}\label{exa:prodnaogera}
Sejam $(X,\mathcal A)$, $(Y,\mathcal B)$ espaços mensuráveis e $\mathcal C$, $\mathcal D$
conjuntos de geradores para as $\sigma$-álgebras $\mathcal A$ e $\mathcal B$ respectivamente. Em geral,
não é verdade que $\mathcal C\Times\mathcal D$ é um conjunto de geradores para
a $\sigma$-álgebra produto $\mathcal A\otimes\mathcal B$. Por exemplo, se $X=Y=\R$,
$\mathcal A=\mathcal B=\big\{\emptyset,[0,1],[0,1]^\compl,\R\big\}$ e $\mathcal C=\mathcal D=\big\{[0,1]\big\}$
então:
\[\mathcal C\Times\mathcal D=\big\{[0,1]\times[0,1]\big\}\]
e a $\sigma$-álgebra gerada por $\mathcal C\Times\mathcal D$ é igual a:
\[\sigma(\mathcal C\Times\mathcal D)=\big\{\emptyset,[0,1]\times[0,1],
\big([0,1]\times[0,1]\big)^\compl,\R^2\big\}.\]
No entanto, a $\sigma$-álgebra produto $\mathcal A\otimes\mathcal B$ é igual a:
\begin{multline*}
\mathcal A\otimes\mathcal B=\big\{\emptyset,[0,1]\times[0,1],[0,1]\times[0,1]^\compl,
[0,1]\times\R,[0,1]^\compl\times[0,1],\\
[0,1]^\compl\times[0,1]^\compl,[0,1]^\compl\times\R,\R\times[0,1],\R\times[0,1]^\compl,\\
\big([0,1]\times[0,1]\big)^\compl,\big([0,1]\times[0,1]^\compl\big)^\compl,
\big([0,1]^\compl\times[0,1]\big)^\compl,\\
\big(\R\times[0,1]\big)\cup\big([0,1]\times\R\big),\\
\big([0,1]\times[0,1]\big)\cup\big([0,1]^\compl\times[0,1]^\compl\big),
\big([0,1]^\compl\times[0,1]\big)\cup\big([0,1]\times[0,1]^\compl\big),\R^2\big\}.
\end{multline*}
\end{example}

Apesar do que vimos no Exemplo~\ref{exa:prodnaogera}, temos o seguinte:
\begin{lem}\label{thm:prodgeradores}
Sejam $(X,\mathcal A)$, $(Y,\mathcal B)$ espaços mensuráveis e $\mathcal C$, $\mathcal D$
conjuntos de geradores para as $\sigma$-álgebras $\mathcal A$ e $\mathcal B$ respectivamente. Suponha que
$X$ é igual a uma união enumerável de elementos de $\mathcal C$ e que $Y$ é igual
a uma união enumerável de elementos de $\mathcal D$ (esse é o caso, por exemplo,
se $X\in\mathcal C$ e $Y\in\mathcal D$). Então $\mathcal C\Times\mathcal D$ é um conjunto
de geradores para a $\sigma$-álgebra $\mathcal A\otimes\mathcal B$.
\end{lem}
\begin{proof}
Seja $\mathcal P$ a $\sigma$-álgebra gerada por $\mathcal C\Times\mathcal D$. Como
$\mathcal C\Times\mathcal D$ está contido em $\mathcal A\otimes\mathcal B$, temos que
$\mathcal P\subset\mathcal A\otimes\mathcal B$. Pelo Lema~\ref{thm:sigmaprodmin}, para provar a inclusão
oposta é suficiente verificar que as projeções $\pi_1:X\times Y\to X$, $\pi_2:X\times Y\to Y$
são mensuráveis quando $X\times Y$ é munido da $\sigma$-álgebra $\mathcal P$.
Para todo $A\in\mathcal C$, temos:
\[\pi_1^{-1}(A)=A\times Y;\]
por hipótese, existe uma família enumerável $(Y_i)_{i\in I}$ de elementos de $\mathcal D$
tal que $Y=\bigcup_{i\in I}Y_i$. Daí $A\times Y_i\in\mathcal C\Times\mathcal D$, para
todo $i\in I$ e:
\[\pi_1^{-1}(A)=A\times Y=\bigcup_{i\in I}(A\times Y_i)\in\mathcal P.\]
Segue do Lema~\ref{thm:funcmensgeradores} que a função $\pi_1$ é mensurável quando $X\times Y$
é munido da $\sigma$-álgebra $\mathcal P$. De modo análogo, verifica-se que
$\pi_2$ é mensurável quando $X\times Y$ é munido da $\sigma$-álgebra $\mathcal P$.
Isso completa a demonstração.
\end{proof}

\begin{cor}\label{thm:prodsigmaassoc}
Dados espaços mensuráveis $(X_1,\mathcal A_1)$, $(X_2,\mathcal A_2)$ e $(X_3,\mathcal A_3)$ então:
\[(\mathcal A_1\otimes\mathcal A_2)\otimes\mathcal A_3=\mathcal A_1\otimes(\mathcal A_2\otimes\mathcal A_3).\]
\end{cor}
\begin{proof}
Temos que $\mathcal A_1\Times\mathcal A_2$ é um conjunto de geradores para $\mathcal A_1\otimes\mathcal A_2$ que possui
o conjunto $X_1\times X_2$ como elemento; além do mais, $\mathcal A_3$ é (trivialmente) um conjunto de geradores
para $\mathcal A_3$ que possui o conjunto $X_3$ como elemento. Segue então do Lema~\ref{thm:prodgeradores} que
$(\mathcal A_1\Times\mathcal A_2)\Times\mathcal A_3$ é um conjunto de geradores para $(\mathcal A_1\otimes\mathcal A_2)\otimes\mathcal A_3$.
De modo análogo, vê-se que $\mathcal A_1\Times(\mathcal A_2\Times\mathcal A_3)$ é um conjunto de geradores
para $\mathcal A_1\otimes(\mathcal A_2\otimes\mathcal A_3)$. Obviamente:
\[(\mathcal A_1\Times\mathcal A_2)\Times\mathcal A_3=\mathcal A_1\Times(\mathcal A_2\Times\mathcal A_3)
=\big\{A_1\times A_2\times A_3:A_i\in\mathcal A_i,\ i=1,2,3\big\}.\]
A conclusão segue.
\end{proof}

Em vista do Corolário~\ref{thm:prodsigmaassoc}, podemos escrever expressões como:
\[\mathcal A_1\otimes\cdots\otimes\mathcal A_n\]
sem nos preocuparmos com a colocação de parênteses.
\begin{cor}
Sejam $(X_1,\mathcal A_1)$, \dots, $(X_n,\mathcal A_n)$ espaços men\-su\-rá\-veis. Então $\mathcal A_1\otimes\cdots\otimes\mathcal A_n$
coincide com a $\sigma$-álgebra gerada pela classe de conjuntos:
\[\mathcal A_1\Times\cdots\Times\mathcal A_n=\big\{A_1\times\cdots\times A_n:A_i\in\mathcal A_i,\ i=1,\ldots,n\big\}.\]
\end{cor}
\begin{proof}
Segue facilmente do Lema~\ref{thm:prodgeradores} usando indução.
\end{proof}

\begin{notation}
Dados conjuntos $X$, $Y$ e um subconjunto $U$ de $X\times Y$ então para todo $x\in X$ nós denotamos
por $U_x\subset Y$ a {\em fatia vertical\/}\index[indice]{fatia!vertical} de $U$ definida por:\index[simbolos]{$U_x$}
\begin{equation}\label{eq:fatiaverticalUx}
U_x=\big\{y\in Y:(x,y)\in U\big\}
\end{equation}
e para todo $y\in Y$ nós denotamos por $U^y\subset X$ a {\em fatia horizontal\/}\index[indice]{fatia!horizontal} de $U$ definida por:\index[simbolos]{$U^y$}
\[U^y=\big\{x\in X:(x,y)\in U\big\}.\]
\end{notation}

\begin{rem}\label{thm:obsfatiasmens}
Se as aplicações $i_x:Y\to X\times Y$ e $i^y:X\to X\times Y$ são definidas como no Exemplo~\ref{exa:ixiy} então:
\[U_x=i_x^{-1}(U),\quad U^y=(i^y)^{-1}(U),\]
para todos $x\in X$, $y\in Y$ e todo $U\subset X\times Y$. Se $(X,\mathcal A)$ e $(Y,\mathcal B)$ são espaços
mensuráveis e $U\in\mathcal A\otimes\mathcal B$, nós concluímos então que $U_x\in\mathcal A$ e $U^y\in\mathcal B$,
para todos $x\in X$, $y\in Y$. Se $\nu:\mathcal B\to[0,+\infty]$ é uma medida na $\sigma$-álgebra $\mathcal B$ então
para todo $U\in\mathcal A\otimes\mathcal B$ faz sentido considerar a função:
\begin{equation}\label{eq:medefatia}
X\ni x\longmapsto\nu(U_x)\in[0,+\infty].\
\end{equation}
\end{rem}

Temos o seguinte:
\begin{lem}\label{thm:medefatiasmens}
Sejam $(X,\mathcal A)$ um espaço mensurável e $(Y,\mathcal B,\nu)$ um espaço de medida. Se $U\in\mathcal A\otimes\mathcal B$
e se a medida $\nu$ é $\sigma$-finita então a função \eqref{eq:medefatia} é mensurável.
\end{lem}
\begin{proof}
Assuma primeiramente que a medida $\nu$ é finita. Nós vamos mostrar que:
\begin{equation}\label{eq:medefatia2}
\big\{U\in\mathcal A\otimes\mathcal B:\text{a função \eqref{eq:medefatia} é mensurável}\big\}
\end{equation}
é uma classe $\sigma$-aditiva que contém $\mathcal A\Times\mathcal B$. Como $\mathcal A\Times\mathcal B$
é uma classe de conjuntos fechada por interseções finitas (na verdade, pelo Lema~\ref{thm:prodsemianel}, $\mathcal A\Times\mathcal B$
é até mesmo um semi-anel), seguirá do lema da classe $\sigma$-aditiva (Lema~\ref{thm:lemsigmaclass}) que
\eqref{eq:medefatia2} contém $\mathcal A\otimes\mathcal B$. Isso implicará que a função \eqref{eq:medefatia} é mensurável
para todo $U\in\mathcal A\otimes\mathcal B$, sob a hipótese que a medida $\nu$ é finita.
O fato que \eqref{eq:medefatia2} é uma classe $\sigma$-aditiva segue diretamente das seguintes observações:
\begin{itemize}
\item dados $U,V\in\mathcal A\otimes\mathcal B$ com $U\cap V=\emptyset$ então $(U\cup V)_x=U_x\cup V_x$,
$U_x\cap V_x=\emptyset$ e:
\[\nu\big((U\cup V)_x\big)=\nu(U_x)+\nu(V_x),\]
para todo $x\in X$;
\item dados $U,V\in\mathcal A\otimes\mathcal B$ com $V\subset U$ então $V_x\subset U_x$, $(U\setminus V)_x=U_x\setminus V_x$
e:
\[\nu\big((U\setminus V)_x\big)=\nu(U_x)-\nu(V_x),\]
para todo $x\in X$;
\item se $(U^k)_{k\ge1}$ é uma seqüência em $\mathcal A\otimes\mathcal B$ com $U^k\nearrow U$ então
$U^k_x\nearrow U_x$ e:
\[\nu(U_x)=\lim_{k\to\infty}\nu(U^k_x),\]
para todo $x\in U$.
\end{itemize}
Para ver que \eqref{eq:medefatia2} contém $\mathcal A\Times\mathcal B$, sejam $A\in\mathcal A$, $B\in\mathcal B$ e
$U=A\times B$; temos:
\[\nu(U_x)=\nu(B)\,\chilow A(x),\]
para todo $x\in X$. Logo \eqref{eq:medefatia} é mensurável e $U=A\times B$ está em \eqref{eq:medefatia2}.
Isso completa a demonstração do lema no caso em que a medida $\nu$ é finita. Passemos ao caso geral.
Como a medida $\nu$ é $\sigma$-finita, existe uma seqüência $(Y_k)_{k\ge1}$ em $\mathcal B$ tal que
$Y=\bigcup_{k=1}^\infty Y_k$ e $\nu(Y_k)<+\infty$, para todo $k\ge1$; substituindo $Y_k$
por $Y_k\setminus\bigcup_{i=1}^{k-1}Y_i$ para $k\ge2$, nós podemos supor que os conjuntos $(Y_k)_{k\ge1}$ são
dois a dois disjuntos (veja Exercício~\ref{exe:disjuntar}). Daí, para todo $U\in\mathcal A\otimes\mathcal B$
e todo $x\in X$ temos $U_x=\bigcup_{k=1}^\infty(U_x\cap Y_k)$ e:
\begin{equation}\label{eq:nuUxYk}
\nu(U_x)=\sum_{k=1}^\infty\nu(U_x\cap Y_k).
\end{equation}
Mas $U_x\cap Y_k=\big(U\cap(X\times Y_k)\big)_x$, para todo $x\in X$ e:
\[U\cap(X\times Y_k)\in(\mathcal A\otimes\mathcal B)\vert_{X\times Y_k}=\mathcal A\otimes(\mathcal B\vert_{Y_k}),\]
onde na última igualdade usamos o Lema~\ref{thm:lemarestrprod}. Como a medida $\nu\vert_{\mathcal B\vert_{Y_k}}$
é finita, a primeira parte da demonstração implica que a função:
\[X\ni x\longmapsto\nu\Big(\big(U\cap(X\times Y_k)\big)_x\Big)=\nu(U_x\cap Y_k)\in[0,+\infty]\]
é mensurável para todo $k\ge1$; segue então de \eqref{eq:nuUxYk} que a função \eqref{eq:medefatia} é mensurável. Isso completa
a demonstração.
\end{proof}

\begin{rem}
Se $(X,\mathcal A,\mu)$ é um espaço de medida, $(Y,\mathcal B)$ é um espaço mensurável e se a medida $\mu$ é
$\sigma$-finita então evidentemente para todo $U\in\mathcal A\otimes\mathcal B$ temos que a função:
\[Y\ni y\longmapsto\mu(U^y)\in[0,+\infty]\]
é mensurável. Isso pode ser demonstrado fazendo as modificações óbvias na demonstração do Lema~\ref{thm:medefatiasmens}
ou aplicando o resultado do Lema~\ref{thm:medefatiasmens} ao conjunto $\sigma(U)\in\mathcal B\otimes\mathcal A$,
onde $\sigma$ é definida como no Exemplo~\ref{thm:funcaotroca}.
\end{rem}

\end{section}

\begin{section}{Medidas Produto}

Sejam $(X,\mathcal A,\mu)$, $(Y,\mathcal B,\nu)$ espaços de medida. Nós definimos
uma aplicação $\mu\times\nu:\mathcal A\Times\mathcal B\to[0,+\infty]$ fazendo:
\begin{equation}\label{eq:defmutimesnu}
(\mu\times\nu)(A\times B)=\mu(A)\nu(B),
\end{equation}
para todos $A\in\mathcal A$, $B\in\mathcal B$; recorde da Seção~\ref{sec:AritRetaEstend} que
$x\cdot 0=0\cdot x=0$, para todo $x\in\overline\R$ (mesmo
para $x=\pm\infty$). Observamos que a aplicação $\mu\times\nu$ está de fato bem
definida, pois todo elemento não vazio de $\mathcal A\Times\mathcal B$ escreve-se de modo
único na forma $A\times B$ com $A\in\mathcal A$, $B\in\mathcal B$ e
$\mu(A)\nu(B)=0$, se $A=\emptyset$ ou $B=\emptyset$.

Temos o seguinte:
\begin{lem}\label{thm:mutimesnuehmedida}
Sejam $(X,\mathcal A,\mu)$, $(Y,\mathcal B,\nu)$ espaços de medida.
A aplicação $\mu\times\nu:\mathcal A\Times\mathcal B\to[0,+\infty]$ definida
em \eqref{eq:defmutimesnu} é uma medida em $\mathcal A\Times\mathcal B$.
\end{lem}
\begin{proof}
Evidentemente $(\mu\times\nu)(\emptyset)=0$. Seja $(E^k)_{k\ge1}$ uma seqüência
de elementos dois a dois disjuntos de $\mathcal A\Times\mathcal B$ tal que
$E=\bigcup_{k=1}^\infty E^k$ está em $\mathcal A\Times\mathcal B$. Temos
que a aplicação (recorde \eqref{eq:fatiaverticalUx}):
\[X\ni x\longmapsto\nu(E_x)\in[0,+\infty]\]
é mensurável e:
\[\int_X\nu(E_x)\,\dd\mu(x)=(\mu\times\nu)(E);\]
de fato, basta observar que se $E=A\times B$ com $A\in\mathcal A$, $B\in\mathcal B$ então
\[\nu(E_x)=\nu(B)\,\chilow A(x),\]
para todo $x\in X$ e:
\[\int_X\nu(E_x)\,\dd\mu(x)=\int_X\nu(B)\,\chilow A(x)\,\dd\mu(x)=\mu(A)\nu(B)=(\mu\times\nu)(E).\]
Similarmente, para todo $k\ge1$ a função $x\mapsto\nu(E^k_x)$ é mensurável
e sua integral é igual a $(\mu\times\nu)(E^k)$. Como para todo $x\in X$ a fatia vertical
$E_x$ é igual à união disjunta das fatias verticais $E^k_x$, temos:
\[\nu(E_x)=\sum_{k=1}^\infty\nu(E^k_x).\]
Integrando dos dois lados e usando o resultado do Exercício~\ref{exe:intserienneg} obtemos:
\[(\mu\times\nu)(E)=\int_X\nu(E_x)\,\dd\mu(x)=\sum_{k=1}^\infty\int_X\nu(E^k_x)\,\dd\mu(x)
=\sum_{k=1}^\infty(\mu\times\nu)(E^k).\]
Logo $\mu\times\nu$ é uma medida em $\mathcal A\Times\mathcal B$.
\end{proof}

Se $(X,\mathcal A,\mu)$, $(Y,\mathcal B,\nu)$ são espaços de medida então
$\mathcal A\Times\mathcal B$ é um semi-anel (Lema~\ref{thm:prodsemianel})
e o $\sigma$-anel gerado por $\mathcal A\Times\mathcal B$ coincide com a $\sigma$-álgebra
de partes de $X\times Y$ gerada por $\mathcal A\Times\mathcal B$, já que
$X\times Y\in\mathcal A\Times\mathcal B$ (Exercício~\ref{exe:anelalgerada}).
Segue então do Teorema~\ref{thm:teoextensao} que a medida $\mu\times\nu$ em $\mathcal A\Times\mathcal B$
estende-se a uma medida na $\sigma$-álgebra produto $\mathcal A\otimes\mathcal B$.
Tal extensão não é única em geral. No entanto, se as medidas $\mu$ e $\nu$ são ambas $\sigma$-finitas
então a medida $\mu\times\nu$ em $\mathcal A\Times\mathcal B$ também é $\sigma$-finita;
de fato, se $X=\bigcup_{k=1}^\infty X_k$, $Y=\bigcup_{k=1}^\infty Y_k$ com
$X_k\in\mathcal A$, $Y_k\in\mathcal B$, $\mu(X_k)<+\infty$ e $\nu(Y_k)<+\infty$
para todo $k\ge1$ então $X\times Y=\bigcup_{k=1}^\infty\bigcup_{l=1}^\infty(X_k\times Y_l)$
e $(\mu\times\nu)(X_k\times Y_l)<+\infty$, para todos $k,l\ge1$. Nesse caso,
o Teorema~\ref{thm:teoextensao} nos diz que $\mu\times\nu$ estende-se {\em de modo único\/}
a uma medida em $\mathcal A\otimes\mathcal B$ e essa extensão também é $\sigma$-finita.

\begin{defin}
Sejam $(X,\mathcal A,\mu)$, $(Y,\mathcal B,\nu)$ espaços de medida e suponha que
$\mu$ e $\nu$ são $\sigma$-finitas. A {\em medida produto\/}\index[indice]{medida!produto}\index[indice]{produto!de medidas}
de $\mu$ por $\nu$ é definida como sendo a única medida em $\mathcal A\otimes\mathcal B$ que estende
$\mu\times\nu$. A medida produto será também denotada por $\mu\times\nu$\index[simbolos]{$\mu\times\nu$}
e o espaço de medida $(X\times Y,\mathcal A\otimes\mathcal B,\mu\times\nu)$ é chamado
o {\em produto\/}\index[indice]{produto!de espacos de medida@de espaços de medida}%
\index[indice]{espaco@espaço!de medida!produto} de $(X,\mathcal A,\mu)$ por $(Y,\mathcal B,\nu)$.
\end{defin}
Quando as medidas $\mu$ e $\nu$ não são $\sigma$-finitas, apesar da extensão da medida $\mu\times\nu$
à $\sigma$-álgebra produto $\mathcal A\otimes\mathcal B$ não ser em geral única (veja Exemplo~\ref{exa:mutimesnunaounica} abaixo),
a teoria desenvolvida na Seção~\ref{sec:TeoExtend} nos dá uma extensão natural de $\mu\times\nu$ a $\mathcal A\otimes\mathcal B$
(obtida por restrição da medida exterior $(\mu\times\nu)^*$ determinada por $\mu\times\nu$).
No entanto, os principais teoremas da teoria das medidas produto não são válidos no caso
de medidas não $\sigma$-finitas; optamos então por usar a terminologia ``medida produto'' apenas
no caso em que $\mu$ e $\nu$ são $\sigma$-finitas.

\begin{example}\label{exa:mutimesnunaounica}
Sejam $X=Y=\R$, $\mathcal A=\Lebmens(\R)$ a $\sigma$-álgebra de subconjuntos Lebesgue mensuráveis da reta,
$\mathcal B=\wp(\R)$, $\mu=\leb$ a medida de Lebesgue e $\nu:\wp(\R)\to[0,+\infty]$ a medida de contagem
(veja Definição~\ref{thm:defmedcont}). A medida $\mu$ é $\sigma$-finita, mas a medida $\nu$ não é.
Vamos mostrar que a medida $\mu\times\nu$ em $\mathcal A\Times\mathcal B$ possui ao menos duas extensões
distintas para a $\sigma$-álgebra produto $\mathcal A\otimes\mathcal B$. Seja
$(\mu\times\nu)^*:\wp(\R^2)\to[0,+\infty]$ a medida exterior determinada por $\mu\times\nu$;
segue dos Lemas~\ref{thm:prodsemianel}, \ref{thm:mutimesnuehmedida},
\ref{thm:mustarehmunoS} e do Teorema~\ref{thm:teoextensao} que a restrição de $(\mu\times\nu)^*$
a $\mathcal A\otimes\mathcal B$ é uma medida que estende $\mu\times\nu$. Seja $\Delta$ a diagonal de $\R^2$, isto é:
\[\Delta=\big\{(x,x):x\in\R\big\}.\]
Temos $\Borel(\R^2)=\Borel(\R)\otimes\Borel(\R)$ (veja Exemplo~\ref{exa:prodBorelRn}) e
obviamente:
\[\Borel(\R)\otimes\Borel(\R)\subset\Lebmens(\R)\otimes\wp(\R)=\mathcal A\otimes\mathcal B.\]
Como $\Delta$ é (fechado e portanto) Boreleano em $\R^2$, segue que $\Delta\in\mathcal A\otimes\mathcal B$.
Afirmamos que $(\mu\times\nu)^*(\Delta)=+\infty$. De fato, seja $(A_k\times B_k)_{k\ge1}$ uma seqüência
com $A_k\in\Lebmens(\R)$ e $B_k\in\wp(\R)$, para todo $k\ge1$ e suponha que:
\[\Delta\subset\bigcup_{k=1}^\infty(A_k\times B_k).\]
Para todo $x\in\R$, existe $k\ge1$ tal que $(x,x)\in A_k\times B_k$, i.e., tal que $x\in A_k\cap B_k$; logo:
\[\R=\bigcup_{k=1}^\infty(A_k\cap B_k).\]
Afirmamos que existe algum índice $i\ge1$ tal que $\leb(A_i)>0$ e tal que o conjunto $B_i$ é infinito. De fato, caso contrário,
teríamos que para todo $i\ge1$, $\leb(A_i)=0$ ou $B_i$ é finito; mas isso implicaria que $\leb(A_i\cap B_i)=0$,
para todo $i\ge1$ e portanto $\leb(\R)=0$, uma contradição. Se $i\ge1$ é tal que $\leb(A_i)>0$ e tal que
$B_i$ é infinito então $(\mu\times\nu)(A_i\times B_i)=+\infty$ e {\it a fortiori}:
\[\sum_{k=1}^\infty(\mu\times\nu)(A_k\times B_k)=+\infty.\]
Isso prova que $(\mu\times\nu)^*(\Delta)=+\infty$. Vamos agora exibir uma outra medida $\rho$
em $\mathcal A\otimes\mathcal B$ que estende $\mu\times\nu$ e tal que $\rho(\Delta)=0$. Dado
$E\in\mathcal A\otimes\mathcal B$ então segue da Observação~\ref{thm:obsfatiasmens} que para todo $y\in\R$
a fatia horizontal $E^y\subset\R$ é Lebesgue mensurável e portanto podemos definir:
\[\rho(E)=\sum_{y\in\R}\leb(E^y),\]
para todo $E\in\mathcal A\otimes\mathcal B$. Se $E=A\times B$ com $A\in\mathcal A$, $B\in\mathcal B$ então:
\[\rho(E)=\sum_{y\in B}\leb(A)=\mu(A)\nu(B)=(\mu\times\nu)(E),\]
donde $\rho$ estende $\mu\times\nu$. Afirmamos que $\rho$ é uma medida; de fato, se $(E_k)_{k\ge1}$
é uma seqüência de elementos dois a dois disjuntos de $\mathcal A\otimes\mathcal B$ e se $E=\bigcup_{k=1}^\infty E_k$
então, para todo $y\in\R$, $E^y=\bigcup_{k=1}^\infty E^y_k$ e:
\[\leb(E^y)=\sum_{k=1}^\infty\leb(E^y_k).\]
Daí:
\[\rho(E)=\sum_{y\in\R}\leb(E^y)=\sum_{y\in\R}\sum_{k=1}^\infty\leb(E^y_k)
=\sum_{k=1}^\infty\sum_{y\in\R}\leb(E^y_k)=\sum_{k=1}^\infty\rho(E_k).\]
Finalmente, observe que:
\[\rho(\Delta)=\sum_{y\in\R}\leb\big(\{y\}\big)=0.\]
Daí $\rho$ e $(\mu\times\nu)^*\vert_{\mathcal A\otimes\mathcal B}$ são duas medidas distintas em $\mathcal A\otimes\mathcal B$
que estendem $\mu\times\nu$.
\end{example}

\begin{lem}\label{thm:ehproduto}
Sejam $(X,\mathcal A,\mu)$, $(Y,\mathcal B,\nu)$ espaços de medida e $\mathcal C\subset\mathcal A$,
$\mathcal D\subset\mathcal B$ coleções de conjuntos tais que:
\begin{itemize}
\item[(a)] $\mathcal C$ é um conjunto de geradores para a $\sigma$-álgebra $\mathcal A$ e
$\mathcal D$ é um conjunto de geradores para a $\sigma$-álgebra $\mathcal B$;
\item[(b)] $X$ é uma união enumerável de elementos de $\mathcal C$ e
$Y$ é uma união enumerável de elementos de $\mathcal D$ (esse é o caso, por exemplo, se $X$ está em $\mathcal C$
e $Y$ está em $\mathcal D$);
\item[(c)] $\mathcal C$ e $\mathcal D$ são fechadas por interseções finitas;
\item[(d)] $\emptyset\in\mathcal C$, $\emptyset\in\mathcal D$ e as medidas $\mu\vert_{\mathcal C}$ e $\nu\vert_{\mathcal D}$
são $\sigma$-finitas.
\end{itemize}
Se $\rho:\mathcal A\otimes\mathcal B\to[0,+\infty]$ é uma medida em $\mathcal A\otimes\mathcal B$ tal que
$\rho(A\times B)=\mu(A)\nu(B)$, para todos $A\in\mathcal C$, $B\in\mathcal D$ então $\rho$ é igual à medida
produto $\mu\times\nu$.
\end{lem}

\begin{rem}
As hipóteses~(a) e (b) no Lema~\ref{thm:ehproduto} são equivalentes à condição de que $\mathcal A$ é o $\sigma$-anel
gerado por $\mathcal C$ e $\mathcal B$ é o $\sigma$-anel gerado por $\mathcal D$ (veja Exercícios~\ref{exe:anelalgerada}
e \ref{exe:Xnosigmaneluncount}). Em vista disso, a hipótese~(d) implica que as medidas $\mu$ e $\nu$ são $\sigma$-finitas,
de modo que está bem definida a medida produto $\mu\times\nu$ (veja Observação~\ref{thm:tambemehsigmafin}).
Sobre a hipótese $\emptyset\in\mathcal C$, $\emptyset\in\mathcal D$, veja a nota de rodapé na página~\pageref{foot:sigmauniqueext}.
\end{rem}

\begin{proof}[Demonstração do Lema~\ref{thm:ehproduto}]
Vamos aplicar o Lema~\ref{thm:sigmauniqueext} às medidas $\mu\times\nu$ e $\rho$. Como $X$ é uma união enumerável
de elementos de $\mathcal C$ e cada elemento de $\mathcal C$ está contido numa união enumerável de elementos
de $\mathcal C$ de medida finita, podemos escrever $X=\bigcup_{k=1}^\infty X_k$, com $X_k\in\mathcal C$ e $\mu(X_k)<+\infty$,
para todo $k\ge1$; similarmente, escrevemos $Y=\bigcup_{k=1}^\infty Y_k$, com $Y_k\in\mathcal D$ e $\nu(Y_k)<+\infty$,
para todo $k\ge1$. Temos que todo elemento de $\mathcal C\Times\mathcal D$ está contido em $X\times Y$ e:
\begin{equation}\label{eq:XtimesYXkYl}
X\times Y=\bigcup_{k=1}^\infty\bigcup_{l=1}^\infty(X_k\times Y_l),
\end{equation}
com $(\mu\times\nu)(X_k\times Y_l)=\mu(X_k)\nu(Y_l)<+\infty$, para todos $k,l\ge1$; isso prova que a medida
$(\mu\times\nu)\vert_{\mathcal C\Times\mathcal D}$ é $\sigma$-finita. A igualdade \eqref{eq:XtimesYXkYl} mostra
também que o $\sigma$-anel gerado por $\mathcal C\Times\mathcal D$ coincide com a $\sigma$-álgebra gerada por
$\mathcal C\Times\mathcal D$ (veja Exercício~\ref{exe:anelalgerada}); mas, pelo Lema~\ref{thm:prodgeradores},
a $\sigma$-álgebra gerada por $\mathcal C\Times\mathcal D$ é $\mathcal A\otimes\mathcal B$. Como a classe
de conjuntos $\mathcal C\Times\mathcal D$ é fechada por interseções finitas, o Lema~\ref{thm:sigmauniqueext}
implica que $\mu\times\nu=\rho$.
\end{proof}

\begin{cor}\label{thm:corprodmuassoc}
Dados espaços de medida $(X_1,\mathcal A_1,\mu_1)$, $(X_2,\mathcal A_2,\mu_2)$ e $(X_3,\mathcal A_3,\mu_3)$
com $\mu_1$, $\mu_2$ e $\mu_3$ $\sigma$-finitas então:
\[(\mu_1\times\mu_2)\times\mu_3=\mu_1\times(\mu_2\times\mu_3).\]
\end{cor}
\begin{proof}
Aplique o Lema~\ref{thm:ehproduto} com $X=X_1\times X_2$, $\mathcal A=\mathcal A_1\otimes\mathcal A_2$, $\mu=\mu_1\times\mu_2$,
$\mathcal C=\mathcal A_1\Times\mathcal A_2$, $Y=X_3$, $\mathcal B=\mathcal A_3$, $\nu=\mu_3$, $\mathcal D=\mathcal A_3$
e
\[\rho=\mu_1\times(\mu_2\times\mu_3),\]
notando que $(\mathcal A_1\otimes\mathcal A_2)\otimes\mathcal A_3=\mathcal A_1\otimes(\mathcal A_2\otimes\mathcal A_3)$
(Corolário~\ref{thm:prodsigmaassoc}) e que:
\begin{multline*}
\rho\big((A_1\times A_2)\times A_3\big)=\rho\big(A_1\times(A_2\times A_3)\big)=\mu_1(A_1)\,(\mu_2\times\mu_3)(A_2\times A_3)\\
=\mu_1(A_1)\mu_2(A_2)\mu_3(A_3)=(\mu_1\times\mu_2)(A_1\times A_2)\,\mu_3(A_3)\\
=\big((\mu_1\times\mu_2)\times\mu_3\big)\big((A_1\times A_2)\times A_3\big),
\end{multline*}
para todos $A_1\in\mathcal A_1$, $A_2\in\mathcal A_2$, $A_3\in\mathcal A_3$.
\end{proof}

Em vista do Corolário~\ref{thm:corprodmuassoc}, podemos escrever expressões como:
\[\mu_1\times\cdots\times\mu_n\]
sem nos preocupar com a colocação de parênteses. No Exercício~\ref{exe:muprodvarios} nós pedimos ao leitor
para demonstrar uma versão do Lema~\ref{thm:ehproduto} para o caso de um produto de um número finito arbitrário
de medidas.

\begin{rem}
Sejam $(X,\mathcal A,\mu)$, $(Y,\mathcal B,\nu)$ espaços de medida com $\mu$ e $\nu$
$\sigma$-finitas e sejam $X_0\in\mathcal A$, $Y_0\in\mathcal B$. As medidas
$\mu\vert_{\mathcal A\vert_{X_0}}$ e $\nu\vert_{\mathcal B\vert_{Y_0}}$ também são
$\sigma$-finitas (veja Exercício~\ref{exe:restrsigmafin}) e a medida produto
$(\mu\vert_{\mathcal A\vert_{X_0}})\times(\nu\vert_{\mathcal B\vert_{Y_0}})$
coincide com a restrição a $(\mathcal A\otimes\mathcal B)\vert_{X_0\times Y_0}$
de $\mu\times\nu$. De fato, o Lema~\ref{thm:lemarestrprod} nos diz que:
\[(\mathcal A\vert_{X_0})\otimes(\mathcal B\vert_{Y_0})=(\mathcal A\otimes\mathcal B)\vert_{X_0\times Y_0}\]
e obviamente a restrição de $\mu\times\nu$ a $(\mathcal A\otimes\mathcal B)\vert_{X_0\times Y_0}$
é uma medida que avaliada em $A\times B$ dá $\mu(A)\nu(B)$, para todo $A\in\mathcal A\vert_{X_0}$
e todo $B\in\mathcal B\vert_{Y_0}$.
\end{rem}

\begin{example}\label{exa:prodmedLebesgue}
Se $\mu$ denota a restrição à $\Borel(\R^m)$ da medida de Lebesgue de $\R^m$ e $\nu$ denota a restrição à $\Borel(\R^n)$
da medida de Lebesgue de $\R^n$ então $\mu\times\nu$ é igual à restrição à $\Borel(\R^{m+n})$ da medida de Lebesgue de $\R^{m+n}$.
De fato, vimos no Exemplo~\ref{exa:prodBorelRn} que $\Borel(\R^{m+n})=\Borel(\R^m)\otimes\Borel(\R^n)$; o fato que $\mu\times\nu$
é exatamente a restrição da medida de Lebesgue segue do Lema~\ref{thm:ehproduto}, tomando $X=\R^m$, $\mathcal A=\Borel(\R^m)$,
$Y=\R^n$, $\mathcal B=\Borel(\R^n)$, $\mathcal C$ como sendo a classe dos blocos retangulares $m$-dimensionais,
$\mathcal D$ como sendo a classe dos blocos retangulares $n$-dimensionais e $\rho$ como sendo a restrição a
$\Borel(\R^{m+n})$ da medida de Lebesgue de $\R^{m+n}$ (tenha em mente que, pelo Lema~\ref{thm:abertocubos},
a $\sigma$-álgebra de Borel é gerada pelos blocos retangulares).
\end{example}
Em vista do Exemplo~\ref{exa:prodmedLebesgue} e do resultado do Exercício~\ref{exe:LebcomplBorel},
vemos que poderíamos definir a medida de Lebesgue em $\R^n$ como sendo o completamento do produto de $n$ cópias
da restrição a $\Borel(\R)$ da medida de Lebesgue de $\R$. Podemos então definir a medida de Lebesgue em $\R^n$
sem usar a teoria desenvolvida no Capítulo~\ref{CHP:LEBESGUE}.

Uma pergunta natural agora seria: que resultado obtemos se fizermos o produto da medida de Lebesgue de $\R^m$
pela medida de Lebesgue de $\R^n$ (sem tomar restrições às $\sigma$-álgebras de Borel)? A resposta é que nesse
caso obtemos a restrição da medida de Lebesgue a uma $\sigma$-álgebra intermediária entre $\Borel(\R^{m+n})$
e $\Lebmens(\R^{m+n})$ (veja Lema~\ref{thm:prodcomplet}).

Vejamos agora como a medida produto $\mu\times\nu$ pode ser escrita usando uma integral.
\begin{prop}\label{thm:Fubinicarac}
Sejam $(X,\mathcal A,\mu)$, $(Y,\mathcal B,\nu)$ espaços de medida, com $\mu$ e $\nu$ $\sigma$-finitas.
Dado $U\in\mathcal A\otimes\mathcal B$ então:
\begin{equation}\label{eq:medprodinteg}
(\mu\times\nu)(U)=\int_X\nu(U_x)\,\dd\mu(x).
\end{equation}
\end{prop}
\begin{proof}
Recorde do Lema~\ref{thm:medefatiasmens} que para todo $U\in\mathcal A\otimes\mathcal B$ a função
$X\ni x\mapsto\nu(U_x)\in[0,+\infty]$ é mensurável, de modo que a integral em \eqref{eq:medprodinteg}
está bem definida. Defina uma aplicação $\rho:\mathcal A\otimes\mathcal B\to[0,+\infty]$ fazendo:
\[\rho(U)=\int_X\nu(U_x)\,\dd\mu(x),\]
para todo $U\in\mathcal A\otimes\mathcal B$. A demonstração da proposição estará completa se verificarmos que
$\rho$ é uma medida e que $\rho$ coincide com $\mu\times\nu$ em $\mathcal A\Times\mathcal B$.
A demonstração de que $\rho$ coincide com $\mu\times\nu$ em $\mathcal A\Times\mathcal B$
foi feita durante a prova do Lema~\ref{thm:mutimesnuehmedida}. Vamos então provar que $\rho$
é uma medida. Seja $(U^k)_{k\ge1}$ uma seqüência de elementos dois a dois disjuntos de $\mathcal A\otimes\mathcal B$ e seja
$U=\bigcup_{k=1}^\infty U^k$. Para cada $x\in X$, temos que a fatia vertical $U_x$ é igual à união disjunta das fatias
verticais $U^k_x$, $k\ge1$, e portanto:
\[\nu(U_x)=\sum_{k=1}^\infty\nu(U^k_x),\]
para todo $x\in X$. Integrando dos dois lados e usando o resultado do Exercício~\ref{exe:intserienneg} obtemos:
\[\rho(E)=\int_X\nu(E_x)\,\dd\mu(x)=\sum_{k=1}^\infty\int_X\nu(E^k_x)\,\dd\mu(x)
=\sum_{k=1}^\infty\rho(E^k).\]
Isso prova que $\rho$ é uma medida e completa a demonstração.
\end{proof}

\begin{rem}
A tese da Proposição~\ref{thm:Fubinicarac} poderia ser substituída por:
\[(\mu\times\nu)(U)=\int_Y\mu(U^y)\,\dd\nu(y),\]
para todo $U\in\mathcal A\otimes\mathcal B$.
Isso pode ser demonstrado fazendo as modificações óbvias na demonstração da Proposição~\ref{thm:Fubinicarac}
ou aplicando o resultado dessa proposição ao conjunto $\sigma(U)\in\mathcal B\otimes\mathcal A$,
onde $\sigma$ é definida como no Exemplo~\ref{thm:funcaotroca} (veja também o resultado do Exercício~\ref{exe:trocapresmed}).

Em particular, se $\mu$ e $\nu$ são $\sigma$-finitas, então para todo $U\in\mathcal A\otimes\mathcal B$, temos:
\[\int_X\nu(U_x)\,\dd\mu(x)=\int_Y\mu(U^y)\,\dd\nu(y).\]
\end{rem}

\end{section}

\begin{section}{O Teorema de Fubini}

\begin{teo}[Fubini--Tonelli]\index[indice]{teorema!de Fubini--Tonelli!abstrato}\index[indice]{Fubini!teorema de!abstrato}\index[indice]{Tonelli!teorema de!abstrato}
\label{thm:Fubiniabs}
Sejam $(X,\mathcal A,\mu)$, $(Y,\mathcal B,\nu)$ espaços de medida, com $\mu$ e $\nu$ $\sigma$-finitas.
Seja $f:X\times Y\to\overline\R$ uma função mensurável, onde $X\times Y$ é munido da $\sigma$-álgebra produto $\mathcal A\otimes\mathcal B$.
Se $f$ é quase integrável então:
\begin{itemize}
\item[(a)] para todo $x\in X$, a função $Y\ni y\mapsto f(x,y)\in\overline\R$ é mensurável;
\item[(b)] o conjunto:
\[X_0=\big\{x\in X:\text{a função $Y\ni y\mapsto f(x,y)\in\overline\R$ é quase integrável}\big\}\]
é mensurável e $\mu(X\setminus X_0)=0$;
\item[(c)] a função $X_0\ni x\mapsto\int_Yf(x,y)\,\dd\nu(y)\in\overline\R$ é
quase integrável;
\item[(d)] vale a igualdade:
\[\int_{X_0}\Big(\int_Yf(x,y)\,\dd\nu(y)\Big)\,\dd\mu(x)=\int_{X\times Y}f(x,y)\,\dd(\mu\times\nu)(x,y).\]
\end{itemize}
\end{teo}

Observe que se a função $f$ é não negativa então $X_0=X$ e a afirmação que aparece
no item~(b) é trivial.

\begin{proof}
A validade do item~(a) segue diretamente do Exemplo~\ref{exa:fvarmens}.
Dividimos o restante da demonstração em itens.
\begin{bulletindent}
\item {\em O teorema vale se $f$ é simples, mensurável e não negativa}.

Podemos escrever $f=\sum_{i=1}^kc_i\chilow{A^i}$, com $c_i\in[0,+\infty]$ e $A^i$
um subconjunto mensurável de $X\times Y$, para $i=1,\ldots,k$. Note que,
se $x\in X$, temos:
\begin{equation}\label{eq:fxysimples2}
f(x,y)=\sum_{i=1}^kc_i\chilow{A^i_x}(y),
\end{equation}
para todo $y\in Y$. Logo, usando a Proposição~\ref{thm:Fubinicarac}, obtemos:
\begin{multline*}
\int_X\Big(\int_Yf(x,y)\,\dd\nu(y)\Big)\,\dd\mu(x)=\int_X\sum_{i=1}^kc_i\nu(A^i_x)\,\dd\mu(x)\\
=\sum_{i=1}^kc_i\int_X\nu(A^i_x)\,\dd\mu(x)
=\sum_{i=1}^kc_i\,(\mu\times\nu)(A^i)\\
=\int_{X\times Y}f(x,y)\,\dd(\mu\times\nu)(x,y).
\end{multline*}

\item {\em O teorema vale se $f$ é mensurável e não negativa}.

Seja $(f_k)_{k\ge1}$ uma seqüências de funções $f_k:X\times Y\to[0,+\infty]$
simples e mensuráveis com $f_k\nearrow f$. Pelo Teorema da Convergência Monotônica,
temos:
\[\int_Yf(x,y)\,\dd\nu(y)=\lim_{k\to\infty}\int_Yf_k(x,y)\,\dd\nu(y),\]
para todo $x\in X$. Logo a função $x\mapsto\int_Yf(x,y)\,\dd\nu(y)$
é mensurável e, usando novamente o Teorema da Convergência Monotônica, obtemos:
\begin{multline*}
\int_X\Big(\int_Yf(x,y)\,\dd\nu(y)\Big)\,\dd\mu(x)=\lim_{k\to\infty}
\int_X\Big(\int_Yf_k(x,y)\,\dd\nu(y)\Big)\,\dd\mu(x)\\
=\lim_{k\to\infty}\int_{X\times Y}f_k(x,y)\,\dd(\mu\times\nu)(x,y)
=\int_{X\times Y}f(x,y)\,\dd(\mu\times\nu)(x,y).
\end{multline*}

\item {\em O teorema vale se $f$ é quase integrável}.

Como $f^+$ e $f^-$ são funções mensuráveis não negativas, temos:
\begin{gather}
\int_X\Big(\int_Yf^+(x,y)\,\dd\nu(y)\Big)\,\dd\mu(x)=\int_{X\times Y}f^+(x,y)\,\dd(\mu\times\nu)(x,y),\label{eq:Fubinif+2}\\
\int_X\Big(\int_Yf^-(x,y)\,\dd\nu(y)\Big)\,\dd\mu(x)=\int_{X\times Y}f^-(x,y)\,\dd(\mu\times\nu)(x,y).\label{eq:Fubinif-2}
\end{gather}
O conjunto $X_0$ é igual ao conjunto dos pontos de $X$ onde ao menos uma das funções:
\[X\ni x\longmapsto\int_Yf^+(x,y)\,\dd\nu(y),\quad
X\ni x\longmapsto\int_Yf^-(x,y)\,\dd\nu(y),\]
é finita; como essas funções são ambas mensuráveis, segue que o conjunto $X_0$ é mensurável.
Como $f$ é quase integrável, temos que $f^+$ é integrável ou $f^-$ é integrável;
para fixar as idéias, vamos supor que:
\[\int_{X\times Y}f^-\,\dd(\mu\times\nu)<+\infty.\]
Tendo em mente o resultado do Exercício~\ref{exe:finitaqs}, segue de \eqref{eq:Fubinif-2} que:
\begin{equation}\label{eq:intYf-finita}
\int_Yf^-(x,y)\,\dd\nu(y)<+\infty,
\end{equation}
para quase todo $x\in X$. Como o conjunto $X_0$ contém os pontos $x\in X$ tais que
\eqref{eq:intYf-finita} vale, concluímos que $\mu(X\setminus X_0)=0$. Além do mais,
de \eqref{eq:Fubinif+2} e \eqref{eq:Fubinif-2} vem:
\begin{multline*}
\hfil\int_{X_0}\Big(\int_Yf(x,y)\,\dd\nu(y)\Big)\,\dd\mu(x)=\\
\int_{X_0}\Big(\int_Yf^+(x,y)\,\dd\nu(y)\Big)\,\dd\mu(x)
-\int_{X_0}\Big(\int_Yf^-(x,y)\,\dd\nu(y)\Big)\,\dd\mu(x)\\
=\int_X\Big(\int_Yf^+(x,y)\,\dd\nu(y)\Big)\,\dd\mu(x)
-\int_X\Big(\int_Yf^-(x,y)\,\dd\nu(y)\Big)\,\dd\mu(x)\\
=\int_{X\times Y}f^+(x,y)\,\dd(\mu\times\nu)(x,y)-\int_{X\times Y}f^-(x,y)\,\dd(\mu\times\nu)(x,y)\\
=\int_{X\times Y}f(x,y)\,\dd(\mu\times\nu)(x,y),
\end{multline*}
onde usamos também o Corolário~\ref{thm:corXlinhazero}.\qedhere
\end{bulletindent}
\end{proof}

\end{section}

\begin{section}{O Completamento da Medida Produto}
\label{sec:complFubini}

\begin{lem}\label{thm:prodcomplet}
Sejam $(X,\mathcal A,\mu)$, $(Y,\mathcal B,\nu)$ espaços de medida, com $\mu$ e $\nu$
$\sigma$-finitas. Denote por $\bar\mu:\overline{\mathcal A}\to[0,+\infty]$,
$\bar\nu:\overline{\mathcal B}\to[0,+\infty]$, respectivamente os completamentos
de $\mu$ e de $\nu$. Então $\bar\mu\times\bar\nu$ é uma extensão de $\mu\times\nu$
e o completamento $\overline{\mu\times\nu}$ de $\mu\times\nu$ é uma extensão
de $\bar\mu\times\bar\nu$.
\end{lem}

Note que, pelo resultado do Exercício~\ref{exe:complsigmafin}, o completamento
de uma medida $\sigma$-finita ainda é $\sigma$-finita, de modo que faz sentido
considerar a medida produto $\bar\mu\times\bar\nu$.

\begin{proof}
Segue do resultado do Exercício~\ref{exe:extendprod} que $\bar\mu\times\bar\nu$
é uma extensão de $\mu\times\nu$. Denote por $\overline{\mathcal A\otimes\mathcal B}$
o domínio de $\overline{\mu\times\nu}$. Para mostrar que $\overline{\mu\times\nu}$
é uma extensão de $\bar\mu\times\bar\nu$, é suficiente mostrar que:
\[\overline{\mathcal A}\Times\overline{\mathcal B}\subset\overline{\mathcal A\otimes\mathcal B}\]
e que:
\[\overline{\mu\times\nu}\,(E\times F)=\bar\mu(E)\bar\nu(F),\]
para todos $E\in\overline{\mathcal A}$, $F\in\overline{\mathcal B}$.
Dados $E\in\overline{\mathcal A}$, $F\in\overline{\mathcal B}$, podemos escrever
$E=A\cup N$, $F=B\cup N'$ com $N\subset M$, $N'\subset M'$, $A,M\in\mathcal A$,
$B,M'\in\mathcal B$, $\mu(M)=0$ e $\nu(M')=0$. Temos:
\[E\times F=(A\times B)\cup\big((A\times N')\cup(N\times B)\cup(N\times N')\big),\]
com:
\[(A\times N')\cup(N\times B)\cup(N\times N')\subset
(A\times M')\cup(M\times B)\cup(M\times M'),\]
e:
\begin{multline*}
(\mu\times\nu)\big((A\times M')\cup(M\times B)\cup(M\times M')\big)
\le\mu(A)\nu(M')+\mu(M)\nu(B)\\+\mu(M)\nu(M')=0.
\end{multline*}
Isso mostra que $E\times F\in\overline{\mathcal A\otimes\mathcal B}$ e que:
\[\overline{\mu\times\nu}\,(E\times F)=(\mu\times\nu)(A\times B)=\mu(A)\nu(B)
=\bar\mu(E)\bar\nu(F).\qedhere\]
\end{proof}

\begin{cor}
Nas condições do Lema~\ref{thm:prodcomplet}, temos que $\overline{\mu\times\nu}$ é o completamento
da medida $\bar\mu\times\bar\nu$.
\end{cor}
\begin{proof}
Segue diretamente do Lema~\ref{thm:prodcomplet} e do resultado do Exercício~\ref{exe:complintermed}.
\end{proof}

\begin{teo}[Fubini--Tonelli]\index[indice]{teorema!de Fubini--Tonelli!para o completamento}%
\index[indice]{Fubini!teorema de!para o completamento}\index[indice]{Tonelli!teorema de!para o completamento}
\label{thm:Fubinicompl}
Sejam $(X,\mathcal A,\mu)$, $(Y,\mathcal B,\nu)$ espaços de medida, com $\mu$ e $\nu$ $\sigma$-finitas
e completas. Seja $\overline{\mu\times\nu}:\overline{\mathcal A\otimes\mathcal B}\to[0,+\infty]$
o completamento da medida produto $\mu\times\nu$. Se $f:(X\times Y,\overline{\mathcal A\otimes\mathcal B})\to\overline\R$
é uma função mensurável, então os itens~(b), (c) e (d) da tese do Teorema~\ref{thm:Fubiniabs} valem.
\end{teo}
\begin{proof}
Segue dos resultados dos Exercícios~\ref{exe:Borelsepar} e \ref{exe:quaseigualmens}
que existem uma função mensurável $g:(X\times Y,\mathcal A\otimes\mathcal B)\to\overline\R$
e um conjunto $U\in\mathcal A\otimes\mathcal B$ de modo que $f(x,y)=g(x,y)$ para todo $(x,y)\in(X\times Y)\setminus U$
e $(\mu\times\nu)(U)=0$. Pela Proposição~\ref{thm:Fubinicarac}, temos:
\[\int_X\nu(U_x)\,\dd\mu(x)=(\mu\times\nu)(U)=0,\]
e portanto (veja Exercício~\ref{exe:intzerofzeroqs}) $\nu(U_x)=0$, para quase todo $x\in X$. Seja $N\in\mathcal A$
um conjunto tal que $\mu(N)=0$ e tal que $\nu(U_x)=0$, para todo $x\in X\setminus N$. Dado $x\in X$, temos que
$f(x,y)=g(x,y)$ para todo $y\in Y\setminus U_x$ e portanto, se $x\in X\setminus N$, temos $f(x,y)=g(x,y)$
para quase todo $y\in Y$. Sejam:
\begin{gather*}
X_0=\big\{x\in X:\text{a função $Y\ni y\mapsto f(x,y)\in\overline\R$ é quase integrável}\big\},\\
X_1=\big\{x\in X:\text{a função $Y\ni y\mapsto g(x,y)\in\overline\R$ é quase integrável}\big\};
\end{gather*}
se $x\in X\setminus N$ temos que $x\in X_0$ se e somente se $x\in X_1$ (veja Exercício~\ref{exe:fquaseiggmens}
e Corolário~\ref{thm:fquaseiggintr})) e:
\begin{equation}\label{eq:intfequalintgdy}
\int_Yf(x,y)\,\dd\nu(y)=\int_Yg(x,y)\,\dd\nu(y),
\end{equation}
para todo $x\in(X_0\cap X_1)\setminus N$. Aplicando o Teorema~\ref{thm:Fubiniabs} para a função $g$, vemos
que o conjunto $X_1$ é mensurável, $\mu(X\setminus X_1)=0$ e:
\begin{equation}\label{eq:teoFubprag}
\int_{X_1}\Big(\int_Yg(x,y)\,\dd\nu(y)\Big)\,\dd\mu(x)=\int_{X\times Y}g(x,y)\,\dd(\mu\times\nu)(x,y).
\end{equation}
Temos $X\setminus X_0\subset(X\setminus X_1)\cup N$ e $\mu\big((X\setminus X_1)\cup N\big)=0$; portanto,
como $\mu$ é completa, $X\setminus X_0$ e $X_0$ são mensuráveis e $\mu(X\setminus X_0)=0$. Seja:
\[R=(X_0\cap X_1)\setminus N;\]
temos $X\setminus R\subset(X\setminus X_0)\cup(X\setminus X_1)\cup N$,
donde $\mu(R)=0$. Daí (veja Corolário~\ref{thm:corXlinhazero}):
\begin{multline}\label{eq:gvirafX0X1}
\int_{X_1}\Big(\int_Yg(x,y)\,\dd\nu(y)\Big)\,\dd\mu(x)=
\int_R\Big(\int_Yg(x,y)\,\dd\nu(y)\Big)\,\dd\mu(x)\\
\stackrel{\eqref{eq:intfequalintgdy}}=\int_R\Big(\int_Yf(x,y)\,\dd\nu(y)\Big)\,\dd\mu(x)=
\int_{X_0}\Big(\int_Yf(x,y)\,\dd\nu(y)\Big)\,\dd\mu(x).
\end{multline}
A conclusão segue de \eqref{eq:teoFubprag} e \eqref{eq:gvirafX0X1}, observando que:
\begin{multline*}
\int_{X\times Y}g(x,y)\,\dd(\mu\times\nu)(x,y)=
\int_{X\times Y}g(x,y)\,\dd(\overline{\mu\times\nu})(x,y)\\
=\int_{X\times Y}f(x,y)\,\dd(\overline{\mu\times\nu})(x,y),
\end{multline*}
onde na primeira igualdade usamos o resultado do Exercício~\ref{exe:sigmaalgmenor}
e na segunda usamos o Corolário~\ref{thm:fquaseiggintr}.
\end{proof}
\end{section}

\section*{Exercícios para o Capítulo~\ref{CHP:PRODUTOS}}

\subsection*{Produto de ${\sigma}$-Álgebras}

\begin{exercise}\label{exe:universalprod}
Sejam $(X,\mathcal A)$, $(Y,\mathcal B)$ espaços mensuráveis e seja $\mathcal P$ uma
$\sigma$-álgebra de partes de $X\times Y$. Mostre que as seguintes condições
são equivalentes:
\begin{itemize}
\item[(a)] $\mathcal P=\mathcal A\otimes\mathcal B$;
\item[(b)] para todo espaço mensurável $(Z,\mathfrak C)$ e toda função
$f:Z\to X\times Y$ com funções coordenadas $f_1:Z\to X$, $f_2:Z\to Y$, temos que
$f:Z\to(X\times Y,\mathcal P)$ é mensurável se e somente se $f_1$ e $f_2$ são ambas mensuráveis.
\end{itemize}
\end{exercise}

\begin{exercise}\label{exe:melhorprodgeradores}
Sejam $\mathcal C$, $\mathcal D$ classes de conjuntos e $\mathcal A$, $\mathcal B$
respectivamente os $\sigma$-anéis gerados por $\mathcal C$ e por $\mathcal D$.
Seja $\mathcal P$ o $\sigma$-anel gerado por $\mathcal C\Times\mathcal D$.
\begin{itemize}
\item[(a)] Dados conjuntos $A_0$ e $B_0$, mostre que as classes de conjuntos:
\begin{gather*}
\big\{A\in\mathcal A:A\times B_0\in\mathcal P\big\},\\
\big\{B\in\mathcal B:A_0\times B\in\mathcal P\big\},
\end{gather*}
são $\sigma$-anéis.
\item[(b)] Mostre que $A\times B_0\in\mathcal P$, para todos $A\in\mathcal A$, $B_0\in\mathcal D$.
\item[(c)] Mostre que $A\times B\in\mathcal P$, para todos $A\in\mathcal A$, $B\in\mathcal B$.
\item[(d)] Conclua que o $\sigma$-anel gerado por $\mathcal A\Times\mathcal B$ é igual
ao $\sigma$-anel gerado por $\mathcal C\Times\mathcal D$.
\end{itemize}
\end{exercise}

\begin{exercise}
Sejam $(X_1,\mathcal A_1)$, \dots, $(X_n,\mathcal A_n)$ espaços mensuráveis. Mostre que
$\mathcal A_1\otimes\cdots\otimes\mathcal A_n$ é a menor $\sigma$-álgebra de partes de $X_1\times\cdots\times X_n$
que torna todas as projeções $\pi_i:X_1\times\cdots\times X_n\to X_i$, $i=1,\ldots,n$, mensuráveis.
\end{exercise}

\begin{exercise}
Sejam $(X_1,\mathcal A_1)$, \dots, $(X_n,\mathcal A_n)$, $(Y,\mathcal B)$ espaços mensuráveis e
$f:Y\to X_1\times\cdots\times X_n$ uma função com funções coordenadas $f_i:Y\to X_i$, $i=1,\ldots,n$.
Se $X_1\times\cdots\times X_n$ é munido da $\sigma$-álgebra produto $\mathcal A_1\otimes\cdots\otimes\mathcal A_n$,
mostre que $f$ é mensurável se e somente se todas as funções coordenadas $f_i$, $i=1,\ldots,n$, são mensuráveis.
\end{exercise}

\subsection*{Medidas Produto}

\begin{exercise}\label{exe:muprodvarios}
Sejam $(X_1,\mathcal A_1,\mu_1)$, \dots, $(X_n,\mathcal A_n,\mu_n)$ espaços de medidas, com $\mu_1$, \dots, $\mu_n$
$\sigma$-finitas. Para cada $i=1,\ldots,n$, seja $\mathcal C_i\subset\mathcal A_i$ uma coleção de conjuntos tal que:
\begin{itemize}
\item $\mathcal C_i$ é um conjunto de geradores para a $\sigma$-álgebra $\mathcal A_i$;
\item $X_i$ é uma união enumerável de elementos de $\mathcal C_i$ (esse é o caso, por exemplo, se $X_i\in\mathcal C_i$);
\item $\mathcal C_i$ é fechado por interseções finitas;
\item $\emptyset\in\mathcal C_i$ e a medida $\mu_i\vert_{\mathcal C_i}$ é $\sigma$-finita.
\end{itemize}
Se $\rho:\mathcal A_1\otimes\cdots\otimes\mathcal A_n\to[0,+\infty]$ é uma medida tal que
\[\rho(A_1\times\cdots\times A_n)=\mu_1(A_1)\cdots\mu_n(A_n),\]
para todos $A_1\in\mathcal C_1$, \dots, $A_n\in\mathcal C_n$,
mostre que $\rho$ é igual à medida produto $\mu_1\times\cdots\times\mu_n$.
\end{exercise}

\begin{exercise}\label{exe:trocapresmed}
Sejam $(X,\mathcal A,\mu)$, $(Y,\mathcal B,\nu)$ espaços de medida, com $\mu$ e $\nu$ $\sigma$-finitas.
Denote por $\sigma:X\times Y\to Y\times X$ a aplicação definida no Exemplo~\ref{thm:funcaotroca}. Se $X\times Y$
e $Y\times X$ são munidos respectivamente das medidas produto $\mu\times\nu$ e $\nu\times\mu$, mostre
que a aplicação $\sigma$ preserva medida (veja Definição~\ref{thm:measurepres}).
\end{exercise}

\begin{exercise}\label{exe:extendprod}
Sejam $(X,\mathcal A,\mu)$, $(Y,\mathcal B,\nu)$ espaços de medida e sejam $\mathcal A_0\subset\mathcal A$,
$\mathcal B_0\subset\mathcal B$ $\sigma$-álgebras. Assuma que as medidas
$\mu$, $\nu$, $\mu\vert_{\mathcal A_0}$ e $\nu\vert_{\mathcal B_0}$ sejam todas
$\sigma$-finitas. Mostre que a medida produto $\mu\times\nu$ é uma extensão
da medida $(\mu\vert_{\mathcal A_0})\times(\nu\vert_{\mathcal B_0})$.
\end{exercise}

\end{chapter}

\begin{chapter}{Conjuntos Analíticos e o Teorema de Choquet}

\begin{section}{Espaços Poloneses e seus Boreleanos}

Dado um conjunto $X$, denotamos por $\Delta_X$\index[simbolos]{$\Delta_X$} a diagonal\index[indice]{diagonal} do produto cartesiano
$X\times X$, isto é:
\[\Delta_X=\big\{(x,x):x\in X\big\}.\]

\begin{lem}
Para todo subconjunto $S$ de $\N^\N$ o conjunto $\Delta_S$ pertence à $\sigma$-álgebra
$\Borel(\N^\N)\otimes\wp(\N^\N)$; mais precisamente, $\Delta_S$ é uma interseção enumerável de uniões enumeráveis
de elementos de $\Borel(\N^\N)\Times\wp(\N^\N)$.
\end{lem}
\begin{proof}
Dados $n,m\in\N$, seja:
\[A_{nm}=\big\{\alpha\in\N^\N:m=\alpha(n)\big\};\]
temos que $A_{nm}$ é fechado em $\N^\N$, sendo a imagem inversa do ponto $m$ pela função contínua
$\N^\N\ni\alpha\mapsto\alpha(n)\in\N$. Em particular, temos $A_{nm}\in\Borel(\N^\N)$. Seja também:
\[B_{nm}=A_{nm}\cap S\in\wp(\N^\N),\]
para todos $n,m\in\N$. Afirmamos que o conjunto $\Delta_S$ é igual a:
\begin{equation}\label{eq:capcupAnmBnm}
\bigcap_{n\in\N}\bigcup_{m\in\N}(A_{nm}\times B_{nm}).
\end{equation}
De fato, dado $(\alpha,\beta)\in(\N^\N)\times(\N^\N)$ então $(\alpha,\beta)$ pertence a \eqref{eq:capcupAnmBnm}
se e somente se para todo $n\in\N$ existe $m\in\N$ tal que $m=\alpha(n)$, $m=\beta(n)$ e $\beta\in S$;
mas temos que existe $m\in\N$ tal que $m=\alpha(n)$, $m=\beta(n)$ e $\beta\in S$ se e somente se $\alpha(n)=\beta(n)$
e $\beta\in S$. Concluímos então que $(\alpha,\beta)$ está em \eqref{eq:capcupAnmBnm}
se e somente se $\beta\in S$ e $\alpha(n)=\beta(n)$, para todo $n\in\N$, isto é, se e somente se $\alpha=\beta$
e $\beta\in S$. Logo \eqref{eq:capcupAnmBnm} é igual a $\Delta_S$.
\end{proof}

\end{section}

\end{chapter}

\begingroup
\appendix

\begin{chapter}{Soluções para os Exercícios Propostos}

\begin{section}{Exercícios do Capítulo~\ref{CHP:LEBESGUE}}

\textbf{Exercício~\ref{exe:lebouterreg}.}\enspace Pelo Lema~\ref{thm:lebmonotonica}, temos
$\leb^*(A)\le\leb^*(U)=\leb(U)$, para todo aberto $U\subset\R^n$ contendo $A$. Logo
$\leb^*(A)$ é uma conta inferior do conjunto $\big\{\leb(U):\text{$U\supset A$ aberto}\big\}$.
Para ver que $\leb^*(A)$ é a maior cota inferior desse conjunto, devemos mostrar que para todo
$\varepsilon>0$ existe $U\supset A$ aberto com $\leb(U)\le\leb^*(A)+\varepsilon$. Mas esse é precisamente
o resultado do Lema~\ref{thm:aproxaberto}.

\medskip

\textbf{Exercício~\ref{exe:translmens}.}\enspace Como $A$ é mensurável então, para todo $\varepsilon>0$ existe
um aberto $U\supset A$ com $\leb^*(U\setminus A)<\varepsilon$.
Daí $U+x$ é um aberto em $\R^n$ contendo
$A+x$ e $(U+x)\setminus(A+x)=(U\setminus A)+x$. Logo, pelo Lema~\ref{thm:extmeastransinv}, temos
$\leb^*\big((U+x)\setminus(A+x)\big)=\leb^*(U\setminus A)<\varepsilon$.

\medskip

\textbf{Exercício~\ref{exe:permutacao}.}
\begin{itemize}
\item[(a)] O resultado é claro se $B$ é vazio. Senão, $B=\prod_{i=1}^n[a_i,b_i]$ e
\[\widehat\sigma(B)=\prod_{i=1}^n[a_{\sigma(i)},b_{\sigma(i)}]\]
também é um bloco retangular $n$-dimensional e:
\[\vert\widehat\sigma(B)\vert=\prod_{i=1}^n(b_{\sigma(i)}-a_{\sigma(i)})=\prod_{i=1}^n(b_i-a_i)=\vert B\vert.\]

\smallskip

\item[(b)] Se $A\subset\bigcup_{k=1}^\infty B_k$ é uma cobertura de $A$ por blocos ratangulares
$n$-dimensionais então $\widehat\sigma(A)\subset\bigcup_{k=1}^\infty\widehat\sigma(B_k)$
é uma cobertura de $\widehat\sigma(A)$ por blocos retangulares $n$-dimensionais e
\[\sum_{k=1}^\infty\vert\widehat\sigma(B_k)\vert=\sum_{k=1}^\infty\vert B_k\vert.\]
Isso mostra que $\mathcal C(A)\subset\mathcal C\big(\widehat\sigma(A)\big)$
(recorde \eqref{eq:defCA}). Por outro lado, se $\tau=\sigma^{-1}$
então $A=\widehat\tau\big(\widehat\sigma(A)\big)$ e daí o mesmo argumento mostra que
$\mathcal C\big(\widehat\sigma(A)\big)\subset\mathcal C(A)$; logo:
\[\leb^*(A)=\inf\,\mathcal C(A)=\inf\,\mathcal C\big(\widehat\sigma(A)\big)=\leb^*\big(\widehat\sigma(A)\big).\]

\smallskip

\item[(c)] Se $A$ é mensurável então para todo $\varepsilon>0$ existe um aberto $U\subset\R^n$
contendo $A$ tal que $\leb^*(U\setminus A)<\varepsilon$. Daí $\widehat\sigma(U)$ é um aberto
contendo $\widehat\sigma(A)$ e:
\[\leb^*\big(\widehat\sigma(U)\setminus\widehat\sigma(A)\big)=\leb^*\big(\widehat\sigma(U\setminus A)\big)
=\leb^*(U\setminus A)<\varepsilon,\]
provando que $\widehat\sigma(A)$ é mensurável.
\end{itemize}

\medskip

\textbf{Exercício~\ref{exe:diagonal}.}
\begin{itemize}
\item[(a)] O resultado é claro se $B$ é vazio. Senão, $B=\prod_{i=1}^n[a_i,b_i]$ e
\[D_\lambda(B)=\prod_{i=1}^n[a'_i,b'_i],\]
onde $a'_i=\lambda_ia_i$, $b'_i=\lambda_ib_i$ se $\lambda_i>0$ e $a'_i=\lambda_ib_i$,
$b'_i=\lambda_ia_i$ se $\lambda_i<0$; em todo caso:
\[\vert D_\lambda(B)\vert=\prod_{i=1}^n(b'_i-a'_i)=\prod_{i=1}^n\vert\lambda_i\vert(b_i-a_i)
=\vert\det D_\lambda\vert\,\vert B\vert.\]

\smallskip

\item[(b)] Se $A\subset\bigcup_{k=1}^\infty B_k$ é uma cobertura de $A$ por blocos retangulares
$n$-dimensionais então $D_\lambda(A)\subset\bigcup_{k=1}^\infty D_\lambda(B_k)$
é uma cobertura de $D_\lambda(A)$ por blocos retangulares $n$-dimensionais
e
\[\sum_{k=1}^\infty\vert D_\lambda(B_k)\vert=\vert\det D_\lambda\vert\sum_{k=1}^\infty\vert B_k\vert.\]
Isso mostra que (recorde \eqref{eq:defCA}):
\begin{equation}\label{eq:incDlambdaCA1}
\vert\det D_\lambda\vert\,\mathcal C(A)=\big\{\vert\det D_\lambda\vert\,a:a\in\mathcal C(A)\big\}
\subset\mathcal C\big(D_\lambda(A)\big).
\end{equation}
Por outro lado, se $\mu=\big(\frac1{\lambda_1},\ldots,\frac1{\lambda_n}\big)$ então
$A=D_\mu\big(D_\lambda(A)\big)$ e daí o mesmo argumento mostra que:
\begin{equation}\label{eq:incDlambdaCA2}
\vert\det D_\mu\vert\,\mathcal C\big(D_\lambda(A)\big)\subset\mathcal C(A).
\end{equation}
Como $\vert\det D_\mu\vert=\vert\det D_\lambda\vert^{-1}$, de \eqref{eq:incDlambdaCA1}
e \eqref{eq:incDlambdaCA2} vem:
\[\mathcal C\big(D_\lambda(A)\big)=\vert\det D_\lambda\vert\,\mathcal C(A).\]
Concluímos então que:
\[\leb^*\big(D_\lambda(A)\big)=\inf\,\mathcal C\big(D_\lambda(A)\big)
=\vert\det D_\lambda\vert\inf\,\mathcal C(A)=\vert\det D_\lambda\vert\,\leb^*(A).\]

\smallskip

\item[(c)] Se $A$ é mensurável então para todo $\varepsilon>0$ existe um aberto $U\subset\R^n$
contendo $A$ tal que $\leb^*(U\setminus A)<\varepsilon\,\vert\det D_\lambda\vert^{-1}$.
Daí $D_\lambda(U)$ é um aberto que contém $D_\lambda(A)$ e:
\[\leb^*\big(D_\lambda(U)\setminus D_\lambda(A)\big)=\leb^*\big(D_\lambda(U\setminus A)\big)
=\vert\det D_\lambda\vert\,\leb^*(U\setminus A)<\varepsilon,\]
provando que $D_\lambda(A)$ é mensurável.
\end{itemize}

\medskip

\textbf{Exercício~\ref{exe:AtriangleBzero}.}\enspace Temos $B\subset A\cup(B\setminus A)\subset
A\cup(A\bigtriangleup B)$ e portanto $\leb^*(B)\le\leb^*(A)+\leb^*(A\bigtriangleup B)=\leb^*(A)$.
De modo análogo mostra-se que $\leb^*(A)\le\leb^*(B)$ e portanto $\leb^*(A)=\leb^*(B)$.
Suponha agora que $A$ é mensurável. Então:
\begin{equation}\label{eq:BAABBA}
B=\big(A\setminus(A\setminus B)\big)\cup(B\setminus A).
\end{equation}
Como $A\setminus B\subset A\bigtriangleup B$ e $B\setminus A\subset A\bigtriangleup B$ então
$\leb^*(A\setminus B)=0$ e $\leb^*(B\setminus A)=0$. Segue do Lema~\ref{thm:nulamens}
que $A\setminus B$ e $B\setminus A$ são ambos mensuráveis; logo \eqref{eq:BAABBA} implica
que $B$ é mensurável. Da mesma forma mostra-se que a mensurabilidade de $B$ implica
na mensurabilidade de $A$.

\medskip

\textbf{Exercício~\ref{exe:mensaproxblocos}.}\enspace Seja $U\supset A$ um aberto tal que
$\leb(U\setminus A)<\frac\varepsilon2$. Pelo Lema~\ref{thm:abertocubos} podemos escrever
$U=\bigcup_{k=1}^\infty B_k$, onde $(B_k)_{k\ge1}$ é uma seqüência de blocos retangulares
$n$-dimensionais com interiores dois a dois disjuntos; pelo Corolário~\ref{thm:corinfinitosblocos}
temos:
\[\leb(U)=\sum_{k=1}^\infty\vert B_k\vert.\]
Note que $\leb(U)=\leb(U\setminus A)+\leb(A)<+\infty$
e portanto a série $\sum_{k=1}^\infty\vert B_k\vert$ é convergente; existe portanto
$t\ge1$ tal que $\sum_{k>t}\vert B_k\vert<\frac\varepsilon2$. Observe agora que:
\[\Big(\bigcup_{k=1}^tB_k\Big)\bigtriangleup A\subset(U\setminus A)\cup\Big(\bigcup_{k>t}B_k\Big)\]
e portanto:
\[\leb\Big(\big({\textstyle\bigcup_{k=1}^t B_k}\big)\bigtriangleup A\Big)\le
\leb(U\setminus A)+\sum_{k>t}\vert B_k\vert<\frac\varepsilon2+\frac\varepsilon2=\varepsilon.\]

\medskip

\textbf{Exercício~\ref{exe:mstarweakcontr}.}\enspace Temos $A\subset B\cup(A\setminus B)
\subset B\cup(A\bigtriangleup B)$ e portanto:
\[\leb^*(A)\le\leb^*(B)+\leb^*(A\bigtriangleup B).\]
Se $\leb^*(B)<+\infty$ segue que:
\begin{equation}\label{eq:mstarweakcontr1}
\leb^*(A)-\leb^*(B)\le\leb^*(A\bigtriangleup B);
\end{equation}
note que \eqref{eq:mstarweakcontr1} também é válida se $\leb^*(B)=+\infty$ já que, nesse caso,
$\leb^*(A)<+\infty$ e $\leb^*(A)-\leb^*(B)=-\infty$. Trocando os papéis de $A$ e $B$
em \eqref{eq:mstarweakcontr1} obtemos:
\begin{equation}\label{eq:mstarweakcontr2}
\leb^*(B)-\leb^*(A)\le\leb^*(A\bigtriangleup B).
\end{equation}
A conclusão segue de \eqref{eq:mstarweakcontr1} e \eqref{eq:mstarweakcontr2}.

\medskip

\textbf{Exercício~\ref{exe:novoenvelope}.}\enspace Temos:
\[\leb^*(A)\le\leb^*(E')\le\leb^*(E)=\leb(E)\]
com $\leb^*(A)=\leb(E)$ e portanto $\leb(E')=\leb^*(E')=\leb^*(A)$. Como $E'$ é mensurável e contém $A$,
segue que $E'$ é um envelope mensurável de $A$.

\medskip

\textbf{Exercício~\ref{exe:LebcomplBorel}.}\enspace Assuma que o conjunto $E$ é Lebesgue mensurável.
Pelo Corolário~\ref{thm:innerFsigma}, existe um subconjunto $A$ de $E$ de tipo $F_\sigma$
tal que $E\setminus A$ tem medida nula. Tome $N=E\setminus A$. Daí $E=A\cup N$ e pelo Lema~\ref{thm:aproxGdelta}
existe um subconjunto $M$ de $\R^n$ de tipo $G_\delta$ tal que $N\subset M$ e
$\leb(M)=\leb(N)=0$. Os conjuntos $A$ e $M$ são Boreleanos e portanto a condição~(b)
é satisfeita. Agora assuma que a condição~(b) é satisfeita. Temos que o conjunto $A$ é mensurável,
por ser Boreleano (Corolário~\ref{thm:corBormens}) e que o conjunto $N$ é mensurável,
já que $\leb^*(N)\le\leb(M)=0$ (Lema~\ref{thm:nulamens}). Logo $E=A\cup N$ é mensurável.

\medskip

\textbf{Exercício~\ref{exe:muAcupB}.}\enspace Temos que $A\cup B$ é união disjunta
dos conjuntos $A\setminus B$, $A\cap B$ e $B\setminus A$; logo:
\[\mu(A\cup B)=\mu(A\setminus B)+\mu(A\cap B)+\mu(B\setminus A).\]
Como $\mu(A\cap B)<+\infty$, segue do Lema~\ref{thm:muAminusB} que:
\[\mu(A\setminus B)=\mu\big(A\setminus(A\cap B)\big)=\mu(A)-\mu(A\cap B),\]
e similarmente $\mu(B\setminus A)=\mu(B)-\mu(A\cap B)$. Logo:
\begin{multline*}
\mu(A\cup B)=\mu(A)-\mu(A\cap B)+\mu(A\cap B)+\mu(B)-\mu(A\cap B)\\=\mu(A)+\mu(B)-\mu(A\cap B).
\end{multline*}

\medskip

\textbf{Exercício~\ref{exe:disjuntar}.}\enspace Note que $B_k\subset A_k$, para todo
$k\ge1$. Sejam $k,l\ge1$ com $k\ne l$, digamos, $k>l$.
Temos $B_k\cap A_l=\emptyset$ e $B_l\subset A_l$, de modo que $B_k\cap B_l=\emptyset$.
Isso prova que os conjuntos $(B_k)_{k\ge1}$ são dois a dois disjuntos.
Vamos mostrar que $\bigcup_{k=1}^\infty A_k=\bigcup_{k=1}^\infty B_k$. Obviamente,
$\bigcup_{k=1}^\infty B_k\subset\bigcup_{k=1}^\infty A_k$. Por outro lado, se $x\in\bigcup_{k=1}^\infty A_k$,
seja $k\ge1$ o menor inteiro tal que $x\in A_k$; daí $x\in A_k$ e $x\not\in\bigcup_{i=0}^{k-1}A_i$,
de modo que, $x\in B_k$.

\textbf{Exercício~\ref{exe:musubad}.}\enspace Sejam $B_k=A_k\setminus\bigcup_{i=0}^{k-1}A_i$,
para todo $k\ge1$, onde $A_0=\emptyset$. Note que $B_k\subset A_k$ e $B_k\in\mathcal A$ para todo $k\ge1$.
Pelo resultado do Exercício~\ref{exe:disjuntar}, os conjuntos $(B_k)_{k\ge1}$ são
dois a dois disjuntos e:
\[\bigcup_{k=1}^\infty A_k=\bigcup_{k=1}^\infty B_k.\]
Daí:
\begin{equation}\label{eq:musubad}
\mu\Big(\bigcup_{k=1}^\infty A_k\Big)=\mu\Big(\bigcup_{k=1}^\infty B_k\Big)
=\sum_{k=1}^\infty\mu(B_k)\le\sum_{k=1}^\infty\mu(A_k).
\end{equation}

\medskip

\textbf{Exercício~\ref{exe:quasedisjuntos}.}\enspace Definimos os conjuntos $B_k$, $k\ge1$, como na
resolução do Exercício~\ref{exe:musubad}. Por \eqref{eq:musubad}, é suficiente mostrarmos
que $\mu(B_k)=\mu(A_k)$ para todo $k\ge1$. Obviamente $\mu(B_k)\le\mu(A_k)$. Por outro lado, temos:
\[A_k\subset B_k\cup\bigcup_{i=0}^{k-1}(A_i\cap A_k);\]
aplicando o resultado do Exercício~\ref{exe:musubad} obtemos:
\[\mu(A_k)\le\mu(B_k)+\sum_{i=0}^{k-1}\mu(A_i\cap A_k)=\mu(B_k),\]
o que completa a demonstração.

\medskip

\textbf{Exercício~\ref{exe:sigmagerada}.}
\begin{itemize}
\item[(a)] Temos $X\in\mathcal A_i$ para todo $i\in I$, de modo que $X\in\mathcal A$ e $\mathcal A\ne\emptyset$.
Dado $A\in\mathcal A$ temos $A\in\mathcal A_i$ para todo $i\in I$ e portanto $A^\compl\in\mathcal A_i$,
para todo $i\in I$; segue que $A^\compl\in\mathcal A$. Seja $(A_k)_{k\ge1}$ uma seqüência de elementos
de $\mathcal A$. Daí $A_k\in\mathcal A_i$ para todo $k\ge1$ e todo $i\in I$, de modo que
$\bigcup_{k=1}^\infty A_k\in\mathcal A_i$ para todo $i\in I$ e portanto $\bigcup_{k=1}^\infty A_k\in\mathcal A$.

\smallskip

\item[(b)] Se $\sigma_1[\mathcal C]$ e $\sigma_2[\mathcal C]$ são ambas $\sigma$-álgebras de
partes de $X$ satisfazendo as propriedades \eqref{itm:sigma1} e \eqref{itm:sigma2} que
aparecem na Definição~\ref{thm:defsigmagerada}, mostremos que $\sigma_1[\mathcal C]=\sigma_2[\mathcal C]$.
De fato, como $\sigma_1[\mathcal C]$ é uma $\sigma$-álgebra de partes de $X$ que contém $\mathcal C$
e como $\sigma_2[\mathcal C]$ satisfaz a propriedade \eqref{itm:sigma2}, temos que
$\sigma_2[\mathcal C]\subset\sigma_1[\mathcal C]$. De modo similar mostra-se que
$\sigma_1[\mathcal C]\subset\sigma_2[\mathcal C]$.

\smallskip

\item[(c)] Seja $\sigma[\mathcal C]$ a interseção de todas as $\sigma$-álgebras de partes
de $X$ que contém $\mathcal C$; pelo resultado do item (a), $\sigma[\mathcal C]$ é uma $\sigma$-álgebra
de partes de $X$ e obviamente $\mathcal C\subset\sigma[\mathcal C]$, já que $\sigma[\mathcal C]$
é a interseção de uma coleção de conjuntos que contém $\mathcal C$. Além do mais, se $\mathcal A$
é uma $\sigma$-álgebra de partes de $X$ que contém $\mathcal C$ então $\mathcal A$ é um dos membros
da coleção cuja interseção resultou em $\sigma[\mathcal C]$; logo $\sigma[\mathcal C]\subset\mathcal A$.
\end{itemize}

\medskip

\textbf{Exercício~\ref{exe:relacionargeradas}.}\enspace Como $\sigma[\mathcal C_2]$
é uma $\sigma$-álgebra de partes de $X$ que contém $\mathcal C_1$ e como $\sigma[\mathcal C_1]$
satisfaz a propriedade \eqref{itm:sigma2} que aparece na Definição~\ref{thm:defsigmagerada}
temos que $\sigma[\mathcal C_1]\subset\sigma[\mathcal C_2]$. Similarmente,
$\mathcal C_2\subset\sigma[\mathcal C_1]$ implica que $\sigma[\mathcal C_2]\subset\sigma[\mathcal C_1]$.

\medskip

\textbf{Exercício~\ref{exe:GdeltaFsigmaBorel}.}\enspace A $\sigma$-álgebra de Borel de $\R^n$
é uma $\sigma$-álgebra de partes de $\R^n$ que contém os abertos de $\R^n$. Logo todo aberto de $\R^n$ e
toda interseção enumerável de abertos de $\R^n$ pertence à $\sigma$-álgebra de Borel de $\R^n$
(veja Lema~\ref{thm:propalgebras}). Como todo fechado é complementar de um aberto, segue que
os fechados de $\R^n$ e as uniões enumeráveis de fechados de $\R^n$ pertencem à $\sigma$-álgebra
de Borel de $\R^n$.

\medskip

\textbf{Exercício~\ref{exe:BorelRgeradores}.}\enspace Seja $\mathcal A$ a $\sigma$-álgebra
gerada pelos intervalos da forma $\left]a,b\right]$, com $a<b$, $a,b\in\R$. Como a $\sigma$-álgebra de
Borel $\Borel(\R)$ é a $\sigma$-álgebra gerada pelos abertos de $\R$, o resultado do Exercício~\ref{exe:relacionargeradas}
nos diz que, para mostrar que $\mathcal A=\Borel(\R)$, é suficiente mostrar as seguintes afirmações:
\begin{itemize}
\item[(i)] todo intervalo da forma $\left]a,b\right]$ é um Boreleano de $\R$;
\item[(ii)] todo aberto de $\R$ pertence a $\mathcal A$.
\end{itemize}
A afirmação (i) é trivial, já que $\left]a,b\right]=\left]a,b\right[\cup\{b\}$,
onde $\left]a,b\right[$ é um subconjunto aberto de $\R$ e $\{b\}$ é um subconjunto fechado de $\R$.
Para mostrar a afirmação (ii), observe que o Lema~\ref{thm:abertocubos} implica que todo aberto
de $\R$ é uma união enumerável de intervalos compactos; é suficiente mostrar então que
$[a,b]\in\mathcal A$, para todos $a,b\in\R$ com $a\le b$. Mas isso segue
da igualdade:
\[[a,b]=\bigcap_{k=1}^\infty\left]a-\tfrac1k,b\right].\]
Isso termina a resolução do item~(a). Para o item~(b), simplesmente observe que:
\[\left]a,b\right]=\left]-\infty,b\right]\setminus\left]-\infty,a\right],\]
e portanto a $\sigma$-álgebra gerada pelos intervalos $\left]-\infty,c\right]$ contém
a $\sigma$-álgebra gerada pelos intervalos $\left]a,b\right]$.

\medskip

\textbf{Exercício~\ref{exe:rarotudonaoda}.}\enspace Suponha por absurdo que $F$ é um fechado
de $\R$ contido propriamente em $I$ com $\leb(F)=\vert I\vert$. Seja $x\in I\setminus F$.
Como $F$ é fechado, existe $\varepsilon>0$ com $[x-\varepsilon,x+\varepsilon]\cap F=\emptyset$.
Se $x$ é um ponto interior de $I$ então podemos escolher $\varepsilon>0$ de modo que
$[x-\varepsilon,x+\varepsilon]\subset I$; senão, se $x$ é uma extremidade de $I$,
podemos ao menos garantir que um dos intervalos $[x-\varepsilon,x]$, $[x,x+\varepsilon]$
está contido em $I$, para $\varepsilon>0$ suficientemente pequeno. Em todo caso,
conseguimos um intervalo $J$ contido em $I$, disjunto de $F$, com $\vert J\vert>0$.
Daí $F$ e $J$ são subconjuntos mensuráveis disjuntos de $I$ e portanto:
\[\vert I\vert=\leb(I)\ge\leb(F\cup J)=\leb(F)+\leb(J)=\vert I\vert+\vert J\vert>\vert I\vert,\]
o que nos dá uma contradição e prova que $F=I$. Em particular, vemos que $F$ não pode ter interior vazio.

\medskip

\textbf{Exercício~\ref{exe:medLebproduto}.}
\begin{itemize}
\item[(a)] Consideramos primeiro o caso em que $A$ e $B$ têm medida exterior finita.
Seja dado $\varepsilon>0$ e sejam $(Q_k)_{k\ge1}$ e $(Q'_l)_{l\ge1}$ respectivamente
uma seqüência de blocos retangulares $m$-dimensionais e uma seqüência de blocos retangulares
$n$-dimensionais tais que:
\[A\subset\bigcup_{k=1}^\infty Q_k,\quad B\subset\bigcup_{l=1}^\infty Q'_l\]
e tais que:
\[\sum_{k=1}^\infty\vert Q_k\vert<\leb^*(A)+\varepsilon,\quad\sum_{l=1}^\infty\vert Q'_l\vert<\leb^*(B)+\varepsilon.\]
Daí $(Q_k\times Q'_l)_{k,l\ge1}$ é uma família enumerável de blocos retangulares $(m+n)$-dimensionais
tal que $A\times B\subset\bigcup_{k,l\ge1}(Q_k\times Q'_l)$. Logo:
\begin{multline*}
\leb^*(A\times B)\le\sum_{k,l\ge1}\vert Q_k\times Q'_l\vert=\sum_{k,l\ge1}\vert Q_k\vert\,\vert Q'_l\vert
=\Big(\sum_{k=1}^\infty\vert Q_k\vert\Big)\Big(\sum_{l=1}^\infty\vert Q'_l\vert\Big)\\
<\big(\leb^*(A)+\varepsilon\big)\big(\leb^*(B)+\varepsilon\big).
\end{multline*}
A conclusão é obtida fazendo $\varepsilon\to0$. Consideramos agora o caso que $\leb^*(A)=+\infty$
ou $\leb^*(B)=+\infty$. Se $\leb^*(A)>0$ e $\leb^*(B)>0$ então $\leb^*(A)\leb^*(B)=+\infty$ e não há nada para mostrar.
Suponha então que $\leb^*(A)=0$ ou $\leb^*(B)=0$, de modo que $\leb^*(A)\leb^*(B)=0$; devemos mostrar então que
$\leb^*(A\times B)=0$ também. Consideraremos apenas o caso que $\leb^*(A)=+\infty$ e $\leb^*(B)=0$ (o caso $\leb^*(A)=0$
e $\leb^*(B)=+\infty$ é análogo). Para cada $k\ge1$, seja $A_k=A\cap[-k,k]^m$. Temos
$A=\bigcup_{k=1}^\infty A_k$ e $\leb^*(A_k)<+\infty$, para todo $k\ge1$. Logo:
\[0\le\leb^*(A_k\times B)\le\leb^*(A_k)\leb^*(B)=0,\]
ou seja, $\leb^*(A_k\times B)=0$, para todo $k\ge1$. Como:
\[A\times B=\bigcup_{k=1}^\infty(A_k\times B),\]
segue que $\leb^*(A\times B)=0$.

\smallskip

\item[(b)] Consideramos primeiro o caso que $\leb(A)<+\infty$ e $\leb(B)<+\infty$.
Dado $\varepsilon>0$, existem abertos $U\subset\R^m$ e $V\subset\R^n$ contendo $A$ e $B$
respectivamente, de modo que $\leb(U)<\leb(A)+1$, $\leb(V)<\leb(B)+1$ e:
\[\leb(U\setminus A)<\frac\varepsilon{2\big(\leb(B)+1\big)},
\quad\leb(V\setminus B)<\frac\varepsilon{2\big(\leb(A)+1\big)}.\]
Daí $U\times V$ é um aberto de $\R^{m+n}$ contendo $A\times B$; além do mais:
\[(U\times V)\setminus(A\times B)\subset\big[(U\setminus A)\times V\big]\cup\big[U\times(V\setminus B)\big].\]
Usando o resultado do item (a) obtemos portanto:
\begin{multline*}
\leb^*\big((U\times V)\setminus(A\times B)\big)\le\leb^*\big((U\setminus A)\times V\big)+\leb^*\big(U\times(V\setminus B)\big)\\
\le\leb(U\setminus A)\leb(V)+\leb(U)\leb(V\setminus B)\\\le
\leb(U\setminus A)\big(\leb(B)+1\big)+\leb(V\setminus B)\big(\leb(A)+1\big)<\varepsilon,
\end{multline*}
o que mostra que $A\times B$ é mensurável. Para o caso geral, definimos $A_k=A\cap[-k,k]^m$, $B_k=B\cap[-k,k]^n$.
Daí $A_k\times B_k$ é mensurável para todo $k\ge1$ e $A\times B=\bigcup_{k=1}^\infty(A_k\times B_k)$; portanto
também $A\times B$ é mensurável.

\smallskip

\item[(c)] Mostremos primeiro que se $U\subset\R^m$, $V\subset\R^n$ são abertos então:
\begin{equation}\label{eq:UtimesV}
\leb(U\times V)=\leb(U)\leb(V).
\end{equation}
Pelo Lema~\ref{thm:abertocubos} podemos escrever $U=\bigcup_{k=1}^\infty Q_k$, onde $(Q_k)_{k\ge1}$ é uma seqüência
de blocos retangulares $m$-dimensionais com interiores dois a dois disjuntos; podemos também escrever $V=\bigcup_{l=1}^\infty Q'_l$,
onde $(Q'_l)_{l\ge1}$ é uma seqüência de blocos retangulares $n$-dimensionais com interiores dois a dois disjuntos.
Note que $(Q_k\times Q'_l)_{k,l\ge1}$ é uma família enumerável de blocos retangulares $(m+n)$-dimensionais com interiores
dois a dois disjuntos e $U\times V=\bigcup_{k,l\ge1}(Q_k\times Q'_l)$. Daí, pelo Corolário~\ref{thm:corinfinitosblocos}, obtemos:
\begin{multline*}
\leb(U\times V)=\sum_{k,l\ge1}\vert Q_k\times Q'_l\vert=\sum_{k,l\ge1}\vert Q_k\vert\,\vert Q'_l\vert
=\Big(\sum_{k=1}^\infty\vert Q_k\vert\Big)\Big(\sum_{l=1}^\infty\vert Q'_l\vert\Big)\\=\leb(U)\leb(V).
\end{multline*}
Isso prova \eqref{eq:UtimesV}.
Dados agora $A\subset\R^m$, $B\subset\R^n$ mensuráveis com $\leb(A)<+\infty$ e $\leb(B)<+\infty$ podemos,
como no item (b), obter abertos $U\subset\R^m$, $V\subset\R^n$ contendo $A$ e $B$ respectivamente de modo que:
\[\leb^*\big((U\times V)\setminus(A\times B)\big)<\varepsilon.\]
Como os conjuntos $U\times V$ e $A\times B$ são mensuráveis e, pelo item (a), $\leb(A\times B)\le\leb(A)\leb(B)<+\infty$,
obtemos:
\[\leb\big((U\times V)\setminus(A\times B)\big)=\leb(U\times V)-\leb(A\times B),\]
e portanto $\leb(U\times V)-\leb(A\times B)<\varepsilon$. Usando agora \eqref{eq:UtimesV} concluímos que:
\[\leb(A\times B)>\leb(U\times V)-\varepsilon=\leb(U)\leb(V)-\varepsilon\ge\leb(A)\leb(B)-\varepsilon;\]
fazendo $\varepsilon\to0$, obtemos $\leb(A\times B)\ge\leb(A)\leb(B)$. Provamos então a igualdade
$\leb(A\times B)=\leb(A)\leb(B)$, já que a desigualdade oposta já foi provada no item (a).
Sejam agora $A\subset\R^m$, $B\subset\R^n$ conjuntos mensuráveis arbitrários e defina:
\[A_k=A\cap[-k,k]^m,\quad B_k=B\cap[-k,k]^n,\]
para todo $k\ge1$. Daí $A_k\nearrow A$, $B_k\nearrow B$, $A_k\times B_k\nearrow A\times B$ e portanto:
\[\leb(A\times B)=\lim_{k\to\infty}\leb(A_k\times B_k)=\lim_{k\to\infty}\leb(A_k)\leb(B_k)
=\leb(A)\leb(B),\]
onde na última igualdade usamos o resultado do Exercício~\ref{exe:prodcresseq}.
\end{itemize}

\medskip

\textbf{Exercício~\ref{exe:lebintlelebext}.}\enspace Se $K\subset A$ é compacto então
$\leb(K)=\leb^*(K)\le\leb^*(A)$, pelo Lema~\ref{thm:lebmonotonica}.
Logo $\leb^*(A)$ é uma cota superior do conjunto:
\[\big\{\leb(K):\text{$K\subset A$ compacto}\big\}\]
e portanto é maior ou igual ao seu supremo, que é $\leb_*(A)$.

\medskip

\textbf{Exercício~\ref{exe:medintmonot}.}\enspace Observe que:
\[\big\{\leb(K):\text{$K\subset A_1$ compacto}\big\}\subset
\big\{\leb(K):\text{$K\subset A_2$ compacto}\big\}\]
e portanto:
\begin{multline*}
\leb_*(A_1)=\sup\big\{\leb(K):\text{$K\subset A_1$ compacto}\big\}\\
\le\sup\big\{\leb(K):\text{$K\subset A_2$ compacto}\big\}=\leb_*(A_2).
\end{multline*}

\medskip

\textbf{Exercício~\ref{exe:alternativamedint}.}\enspace Se $\mathcal M'\subset\Lebmens(\R^n)$
contém todos os subconjuntos compactos de $\R^n$ então:
\[\big\{\leb(K):\text{$K\subset A$ compacto}\big\}\subset\big\{\leb(E):E\subset A,\ E\in\mathcal M'\big\}\]
e portanto:
\[\leb_*(A)=\sup\big\{\leb(K):\text{$K\subset A$ compacto}\big\}\le
\sup\big\{\leb(E):E\subset A,\ E\in\mathcal M'\big\}.\]
Por outro lado, se $E\in\mathcal M'$ e $E\subset A$ então segue do Corolário~\ref{thm:mensdentronaomens} que:
\[\leb(E)\le\leb_*(A);\]
isso mostra que $\leb_*(A)$ é uma cota superior do conjunto:
\[\big\{\leb(E):E\subset A,\ E\in\mathcal M'\big\}\]
e portanto $\leb_*(A)\ge\sup\big\{\leb(E):E\subset A,\ E\in\mathcal M'\big\}$.

\medskip

\textbf{Exercício~\ref{exe:internalmens}.}\enspace Se $\leb_*(A)<+\infty$ então
para todo $r\ge1$ existe um compacto $K_r\subset A$ com $\leb(K_r)>\leb_*(A)-\frac1r$;
daí $W=\bigcup_{r=1}^\infty K_r$ é um $F_\sigma$ contido em $A$ e:
\[\leb_*(A)-\frac1r<\leb(K_r)\le\leb(W)\le\leb_*(A),\]
para todo $r\ge1$, onde usamos o Corolário~\ref{thm:mensdentronaomens}.
Segue que $\leb(W)=\leb_*(A)$. Se $\leb_*(A)=+\infty$ então para todo $r\ge1$ existe um compacto
$K_r\subset A$ com $\leb(K_r)>r$ e daí $W=\bigcup_{r=1}^\infty K_r$ é um $F_\sigma$ contido em
$A$ tal que:
\[\leb(W)\ge\leb(K_r)>r,\]
para todo $r\ge1$; logo $\leb(W)=+\infty=\leb_*(A)$.

\medskip

\textbf{Exercício~\ref{exe:intersuperadd}.}\enspace Para cada $k\ge1$, seja $W_k\subset\R^n$ um subconjunto
de tipo $F_\sigma$ tal que $W_k\subset A_k$ e $\leb(W_k)=\leb_*(A_k)$ (veja Exercício~\ref{exe:internalmens}).
Como os conjuntos $W_k$ são dois a dois disjuntos e mensuráveis, temos:
\[\leb\Big(\bigcup_{k=1}^\infty W_k\Big)=\sum_{k=1}^\infty\leb(W_k)=\sum_{k=1}^\infty\leb_*(A_k).\]
Mas $\bigcup_{k=1}^\infty W_k$ é um subconjunto mensurável de $\bigcup_{k=1}^\infty A_k$ e portanto
o Corolário~\ref{thm:mensdentronaomens} nos dá:
\[\leb_*\Big(\bigcup_{k=1}^\infty A_k\Big)\ge\leb\Big(\bigcup_{k=1}^\infty W_k\Big)=\sum_{k=1}^\infty\leb_*(A_k).\]

\medskip

\textbf{Exercício~\ref{exe:AksearrowA}.}\enspace O resultado do Exercício~\ref{exe:medintmonot}
implica que $\big(\leb_*(A_k)\big)_{k\ge1}$ é uma seqüência decrescente e que
$\leb_*(A_k)\ge\leb_*(A)$, para todo $k\ge1$; logo $\big(\leb_*(A_k)\big)_{k\ge1}$
é convergente e:
\[\lim_{k\to\infty}\leb_*(A_k)\ge\leb_*(A).\]
Para cada $k\ge1$, o resultado do Exercício~\ref{exe:internalmens}
nos dá um subconjunto $W_k$ de $A_k$ de tipo $F_\sigma$ tal que $\leb(W_k)=\leb_*(A_k)$.
Defina $V_k=\bigcup_{r=k}^\infty W_r$. Daí $V_k$ é mensurável e $W_k\subset V_k\subset A_k$,
donde:
\[\leb_*(A_k)=\leb(W_k)\le\leb(V_k)\le\leb_*(A_k),\]
onde na última desigualdade usamos o Corolário~\ref{thm:mensdentronaomens}. Mostramos então que $\leb(V_k)=\leb_*(A_k)$,
para todo $k\ge1$. Obviamente $V_k\supset V_{k+1}$ para todo $k\ge1$ e:
\[\bigcap_{k=1}^\infty V_k\subset\bigcap_{k=1}^\infty A_k=A.\]
Como $\leb(V_k)=\leb_*(A_k)<+\infty$ para algum $k\ge1$, o Lema~\ref{thm:setlimits}
nos dá:
\[\lim_{k\to\infty}\leb(V_k)=\leb\Big(\bigcap_{k=1}^\infty V_k\Big)\le\leb_*(A),\]
e portanto:
\[\lim_{k\to\infty}\leb_*(A_k)\le\leb_*(A).\]

\end{section}

\begin{section}{Exercícios do Capítulo~\ref{CHP:INTEGRAL}}

\textbf{Exercício~\ref{exe:constmens}.}\enspace Se $f:X\to X'$ é constante então para todo subconjunto
$A$ de $X'$ temos $f^{-1}(A)=\emptyset$ ou $f^{-1}(A)=X$; em todo caso, $f^{-1}(A)\in\mathcal A$.

\medskip

\textbf{Exercício~\ref{exe:AvertY}.}\enspace Evidentemente $\mathcal A\vert_Y$ é não
vazia, já que $\mathcal A$ é não vazia. Seja $(A'_k)_{k\ge1}$ uma seqüência em $\mathcal A\vert_Y$;
para cada $k\ge1$ existe $A_k\in\mathcal A$ com $A'_k=A_k\cap Y$. Daí:
\[\bigcup_{k=1}^\infty A'_k=\Big(\bigcup_{k=1}^\infty A_k\Big)\cap Y\]
e $\bigcup_{k=1}^\infty A_k\in\mathcal A$; logo $\bigcup_{k=1}^\infty A'_k\in\mathcal A\vert_Y$.
Agora seja $A'\in\mathcal A\vert_Y$, de modo que $A'=A\cap Y$, com $A\in\mathcal A$.
Temos que o complementar de $A'$ em $Y$ é igual à interseção do complementar de $A$
em $X$ com $Y$, ou seja:
\[Y\setminus A'=Y\setminus(A\cap Y)=(X\setminus A)\cap Y.\]
Como $X\setminus A$ está em $\mathcal A$, segue que $Y\setminus A'\in\mathcal A\vert_Y$.

\medskip

\textbf{Exercício~\ref{exe:gerarestrita}.}\enspace Pelo resultado do Exercício~\ref{exe:AvertY},
temos que $\mathcal A\vert_Y$ é uma $\sigma$-álgebra de partes de $Y$ que contém $\mathcal C\vert_Y$; logo $\mathcal A\vert_Y$
contém $\sigma[\mathcal C\vert_Y]$. Para mostrar que $\mathcal A\vert_Y$ está contido em
$\sigma[\mathcal C\vert_Y]$, considere a coleção:
\[\mathcal A'=\big\{A\subset X:A\cap Y\in\sigma[\mathcal C\vert_Y]\big\}.\]
Verifica-se diretamente que $\mathcal A'$ é uma $\sigma$-álgebra de partes de $X$;
obviamente, $\mathcal C\subset\mathcal A'$. Logo $\mathcal A\subset\mathcal A'$, o que
prova que $A\cap Y\in\sigma[\mathcal C\vert_Y]$, para todo $A\in\mathcal A$, i.e.,
$\mathcal A\vert_Y\subset\sigma[\mathcal C\vert_Y]$.

\medskip

\textbf{Exercício~\ref{exe:BorelbarRrestrR}.}\enspace De acordo com a definição da $\sigma$-álgebra
de Borel de $\overline\R$, se $A\in\Borel(\overline\R)$ então $A\cap\R\in\Borel(\R)$;
logo $\Borel(\overline\R)\vert_{\R}\subset\Borel(\R)$. Por outro lado, se $A\in\Borel(\R)$ então
também $A\in\Borel(\overline\R)$ (já que $A\cap\R=A$ é um Boreleano de $\R$) e portanto
$A\cap\R=A\in\Borel(\overline\R)\vert_{\R}$.

\medskip

\textbf{Exercício~\ref{exe:geradoresbarR}.}\enspace Seja $\mathcal C$ a coleção formada pelos
intervalos da forma $[-\infty,c]$, $c\in\R$. Claramente $\mathcal C\subset\Borel(\overline\R)$ e
portanto $\sigma[\mathcal C]\subset\Borel(\overline\R)$. Vamos mostrar então que $\Borel(\overline\R)\subset\sigma[\mathcal C]$.
Em primeiro lugar, afirmamos que:
\begin{gather}
\emptyset,\{+\infty\},\{-\infty\},\{+\infty,-\infty\}\in\sigma[\mathcal C],\label{eq:exe:geradoresbarR1}\\
\R\in\sigma[\mathcal C].\label{eq:exe:geradoresbarR2}
\end{gather}
De fato, \eqref{eq:exe:geradoresbarR1} segue das igualdades:
\[\{-\infty\}=\bigcap_{k=1}^\infty[-\infty,-k],\quad\{+\infty\}=\bigcap_{k=1}^\infty[-\infty,k]^\compl,\]
e \eqref{eq:exe:geradoresbarR2} segue de \eqref{eq:exe:geradoresbarR1}, já que
$\R=\{+\infty,-\infty\}^\compl$.
Note que:
\[\mathcal C\vert_{\R}=\big\{\left]-\infty,c\right]:c\in\R\big\}\]
e portanto o resultado do Exercício~\ref{exe:BorelRgeradores} nos dá $\sigma[\mathcal C\vert_{\R}]=\Borel(\R)$;
daí, o resultado do Exercício~\ref{exe:gerarestrita} implica que:
\begin{equation}\label{eq:exe:geradoresbarR}
\sigma[\mathcal C]\vert_{\R}=\Borel(\R).
\end{equation}
Seja $A\in\Borel(\overline\R)$, de modo que $A\cap\R\in\Borel(\R)$. Por \eqref{eq:exe:geradoresbarR},
temos que existe $A'\in\sigma[\mathcal C]$ tal que $A\cap\R=A'\cap\R$. Daí \eqref{eq:exe:geradoresbarR2}
implica que $A\cap\R\in\sigma[\mathcal C]$. Finalmente, \eqref{eq:exe:geradoresbarR1} implica
que $A\cap\{+\infty,-\infty\}\in\sigma[\mathcal C]$, o que prova que
$A=(A\cap\R)\cup\big(A\cap\{+\infty,-\infty\}\big)\in\sigma[\mathcal C]$.

\medskip

\textbf{Exercício~\ref{exe:figualg}.}\enspace Pelo Corolário~\ref{thm:corsomaprodmens}, a função
\[h:\big(f^{-1}(\R)\cap g^{-1}(\R)\big)\longrightarrow\R\]
definida por $h(x)=f(x)-g(x)$ é mensurável. Logo o conjunto:
\[h^{-1}(0)=\big\{x\in f^{-1}(\R)\cap g^{-1}(\R):f(x)=g(x)\big\}\]
é mensurável. A conclusão segue da igualdade:
\begin{multline*}
\big\{x\in X:f(x)=g(x)\big\}=\big(f^{-1}(+\infty)\cap g^{-1}(+\infty)\big)\cup
\big(f^{-1}(-\infty)\cap g^{-1}(-\infty)\big)\\
\cup\big\{x\in f^{-1}(\R)\cap g^{-1}(\R):f(x)=g(x)\big\}.
\end{multline*}

\medskip

\textbf{Exercício~\ref{exe:funcaomaluca}.}\enspace Vamos usar o Lema~\ref{thm:cobremensf}.
Temos que os conjuntos:
\begin{subequations}
\begin{gather}
\big\{(x,y)\in\R^2:y\ge1\big\}\label{eq:funcaomalucaa},\\
\big\{(x,y)\in\R^2:-1<y<1\big\}\label{eq:funcaomalucab},\\
\big\{(x,y)\in\R^2:y\le-1\big\}\label{eq:funcaomalucac},
\end{gather}
\end{subequations}
constituem uma cobertura enumerável de $\R^2$ por Boreleanos. É suficiente então mostrar
que a restrição de $f$ a cada um desses Boreleanos é Borel mensurável. A restrição de $f$
ao conjunto \eqref{eq:funcaomalucaa} é contínua, e portanto Borel mensurável (veja Lema~\ref{thm:contmens}).
A restrição de $f$ ao conjunto \eqref{eq:funcaomalucab} é um limite pontual de funções contínuas
e portanto é Borel mensurável, pelo Corolário~\ref{thm:limmensmens} (na verdade, essa restrição de $f$
também é contínua, já que a série em questão converge uniformemente, pelo teste $M$ de Weierstrass).
Finalmente, a restrição de $f$ ao conjunto \eqref{eq:funcaomalucac} é Borel mensurável, sendo
igual à composição da função contínua $(x,y)\mapsto x+y$ com a função Borel mensurável
$\chilow{\mathbb Q}$.

\medskip

\textbf{Exercício~\ref{exe:mensqs}.}
\begin{itemize}
\item[(a)] Como $X\setminus X_1$ tem medida nula, temos que todo subconjunto de $X\setminus X_1$
é mensurável (recorde Lema~\ref{thm:nulamens}). Portanto, a restrição de $f$ a $X\setminus X_1$
é automaticamente mensurável (seja lá qual for a função $f$). Como os conjuntos $X\setminus X_1$ e
$X_1=X\setminus(X\setminus X_1)$ são mensuráveis, segue do Lema~\ref{thm:cobremensf} que
$f$ é mensurável.

\smallskip

\item[(b)] Como $f=g$ quase sempre, existe um subconjunto $X_1$ de $X$ tal que
$X\setminus X_1$ tem medida nula e tal que $f$ e $g$ coincidem em $X_1$. Como
$f$ é mensurável, segue que $g\vert_{X_1}=f\vert_{X_1}$ também é mensurável; logo,
o resultado do item (a) implica que $g$ é mensurável.

\smallskip

\item[(c)] Basta observar que $g=\liminf_{k\to\infty}f_k$ quase sempre e usar o resultado
do item (b) juntamente com o Corolário~\ref{thm:liminfsupmens}.
\end{itemize}

\medskip

\textbf{Exercício~\ref{thm:projLebmens}.}\enspace Devemos mostrar que se $A$ é um subconjunto
Lebesgue mensurável de $\R^m$ então $\pi^{-1}(A)$ é um subconjunto Lebesgue mensurável
de $\R^{m+n}$. Mas $\pi^{-1}(A)=A\times\R^n$ e portanto a conclusão segue do
resultado do item (b) do Exercício~\ref{exe:medLebproduto}.

\medskip

\textbf{Exercício~\ref{thm:graficomens}.}\enspace Considere a função $\phi:X\times\R^n\to\R^n$
definida por $\phi(x,y)=y-f(x)$, para todos $x\in X$, $y\in\R^n$. Obviamente:
\[\Gr(f)=\phi^{-1}(0).\]
Considere a projeção $\pi:\R^{m+n}\to\R^m$ nas primeiras $m$ coordenadas. Temos que $\pi$
é contínua e portanto Borel mensurável; daí $X\times\R^n=\pi^{-1}(X)$ é Boreleano, caso $X$
seja Boreleano. Além do mais, pelo resultado do Exercício~\ref{thm:projLebmens}, $X\times\R^n$
é Lebesgue mensurável, caso $X$ seja Lebesgue mensurável. Para concluir a demonstração, vamos
verificar que:
\begin{itemize}
\item $\phi$ é Borel mensurável se $f$ for Borel mensurável;
\item $\phi$ é mensurável se $f$ for mensurável.
\end{itemize}
De fato, temos que $\phi$ é igual à diferença entre a função contínua $(x,y)\mapsto y$
e a função $(x,y)\mapsto f(x)$, que é simplesmente a composição da restrição de $\pi$
a $X\times\R^n$ com $f$. A conclusão segue do resultado do Exercício~\ref{thm:projLebmens}.

\medskip

\textbf{Exercício~\ref{exe:moduloint}.}
\begin{itemize}
\item[(a)] Se $f$ é integrável então, por definição, $f^+$ e $f^-$ são integráveis,
donde $\vert f\vert=f^++f^-$ é integrável. Reciprocamente, se $\vert f\vert$ é integrável
então $f^+$ e $f^-$ são integráveis, já que $0\le f^+\le\vert f\vert$ e $0\le f^-\le\vert f\vert$.
Segue que $f$ é integrável.

\smallskip

\item[(b)] Temos:
\begin{multline*}
\Big\vert\int_Xf\,\dd\mu\Big\vert=\Big\vert\int_Xf^+\,\dd\mu-\int_Xf^-\,\dd\mu\Big\vert
\le\Big\vert\int_Xf^+\,\dd\mu\Big\vert+\Big\vert\int_Xf^-\,\dd\mu\Big\vert\\
=\int_Xf^++f^-\,\dd\mu=\int_X\vert f\vert\,\dd\mu.
\end{multline*}
\end{itemize}

\medskip

\textbf{Exercício~\ref{exe:intserienneg}.}\enspace Seja $g_n=\sum_{k=1}^nf_k$. Daí
$(g_n)_{n\ge1}$ é uma seqüência de funções mensuráveis não negativas com $g_n\nearrow f$.
Segue do Teorema~\ref{thm:monotonicanneg} que:
\[\sum_{k=1}^\infty\int_Xf_k\,\dd\mu=\lim_{n\to\infty}\sum_{k=1}^n\int_Xf_k\,\dd\mu
=\lim_{n\to\infty}\int_Xg_n\,\dd\mu=\int_Xf\,\dd\mu.\]

\medskip

\textbf{Exercício~\ref{exe:integralindef}.}\enspace Obviamente $\nu_f(\emptyset)=0$,
pelo Lema~\ref{thm:muXzero}. Seja $(E_k)_{k\ge1}$ uma seqüência de subconjuntos mensuráveis
dois a dois disjuntos de $X$. Temos:
\[f\chilow E=\sum_{k=1}^\infty f\chilow{E_k},\]
e portanto o Lema~\ref{thm:vertYchiYnneg} e o resultado do Exercício~\ref{exe:intserienneg}
implicam:
\[\sum_{k=1}^\infty\nu_f(E_k)=\sum_{k=1}^\infty\int_Xf\chilow{E_k}\,\dd\mu
=\int_Xf\chilow E\,\dd\mu=\nu_f(E).\]

\medskip

\textbf{Exercício~\ref{exe:propintegralmedida}.}
\begin{itemize}
\item[(a)] Se a função $f$ é não negativa, a afirmação segue do resultado do Exercício~\ref{exe:integralindef}.
No caso geral, temos:
\[\int_Af^+\,\dd\mu=\sum_{k=1}^\infty\int_{A_k}f^+\,\dd\mu,\quad
\int_Af^-\,\dd\mu=\sum_{k=1}^\infty\int_{A_k}f^-\,\dd\mu,\]
e a conclusão segue subtraindo as duas igualdades acima.

\smallskip

\item[(b)] Se a função $f$ é não negativa, a afirmação segue do resultado do Exercício~\ref{exe:integralindef}
e do Lema~\ref{thm:setlimits}. No caso geral, temos:
\[\int_Af^+\,\dd\mu=\lim_{k\to\infty}\int_{A_k}f^+\,\dd\mu,\quad
\int_Af^-\,\dd\mu=\lim_{k\to\infty}\int_{A_k}f^-\,\dd\mu,\]
e a conclusão segue subtraindo as duas igualdades acima.

\smallskip

\item[(c)] Análogo ao item (b), observando que se $f\vert_{A_1}$ é integrável então
$\int_{A_1}f^+\,\dd\mu<+\infty$ e $\int_{A_1}f^-\,\dd\mu<+\infty$.
\end{itemize}

\end{section}

\end{chapter}
\endgroup

\backmatter

% bibliografia

\bibliographystyle{plain}
\begin{thebibliography}{99}

\bibitem{Mendelson} E. Mendelson, {\em Introduction to mathematical logic}, Chapman \& Hall, London, 1997, x+440 pgs.

\end{thebibliography}

\iflatexml
\printindex
\else
\printindex[simbolos]
\printindex[indice]
\fi

%\printindex{A}{Lista de Símbolos}
%\printindex{B}{Índice Remissivo}

\end{document}
